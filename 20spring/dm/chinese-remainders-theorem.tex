\documentclass{article}

\usepackage[T2A]{fontenc}
\usepackage[utf8]{inputenc}
\usepackage[russian]{babel}
\usepackage{amsmath}
\usepackage{hyperref}
\usepackage{framed}
\usepackage{subfiles}

\newenvironment{exercise}{%
    \begin{framed}\par\noindent\slshape%
}%
{\end{framed}}
\newcommand{\eqmod}[1]{\mathbin{\underset{#1}\equiv}}

\title{Китайская теорема об остатках}
\author{}
\date{}

\begin{document}
\maketitle

\section{Постановка задачи}

В общем случае дана система сравнений:

\[
\left\{
\begin{aligned}
x & \eqmod{m_1} a_1 \\
x & \eqmod{m_2} a_2 \\
& \vdots \\
x & \eqmod{m_n} a_n \\
\end{aligned}
\right.
\]

Все $a_i$ даны, нужно подобрать $x$, подходящий сразу под все сравнения.
Например, если дана система

\[
\left\{
\begin{aligned}
x & \eqmod{9} 6 \\
x & \eqmod{10} 2 \\
x & \eqmod{11} 9 \\
\end{aligned}
\right.,
\]
можно подумать, поподбирать и обнаружить, что в нее подходит число 42. Проверьте это.

Китайская теорема об остатках говорит, что, если все $m_i$ попарно взаимно просты,
т.е. НОД любых двух разных $m_i$ и $m_j$ равен $1$, тогда система всегда имеет решение,
причем все решения будут сравнимы по модулю $m_1\cdot m_2\cdots m_n$.

В нашем примере $(9, 10)=1$, $(9, 11)=1$, $(10, 11)=1$, значит, система подходит под условие
теоремы. Следовательно у нее есть решение, что мы и так знаем, потому что уже нашли 42,
и следовательно все решения сравнимы по модулю $9\cdot10\cdot11=990$. Действительно, числа
$42$, $42+990$, $42 + 990 + 990$. $42 - 990$ и т.п. будут все подходить в систему. Убедитесь, что
добавление или вычитание числа 990 не может испортить решение: дело в том, что $990\eqmod{9}0$,
$990\eqmod{10}0$, $990\eqmod{11}0$.

\section{Решение задачи}

Решение задачи состоит из нескольких шагов.

\subsection{Предварительные вычисления}

Сначала вычислим $M = m_1\cdot m_2\cdots m_n$. В нашем примере $M = 990$.
Потом вычислим $M_i = \frac{M}{m_i}$. Или, что тоже самое, произведение всех
$m$ кроме $m_i$. В нашем примере:

\begin{gather*}
M_1 = \frac{990}{9} = 10\cdot11 = 110 \\
M_2 = \frac{990}{10} = 9\cdot11 = 99 \\
M_3 = \frac{990}{11} = 9\cdot10 = 90.
\end{gather*}

\subsection{Решение сравнений}

Необходимо решить $n$ сравнений $M_i x_i \eqmod{m_i} a_i$. В нашем случае это:

\[
\begin{aligned}
110x_1 & \eqmod{9} 6 \\
99x_2 & \eqmod{10} 2 \\
90x_3 & \eqmod{11} 9 \\
\end{aligned}
\]
Обратите внимание, что все сравнения надо решить независимо, и для каждого найти своё решение
$x_i$. Давайте сделаем это. В общем случае вам может потребоваться свести сравнение к диофантовому
уравнению, но здесь числа такие маленькие, что сравнения можно решить вручную. Смотрите:

\noindent\textbf{Решим сравнение} $\boldmath 110x_1 \eqmod{9} 6$

Заменим $110$ по модулю $9$: $110\eqmod9 2$:
\[
2x_1 \eqmod{9} 6
\]
Сократим на 2:
\[
x_1 \eqmod{9} 3.
\]

\noindent\textbf{Решим сравнение} $\boldmath 99x_1 \eqmod{10} 6$


Заменим $99$ по модулю $10$: $99\eqmod{10} -1$:

\[
-x_1 \eqmod{9} 6
\]
Домножим на $-1$:
\[
x_1 \eqmod{9} -6 \eqmod{9} 3.
\]

\end{document}
