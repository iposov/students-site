\documentclass[a4paper, 12pt] {article}
\usepackage{cmap}
\usepackage[T2A]{fontenc}
\usepackage[utf8]{inputenc}
\usepackage[english, russian]{babel}
\usepackage{icomma}
\usepackage{amsmath,amsfonts,amssymb,amsthm,mathtools}
\usepackage{cancel}
\usepackage[normalem]{ulem}
\usepackage[all]{xy}
\usepackage{xcolor}
\usepackage{hyperref}


\title{Комбинаторика и теория графов}
\author{Посов И.А.}
\date{Осень 2021 г.}



\begin{document}
	
	
	\maketitle
	\begin{center}
		Запись конспекта: Спиридонов А.
	\end{center}

	\thispagestyle{empty}
	\newpage
	
	\begin{center}
		\tableofcontents
	\end{center}
	
	\newpage
	
	\begin{center}
		\section{Бинарные отношения}
	\end{center}
	\textbf{Определение.} M --- множество $\ne \varnothing$.
	
	$R\subset M \times M$ --- бинарное отношение.
	
	\textit{Пояснение}
	
	$M \times M$ --- множество пар из элементов R.
	
	Допустим, $M = \{ a, b, c \}$. Тогда $M \times M = \{ (a, a), (a, b), (a, c), (b, a), (b, b), (b, c), $
	
	$(c, a), (c, b), (c, c)\}$
	
	Или $M = \mathbb N$
	
	$M \times M = \mathbb N \times \mathbb N =  \{ (1, 1), (1, 2), \dots, (42, 17), \dots\}$
	
	Отношение R --- это подмножество пар.
	
	\underline{Обозначение.}
	
	$(x, y) \in R$ --- пара (x, y) принадлежит отношению. Вместо $(x, y) \in R$ будем писать $xRy$.
	
	Вместо $(x, y) \notin R$ будем писать $x \bcancel Ry$\\
	
	
	Примеры: 
	
	\textbf{1.} $M = \mathbb R$ ; $> =R = \{ (x, y): x > y \}$
	
	$(3,2) \in R \Leftrightarrow 3R2 \Leftrightarrow 3>2$
	
	$(3,4) \notin R \Leftrightarrow 3 \bcancel R2 \Leftrightarrow 3 \bcancel >2$
	
	\textbf{2.} $M = \mathbb R$ отношение $\ge: 7\ge6; 7\ge7; 7 \bcancel \ge 8$
	
	\textbf{3.} $M = \mathbb R$ отношение $=: 7=7; 7 \bcancel = 8$
	
	\textbf{4.} $M = \mathbb R \approx : x \approx y \Leftrightarrow |x-y| < 1$
	
	\textbf{5.} $M = \mathbb R  \text{ отношение} @ : x @ y \Leftrightarrow x^{2}>y$. $2@2; 1 \bcancel @2$
	
	\textbf{6.} $M = \mathbb N  \text{ отношение} \vdots : x \vdots y \Leftrightarrow \exists k \in \mathbb R: x=ky.$  $ 4 \vdots 2; 10 \bcancel \vdots 3$
	
	\textbf{6.1.} $M = \mathbb Z  \text{ отношение } \vdots : x \vdots y \Leftrightarrow \exists k \in \mathbb R: x=ky.$  $ 7 \bcancel \vdots 0; 0 \vdots 0$
	
	\textbf{7.} $M = \mathbb Z  \text{ отношение} \equiv \text{mod 3} : 0 \equiv 3 \text{ mod 3}; 1 \bcancel \equiv 8 \text{ mod 3}$
	
	\textbf{8.} $M = \mathbb N  \text{ отношение } a \xi b \text{, если в числе $ 'a' $ $ 'b' $ цифр}: 100 \xi3; 238 \bcancel \xi 8$
	
	\textbf{9.} $M =\text{ прямые на } \mathbb R^{2}$

$\xymatrix{
	 \ar@{-}[rd]  &  \\
	 & & \ar@{-}[lu]    \\
}$
	
	$\text{отношение} \parallel : l_{1} \text{ параллельно } l_{2} \text{, если } l_{1} \text{ не пересекает } l_{2}\text{, или } l_{1}=l_{2}$
	
	\textbf{10.} $M =\text{ студент ЛЭТИ}. x \succ  y \text{, если средний балл за последнюю сессию}$ 
		
		$ x $ больше чем $ y $
	
	\textbf{11.} $M =\text{ Пользователи ВКонтакте}.$  $ x \rightarrow  y \text{, если пользователь} $
	 
	 $ x $ находится в друзьях у пользователя $ y $\\
	\newpage 
	
	\subsection{Свойства бинарных отношений}
	\textbf{Определение} 
	$\text{Бинарное отношение R на множестве M называют}$
	
	$\textit{рефлексивным} \text{, если } \forall x \in M: xRx \left((x, x) \in R\right)$
	
	\textit{Замечание.} Отношение не рефлексивно $\Leftrightarrow \exists x: x \bcancel Rx$ --- \underline{контрпример.}\\
	
	
	Примеры:
	
	$=: \forall x: x=x$
	
	$ \ge: \forall x: x \ge x$
	
	$\approx : \forall x: x \approx x (|x-x|=0<1)$
	
	$>: не рефлексивно (2 \cancel > 2)$
	
	$\xi: \text{не рефлексивно} (3 \bcancel \xi 3)$\\
	
	
	\textbf{Определение} 
	$\text{Бинарное отношение R на множестве M называют}$
	
	$\textit{антирефлексивным} \text{, если } \forall x \in M: x \bcancel Rx$
	
	\textit{Замечание.} Отношение не антирефлексивно $\Leftrightarrow \exists x: xRx$ --- \underline{контрпример.}\\
	
	
	Примеры:
	
	>: антирефлексивно $\forall x: x \bcancel Rx$
	
	$\xi: \text{не антирефлексивно } (1\xi 1)$\\
	
	
	\textit{Замечание:} 
	
	$\xi \text{ не рефлексивно и не антирефлексивно.}$
	
	Не бывает R, которое одновременно и рефлексивно, и 
	
	антирефлексивно.
	
	\[
	\text{Рассмотрим} a \in M \begin{cases}
		aRa \Rightarrow \text{ не антирефлексивно} \\
		a \bcancel Ra \Rightarrow \text{ не рефлексивно}
	\end{cases}
	\]\\
	
	\textbf{Определение} 
	$\text{Бинарное отношение R на множестве M называют}$
	
	$\textit{симметричным} \text{, если } \forall x, y \in M: xRy = yRx$
	
	\textit{Замечание.} Отношение не симметрично $\Leftrightarrow \exists x, y: xRy, y \bcancel Rx$ --- \underline{контрпример.}\\
	
	Примеры:
	
	=: --- симметрично $x=y \Leftrightarrow y=x$
	
	$\approx : \text{ --- симметрично } \forall x, y: \text{ если }|x-y|<1 \Rightarrow |y-x|<1$
	
	$\xi: \text{ --- не симметрично } $100$ \xi 3,\text{ } 3 \bcancel \xi100$\\
	
	\textbf{Определение} 
	$\text{Бинарное отношение R на множестве M называют}$
	
	$\textit{антисимметричным} \text{, если } \forall x \bcancel = y \in M: xRy \Rightarrow y \bcancel Rx$
	
	\textit{Замечание.} Отношение не антисимметрично $\Leftrightarrow \exists x \bcancel = y: xRy, yRx$ --- \underline{контрпример.}\\
	
	Примеры:
	
	$>: \text{ --- антисимметрично } x > y \Rightarrow y \cancel  >x$
	
	$\textit{Попробуем построить контрпример:}$
	
	>:$ x \bcancel = y: x>y, y>x \text{ --- невозможно }$
	
	$\Rightarrow \text{ нет контрпримера } \Rightarrow \text{ отношение антисимметрично }$
	
	$\ge : x \bcancel = y: x \ge y, y \ge x \text{ --- невозможно }$
	
	$\Rightarrow \text{ нет контрпримера } \Rightarrow \text{ отношение антисимметрично }$
	
	$= : x \bcancel = y: x =e y, y = x \text{ --- невозможно }$
	
	$\Rightarrow \text{ нет контрпримера } \Rightarrow \text{ отношение антисимметрично }$
	
	$\equiv \text{mod 3} : 1 \equiv 4 \text{ mod 3} \text{ --- контрпример } \Rightarrow \text{ отношение не антисимметрично }$
	
	$\text{ Над }\mathbb N  \text{ отношение } \vdots  : x \bcancel = y: x \vdots y, y \vdots x \text{ --- невозможно }$
	
	$\Rightarrow \text{ нет контрпримера } \Rightarrow \text{ отношение антисимметрично }$
	
	$\text{ Над }\mathbb Z  \text{ отношение } \vdots  : 4 \bcancel = -4: 4 \vdots -4, -4 \vdots 4 \text{ --- контрпример }$
	
	$\Rightarrow \text{ отношение не антисимметрично }$\\
	
	\textbf{Определение} 
	$\text{Бинарное отношение R на множестве M называют}$
	
	$\textit{асимметричным} \text{, если } \forall x, y \in M: xRy \Rightarrow y \bcancel Rx$
	
	\textit{Замечание.} Отношение не асимметрично $\Leftrightarrow \exists x, y: xRy, yRx$ --- \underline{контрпример.}\\
	
	\textbf{Эквивалентное определение} 
	$\text{Бинарное отношение R на множестве M называют}$
	
	$\textit{асимметричным} \text{, если R --- антисимметрично и антирефлексивно.}$\\
	
	Примеры:
	
	$>: \text{ --- асимметрично } x > y \Rightarrow y \cancel  >x$
	
	$\blacksquare : \text{ --- асимметрично (пустое отношение, когда } R = \varnothing \text{)}$
	
	$\text{"Начальник" на множестве тех, кто работает в ЛЭТИ: отношение асимметрично }$
	
	$x \text{ начальник } y \Rightarrow y \text{ не начальник } x$\\
	
	
	\textbf{Определение} 
	$\text{Бинарное отношение R на множестве M называют}$
	
	$\textit{транзитивным} \text{, если } \forall x, y,z \in M: xRy, yRz \Rightarrow xRz$
	
	\textit{Замечание.} Отношение не транзитивно $\Leftrightarrow \exists x, y, z: xRy, yRz, x \bcancel Rz$ --- \underline{контрпример.}\\
	
	Примеры:
	
	$>: \text{ --- транзитивно } x > y, y > z \Rightarrow x>z$
	
	$\ge : \text{ --- транзитивно } x \ge y, y \ge z \Rightarrow x \ge z$
	
	$\vdots : \text{ --- транзитивно } x \vdots y, y \vdots z \Rightarrow x \vdots z \Leftrightarrow (x=ky; y=lz \Rightarrow x=k(lz) \Rightarrow x \vdots z) $
	
	$\xi: \text{ --- не транзитивно } 100 \xi 3, 3 \xi 1, 100 \bcancel \xi 1$\\
	
	\textbf{Определение} 
	Бинарное отношение R на множестве M называют 
	
	отношением \underline{эквивалентности}, если отношение R рефлексивно, 
	
	симметрично и транзитивно.
	
	\textit{Замечение} Вместо \sout{R --- эквивалентно}, нужно говорить  \textit{R --- отношение эквивалентности}.\\
	
	Пример 1:
	
	$=: \text{ --- отношение эквивалентности }$
	
	$\forall x: x=x$ (Рефлексивность)
	
	$\forall x, y: x=y \Rightarrow y=x$ (Симметричность)
	
	$\forall x, y, z: x=y, y=z \Rightarrow x=z$ (Транзитивность)
	
	$ \ge: \text{ --- не отношение эквивалентности }$
	
	$x \ge y \bcancel \Rightarrow y \ge x (4 \ge 2; 2 \bcancel \ge 4)$
	
	Пример 2:
	
	$ \pi \text{(одинаковое количество цифр в числе): --- отношение эквивалентности }$
	
	$\forall x: x \pi x$ (Рефлексивность)
	
	$\forall x, y: x \pi y \Rightarrow y \pi x$ (Симметричность)
	
	$\forall x, y, z: x \pi y, y \pi z \Rightarrow x \pi z$ (Транзитивность)\\
	
	\textbf{Определение} 
	R --- отношение эквивалентности на множестве M, $x \in M - \text{класс эквивалентности } x: M_{x} = \{y | xRy\}$\\
	
	
	Примеры:
	
	$=: M_{5}=\{5\}$
	
	$\equiv \text{mod 3}: M_{2}=\{2, 5, 8, \dots \} = M_{5}$\\
	
	
	\textbf{Утверждение} 
	R --- отношение эквивалентности на множестве M, $ \forall x, y \in M: M_{x} =  M_{y} \text{ или } M_{x} \cap M_{y} = \varnothing$\\
	
	
	\textbf{Доказательство}
	
	$\sqsupset M_{x} \cap M_{y} \ne \varnothing \Rightarrow \exists z \in M_{x}, z \in M_{y} \Rightarrow xRz, yRz \Rightarrow \text{ (Симметричность) } zRy $  
	
	$\Rightarrow \text{ (Транзитивность) } xRy$
	
	Теперь проверим, что класс $M_{x} =  M_{y}$. Возьмём $u \in M_{x} \text{ и проверим }$ 
	
	$u \in (?) M_{y}. u \in M_{x} \Rightarrow xRu; xRy \Rightarrow \text{ (Симметричность) } yRx \Rightarrow $ 
	
	 $ \Rightarrow \text{ (Транзитивность) } yRu \Rightarrow u \in M_{y} $   $\blacksquare $
	
	\textbf{Следствие}
	R --- отношение эквивалентности на множестве M, тогда M \underline{разбито} на несколько классов элементов (\textit{классов эквивалентности})
	
	$M = M_{1} \cup \dots \cup M_{n}$
	
	$M_{i} \cap M_{j} = \varnothing$\\
	
	
	Примеры:
	
	= на $\mathbb N:$
	
	$\mathbb N = \{1\} \cup \{2\} \cup \{3\} \cup \dots$
	
	$\equiv \text{mod 3} \text{ на } \mathbb N:$
	$\mathbb N = \{ \text{0 3 6 9 }\dots \} \cup \{ \text{1 4 7 10 \dots } \} \cup \{ \text{2 5 8 11 \dots } \}$\\
	
	
	\textit{Замечание}
	Если есть $M \ne \varnothing$ разбито на $M_{i} \ne \varnothing:$ 
	
	$M = M_{1} \cup \dots \cup M_{n}; M_{i} \cap M_{j} = \varnothing$
	
	Тогда можно ввести отношение R: xRy, если $\exists M_{i}: x, y \in M_{i}$
	
	\subsection{Отношения порядка}

	\textbf{Определение}
	$\sqsupset \text{ бинарное отношение R транзитивно и антисимметрично:}$
	
	1) рефлексивно - нестрогий порядок $( \succeq  )$
	
	2) антирефлексивно - строгий порядок $( \succ  )$\\
	
	
	Примеры:
	
	> на $\mathbb R$ --- строгий порядок 
	
	$\ge$ на $\mathbb R$ --- нестрогий порядок 
	
	$\vdots$ на $\mathbb N$ --- нестрогий порядок \\
	
	
	\textbf{Определение}
	$\sqsupset R$ --- строгий или нестрогий порядок:
	
	R --- линейный порядок, если $\forall x \ne y: xRy \text{ или } yRx$ 
	
	R --- частичный порядок, если $\exists x \ne y: x \bcancel Ry, \text{} y \bcancel Rx$\\
	
	
	Примеры:
	
	>, $\ge$ --- линейный порядок 
	
	$\vdots$ на $\mathbb N$ --- частичный порядок $(2 \bcancel \vdots 3, 3 \bcancel \vdots 2)$\\
	
	
	\textbf{Утверждение}
	R --- строгий или нестрогий порядок на конечном 
	
	множестве $M (|M| < \infty)$. Тогда $\exists x$ --- минимальный, то есть $\forall y: x \bcancel \succ y$\\
	
	
	Примеры:
	
	$\ge$ на  \{1, 2, 3, 4, 5\}: 
	
	Минимальный - 1, так как $\forall y \ne 1: 1 \bcancel \ge y$
	
	$\vdots$ на  \{2, 3, 4, 5, 6\}: 
	
	Минимальный - 2, так как $\forall y \ne 2: 2 \bcancel \vdots y$
	
	Минимальный - 3, так как $\forall y \ne 3: 3 \bcancel \vdots y$
	
	Минимальный - 5, так как $\forall y \ne 5: 5 \bcancel \vdots y$\\
	
	
	\textbf{Доказательство}
	
	Берём $x_{1}$ - любой элемент множества. 
	
	Если он не минимальный $ \Rightarrow \exists x_{2} \ne x_{1}: x_{1} \succ x_{2}$
	
	Если $x_{2}$ не минимальный $ \Rightarrow \exists x_{3} \ne x_{2}: x_{2} \succ x_{3}$
	
	$\dots$
	
	Если мы не можем найти минимальный элемент, поскольку 
	
	множество конечно $ \Rightarrow $ в какой-то момент элемен повторится ($x_{i}=x_{j}$)
	
	$x_{i} \succ x_{i+1} \succ x_{i+2} \succ \dots \succ x_{j-1} \succ x_{j}= x_{i}$
	
	$\succ$ --- транзитивно $\Rightarrow x_{i} \succ x_{j-1}, x_{j-1} (\ne x_{i}) \succ x_{i}$ (невозможно по 
	
	антисимметричности) 
	
	Противоречие: следовательно, такое невозможно и рано ли поздно мы найдём минимальный элемент $\blacksquare$\\
	
	
	\textbf{Определение}
	Отношенне $R_{1}$ на множестве М расширяет $R_{2}$ на множестве М, если $R_{2} \in R_{1}$
	
	\textit{Замечание}
	$R_{1}$ "добавляет"$  $ пары, где $xRy$
	
	\textit{Замечание}
	$xR_{2}y \Rightarrow xR_{1}y$\\
	
	\subsection{Топологическая сортировка}
	
	\textbf{Теорема} \underline{о топологической сортировке}
	
	Если $\succ$ --- отношение строгого или нестрогого порядка на конечном множестве М, то $\exists \gg$ --- отношение линейного порядка на конечном 
	
	множестве М, такое что $\gg$ расширяет $\succ$
	
	$\xymatrix{
		& & A \ar@{-}[d] \ar@{-}[ddl] \ar@{-}[ddr] \ar@{-}[rd]  \\ 
		& & B \ar@{-}[dl] \ar@{-}[dr] & C \\
		& D & & E \\
	}$
	
	--- нелинейный порядок
	\newpage
	Варианты топологической сортировки:
	
	$\xymatrix{
		A & A\\
		B & B\\
		C & D\\
		D & E\\
		E & C\\
	}$\\

	
	\textbf{Доказательство}
	Найдём минимальный элемент отношения $\succ$. $\sqsupset$ это $x_{1} \in M$. Удаляем $x_{1}$ из М: тогда имеем $M \text{ без }x_{1}$. Очевидно, что новое отношение также антисимметрично, транзитивно и рефлексивно либо антирефлексивно в зависимости от свойств изначального $\Rightarrow$ в нём также есть минимальный элемент $x_{2}$. Удаляем $x_{2}$ и продолжаем до тех пор, пока не получим последовательность $x_{1}, x_{2}, \dots, x_{n}$, где $n=|M|$ 
	
	Вводим новый порядок $ x_{i} \ll x_{j}$ для $i < j: x_{1} \ll x_{2} \ll \dots \ll x_{n}$
	
	Почему $\ll$ расширяет $<$?
	
	Если $x < y \Rightarrow $ x был удалён раньше y $\Rightarrow x \ll y $  $\blacksquare$\\
	
	
	\textit{Замечание}
	Этот алгоритм (поиска минимального элемента и его 
	
	удаление) не самый эффективный. Лучше сделать поиск в глубину с 
	
	обратной нумерацией.
	
	\textit{Замечание}
	Топологическая сортировка - практически очень важная задача.
	
	Например,порядок работы: 
	
	$\xymatrix{
		& B \ar[ld]\\
		D & & A \ar[ld] \ar[lu]\\
		& C \ar[lu] \\
	}$
	
	\subsection{Транзитивное замыкание}
	
Был порядок - расширяли до линейного (Топопологическая сортировка)
Было отношение - расширяем до транзитивного 

(Транзитивное замыкание)
	
	Имеется нетранзитивное отношение:
	
	Добавлением пунктирных рёбер мы делаем 
	
	его транзитивным:
	$\xymatrix{
		& & A \ar@{-}[d] \ar@{--}[ddl] \ar@{--}[ddr] \ar@{-}[rd]  \\ 
		& & B \ar@{-}[dl] \ar@{-}[dr] & C \\
		& D & & E \\
	}$\\
	
	\textbf{ Теорема}
	
	$\sqsupset R$ --- бинарное отношение на множестве M. $\exists \bar R$ --- отношение на М. 
	
	
	1) $\bar R$ расширяет R ($R \in \bar R$)
	
	2) $\bar R$ транзитивно
	
	3) $\bar R$ --- минимальное трнзитивное расширение, то есть если $\widetilde{R}$ --- транзитивное расширение R, то $\widetilde{R} \supset \bar R$\\
	
	
	\textbf{ Доказательство} \underline{(не для алгоритма)}
	Рассмотрим все транзитивные расширения \{$\bar R$\}, возьмём $\bar R \cap \bar R_{i}$, то есть берём только те пары, которые есть во всех транзитивных расширениях.
	
	Пример:
	
	$ M = \{a, b, c, d\} : aRb, bRc, bRd$
	
	$\xymatrix{
		& a \ar[d] \ar[ddr] \ar[ddl] & & & a \ar[d] \ar[ddr] \ar[ddl] \\
		& b \ar[dl] \ar[dr] & & & b  \ar[dl] \ar[dr]\\
		c \ar[rr] & & d  & c & & d \ar[ll]
	}$
	
	Оставим только те пары, которые есть везде --- $\bar R:$
	
	$\xymatrix{
		& a \ar[d] \ar[ddr] \ar[ddl] \\
		& b \ar[dl] \ar[dr]\\
		c & & d  
	}$
	
	Проверим, что $\bar R$ подходит под условия теоремы:
	
	0) Почему $\bar R_{i}$ существует?
	
	$\bar R_{i}$:= полное отношение $M \times M \Rightarrow \exists$
	
	1) $\bar R$ расширяет: $\sqsupset xRy \Rightarrow \forall \bar R_{i}: x \bar R_{i} y \Rightarrow x \bar R y$
	
	2)  $\sqsupset x \bar R y, y \bar R z \Rightarrow \forall \bar R_{i}:x \bar R y, y \bar R z \Rightarrow x \bar R z \Rightarrow$ транзитивно $x \bar R z$
	
	3) $\widetilde{R} = \bar R_{i} \supset R$, так как $R=\bar R_{i} \cap \dots $	$\blacksquare$ 
	
	\newpage
	
	\begin{center}
		\section{Графы}
	\end{center}

\textbf{Определение}
Неориентированный граф G = (V, E),

где V --- множество (вершины); $E \subset \{(u, v)$ (пара неупорядочена), 

где  $u, v\in V\}$

\textit{Замечание}
Вершины - точки или кружочки; рёбра --- линии, неважно какой формы. Важно лишь то, что они соединяют.

\textit{Замечание}
V --- vertex, E - edge\\

Примеры:

$\xymatrix{
	a \ar@{-}[r] & b \ar@{-}[ld]\\
	c \\
}$

$\xymatrix{
	. \ar@{-}[r] \ar@{-}[rd] & . \ar@{-}[d] \ar@{-}[ld]\\
	. \ar@{-}[u] \ar@{-}[r]  & . \\ 
}$
--- полный граф

$\xymatrix{
	.  & . \\
	.   & . \\ 
}$
--- пустой граф\\


\textbf{Определение}
G --- полный граф, если $\forall u, v \in V: (u, v) \in E$\\

	
\textbf{Определение}
Размер(порядок) графа |G| = количество вершин |V|

\textit{Замечание}

|V| = n --- количество вершин

|E| = m --- количество рёбер

G --- (n, m) граф\\


\textbf{Определение}
Степень вершины $v \in V: \{ (v, u) | (v, u) \in E \}$ 

(Количество рёбер с этой вершиной)\\

\textbf{Определение}
K-регулярный граф --- граф, где $\forall v \in V: deg (v) = k $

$\xymatrix{
	. \ar@{-}[r] \ar@{-}[rd] & . \ar@{-}[d] \ar@{-}[ld]\\
	. \ar@{-}[u] \ar@{-}[r]  & . \\ 
}$
--- 3-регулярный граф\\

\textbf{Определение}
Путь в графе --- последовательность вершин-рёбер: $v_{1}, e_{1}, v_{2}, e_{2}, \dots, v_{n-1}, e_{n-1}, v_{n}$

$G=(V, E); v_{i} \in V; e_{i} \in E; e_{i} = (v_{i}, v_{i+1})$

$\xymatrix{
	& d \ar@{-}[rd] \ar@{-}[ld] \\
	e \ar@{-}[rr] & & c \\ 
	& b \ar@{-}[ru] \ar@{-}[lu]\\
	& a \ar@{-}[u]\\
}$

Примеры путей:

1) abcd

2) ab

3) aba

4) abcdecdeba\\

\textbf{Определение}
Замкнутый путь --- путь, если $v_{1} = v_{n}$

Не замкнутый, открытый путь --- путь, если $v_{1} \ne v_{n}$\\

\textbf{Определение}
Простой путь - если $e_{i} \ne e_{j}$ при $i \ne j$\\

Примеры:

4) abcdecdeba: путь замкнутый (начинается и заканчивается в вершине a), и непростой (два раза встречается ребро de)

5) bedce: путь простой, но не замкнутый


\begin{tabular}{ | l |l | l | l | }
	\hline
	Путь & & Все рёбра разные & Все вершины разные \\ \hline
	Замкнутый & Замкнутый путь & Простой замкнутый путь & Цикл\\ \hline
	Открытый & Открытый путь & Простой открытый путь & Цепь \\ \hline
\end{tabular}
\\
\\
\textbf{Теорема}
Если $\exists$ путь между вершинами u, v $\Rightarrow \exists$ цепь от u до v\\

\textbf{Доказательство}
$\sqsupset \text{ путь } v_{1}, e_{1}, v_{2}, e_{2}, \dots, v_{n-1}, e_{n-1}, v_{n}$. Рассмотрим все пути из этих рёбер и выберем минимальный - это будет цепь.

Иначе $v_{i}, \dots, v_{j},  \dots, v_{n}.$ 

$\sqsupset v_{i} = v_{j}.$ 

$\text{Укоротим } U: v_{i} = v_{j}, \dots, v_{n}$ \textit{Противоречие!!!}\\

\textbf{Теорема}
Если есть простой замкнутый путь через ребро e, $\Rightarrow \exists$ цикл через e \\

\textbf{Доказательство}
Аналогично $\blacksquare$
\newpage
\textit{Замечание}

$\xymatrix{
	a \ar@{-}[dd] \ar@{-}[rd] & & & k \\
	& b \ar@{-}[r]_e \ar@{-}[dl] & d \ar@{-}[dr] \ar@{-}[ru] \\
	c & & & l \ar@{-}[uu] \\
}$

dbacbd --- не простой путь (e повторяется). Цикла через e нет.

\subsection{Связность графа}

\textbf{Определение}
Граф G = (V, E) связен, если $\forall u, v \in V \exists$ цепь(путь) из U в V

$\xymatrix{
	a \ar@{-}[d] \ar@{-}[r] & b \ar@{-}[d] \\
	c  \ar@{-}[r] & d \\
}$

--- связный граф

$\xymatrix{
	a \ar@{-}[d] \ar@{-}[r] & b \ar@{-}[d]  & & a \ar@{-}[d] \ar@{-}[dl]\\
	c  \ar@{-}[r] & d & b \ar@{-}[r] & c\\
}$

--- не связный граф\\

Введём отношение $\equiv$ на вершинах графа: $U \equiv V$, если существует путь из U в V.

Проверим, что $\equiv$ --- отношение эквивалентности:

1) рефлексивность $U \equiv U$ --- верно, путь U

2) симетричность.

Путь U: $v_{1}, e_{1}, v_{2}, e_{2}, \dots, v_{n-1}, e_{n-1}, v_{n}$. Путь V получем переворотом пути U: $v_{n}, \dots, v_{1}, e_{1}, u$. Поскольку мы считаем граф неориентированным, $\Rightarrow U \equiv V \Leftrightarrow V \equiv U$

3) транзитивность: $ U \equiv V, V \equiv W$ --- путь $ U \equiv W$ получен 

объединением первого и второго $\blacksquare$\\

\textbf{Определение}
Класс эквивалентности $  \equiv $ ---" компонента связности"

$\xymatrix{
	a \ar@{-}[d] \ar@{-}[r] & b \ar@{-}[d]  & & a \ar@{-}[d] \ar@{-}[dl]\\
	c  \ar@{-}[r] & d & b \ar@{-}[r] & c\\
}$
--- не связный граф


\fbox{$\xymatrix{
	a \ar@{-}[d] \ar@{-}[r] & b \ar@{-}[d] \\
	c  \ar@{-}[r] & d \\
}$}
$\leftarrow$ две компоненты связности $\rightarrow$
\fbox{$\xymatrix{
	& a  \ar@{-}[dl] \ar@{-}[d]\\
	b \ar@{-}[r] & c\\
	}$ }\\

\textbf{Определение}
$G_{1} = (V_{1}, E_{1})$ --- подграф $ G = (V, E) $, если $ V_{1} \subset V,$ 

$E_{1} \subset E $

$\xymatrix{
	a  \\
	& b \ar@{--}[ul] \ar@{--}[dl] \\
	c\\
}$
(1) --- подграф (2)
$\xymatrix{
	a \ar@{-}[dd] \\
	& b \ar@{-}[ul] \ar@{-}[dl] & d \ar@{-}[l]\\
	c\\
}$

\textit{Замечание} 

G всегда является своим подграфом

$ \varnothing $ --- подграф чего угодно\\

\textbf{Определение}
$G = (V, E)$. Ребро e называется \underline{мостом}, если количество компонент связности G < количества компонент связности

 (V, E $\text{/} \{e\})$
 
 $\xymatrix{
 	a \ar@{-}[d] \ar@{-}[r] & b \ar@{-}[d]  & & a \ar@{-}[d] \ar@{-}[dl]\\
 	c  \ar@{-}[r] & d \ar@{-}[r]_l & b \ar@{-}[r] & c\\
 }$
 \\
 l --- мост\\
 
\textbf{Определение}
Степень связности графа G --- минимальное количество рёбер, которые надо выкинуть, чтобы граф стал не связным.\\
\textbf{Определение}
Двусвязный граф --- граф, из которого нужно выкинуть хотя бы 2 ребра, чтобы он стал не связным.\\
\textit{Замечание} Двусвязный граф $ \Leftrightarrow $ нет мостов

$\xymatrix{
	a \ar@{-}[d] \ar@{-}[r] & b \ar@{-}[d] \\
	c  \ar@{-}[r] & d \\
}$

--- двусвязный граф\\

\textbf{Определение}
Вершина $ u  \in U $ называется \underline{точкой сочленения},  если количество компонент связности G < количества компонент связности $ \bar G = (V / \{v\}, E / \{(v, u) | (v, u) \in E\}) $\\
\subsection{Количество рёбер, вершин}

\textbf{Теорема}
В графе $G = (V, E)$, если $deg(u)$ --- степень вершины $ u $,

 количество рёбер 


\begin{equation*}
	|E| = \dfrac{1}{2} \sum_v deg(v)
\end{equation*}

Пример:

$\xymatrix{
	& 2 \ar@{-}[dl] \ar@{-}[dr] \\
	3 \ar@{-}[rr] & &  4 \ar@{-}[ul] \ar@{-}[dl] & 1 \ar@{-}[l]\\
	& 2 \ar@{-}[ul] \ar@{-}[ur]\\
}$
Количество рёбер: 6 = $ \dfrac{1}{2} (3+2+2+4+1)$

\textbf{Доказательство}
$deg(v)$ --- количество рёбер, выходящих из вершины

$ \sum_v deg(v) $ --- все рёбра посчитаны дважды = $ 2|E| $ $ \blacksquare $

Следствие:

1) Сумма степеней вершин всегда чётная

2) Вершин нечётной степени - чётное количество\\

\underline{Задача}
У каждого из 15 инопланетян по 3 руки. Могут ли они взяться за руки так, чтобы ни у кого не осталось своодной руки?

\underline{Решение} Нет, поскольку это граф из 15(нечётного числа) вершин 

степени 3 (нечётной)\\

\textbf{Определение}
Висячая вершина - вершина степени 1

$\xymatrix{
	. \ar@{-}[dd] \\
	& . \ar@{-}[ul] \ar@{-}[dl] & d \ar@{-}[l]\\
	.\\
}$ 

d --- висячая вершина\\ \\
\textbf{Теорема}
Если в графе есть рёбра, но нет висячих вершин $ \Rightarrow \exists $ цикл
Доказательство
Берём ребро $ e(u_{1}, u_{2}) $. 

$u_{2} $ --- не висячая $ \Rightarrow $ из неё есть ещё ребро $ e(u_{2}, u_{3})$.

$u_{3} $ --- не висячая $ \Rightarrow $ из неё есть ещё ребро $ e(u_{3}, u_{4})$.

Продолжаем, пока очередной $u_{n} $ не будет равен $u_{i}: 1 \le i < n $.

Путь $ u_{i}, u_{i+1}, \dots, u_{n} $ --- цикл, поскольку все рёбра и вершины разные (при первом повторе мы замкнули цикл) $ \blacksquare $\\

\textbf{Определение}
Дерево - связный граф без циклов

Пример:
$\xymatrix{
	. \ar@{-}[r] & . \ar@{-}[r] & . \ar@{-}[r] & . 
} $\\

\textbf{Теорема}
В любом дереве есть хотя бы висячих 2 вершины

\textbf{Доказательство}
Берём любую вершину: если она не висячая --- идём по ребру; если опять не висячая - есть ещё ребро $ \dots $ Циклов нет $ \Leftrightarrow $ будет конец --- это и есть висячая вершина.

Чтобы найти вторую висячую вершину, нужно начать путь из первой $ \blacksquare $\\

\textbf{Теорема}
Если $G = (V, E)$ - дерево, $ |V| = 1 + |E| $

Пример:

$\xymatrix{
	& . \ar@{-}[r] \ar@{-}[rd]. & .\\
	. \ar@{-}[ru] \ar@{-}[rd] & & .\\
	& . \ar@{-}[r] & . \\
}$ 
--- 6 вершин, 5 рёбер\\

\textbf{Доказательство} по индукции (количество вершин)\\

База: $ |V| = 1, |E| = 0 $
$\xymatrix{
	& \fbox{.} \\
}$ $ |V| = 1 + |E| $\\

Переход: $ \sqsupset  |V| = n+1$
$\xymatrix{
	& . \ar@{-}[r] \ar@{-}[rd]. & .\\
	. \ar@{-}[ru] \ar@{-}[rd] & & .\\
	& . \ar@{-}[r] & . \\
}$ n+1 вершина

Найдём висячую вершину и удалим её, с её единственным ребром:

$\xymatrix{
	& . \ar@{-}[r] \ar@{-}[rd]. & .\\
	. \ar@{-}[ru] \ar@{-}[rd] & & .\\
	& .  \\
}$

$ \bar G = (V / \{v\}, E / \{e\}) $ --- тоже дерево, поскольку граф связен и нет 

циклов $ \Rightarrow |\bar V| = 1 + |\bar E| $

$ |V| = |\bar V| + 1; |E| = |\bar E| + 1 $

$ \Rightarrow  |V| = 1 + |E| $ $ \blacksquare  $

\newpage
$ \sqsupset G$ --- полный граф, $ \forall u \ne v \in V $ соединены ребром

$\xymatrix{
	. \ar@{-}[r] \ar@{-}[rd] & . \ar@{-}[d] \ar@{-}[ld]\\
    . \ar@{-}[u] \ar@{-}[r]  & . \\ 
}$

если n вершин ($ |V| = n $), то рёбер:

1) $C_{n}^{2}=\frac{n(n-1)}{2}$

2) степень всех вершин n-1

$ \sum_v deg(v) =	2|E| \Rightarrow n(n-1) = 2|E| $\\


Ответ: $ \frac{n(n-1)}{2} $

\subsection{Планарные графы}

\textbf{Определение} G --- планарный граф, если его можно нарисовать 

на плоскости так, чтобы рёбра не пересекались

$\xymatrix{
	A \ar@{-}[r]  & B \ar@{-}[d]  & & A_{1} \ar@{-}[r] \ar@{-}[rd] & B_{1}  \ar@{-}[ld]\\
	C \ar@{-}[u] \ar@{-}[r]  & D & & C_{1}  \ar@{-}[r]  & D_{1}\\ 
}$

Оба графа планарны, поскольку планарность --- свойство графа, а не рисунка. Можно сказать, что граф $A_{1}, B_{1}, C_{1}, D_{1}$ 

"неправильно"$  $ нарисованный\\


\textbf{Формула Эйлера} Если связный планарный граф $G = (V, E)$ нарисован на плоскости, у него можно посчитать грани. $ \sqsupset $ их $ f, |V|=n, |E|=m $\\


$\xymatrix{
	. \ar@{-}[r] & . \ar@{-}[d] \\
	. \ar@{-}[u] \ar@{-}[r]  & . \\ 
}$ I граф --- 2 грани: внутри квадрата и вне его контура\\


$\xymatrix{
	. \ar@{-}[r] & . \ar@{-}[d]  . \ar@{-}[r] & . \ar@{-}[d]   . \ar@{-}[r] & . \ar@{-}[d] \\
	. \ar@{-}[u] \ar@{-}[r]  & .   \ar@{-}[u] \ar@{-}[r]  & .   \ar@{-}[u] \ar@{-}[r]  & . \\
}$ II граф --- 4 грани\\

Тогда $ n-m+f=2 $\\

\underline{Проверим:}

I граф: $4-4+2=2$

II граф: $8-10+4=2$\\

\newpage
\textbf{Доказательство} (индукция по количеству рёбер)\\

База: G --- дерево. У дерева всега 1 грань. Поскольку вокруг грани всегда цикл, а в дереве циклов нет, то и грань будет только одна - внешняя: $ n-(n-1)+1=2 $

Переход: возьмём граф G, для которого мы не знаем, верна ли формула. Если она верна для графа $ \bar G \Rightarrow $ будет верна и для G

(G и $ \bar G $ --- связные планарные графы)

G --- не дерево $ \Rightarrow $ есть цикл. Берём любое ребро цикла - вокруг него две грани. Удалим ребро, получим $ \bar G $ --- тоже связен и планарен.

$ \bar n, \bar m, \bar f$ --- вершины, рёбра, грани $ \bar G $

$ \bar n = n; \bar m = m-1; \bar f = f-1$

По индукционному предположению $ \bar n - \bar m + \bar f =2$

$ \Rightarrow n-(m-1)+(f-1)=2 $

$ \Rightarrow n-m+f=2 $ $ \blacksquare $\\

\textit{Следствия}

\textbf{1.} Неважно, как рисовать планарный граф - количество граней 

постоянно

\textbf{2.} Про многогранники так же: кубик на плоскости может быть 

представлен следующим образом

$\xymatrix{
. \ar@{-}[rrr] \ar@{-}[ddd] \ar@{-}[rd] & & & . \\
& . \ar@{-}[r] \ar@{-}[d] & . \ar@{-}[ru] & \\
& . & . \ar@{-}[l] \ar@{-}[u] \ar@{-}[rd] &  \\
. \ar@{-}[ru] & & & . \ar@{-}[lll] \ar@{-}[uuu] \\
}$ $ 8-12+6=2 $

\textbf{3.} Если G --- планарный (не обязательно связный) граф, то 

$ n-m+f=1+ |\text{количество компонент связноси } G| $

\textbf{4.} $ \sqsupset$ у каждой грани вокруг $ \ge $ 3 ребра

$ 3f \le \sum_g \text{количество рёбер вокруг g} \le	2m$ (каждое ребро посчитано дважды) $ \Rightarrow $ \begin{center}
	\fbox{$ 3f \le 2m $}
\end{center}

Домножаем выражение \fbox{$ n-m+f=2 $} на 3, получаем: 

$ 3n-3m+3f=6 \Rightarrow 3n-3m+2m \ge 6 \Rightarrow 3n-m \ge 6 \Rightarrow$

\begin{equation}\label{eq1}
	\fbox{$ m \le 3n-6 $}
\end{equation}

Итого, $ m \le 3n-6 $ в связном планарном графе\\


\textit{Следствие} Полный граф при $ n = 5 $ не планарен

\underline{Доказательство} $ n = 5, m = \frac{n(n-1)}{2} = \frac{5 \times 4}{2} = 10$

$ 10 \le 3 \times 5 - 6 = 9 $ --- неверно \textbf{Противоречие}\\


\textit{Замечание} $ K_{n} $ --- полный граф на n вершинах\\

\textbf{Утверждение} Граф $ K_{3,3} $ тоже не планарный

$\xymatrix{
	. \ar@{-}[rr] \ar@{-}[rrd] \ar@{-}[rrdd] & & . \ar@{-}[ll] \ar@{-}[lld] \ar@{-}[lldd]\\
	. \ar@{-}[rr] \ar@{-}[rrd] \ar@{-}[rru] & & . \ar@{-}[ll] \ar@{-}[lld] \ar@{-}[llu]\\
	. \ar@{-}[rr] \ar@{-}[rru] \ar@{-}[rruu] & & . \ar@{-}[ll] \ar@{-}[llu] \ar@{-}[lluu]\\
}$

\textbf{Доказательство} $ n=6, m=9: 9 \le 3 \times 6 - 6 $  верно

Сколько у него должно быть граней, чтобы он был планарным?

$ 6-9+f=2 \Rightarrow f=5 $ граней

В $ K_{3,3} $ все циклы --- чётные (ходим слева направо и справа налево)

$ \Rightarrow $ у любой грани $ \ge 4 $ ребра

$ 4f \le \sum_g \text{количество рёбер вокруг g} \le	2m \Rightarrow m \ge 2f$ 

Однако $ 9 \ge 2 \times 5 $ --- неверно \textbf{Противоречие}\\


\textbf{Теорема Понтрягина - Куратовского}
Граф G планарен $ \Leftrightarrow $ он не содержит подграфов, "стягивающихся" к $ K_{5} $ и $ K_{3,3} $ 

$\xymatrix{
	. \ar@{-}[rr] \ar@{-}[rrd] \ar@{-}[rrdd] & & . \ar@{-}[ll] \ar@{-}[lld] \ar@{-}[lldd]\\
	. \ar@{-}[rr] \ar@{-}[rrd] \ar@{-}[rru] & & . \ar@{-}[ll] \ar@{-}[lld] \ar@{-}[llu]\\
	. \ar@{-}[rd] \ar@{-}[rru] \ar@{-}[rruu] & & . \ar@{-}[ld] \ar@{-}[llu] \ar@{-}[lluu]\\
	& . &\\
}$ $ \leftarrow $ стягивается к $ K_{3,3} \rightarrow $
$\xymatrix{
	. \ar@{-}[rr] \ar@{-}[rrd] \ar@{-}[rrdd] & & . \ar@{-}[ll] \ar@{-}[lld] \ar@{-}[lldd]\\
	. \ar@{-}[rr] \ar@{-}[rrd] \ar@{-}[rru] & & . \ar@{-}[ll] \ar@{-}[lld] \ar@{-}[llu]\\
	. \ar@{-}[rr] \ar@{-}[rru] \ar@{-}[rruu] & & . \ar@{-}[ll] \ar@{-}[llu] \ar@{-}[lluu]\\
}$

\begin{equation}\label{eq2}
\end{equation}
\subsection{Хроматизм}

\textbf{Определение} $ \sqsupset G = (V, E)$ --- граф.

Раскраска графа G в k цветов --- это функция $c: V \rightarrow \{1 \dots k\}$

Причём если есть ребро (u, v), то $ c(u) \ne c(v)$

$\xymatrix{
	1 \ar@{-}[r] & 2 \ar@{-}[d]  \ar@{-}[r] & 3 \ar@{-}[d] \\
	2 \ar@{-}[u] \ar@{-}[r]  & 1   \ar@{-}[u] \ar@{-}[r]  & 2   \ar@{-}[u] \\
}$ --- ракраска

$\xymatrix{
	1 \ar@{-}[r] & 2 \ar@{-}[d]  \ar@{-}[r] & 3 \ar@{-}[d] \\
	3 \ar@{-}[u] \ar@{-}[r]  & 2   \ar@{-}[u] \ar@{-}[r]  & 1   \ar@{-}[u] \\
}$ --- не ракраска (существует ребро 2-2)\\

Какие графы можно раскрасить в 1 цвет?

$\xymatrix{
	1  & 1  \\
	1  & 1  \\
}$ --- графы без рёбер\\

Какие графы можно раскрасить в 2 цвета?\\

\textbf{Определение} Граф называется \textit{двудольным}, 

если его можно раскрасить в 2 цвета

$\xymatrix{
	1 \ar@{-}[r] \ar@{-}[d] & 2  \\
	2  & 1 \ar@{-}[l] \ar@{-}[u] \\
}$ --- двудольный граф
$\xymatrix{
	& 1 \ar@{-}[d]   \\
	2 \ar@{-}[ru] &  1 \ar@{-}[l] \\
}$--- не двудольный граф\\


$ K_{3,3}$ --- двудольный:

$\xymatrix{
	1 \ar@{-}[rr] \ar@{-}[rrd] \ar@{-}[rrdd] & & 2 \ar@{-}[ll] \ar@{-}[lld] \ar@{-}[lldd]\\
	1 \ar@{-}[rr] \ar@{-}[rrd] \ar@{-}[rru] & & 2 \ar@{-}[ll] \ar@{-}[lld] \ar@{-}[llu]\\
	1 \ar@{-}[rr] \ar@{-}[rru] \ar@{-}[rruu] & & 2 \ar@{-}[ll] \ar@{-}[llu] \ar@{-}[lluu]\\
}$\\

\textit{Замечание} Двудольные графы часто рисуют из двух частей (долей (принцип построения, как $ K_{3,3}$))\\

\textbf{Теорема} G --- двудолен $ \Leftrightarrow $ все циклы G имеют чётную длину

\textbf{Доказательство}

\textbf{1)} Граф двудолен $ \Rightarrow $ циклы чётные, иначе получаются две соседние вершины с одним цветом:

$\xymatrix{
& 1 \ar@{-}[rr] & & 2 \ar@{-}[rd]\\
& & & & 1 \ar@{-}[ld]\\
& 1 \ar@{-}[rr] \ar@{--}[uu] & & 2 \\
}$ --- нечётный цикл

$\xymatrix{
	& 1 \ar@{-}[rr] & & 2 \ar@{-}[rd]\\
	2 \ar@{-}[ru] & & & & 1 \ar@{-}[ld]\\
	& 1 \ar@{-}[rr] \ar@{-}[lu] & & 2 \\
}$ --- чётный цикл\\

\textbf{2)} Циклы чётные $ \Rightarrow $ граф двудолен: "подвесим граф за вершину"

Выбираем любую вершину и назначаем ей цвет 1. Тогда все вершины, соединённые с ней, будут иметь цвет 2:

$\xymatrix{
	& 1 \ar@{-}[rd] \ar@{-}[ld]& \\
	2  &  & 2\\
}$

Теперь из всех вершин цвета 2 будем также проводить все рёбра туда, куда их ещё не проводили (рёбра, которые направлены на уже нарисованную ранее вершину, нас не интересуют) - новые рёбра будут иметь цвет 1:

$\xymatrix{
	& 1 \ar@{-}[rd] \ar@{-}[ld]& \\
	2  \ar@{-}[d] &  & 2\ar@{-}[d]\\
	1  &  & 1\\
}$

Назначаем цвета "по уровням"

Почему "обратное ребро" (ведущее в ранее существовавшую вершину) не соединяет одинаковые цвета?

Поскольку в таком случае будет нечётный цикл, а по условию все циклы --- чётные:

$\xymatrix{
	1 \ar@{-}[d] \ar@{-}[rrd]  \\
	2 \ar@{-}[d] & & 2 \ar@{--}[ddll] \\
	1 \ar@{-}[d]\\
	2 \\
}$

\newpage

\textbf{Определение} $ \sqsupset G = (V, E)$ --- граф.

$ \psi (G) $ --- хроматическое число графа, минимальное количество цветов, в котрое его можно раскрасить

$\psi $ 
$\begin{pmatrix}
 \xymatrix{
	. \ar@{-}[d]& . \ar@{-}[ld] \ar@{-}[d] \\
	.  & . \ar@{-}[l] \\
}
\end{pmatrix}
$ = 3 ; 
$\psi $ 
$\begin{pmatrix}
	\xymatrix{
		. \ar@{-}[d]& . \ar@{-}[l] \ar@{-}[d] \\
		.  & . \ar@{-}[l] \\
	}
\end{pmatrix}
$ = 2\\

$\psi $ 
$\begin{pmatrix}
	\xymatrix{
	& . \ar@{-}[ld] \ar@{-}[ldd] \ar@{-}[rd] \ar@{-}[rdd]&\\
	. \ar@{-}[d] \ar@{-}[drr] \ar@{-}[rr] & & . \ar@{-}[dll] \ar@{-}[d] \\
	.  \ar@{-}[rr] & & . \\
	}
\end{pmatrix}
$ = 5 (все должны быть разные)\\

$\psi (K_{n} = n)$, где  $K_{n}$ --- полный граф на $ n $ вершиннах\\

\textit{Замечание} Если $ k\ge \psi (G) $, то G можно покрасить в $ k $ цветов\\

\textbf{Утверждение} $ \psi (G) \le (max \text{ } deg \text{ } V +1)$

\underline{Пример:}

$G = $ 
$\begin{matrix}
	\xymatrix{
		1 \ar@{-}[d]& 2 \ar@{-}[ld] \ar@{-}[d] \\
		3  & 2 \ar@{-}[l] \\
	}
\end{matrix}
$ --- степени вершин; $ max \text{ } deg = 3;  \psi (G) \le 3+1 =4$

\textbf{Доказательство} индукция по количеству вершин

База: $ n =0, m = 1 $ --- верно ($ max \text{ } deg = 0,  \psi (G) \ge 1 $)

Переход: $ G,  v - max \text{ } deg = \Delta$, уберём её --- получим $ \bar G = G \text{ без }v $

$ max \text{ } deg \text{ } \bar G \le max \text{ } deg \text{ } G = \Delta$

Раскрасим $ \bar G $ в $ \Delta+1 $ цвет

Для $ v $ запрещены $ \le \Delta $ цветов $ \Rightarrow \ge 1 $ цвет разрешены, покрасим в него $ \blacksquare $\\

\textbf{Утверждение} $ G $ --- планарный граф $ \Rightarrow \psi (G) \le 5 $

\textbf{Доказательство}

\textbf{1)} В $ G $ есть вершина степени $ \le 5 $

$ |V|=n; |E|=m $

Если такой нет $ \Rightarrow deg \text{ } V \ge 6 \Rightarrow \sum deg \text{ } V \ge 6n$

$ \Rightarrow 2m \ge 6n \Rightarrow m \ge 3n $, но в планарном графе $ m \le 3n-6 $, согласно $ \hyperref[eq1]{\text{доказанному ранее}} $ \textbf{Противоречие} $ \Rightarrow $ точно есть вершина степени $ \le 5 $

\textbf{Утверждение} Раскрасим в 5 цветов по индукции

База: графы из 1, 2, 3, 4, 5 вершин точно можно раскрасить

Переход: $ \sqsupset n $ вершин. По индукционному предположению, для графа с ($ n -1$) вершинами раскраска есть

Берём $ v: deg \text{ } v \le 5 $


$\begin{matrix}
	\xymatrix{
		& . \ar@{--}[rd] \ar@{--}[ld]&\\
		. \ar@{--}[d] & v \ar@{-}[u] \ar@{-}[l] \ar@{-}[r] \ar@{-}[ld] \ar@{-}[rd]& . \ar@{--}[d]\\
		. \ar@{--}[rr]  & & . \\
	}
\end{matrix}
$ 

Раскрасим $ \bar G $ без $ v $: если соседи $ v $ используют $ \le 4 $ цветов, $ \Rightarrow $ для $ v $ есть цвет

Остался случай
$\begin{matrix}
	\xymatrix{
		& 1 \ar@{--}[rd] \ar@{--}[ld]&\\
		2 \ar@{--}[d] & v \ar@{-}[u] \ar@{-}[l] \ar@{-}[r] \ar@{-}[ld] \ar@{-}[rd]& 3 \ar@{--}[d]\\
		4 \ar@{--}[rr]  & & 5 \\
	}
\end{matrix}
$ 

Попробуем убрать $ v $:
$\begin{matrix}
	\xymatrix{
		& 1 \ar@{-}[rd] \ar@{-}[ld]&\\
		2 \ar@{-}[d] &   \ar@{--}[u] \ar@{--}[l] \ar@{--}[r] \ar@{--}[ld] \ar@{--}[rd]& 3 \ar@{-}[d]\\
		4 \ar@{-}[rr]  & & 5 \\
	}
\end{matrix}
$ --- получится 1 грань без $ v $ 

Стянем в одну две вершины, которые не соединены ребром и имеют одинковый цвет --- сделаем $ \tilde G $

Если $ v_{i}, v_{j} $ все соединены ребром $ \Rightarrow $ есть $ K_{5} \Rightarrow G$ ---  $ \hyperref[eq2]{\text{не планарен}} $

$ \tilde G $ имеет $ (n-2) $ вершин $ \Rightarrow $ можно раскрасить

Если мы стянули две вершины, значит они имеют однаковый цвет $ \Rightarrow $ есть цвет для $ v $ $ \blacksquare $\\

\textbf{Утверждение} 
$ \psi (G) \le 4 $ $ (\href{https://ru.wikipedia.org/wiki/%D0%A2%D0%B5%D0%BE%D1%80%D0%B5%D0%BC%D0%B0_%D0%BE_%D1%87%D0%B5%D1%82%D1%8B%D1%80%D1%91%D1%85_%D0%BA%D1%80%D0%B0%D1%81%D0%BA%D0%B0%D1%85}{\text{Проблема 4 красок}}) $

Это первая крупная математическая теорема, доказанная с помощью компьютера

\subsection{Хроматические многочлены}

\textbf{Определение} $ \sqsupset \chi (G, k) $ --- функция "сколько способов раскрасить $ G $ в $ k $ цветов" 

$ \chi \begin{pmatrix}
	\xymatrix{
		. \ar@{-}[r]& . \\
	}, k
\end{pmatrix}$ =
$ \left\{
\begin{array}{ccc}
	k=0 & 0 \\
	k=1 & 0 \\
	k=2 & 2 \\
	k=3 & 6 \\
	\text{иначе} & k(k-1) \\
\end{array}
\right. $

$ \chi \begin{pmatrix}
	\xymatrix{
		. \ar@{-}[r]& . \\
	}, k
\end{pmatrix}$ = $ k(k-1)$

$ \chi \begin{pmatrix}
	\xymatrix{
		. & . \\
	}, k
\end{pmatrix}$ = $ k^{2}$

\textbf{1)} (Изолированный граф) $ \chi \begin{pmatrix}
	\varnothing_{n}, k
\end{pmatrix}$ = $ k^{n}$

\textbf{2)} (Полный граф) $ \chi \begin{pmatrix}
	K_{n}, k
\end{pmatrix}$ = $ \underbrace{k(k-1)(k-2)\dots(k-n+1)}_{n \text{ множителей}}=k^{\underline{n}}$

\textbf{3)} (Дерево)
$ \chi \begin{pmatrix}
	T_{n}, k
\end{pmatrix}$ = $ k(k-1)^{n-1}$

\textbf{4)} $ \bar G $ --- граф, $ u, v $ --- вершины, соединены ребром $ (u, v) $

$ G $ ---  $ \bar G $ без вершин $ u, v $

$ \tilde G $ --- $ \bar G $ с соединёнными в одну вершинами $ u, v $

\underline{Пример:}

$ \bar G $
$\begin{matrix}
	\xymatrix{
		. \ar@{-}[rr] \ar@{-}[rd] & & v \ar@{-}[ld] \ar@{-}[rd] \ar@{-}[r] & . \ar@{-}[d]\\
		& u \ar@{-}[rr]& & .\\
	}
\end{matrix}$;
$ G $
$\begin{matrix}
	\xymatrix{
		. \ar@{-}[rr] \ar@{-}[rd] & & v  \ar@{-}[rd] \ar@{-}[r] & . \ar@{-}[d]\\
		& u \ar@{-}[rr]& & .\\
	}
\end{matrix}$;
$ \tilde G $
$\begin{matrix}
	\xymatrix{
		. \ar@{-}[rr] & & u=v  \ar@{-}[rd] \ar@{-}[r] & . \ar@{-}[d]\\
		& & & .\\
	}
\end{matrix}$\\
\begin{center}
	 \begin{equation}\label{eq3}
	\fbox{$\chi (G, k)= \chi (\bar G, k) + \chi (\tilde G, k)$}
\end{equation}
\end{center}
= (способы раскрасить $ G $, если $ u, v $ --- разный цвет) +

+ (способы раскрасить $ G $, если $ u, v $ --- одинаковый цвет)

\textit{Следствие}
\begin{center}
	\fbox{$ \chi (\bar G, k) = \chi (G, k) - \chi (\tilde G, k)$}
\end{center}

\underline{Примеры:}

$ \chi \Bigg( \begin{matrix}
	\xymatrix{
		. \ar@{-}[r]& . \ar@{-}[ld] \\
		. \ar@{-}[r]& . \\
	}
\end{matrix}, k \Bigg)$=
$ \chi \Bigg( \begin{matrix}
	\xymatrix{
		. \ar@{-}[r] \ar@{-}[rd]& . \ar@{-}[ld] \\
		. \ar@{-}[r]& . \\
	}
\end{matrix}, k \Bigg)$+
$ \chi \Bigg( \begin{matrix}
	\xymatrix{
		& . \ar@{-}[ld] \ar@{-}[d]\\
		. \ar@{-}[r]& . \\
	}
\end{matrix}, k \Bigg)$=\\

=$ k(k-1)(k-2)(k-3)  + k(k-1)(k-2)$

\textbf{5)} 
$ \chi \Bigg( \begin{matrix}
	\xymatrix{
		& . \ar@{-}[rr] \ar@{-}[ld]& & . \ar@{-}[rd] \\
		.\ar@{-}[rd] & & C_{n} & & .\ar@{-}[ld] \\
		& . \ar@{-}[rr]& & .  \\
	}
\end{matrix}, k \Bigg)$ $ C_{n} - n \text{ вершин} $\\

$ c_{n} = \chi ( C_{n}, k)= k(k-1)^{n-1}-k(k-1)^{n-2}+k(k-1)^{n-3}+\dots$ \\
 $\dots + (-1)^{n}k(k-1) +(-1)^{n+1}k$
 
 Это \underline{геометрическая прогрессия:}
 
 Начало: $ (-1)^{n+1}k $
 
 Множитель $ q $: $ -(k-1) $
 
 Слагаемых: $ n $ штук

$ S =  (-1)^{n+1}k \frac{q^{n}-1} {q-1}= (-1)^{n+1}k \frac{(1-k)^{n}-1} {1-k-1}=(-1)^{n+1}k \frac{(-1)^{n}(k-1)^{n}-1} {-k}=$ 

 $(-1)^{n} \Big((-1)^{n}(k-1)^{n}-1)\Big)=(k-1)^{n}+(-1)^{n}(k-1)$ $ \blacksquare $
 
\textbf{ Утверждения} 

\textbf{1)} $  \sqsupset G $ имеет висячую вершину $ u $ 

$ \underbrace{\begin{matrix}
	\xymatrix{
		\bar G \ar@{-}[r]& u \\
	}
\end{matrix}}_{G}$, $ \chi (G, k) = \chi (\bar G, k) +\underbrace{(k-1)}_{\text{раскрасить } u} $

\textbf{2)}

$ \underbrace{\begin{matrix}
		\xymatrix{
			. \ar@{-}[rd] \ar@{-}[dd]_{\bar G}& \\
			& u\\
			.\ar@{-}[ru] &
		}
\end{matrix}}_{G}$, $ \chi (G, k) = \chi (\bar G, k) +\underbrace{(k-2)}_{\text{раскрасить } u} $

\textbf{3)} $ G = \bar G \cup \tilde G $ ; между $ \bar G $ и $ \tilde G $ нет рёбер

$ \underbrace{\begin{matrix}
		\xymatrix{
			. \ar@{-}[r]_{\bar G} & . \\
		}
\end{matrix} \text{ нет рёбер }
\begin{matrix}
	\xymatrix{
		. \ar@{-}[r]_{\tilde G} & . \\
	}
\end{matrix}
}_{G}$, $  \chi (G, k) = \chi (\bar G, k) \times \chi (\tilde G, k)$\\
Общий случай рассмотрен $ \hyperref[eq3]{\text{здесь}} $\\

\textbf{Утверждение}
 $ \chi (G, k) $ --- это многочлен:
 
\textbf{ 1)} Старший коэффициент = 1

\textbf{ 2)} Степени = $ n $(количество вершин)

\textbf{ 3)} Знаки чередуются: $ k^{n}-k^{n-1}+k^{n-2}-\dots -0$
 
\textbf{ 4)} Младший коэффициент = 0

\textbf{ 5)} Коэффициент при $ k^{n-1} = \pm m $ (количество рёбер)

\textbf{Доказательство} Индукция по количеству вершин, при равном

 количестве вершин --- по количеству рёбер
 
 База: возьмём пустой граф из $ n $ вершин: $ \chi \begin{pmatrix}
 	\varnothing_{n}, k
 \end{pmatrix}$ = $ k^{n}=0\times k^{n-1}$

Переход: 
$ \bar G $ с рёбрам

$ \chi \begin{pmatrix}
	\xymatrix{
		. \ar@{-}[r] & . \\
	}, k
\end{pmatrix} =
\underbrace{\chi \underbrace{\begin{pmatrix}
	\xymatrix{
		. & . \\
	}, k
\end{pmatrix}}_{\text{мало рёбер }} -
\chi \underbrace{\begin{pmatrix}
	\xymatrix{
		. \\
	}, k
\end{pmatrix}}_{\text{мало вершин }}}_{\text{Работает индукционное предположение }}
$

\textbf{1)} Старший коэффициент = $ (1\times k^{n}-\dots)-( k^{n-1}-\dots) $

\textbf{ 2)} Степень = $ n $

\textbf{ 3)} Знаки чередуются: $ (k^{n}-k^{n-1}+k^{n-2}-\dots)-(k^{n-1}+k^{n-2}-\dots)$

\textbf{ 4)} Младший коэффициент $ = 0-0=0 $

\textbf{ 5)} (Количество рёбер $ G) \times k^{n-1}-k^{n-1}=$ 

 $-(\text{Количество рёбер }G+1) \times k^{n-1}$ = -(Количество рёбер $ \bar G) \times k^{n-1}$
 \newpage 
 На примере:
 
 $ \chi \Bigg( \begin{matrix}
 	\xymatrix{
 		& . \ar@{-}[ld] \ar@{-}[d]  \\
 		. \ar@{-}[r] \ar@{-}[rd]& . \\
 		 & .   \\
 	}
 \end{matrix}, k \Bigg)=(k-1)\times
 \chi \Bigg( \begin{matrix}
	\xymatrix{
		& . \ar@{-}[ld] \ar@{-}[d]  \\
		. \ar@{-}[r] & . \\
	}
\end{matrix}, k \Bigg)=\\=(k-1)k(k-1)(k-2)=k^{4}-4k^{3}+5k^{2}-2k
 $\\
 
\textbf{Утверждение} $ \chi (G) $ --- хроматическое число (минимальное число

 цветов для раскраски)

$ \chi (G, k), k = 0, 1, 2, \dots, \chi (G)-1 $ --- корни, $  \chi (G) $ --- не корень

\subsection{Эйлеровы графы}

$ \begin{matrix}
	\xymatrix{
		. \ar@{-}[rd]  \ar@{-}[d]  \\
		. \ar@{-}[d] \ar@{-}[r] \ar@{-}[rd]& . \ar@{-}[ld] \ar@{-}[d]  \\
		. \ar@{-}[r] & . \\
	}
\end{matrix} $ 

Задача --- нарисовать, не проводя жважды по одному ребру 

(по вершине можно):

$ \begin{matrix}
	\xymatrix{
		5 \ar@{-}[rd]  \ar@{-}[d]  \\
		4, 7 \ar@{-}[d] \ar@{-}[r] \ar@{-}[rd]& 2, 6 \ar@{-}[ld] \ar@{-}[d]  \\
		3, 9 \ar@{-}[r] & 1, 8 \\
	}
\end{matrix} $ Номера --- поряжок обхода\\


\textbf{Определение} Эйлеров путь --- простой путь, содержащий все рёбра\\

\textbf{Определение} Эйлеров цикл --- цикл, содержащий все рёбра\\

(В обоих случаях не проходим дважды по одному ребру)\\

\textbf{Утверждение} Граф $ G $ содержит Эйлеров цикл $ \Leftrightarrow $ граф связен и все степени вершин --- чётные

Пример:

$ \begin{matrix}
	\xymatrix{
		2 \ar@{-}[rd]  \ar@{-}[d]  \\
		4 \ar@{-}[d] \ar@{-}[r] \ar@{-}[rd]& 4 \ar@{-}[ld] \ar@{-}[d]  \\
		 3 \ar@{-}[r] & 3 \\
	}
\end{matrix} $ Номера --- степени вершин $ \Rightarrow $ нет Эйлерова цикла\\

$ \begin{matrix}
	\xymatrix{
		2 \ar@{-}[rd]  \ar@{-}[d]  \\
		4 \ar@{-}[d] \ar@{-}[r] \ar@{-}[rd]& 4 \ar@{-}[ld] \ar@{-}[d]  \\
		4 \ar@{-}[r] & 4 \\
		2 \ar@{-}[ru]  \ar@{-}[u]  \\
	}
\end{matrix} $ Номера --- степени вершин $ \Rightarrow $ есть Эйлеров цикл\\

$ \begin{matrix}
	\xymatrix{
		2 \ar@{-}[rd]  \ar@{-}[d]  \\
		1, 7, 11 \ar@{-}[d] \ar@{-}[r] \ar@{-}[rd]& 3, 10 \ar@{-}[ld] \ar@{-}[d]  \\
		6, 9 \ar@{-}[r] & 4, 8 \\
		5 \ar@{-}[ru]  \ar@{-}[u]  \\
	}
\end{matrix} $ Номера --- порядок обхода\\

\textbf{Доказательство}

$ \Rightarrow $ 

Граф связен, потому что мы прошлись по всем вершинам, значит от каждой до каждой можно дойти

Каждая вершина чётна, потому что количество входов в неё за время цикла = количеству выходов

$ \Leftarrow $ 

Начнём строить цикл: идём из любой вершины, выбираем ребро на каждом шаге,  которое ещё не использовали. В каждой вершине по пути использовано чётное число рёбер ($ k $ входов, $ k $ выходов) + 1 рёбро, через которое мы сейчас вошли $ \Rightarrow $  использовано чётное количество рёбер $ \Rightarrow $  есть ещё $ \ge $ 1 ребро, через которое и нужно уйти.

Такая ситуация складывается на всех вершинах, кроме начальной: из неё мы вышли на 1 раз больше $ \Rightarrow $ мы закончим ходить в начальной вершине.

$ \begin{matrix}
	\xymatrix{
		1, 8 \ar@{-}[d] \ar@{-}[r] & 2, 6 \ar@{-}[ld] \ar@{-}[d] & 3 \ar@{~>}[dll] \ar@{-}[dl] \ar@{-}[l]\\
		7 \ar@{-}[r] & 5 & 4 \ar@{-}[u] \ar@{-}[l] \\
	}
\end{matrix} $ номера --- порядок обхода.

Мы обошли не все вершины, однако вернулись в изначальную.

Нужно выкинуть все просмотренные рёбра, которые мы обошли: в оставшейся части тоже все степени чётные.

Поскольку граф $ G $ связен, из изначальной вершины $ X $ мы можем попасть в любую другую вершину и ребро.

Повторим процесс из ребра, входящего в первый цикл, из которой ведёт новое ребро.

Объединим два цикла: проходимся по первому циклу до первого

 пересечения со вторым, проходим по второму, возвращаемся в то же место пересечения с первым и заканчиваем первый.

Продолжаем до тех пор, пока все рёбра не объеднятся в один цикл.

$ \blacksquare $

\textbf{Теорема} Граф $ G $ содержит Эйлеров путь $ \Leftrightarrow $ 

1) Граф $ G $ связен

2) $ \left[
\begin{array}{ccc}
	\text{Степени всех вершин чётны} \\
	\text{Степени всех вершин, кроме двух, чётны} \\
\end{array}
\right. $

Во втором случае нечётные вершины гарантированно являются началом и концом

$ \begin{matrix}
	\xymatrix{
		2 \ar@{-}[rd]  \ar@{-}[d]  \\
		4 \ar@{-}[d] \ar@{-}[r] \ar@{-}[rd]& 4 \ar@{-}[ld] \ar@{-}[d]  \\
		3 \ar@{-}[r] & 3 \\
	}
\end{matrix} $\\

\subsection{Гамильтоновы графы}

\textbf{Определение} Гамильтонов путь/цикл --- простой путь/цикл на всех вершинах.

$ \begin{matrix}
	\xymatrix{
		2 \ar@{->}[rd]  \ar@{<-}[d]  \\
		4 \ar@{<-}[d] \ar@{-}[r] \ar@{-}[rd]& 4 \ar@{-}[ld] \ar@{->}[d]  \\
		3 \ar@{<-}[r] & 3 \\
	}
\end{matrix} $ ;        
$ \begin{matrix}
	\xymatrix{
		. \ar@{->}[d] \ar@{-}[rd] & & . \ar@{->}[d] \ar@{<-}[ld]\\
		. \ar@{->}[r] & . & . \ar@{-}[l] \\
	}
\end{matrix} $ --- Гамильтоновы пути


$ \begin{matrix}
	\xymatrix{
		.\ar@{-}[d] \ar@{-}[r] & . \ar@{-}[d] \ar@{-}[rd] & & .  \ar@{-}[ld]\\
		. \ar@{-}[r] & . \ar@{-}[r] & . & . \ar@{-}[l] \\
	}
\end{matrix} $ --- Не гамильтонов путь

\subsection{Длины путей в графах}

\textbf{Определение} Длина пути в графе --- количество рёбер в пути

\underline{Пример}

$ \begin{matrix}
	\xymatrix{
		 B \ar@{-}[d] \ar@{-}[rd] & & D  \ar@{-}[ld] \ar@{-}[d] \ar@{-}[r] & F\\
		A \ar@{-}[r] & C & E \ar@{-}[l] \\
	}
\end{matrix} $ 

ABCDF --- путь от A до F, длина 4 (4 ребра)

ACEDF --- путь от A до F, длина 4 (4 ребра)

ACDF --- путь от A до F, длина 3 (3 ребра)

ABCEDF --- путь от A до F, длина 5 (5 рёбер)\\

\textbf{Определение} Расстояние между вершинами --- минимальная длинна между вершинами или $ + \infty $, если пути нет

\textit{Обозначение} $ d(X, Y) $ --- расстояние от $ X $ до $ Y $

\underline{В примере выше}: $ d(A, F) =3$ (достигается на AF)

$ \begin{matrix}
	\xymatrix{
	A \ar@{-}[r] \ar@{-}[d] & . \ar@{-}[r] \ar@{-}[d] & . \ar@{-}[d]  \\
	. \ar@{-}[r] \ar@{-}[d] & . \ar@{-}[r] \ar@{-}[d] & . \ar@{-}[d] \\
	. \ar@{-}[r] & . \ar@{-}[r] & Z \\
	}
\end{matrix} $ диаметр 4; все другие расстояния $ \le 4 $\\

\textbf{Определение} Для каждой вершины графа $ G(V, E) $ можно посчитать максимальное расстояние до других вершин

$ r(v):=max \text{ } \{d(v,s) | s \in V \}$

Радиус $ r(G)=min \text{ } \{r(v) | s \in V \} $

Те вершины, на которых достигается минимум --- \underline{центр}

$ \begin{matrix}
	\xymatrix{
		4 \ar@{-}[r] \ar@{-}[d] & 3 \ar@{-}[r] \ar@{-}[d] & 4 \ar@{-}[d]  \\
		3 \ar@{-}[r] \ar@{-}[d] & 2 \ar@{-}[r] \ar@{-}[d] & 3 \ar@{-}[d] \\
		4 \ar@{-}[r] & 3 \ar@{-}[r] & 4 \\
	}
\end{matrix} $ $ r(G) =2 $ --- радиус графа; центр графа --- вершина 2\\

центров может быть много:

$ \begin{matrix}
	\xymatrix{
		6 \ar@{-}[r] \ar@{-}[d] & 5 \ar@{-}[r] \ar@{-}[d] & 5 \ar@{-}[d] \ar@{-}[r] & 6 \ar@{-}[d]\\
		5 \ar@{-}[r] \ar@{-}[d] & 4 \ar@{-}[r] \ar@{-}[d] & 4 \ar@{-}[d] \ar@{-}[r] & 5 \ar@{-}[d]\\
		5 \ar@{-}[r] \ar@{-}[d] & 4 \ar@{-}[r] \ar@{-}[d] & 4 \ar@{-}[d] \ar@{-}[r] & 5 \ar@{-}[d]\\
		6 \ar@{-}[r] & 5 \ar@{-}[r] \ar@{-}[r]& 5 \ar@{-}[r] & 6 \\
	}
\end{matrix} $ 

4 центра в вершинах 4; $ r(G) =4 $ --- радиус графа\\

$ \begin{matrix}
	\xymatrix{
		3 \ar@{-}[r]  & 2 \ar@{-}[r]  & 2  \ar@{-}[rd] \ar@{-}[r] & 3 \\
		 & & & 3 \\
	}
\end{matrix} $ 

2 центра в вершинах 2; $ r(G) =2 $ --- радиус графа\\

\textbf{Утверждение } В графе $ G(V, E): d(G) \le 2r(G) $

\textbf{Доказательство} $  \sqsupset  c $ --- центр графа; $ u, v \in V $

$ \begin{matrix}
	\xymatrix{
		u \ar@{-}[r]_{\le r}  & c \ar@{-}[r]_{\le r}  & v  \\
	}
\end{matrix} $ 

$ d(c,u) \le r;  d(c,v) \le r \Rightarrow d(u,v) \le 2r \Rightarrow d(G) = \underbrace{max}_{u, v} d(u,v) \le 2r $ $ \blacksquare $\\

\textbf{Утверждение } В дереве $ \le 2 $ центров

$ \sqsupset  $ их 3:

$ \begin{matrix}
	\xymatrix{
		c_{1}  \ar@{--}[r] & c_{0} \ar@{--}[r] \ar@{--}[d] & c_{3}  \\
		& c_{2}  \\
	}
\end{matrix} $ 

Построим пути между $ c_{1} $ и $ c_{2} $

Потом между $ c_{2} $ и $ c_{3} $

(В дереве ровно 1 путь между вершинами)

Вершина развилки $ c_{0}: r(c_{0}) < r(c_{1})=r(c_{2})=r(c_{3})=r(G)=r $\\

\textit{Замечание} Будем иногда дальше использовать оринтированные 

графы $ G=(V, E) $ (рёбра в ориентированном графе иногда 

называют дугами)

$ E \subset \{ (u, v) \text{ --- упорядоченная пара} \} $

$ \begin{matrix}
	\xymatrix{
		. \ar@{->}[r] \ar@{<-}[d] & .  \ar@{<-}[d]  \\
		. \ar@/^/[r] \ar@{->}[ru]  & . \ar@/^/[l]  \\
	}
\end{matrix} $\\

\textit{Замечание} У рёбер будет вес

$ G=(V, E) $; вес --- функция $ f: E \rightarrow \mathbb R $

то есть число у каждого ребра 

$ \begin{matrix}
	\xymatrix{
		. \ar@{->}[r]_3  & . \\
	}
\end{matrix} $ ;
$ \begin{matrix}
	\xymatrix{
		. \ar@{-}[r]_7  & . \\
	}
\end{matrix} $

Расстояние на графе считается как минимальная сумма весов по всем путям

$ \begin{matrix}
	\xymatrix{
		& C \ar@{-}[rr]_4 \ar@{-}[ld]_1 & & D \ar@{-}[rd]_2 \\
		A \ar@{-}[rd]_2 & & & & B \ar@{-}[ld]_5 \\
		& E \ar@{-}[rr]_1 \ar@{-}[rruu]_3 & & F  \\
	}
\end{matrix} $

$  d(A, B) = ? $

$ d(ACDEFB)=1+4+3+1+5=14 $

$ d(AEFB)=2+1+5=8 $

$ d(AEDB)=2++3+2=7 $

$ \Rightarrow d(A, B) =7 $\\

\textit{Замечание} Расстояние во взвешенном графе не всегда существует

$ \begin{matrix}
	\xymatrix{
        & & C \ar@{->}[rd]_{2} & &	\\
		& D \ar@{<-}[rr]_2 \ar@{->}[ru]_{-5} \ar@{<-}[ld]_1 & & E \ar@{->}[rd]_1 \\
		A \ar@{->}[rd]_3 & & & & B \ar@{<-}[ld]_7 \\
		& F \ar@{->}[rr]_4  & & G  \\
	}
\end{matrix} $

$  d(F, G) =4 $

$  d(G, F) =+ \infty $

$  d(A, B) = ? $

$  d(ADCEB) = 1-5+2+1 =-1 $

$  d(ADCEDCEB) = 1-5+2+2-5+2+1 =-2 $ 

и так далее, минимально $ -\infty $

\textbf{Утверждение} В графе есть все расстояния $ \Leftrightarrow $ в графе нет циклов отрицательной длины

\textbf{Доказательство} Если есть цикл отрицательной длины $ \Rightarrow \forall $ две 

вершины этого цикла не иммеют расстояния (или $ -\infty $)

Если нет расстояния, то есть для $ u, v $ есть пути без положительных циклов сколь угодно маленькие. $ \sqsupset  $ есть путь длинее $ n=|V| $ рёбер $ \Rightarrow $ повторяются вершины в пути --- это и будет отрицательный цикл

Как хранятся графы в компьютере?

(Представление графа в компьютере)

\textbf{1.} Матрица смежности: таблица вершины $ \times $ вершины

$ a(i, j) $ =
$ \left\{
\begin{array}{ccc}
    0, \text{ если нет ребра} \\
	1, \text{ если есть ребро} \\
\end{array}
\right. $

$ \begin{matrix}
	\xymatrix{
		1 \ar@{-}[d] \ar@{-}[r] \ar@{-}[rd]& 2 \ar@{-}[d]  \\
		4 & 3 \\
	}
\end{matrix} $ 

\begin{tabular}{ | l | l | l |  l | l | }
\hline
	 & 1 & 2 & 3  & 4 \\ \hline
	1 & 0 & 1 & 1 & 1 \\ \hline
	2 & 1 & 0 & 1 & 0 \\ \hline
	3 & 1 & 1 & 0 & 0 \\ \hline
	4 & 1 & 0 & 0 & 0 \\
\hline
\end{tabular} --- симметрична для неориентироваанных графов\\

Для графов с весами $ a(i, j) $ = вес ребра $ (i, j) $ или $ + \infty $, если ребра нет\\

$ \begin{matrix}
	\xymatrix{
		2 \ar@{->}[dd]_{10} \ar@{->}[r]_{21} \ar@{<-}[rd]_{42} & 3 \ar@{->}[d]_{14}  \\
		 & 4 \\
		 1 \ar@/^/[uu]_{15} \\
	}
\end{matrix} $ 


\begin{tabular}{ | l | l | l |  l | l | }
	\hline
	& 1 & 2 & 3  & 4 \\ \hline
	1 & $ + \infty $ & 15 & $ + \infty $ & $ + \infty $ \\ \hline
	2 & 10 & $ + \infty $ & 21 & $ + \infty $ \\ \hline
	3 & $ + \infty $ & $ + \infty $ & $ + \infty $ & 14 \\ \hline
	4 & $ + \infty $ & 42 & $ + \infty $ & $ + \infty $ \\
	\hline
\end{tabular}\\

Объём памяти: $ n^{2} = |V|^{2} $\\

\textbf{2.} Списки смежности --- для каждой вершины храним список соседей

$ \begin{matrix}
	\xymatrix{
		2 \ar@{->}[dd]_{10} \ar@{->}[r]_{21} \ar@{<-}[rd]_{42} & 3 \ar@{->}[d]_{14}  \\
		& 4 \\
		1 \ar@/^/[uu]_{15} \\
	}
\end{matrix} $ 

$ 1: 2(15) $

$ 2: 1(10), 3(21) $

$ 3: 4(14) $

$ 4: 2(42) $

$ \begin{matrix}
	\xymatrix{
		. \ar@{-}[r] \ar@{-}[rd] \ar@{-}[d] & .  \ar@{-}[d]  \\
		. \ar@{-}[r]   & .   \\
	}
\end{matrix} $\\

$ 1:2,3,4 $

$ 2:1,3 $

$ 3:2,4 $

$ 4:1,3 $\\

Память $ \approx |E| $ --- количество рёбер\\

\textbf{3.} Неявные способы

Умеем вычислять всех соседей $ \forall $ вершины

\underline{Пример:} обход конём шахматной доски

Граф:

вершины = клетки (64)

рёбра --- вершины, связанные ходом коня

Можно для $ \forall $ клетки (вершины) посчитать, куда из неё можно 

попасть

Задача обхода конём $ \Leftrightarrow $ задача поиска Гамильтонова цикла 

в этом графе\\

\underline{Задача:} даны две вершины $ u, v $: найти $ d(u, v) $ и путь, на котором 

достигается это расстояние

\textit{Замечание} оказывается, что найти путь от $ u $ до $ v $ --- это $ \approx $ то же самое, что найти путь от  $ u $ до всех вершин

\subsection{Алгоритм Беллмана - Форда}

$ G(V, E), f $ --- вес, $ u \in V $. Найти расстояние $ d(u, v) \forall v \in V $

Будем писать $ d(v) = d(u, v)$, так как $ u $ не меняется

Будем хранить в массиве $ d $ текущие найденные расстояния

В начала $ d(u)=0, d(v) =+ \infty $, если $ v \ne u $

Релаксация ребра $ e = (v_{1}, v_{2}) $:

$ \begin{matrix}
	\xymatrix{
		v_{1} \ar@{-}[r]  & v_{2}  \\
	}
\end{matrix} $

если $ d(v_{1}) + f(v_{1}, v_{2}) < d(v_{2}) \Rightarrow d(v_{2}):=d(v_{1}) + f(v_{1}, v_{2})$

Алгоритм: 

повторить $ n-1 $ раз, перебрать все рёбра $ e $

 и каждое релаксировать

В неориентированном графе 

$ \begin{matrix}
	\xymatrix{
		v_{1} \ar@{-}[r]  & v_{2}  \\
	}
\end{matrix} $=
$ \begin{matrix}
	\xymatrix{
		v_{1} \ar@/^/[r]  & v_{2} \ar@/^/[l] \\
	}
\end{matrix} $, то есть две релаксации на ребро 
\newpage
Пример 1:

$ \begin{matrix}
	\xymatrix{
		B \ar@{->}[r]_2  \ar@{<-}[d]_1 & C \ar@{<-}[ld]_5 \ar@{->}[d]_3  \\
		A \ar@{->}[r]_{10}   & D   \\
	}
\end{matrix} $

$ n = 4 $ (4 вершины)

$ A: B(1), C(5), D(10) $

$ B:C(2) $

$ C:D(3) $

$ D: $\\

\begin{tabular}{ | l | l | l |  l | l | }
	\hline
	& A & B & C  & D \\ \hline
	d & 0 & $ + \infty $ & $ + \infty $ & $ + \infty $ \\ \hline

\end{tabular}\\

\textit{Шаг 1:}

\begin{tabular}{ | l | l | l |  l | l | }
	\hline
	& A & B & C  & D \\ \hline
	AB & 0 & 1 & $ + \infty $  & $ + \infty $ \\ \hline
	AC & 0 & 1 & 5 & $ + \infty $ \\ \hline
	AD & 0 & 1 & 5 & 10 \\ \hline
	BC & 0 & 1 & 3 & 10 \\ \hline
	CD & 0 & 1 & 3 & 6 \\
	\hline
\end{tabular}\\

\textit{Шаг 2:}

\begin{tabular}{ | l | l | l |  l | l | }
	\hline
	& A & B & C  & D \\ \hline
	AB & 0 & 1 & $ + \infty $  & $ + \infty $ \\ \hline
	AC & 0 & 1 & 5 & $ + \infty $ \\ \hline
	AD & 0 & 1 & 5 & 10 \\ \hline
	BC & 0 & 1 & 3 & 10 \\ \hline
	CD & 0 & 1 & 3 & 6 \\
	\hline
\end{tabular} --- ничего не изменится\\

\textit{Шаг 3:}

\begin{tabular}{ | l | l | l |  l | l | }
	\hline
	& A & B & C  & D \\ \hline
	AB & 0 & 1 & $ + \infty $  & $ + \infty $ \\ \hline
	AC & 0 & 1 & 5 & $ + \infty $ \\ \hline
	AD & 0 & 1 & 5 & 10 \\ \hline
	BC & 0 & 1 & 3 & 10 \\ \hline
	CD & 0 & 1 & 3 & 6 \\
	\hline
\end{tabular} --- ничего не изменится\\

Время работы $ \approx |V| \times |E| \le |V|^{3} $\\

Ответ
$ \left\{
\begin{array}{ccc}
	d(A)=0 \\
	d(B)=1 \\
	d(C)=3 \\
	d(D)=6 \\
\end{array}
\right. $

\newpage
Пример 2:

$ \begin{matrix}
	\xymatrix{
		A \ar@{->}[rd]_1  \ar@{->}[d]_3 & 1 \ar@{->}[l]  \\
		B \ar@{->}[r]_{1}   & C \ar@{-}[u]   \\
	}
\end{matrix} $

A: 3B, 1C

B: 4C

C: 1B 

$ n = 3 \Rightarrow n - 1 = 2 \Rightarrow  $ 2 раза запускаем циклы  релаксации

Пути из A:

\begin{tabular}{ | l | l | l |  l | }
	\hline
	& A & B & C   \\ \hline
	d & 0 & $ + \infty $ & $ + \infty $  \\ \hline
	
\end{tabular}\\

Шаг 1:
\begin{tabular}{ | l | l | l |  l  | }
	\hline
	& A & B & C   \\ \hline
	AB & 0 & $ + \infty $ & $ + \infty $   \\ \hline
	AC & 0 & 3 & $ + \infty $  \\ \hline
	BC & 0 & 3 & 1  \\ \hline
	CB & 0 & 2 & 1  \\
	\hline
\end{tabular}\\

Шаг 2: AB, AC, BC, CB --- ничего не улучшилось\\

Ответ: \begin{tabular}{ | l | l | l |  }
	\hline
	A & B & C   \\ \hline
	0 & 2 & 1  \\
	\hline
\end{tabular}\\

\underline{Корректность алгоритма}

\textbf{Теорема } В конце массив $ d $ содержит расстояния от A

\textbf{Доказательство} После $ i $-го цикла релаксации всех рёбер $ d $ хранит числа $ d(v) \le min (\text{длин путей, в которых} \le i \text{ рёбер}) $

Действительно:

База:  $ i = 0 = min (\text{путей из 0 рёбер}) $ (Только путь A --- A) 

$ d(A) =0$ --- верно; $ d(u) =+\infty$

Переход: $ \sqsupset $ есть оптимальный путь из $ i+1 $ ребра

$ \underbrace{\begin{matrix}
	\xymatrix{
		A \ar@{--}[rr]  & & C \ar@{-}[r] & B  \\
	}
\end{matrix}}_{i+1 \text{ ребро}} $

$ \underbrace{\begin{matrix}
		\xymatrix{
			A \ar@{--}[rr]  & & C \\
		}
\end{matrix}}_{i \text{ ребер}} $ B

По индукционному предположению $ d(C) = dist(A, C) $

Длина пути $\begin{matrix}
		\xymatrix{
			A \ar@{-}[r] & C \ar@{-}[r] & B  \\
		}
\end{matrix} = dist(C) (= d(C)) + dist(C, B)$

Проверка: $ d(C) + dist(C, B) \le d(B)$ --- верно, так как путь 
$\begin{matrix}
	\xymatrix{
		A \ar@{-}[r] & C \ar@{-}[r] & B  \\
	}
\end{matrix}$ оптимальный = $ d(B) = d(C) + dist(C, B) $

\textit{Почему $ n-1 $ этап?}

Оптимальный путь не содержит циклов $ \Rightarrow \le n-1 $ ребро $ \blacksquare $

\textit{Замечание} Мы выясняем только расстояния, но путь не называется

\underline{Как восстановить путь?}

Будем сохранять информацию об успешных  релаксациях

$ Prev $ --- изначально пустой массив вершин

Если релаксация $\begin{matrix}
	\xymatrix{
		u \ar@{-}[r] & v  \\
	}
\end{matrix}$ успешна, то $ Prev[v] = u $ --- оптимальный путь в $ v $ лежит через $ u $

$\begin{matrix}
	\xymatrix{
		A \ar@{->}[r]_3 \ar@{->}[d]_1 & B  \\
		C \ar@{->}[ru]_1 \\
	}
\end{matrix}$

\begin{tabular}{ | l | l | l |  l | }
	\hline
	& A & B & C   \\ \hline
	d & 0 & $ + \infty $ & $ + \infty $  \\ \hline
	
\end{tabular}\\

\begin{tabular}{ | l | l | l |  l  | }
	\hline
	& A & B & C   \\ \hline
	AB & 0 & $ + \infty $ & $ + \infty $   \\ \hline
	AC & 0 & 3/A & $ + \infty $  \\ \hline
	BC & 0 & 3/A & 1/A  \\ \hline
	CB & 0 & 2/C & 1/A  \\
	\hline
\end{tabular}\\

Восстановить путь в $ B$:

$\begin{matrix}
	\xymatrix{
		A \ar@{->}[r]_{Prev[C]} & C \ar@{->}[r]_{Prev[B]} & B  \\
	}
\end{matrix}$

В общем случае путь $\begin{matrix}
	\xymatrix{
		A \ar@{->}[r] & v  \\
	}
\end{matrix}$ --- это

$\begin{matrix}
	\xymatrix{
		A \ar@{-->}[r] & Prev[Prev[v]] \ar@{->}[r] & [Prev[v]\ar@{->}[r] & v  \\
	}
\end{matrix}$

\subsection{Алгоритм Дейкстры}

В отличие от алгоритма Беллмана - Форда, требуется чтобы все веса $ w(e) \ge 0 $

\textbf{Алгоритм} Дан граф $ G = (V, E), A \in V $

Найти расстояния до всех вершин $ d(u) = dist(A, u) $

\underline{Алгоритм} $ d(A) = 0, d(u \ne A) =  + \infty  $

$ P = \varnothing $ --- обработанные вершины

Повторяем $ n = V $ раз:

Выбрать $ u \in V \textbackslash  P $, где $ d(u) $ --- минимальный 

(выбираем из необработанных вершин ту, у которой минимальный $ d $)

$ \forall e \in \text{ ребро из } u, e = (u, v) $

Релаксируем ребро $ e $

$ P = P \cup \{u\} $

\newpage
\underline{Пример:}

$\begin{matrix}
	\xymatrix{
		A \ar@{->}[r]_1 \ar@{->}[d]_3 & B \ar@{->}[d]_8 \\
		C \ar@{->}[r]_4 & D \ar@{->}[r]_2 & E \\
	}
\end{matrix}$

\begin{tabular}{ | l | l | l |  l |  l |  l | l |}
	\hline
	& A & B & C & D & E &  \\ \hline
	$ d $ & \fbox{0} & $ + \infty $ & $ + \infty $ & $ + \infty $ & $ + \infty $ & $ P = \varnothing $ \\ \hline
	$ u = A $ &  & \fbox{1} & 3 & $ + \infty $ & $ + \infty $ & $ P = \{A\} $ \\ \hline
	$ u = B $ &  &  & \fbox{3} & 9 & $ + \infty $ & $ P = \{A, B\} $ \\ \hline
	$ u = C $ &  &  &  & \fbox{7} & $ + \infty $ & $ P = \{A, B, C\} $ \\ \hline
	$ u = D $ &  &  &  &  & 9 & $ P = \{A, B, C, D\} $ \\ \hline
	$ u = E $ &  &  &  &  & \fbox{9} & $ P = \{A, B, C, D\} $ \\ \hline
\end{tabular}\\

\textbf{Эффективность:} $ |V| \times |E| \times \underbrace{log |V|}_{\text{Выбор } min} $\\

\textbf{Корректность}

\textit{Идея:} На каждом шаге $ d(u) = min \text{ путей }
\begin{matrix}
	\xymatrix{
		A \ar@{-->}[rrr]_{\text{обработанные вершины}} & & & u \\
	}
\end{matrix}$

База: шаг $ = 0 ; d(A)=0; d(u) = + \infty $

Переход: Выбираем $ u =$ минимальная вершина из $ V \textbackslash \{P\} $

$ \sqsupset $ есть оптимальный путь в $ u $

$\begin{matrix}
	\xymatrix{
		A \ar@{-->}[rrr]_{\text{обработанные рёбра}} & & & \bar u \ar@{-->}[r]_x & u\\
	}
\end{matrix}$

$ dist(\bar u) = dist(u) - x $

По индукционному предположению $ dist(\bar u) = d(\bar u) $

$ d(u) \ge dist(u) > dist(\bar u) \Rightarrow d(u) > d(\bar u) $

$ \Rightarrow $ \textbf{Противоречие} ( $ d(u) $ был минимальным)

$ \sqsupset $ оптимальный путь в $ v $ идёт через $ u $:

$ dist(A, u) + w(u, v) = dist (A, v) \Rightarrow $ релаксация
$\begin{matrix}
	\xymatrix{
		 u \ar@{->}[r] & v\\
	}
\end{matrix}$ успешна, $ d(v) $ получит расстояние
$ \blacksquare $

Для восстановления пути нужен аналогичный массив $ Prev $

В случае успешной релаксации $\begin{matrix}
	\xymatrix{
		u \ar@{->}[r] & v\\
	}
\end{matrix}$ $ Prev[v] = u $
\newpage

$\begin{matrix}
	\xymatrix{
		A \ar@{->}[r]_1 \ar@{->}[d]_3 & B \ar@{->}[d]_8 \\
		C \ar@{->}[r]_4 & D \ar@{->}[r]_2 & E \\
	}
\end{matrix}$\\

\begin{tabular}{ | l | l |  l |  l |  l |}
	\hline
	A & B & C & D & E  \\ \hline
	 \fbox{0} & $ + \infty $ & $ + \infty $ & $ + \infty $ & $ + \infty $  \\ \hline
	  & \fbox{1/A} & 3/A & $ + \infty $ & $ + \infty $  \\ \hline
	  &  & \fbox{3/A} & 9/B & $ + \infty $  \\ \hline
	  &  &  & \fbox{7/C} & $ + \infty $  \\ \hline
	  &  &  &  & \fbox{9/D}  \\ \hline
\end{tabular}\\

Путь $\begin{matrix}
	\xymatrix{
		A \ar@{->}[r] & E\\
	}
\end{matrix}$: 
$\begin{matrix}
	\xymatrix{
		A \ar@{->}[r]_{Prev[C]} & C \ar@{->}[r]_{Prev[D]} & D \ar@{->}[r]_{Prev[E]} & E  \\
	}
\end{matrix}$\\

\textbf{Утверждение} В дереве $ 1 \le $ центров $ \le 2 $

$ \begin{matrix}
	\xymatrix{
		4 \ar@{-}[r] & 3 \ar@{-}[r]  & \fbox{2} \ar@{-}[r]  & 3  \ar@{-}[rd] \ar@{-}[r] & 4 \\
		4 \ar@{-}[ru] & & & & 4 \\
	}
\end{matrix} $; Центр --- вершина, максимальное 

растояние из которой минимально\\

\textbf{Доказательство} Если убрать у дерева все висячие вершины,

 максимальные расстояния уменьшатся на 1
 
 $ \begin{matrix}
 	\xymatrix{
 		4 \ar@{-}[r] & 3 \ar@{-}[r]  & \fbox{2} \ar@{-}[r]  & 3  \ar@{-}[rd] \ar@{-}[r] & 4 \\
 		4 \ar@{-}[ru] & & & & 4 \\
 	}
 \end{matrix}
\Rightarrow
 \begin{matrix}
 	\xymatrix{
 		 2 \ar@{-}[r]  & \fbox{1} \ar@{-}[r]  & 2 \\
 		 & &  \\
 	}
 \end{matrix}
 $
 
 Если продолжать убирание висячих вершин, придём 
 
 к одной из ситуаций:
 
 $ \begin{matrix}
 	\xymatrix{
 		 \fbox{.} \\
 	}
 \end{matrix} $ --- центр;

$ \begin{matrix}
	\xymatrix{
		\fbox{.} \ar@{-}[r] & \fbox{.} \\
	}
\end{matrix} $ --- центры

$ \blacksquare $\\

\newpage

\subsection{Алгоритм Флойда}
$  $\\

Дан взвешенный граф $ G = (V, E) $

Вернуть "таблицу"$  $ $ d(v, u) $, содержащую расстояния между всеми

парами вершин

$ Алгоритм $

Составляем матрицу $ d_{0} $:

$ d_{0} (u, u) = 0 $

$ d_{0} (u, v)=
\left[
\begin{array}{ccc}
	\infty \text{, если нет ребра } u-v  \\
	w (u, v)\text{, если есть ребро } u-v
\end{array}
\right.$\\
 
 $ \begin{matrix}
 	\xymatrix{
 		 & C \ar@{->}[rd]_1 \\
 		A \ar@{->}[rr]^5 \ar@{->}[ru]_2 & & B \ar@/^/[ll]^3  \\
 	}
 \end{matrix} $

\begin{tabular}{ | l | l | l |  l |}
	\hline
	 $ d_{0} $ & A & B & C  \\ \hline
	$ A $ & 0 & 5 & 2 \\ \hline
	$ B $ & 3 &0 & $ \infty $ \\ \hline
	$ C $ & $ \infty $ & 1 & 0 \\ \hline
\end{tabular}\\


for $ k \in V $
	
$ \text{		} $for $ u \in V $
		
$ \text{		}  \text{		}$ for $ v \in V $
			
$ \text{		}  \text{		} $ if $ d(u, v) > d(u, k) + d(k, v) \Rightarrow d(u, v) = d(u, k) + d(k, v)$\\

\underline{Пример:}

$ K=A $

\begin{tabular}{ | l | l | l |  l |}
	\hline
	 & A & B & C  \\ \hline
	$ A $ & 0 & 5 & 2 \\ \hline
	$ B $ & 3 &0 & \fbox{5} \\ \hline
	$ C $ & $ \infty $ & 1 & 0 \\ \hline
\end{tabular}\\

$ K=B $

\begin{tabular}{ | l | l | l |  l |}
	\hline
	& A & B & C  \\ \hline
	$ A $ & 0 & 5 & 2 \\ \hline
	$ B $ & 3 &0 & 5 \\ \hline
	$ C $ &  \fbox{4}   & 1 & 0 \\ \hline
\end{tabular}\\

\newpage
$ K=C $

\begin{tabular}{ | l | l | l |  l |}
	\hline
	& A & B & C  \\ \hline
	$ A $ & 0 & \fbox{3} & 2 \\ \hline
	$ B $ & 3 &0 & 5 \\ \hline
	$ C $ &  4   & 1 & 0 \\ \hline
\end{tabular}\\

\textbf{Корректность}

Утверждение После шага $ K d(u, v) = min d(\text{пути})$

$ \begin{matrix}
	\xymatrix{
		& u \ar@{->}[r] & \text{пути от 1 до к} \ar@{->}[r] & v \\
	}
\end{matrix}$

\textbf{Доказательство} по индукции для $ K $

\underline{База:} $ K=0 $

$ d(u, v) = min(
\begin{matrix}
	\xymatrix{
		u \ar@{->}[r] & \text{нет} \ar@{->}[r] & v \\
	}
\end{matrix}
) $

Действительно, сначала $ d $ содержит длины рёбер

\underline{Переход $ k \rightarrow k+1$:}

$ \begin{matrix}
	\xymatrix{
		& u \ar@{->}[r] & \text{вершины от 1 до к+1} \ar@{->}[r] & v \\
	}
\end{matrix}$

$ \sqsupset $ есть оптимальный путь
$ \begin{matrix}
	\xymatrix{
	 u \ar@{->}[r] & \text{вершины от 1 до к+1} \ar@{->}[r] & v \\
	}
\end{matrix}$

$\left[
\begin{array}{ccc}
	\text{1) В нём нет вершины } k+1 \Rightarrow \text{ его длина } =d(u, v) \\
	\text{2) В нём есть вершина } k+1
	\begin{matrix}
		\xymatrix{
			u \ar@{-}[r] & 1 \dots k \ar@{-}[r] & \text{к+1} \ar@{->}[r] & 1 \dots k \ar@{-}[r] & v \\
		}
	\end{matrix}
\end{array}
\right.$\\

В случае 2) его длина $ =d(u, k+1) + d(k+1, v) $

Это являеся ровно проверкой из цикла выше

Наименьший вариант запишется в $ d $ 

В конце $ d(u, v) = min(u \dots \forall \text{ вершины } \dots v) = dist(u, v) $

$ \blacksquare $\\

\textit{Замечание} Чтобы восстановить путь, можно ввести массив $ through: $

$ if d(u, v) > d(u, k) + d(k, v) \Rightarrow d(u, v) = d(u, k) + d(k, v)$

$ through(u, v) = k $

Для восстановления пути:

$ \underbrace{\begin{matrix}
	\xymatrix{
		A \ar@{-->}[r]_{\text{th(A, I)}} & I \ar@{-->}[r] & B \\
	}
\end{matrix}}_{\text{ th(A, B)}} $

И так далее \dots

Если ребро $ (X, Y) $ не записано $ \Rightarrow $ ребро $ X - Y $ --- оптимальный путь\\

\newpage
\textit{Замечание}

Алгоритм Флойда ищет транзитивные замыкания бинарных 

отношений

$ \sqsupset R $ --- бинарное отношение на $ M $

$ \sqsupset \bar R $ --- транзитивное замыкание $ R $, если 

1) $ \bar R \supset R $

2) $ \bar R $ --- транзитивно

3) $ \forall \tilde R: \bar R \supset \tilde R \supset R $ --- не траннзитивно

\underline{Пример:}

$ \begin{matrix}
	\xymatrix{
		& d \ar@{<-}[d] & c  \ar@/^/[d] \\
		& b  & a \ar@{->}[l] \ar@{->}[u] \\
	}
\end{matrix}$ --- не транзитивно: aRb, bRd, a$ \bcancel R$d

делаем транзитивным:

$ \begin{matrix}
	\xymatrix{
		& d \ar@{<-}[d] & c  \ar@/^/[d] \ar@{->}[dl] \ar@{->}[l] \ar@/^/[r] &  \ar@/^/[l] \\
		& b  & a \ar@/^/[d] \ar@{->}[lu] \ar@{->}[l] \ar@{->}[u] \\
		& &  \ar@/^/[u]
	}
\end{matrix}$ --- теперь транзитивно

\textbf{Теорема}

$ \sqsupset R $ --- бинарное отношение на $ M $

$ \sqsupset G=(M, R) $ --- граф отношения

Тогда $ \bar R $ --- это $ x\bar Ry \Leftrightarrow $ есть путь
$ \begin{matrix}
	\xymatrix{
		x \ar@{->}[r] & y  \\
	}
\end{matrix}$ в $ G $

\textbf{Доказательство}

1) $ \bar R \supset R $, так как если $ xRy \Rightarrow$ есть путь из 1 ребра $ \Rightarrow x\bar Ry $

2) $ \bar R $ --- транзитивно, так как $ x\bar Ry, y\bar Rz \Rightarrow x\bar Rz$ (путь $ \begin{matrix}
	\xymatrix{
		x \ar@{->}[r] & z  \\
	}
\end{matrix}$ тоже есть)

3) $ \sqsupset \tilde R \supset R $, $ \tilde R $ --- транзитивно

$ \sqsupset $ есть путь $ x $ в $ y $:
 $ \begin{matrix}
	\xymatrix{
		x \ar@{->}[r] & \dots \ar@{->}[r] & x_{n} \ar@{->}[r] & \dots \ar@{->}[r] & y  \\
	}
\end{matrix}$

$ \left.
\begin{array}{ccc}
	 xRx_{1} \Rightarrow x \tilde Rx_{1}\\
	 x_{1}Rx_{2} \Rightarrow x_{1} \tilde Rx_{2}
\end{array}
\right\}  x \tilde Rx_{2} \Rightarrow x\tilde Rx_{n} \Rightarrow x \tilde Ry$

$ \Rightarrow \tilde R \supset \bar R \supset R$

$ \blacksquare $

\newpage

\textbf{Приминение алгоритма Флойда к графу $ G=(M, R) $}
\\
$ d_{0} (x, y)=
\left[
\begin{array}{ccc}
	1 \text{, если } xRy \\
	\infty \text{, если } x \bcancel Ry
\end{array}
\right.$\\

Замыкание, $ x \bar Ry $, если $ d(x, y) < \infty$

\underline{На практике:}

\textbf{Алгоритм транзитивного замыкания}

$ \bar R = R $

 for $ k < M $

$ \text{		} $ for  $ x < M $

$ \text{		}  \text{		} $ for $ y < M $

if $ xRy$  $ \&$  $ kRy $

$ \tilde R \leftarrow (x, y)$, то есть сделать $ x \bar Ry $

\subsection{Потоки в сетях}

\textbf{Определение} Сеть --- ориентированный граф $ G=(V, E), s \in V, t \in V $

$ \ne \exists e = (u, s)$ --- ничего не входит

 $ \ne \exists e = (t, u)) $ --- ничего не выходит
 
 $ \begin{matrix}
 	\xymatrix{
 		& . \ar@{<-}[rr] \ar@{<-}[ld] & & . \ar@{->}[rd] \\
 		s \ar@{->}[rd] & & & & t \ar@{<-}[ld] \\
 		& . \ar@{-<}[rr] \ar@{->}[rruu] & & .  \\
 	}
 \end{matrix} $\\

$ c: E \rightarrow \mathbb N$ --- пропускная способность рёбер

\underline{Пример:}

$\begin{matrix}
	\xymatrix{
		& & a \ar@{->}[rdr]_3 \ar@{->}[rdd]_6 \ar@{<-}[ldd]_7 \ar@{<-}[lld]_5 \\
		s \ar@{->}[rd]_1 & & & & t \ar@{<-}[ld]_2 \\
		& b \ar@{<-}[rr]_5 &  & c  \\
	}
\end{matrix}$

\textbf{Определение} Поток $ f $ в сети $ G $ --- это 

$ f: E \rightarrow \mathbb R$:

1) $ 0 \le f(e) \le c(e) $

2) $ \forall u \ne s, t: \sum_{e =(v, u) \in E} f(e) = \sum_{e =(u, v) \in E} f(e) $
\newpage

\underline{Пример 1:}

$\begin{matrix}
	\xymatrix{
		& & a \ar@{->}[rdr]_2 \ar@{->}[rdd]_2 \ar@{<-}[ldd]_2 \ar@{<-}[lld]_2 \\
		s \ar@{->}[rd]_1 & & & & t \ar@{<-}[ld]_1 \\
		& b \ar@{<-}[rr]_1 &  & c  \\
	}
\end{matrix}$\\

\underline{Пример 2:}

$\begin{matrix}
	\xymatrix{
		& & b \ar@{->}[rd]_4 \\
	& a \ar@{->}[rd]_8 \ar@{->}[ru]_7 & & d \ar@{->}[rd]_5 \ar@/^/[dl]_4 \\
	s \ar@{->}[rd]_8 \ar@{->}[ru]_3 & & c \ar@{->}[uu]_2 \ar@{->}[rd]_2 \ar@/^/[ru]_1 & & t \\
	& e \ar@{->}[rd]_6 \ar@{->}[ru]_5 & & g \ar@{->}[ru]_6 \ar@{->}[uu]_6 \\
	& & f \ar@{->}[ru]_3
	}
\end{matrix}$ --- сеть $ (c) $ \\

$\begin{matrix}
	\xymatrix{
		& & b \ar@{->}[rd]_1 \\
		& a \ar@{->}[rd]_0 \ar@{->}[ru]_1 & & d \ar@{->}[rd]_2 \ar@/^/[dl]_0 \\
		s \ar@{->}[rd]_3 \ar@{->}[ru]_1 & & c \ar@{->}[uu]_0 \ar@{->}[rd]_2 \ar@/^/[ru]_1  & & t \\
		& e \ar@{->}[rd]_0 \ar@{->}[ru]_3 & & g \ar@{->}[ru]_0 \ar@{->}[uu]_2 \\
		& & f \ar@{->}[ru]_0
	}
\end{matrix}$ --- поток $ (f) $

$ 0 \le f(e) \le c(e) $ \\

\underline{Пример 3:}

$\begin{matrix}
	\xymatrix{
		& . \ar@{->}[rd]_1^1 \ar@/^/[dd]^{100} \\
		s \ar@{->}[rd]_1^1 \ar@{->}[ru]_1^1 &  & t \\
		& . \ar@{->}[ru]_1^1 \ar@/^/[uu]^{100} \\
	}
\end{matrix}$ --- поток корректный\\

Величина потока в примере 1: 4, в примере 2 --- 2\\

\textbf{Теорема} 

Дана сеть $ (G(V, E), c) $, поток $ f $ на $ G $

Тогда
$ \sum\limits_{u: e = (s, u)} f(e) = \sum\limits_{u: e = (u, t)} f(e) $

Рассмотрим $ \sum\limits_{e \in E} f(e) =$

$ \sum\limits_{v \in V} \sum\limits_{e: e = (u, v)} f(e) = \underbrace{\sum\limits_{e: e = (u, s)} f(e)}_0 + \sum\limits_{e: e = (u, t)} f(e) + \sum\limits_{v \in V \backslash \{s, t\}} \sum\limits_{e: e = (u, v)} f(e) =$\\

$ = \text{вытекает } + \sum\limits_{v \in V \backslash \{s, t\}} \sum\limits_{e: e = (v, u)} f(e) = $

$ = \text{вытекает } + \sum\limits_{v \in V} \sum\limits_{e: e = (v, u)} f(e) - \sum\limits_{e: e = (s, u)} f(e) - \sum\limits_{e: e = (t, u)} f(e) =$

$ = \text{вытекает } + \sum\limits_{v \in V} \sum\limits_{e: e = (v, u)} f(e) - \text{втекает } -0  =$

$ = \text{вытекает } - \text{втекает } + \sum\limits_{e \in E} f(e) \Rightarrow $

$ \sum\limits_{e \in E} f(e) = \text{вытекает } - \text{втекает } + \sum\limits_{e \in E} f(e) $\\

$ \bcancel{\sum\limits_{e \in E} f(e)} = \text{вытекает } - \text{втекает } + \bcancel{\sum\limits_{e \in E} f(e)} \Rightarrow \text{вытекает } = \text{втекает } $

$ \sum\limits_{e: e = (u, t)} f(e) = \sum\limits_{e: e = (s, u)} f(e) $ 

Эта величина называется \underline{величиной потока} $ w(f) $

$ \blacksquare $\\

\textbf{Определение} Разрез в сети $ (G(V, E), c) $

Разрез $ G(V_{1},V_{2} ): s \in V_{1}, t \in V_{2}, V_{1} \cup V_{2} = V, V_{1} \cap V_{2} = \varnothing $

$\begin{matrix}
	\xymatrix{
		& a \ar@{->}[rdd] \ar@{->}[r] & b \ar@{->}[rd] \\
		s \ar@{->}[rd] \ar@{->}[ru] &  & & t \\
		& c \ar@{->}[r] \ar@{->}[ruu] & d \ar@{->}[ru] \\
	}
\end{matrix}$

Например, $ V_{1} = \{s, a, c\}, V_{2} = \{b, d, t\} $\\

\textbf{Определение} Рёбра разреза $ E_{c} $ --- все рёбра разреза, которые идут из $ V_{1} $ в $ V_{2} $ или наоборот

$ E_{c}^{+} $ --- прямые рёбра разреза (из $ V_{1} $ в $ V_{2} $)

$ E_{c}^{-} $ --- обратные рёбра разреза (из $ V_{2} $ в $ V_{1} $)

\newpage

$\begin{matrix}
	\xymatrix{
		& a \ar@{->}[dd] \ar@{->}[r] & b \ar@{->}[rd] \\
		s \ar@{->}[rd] \ar@{->}[ru] &  & & t \\
		& c \ar@{->}[r] \ar@{->}[ruu] & d \ar@{->}[ru] \ar@{<-}[uu]  \\
	}
\end{matrix}$\\

$ E_{c}^{+} = \{s, a, b\} $

$ E_{c}^{-} = \{c, d, t\}$\\

Прямое $ E_{c}^{+} = \{sc,ac, bd, bt \} $

Обратное $ E_{c}^{-} = \{cb\}$\\

$ E_{c} = E_{c}^{+} \cup E_{c}^{-} $\\

\textbf{Определение} Величина разреза = $ \sum\limits_{e \in E_{c}^{+}} c(e) $

\underline{Обозначение} $ c(G) $

$\begin{matrix}
	\xymatrix{
		& a  \ar@{->}[r]_8 & b \ar@{->}[rd]_2 \\
		s \ar@{->}[rd]_6 \ar@{->}[ru]_5 &  & & t \\
		& c \ar@{->}[r]_4 \ar@{->}[ruu]_3 & d \ar@{->}[ru]_1 \\
	}
\end{matrix}$

$ V_{1} = \{s, a, c\}, V_{2} = \{b, d, t\} $

$ C(V_{1}, V_{2})$ 

 $ c(C)= 8+3+4=15 $

\textbf{Утверждение}
Путь есть сеть $ (G(V, E), c) $, поток $ f $, разрез $ C(V_{1}, V_{2}) $

Тогда $ w(f) = \sum\limits_{e \in E_{c}^{+}} f(e) - \sum\limits_{e \in E_{c}^{-}} f(e)$

\newpage

\underline{Пример:}

$\begin{matrix}
	\xymatrix{
		& & b \ar@{->}[rd]_1 \\
		& a \ar@{->}[rd]_0 \ar@{->}[ru]_1 & & d \ar@{->}[rd]_2 \ar@/^/[dl]_0 \\
		s \ar@{->}[rd]_3 \ar@{->}[ru]_1 & & c \ar@{->}[uu]_0 \ar@{->}[rd]_2 \ar@/^/[ru]_1  & & t \\
		& e \ar@{->}[rd]_0 \ar@{->}[ru]_3 & & g \ar@{->}[ru]_0 \ar@{->}[uu]_2 \\
		& & f \ar@{->}[ru]_0
	}
\end{matrix}$

$ V_{1} = \{s, e, f, g\}, V_{2} = \{a, b, c, d, t\} $

$ w(f) = 1+3 = 2+2=4 $

$ \sum\limits_{e \in E_{c}^{+}} f(e) = 1+3+0+2=6 $\\

$ \sum\limits_{e \in E_{c}^{-}} f(e)=2 $\\

Действительно, $ 4 = 6-2 $

\textbf{Доказательство}

Посчитаем сумму $ \sum\limits_{v \in V_{1}} (\sum\limits_{e: e = (u, v)} f(e)- \sum\limits_{e: e = (v, u)} f(e)): $

1) Для $ \forall v \in V_{1} \backslash \{s\} $ внутренняя $ \sum - \sum =0 $

Для $ v = s: w(f) = \sum\limits_{e: e = (s, u)} f(e) $\\

2) $ \sum\limits_{e: e = (u, v), u \in V_{1}, v \in V_{1}} (f(e)-f(e)) + \underbrace{\sum\limits_{e \in E_{c}^{+}} f(e) + \sum\limits_{e \in E_{c}^{-}} [-f(e)]}_{\text{см. условие }} +$ величина из условия $ \blacksquare $

\underline{Обозначение} $ w(C, f) $ --- величина (размер) потока через разрез 

$ = \sum\limits_{e \in E_{c}^{+}} f(e) - \sum\limits_{e \in E_{c}^{-}} f(e) $\\

\textit{Замечание } $ \forall C \text{	} w(f) = w(C, f) $ --- по Теореме\\

\textit{Замечание } Будем решать задачу о максимльном потоке в сети 

(найти $ f: w(f) \rightarrow max$)\\

\newpage

\textbf{Утверждение} Дано: $ (G(V, E), c) $ --- сеть, $ C $ --- разрез

Тогда $ w(f) \le c(C) $

\textbf{Доказательство}

$ w(f) = w(C, f) = \sum\limits_{e \in E_{c}^{+}} f(e) - \sum\limits_{e \in E_{c}^{-}} f(e) \le \sum\limits_{e \in E_{c}^{+}} f(e) \le \sum\limits_{e \in E_{c}^{+}} c(e) = c(C)$

$ \Rightarrow w(f) \le c(C) $ $ \blacksquare $

\underline{Следствие}

\hypertarget{link2}{В сети $ G: w(f_{max}) \le c(C_{min}) $}

$ w(f_{max}) = max \text{ } w(f) $ (f --- поток)

$ c(C_{min}) = min \text{ } c(C) $ (C --- разрез)\\

\subsection{Теорема Форда — Фалкерсона}
В сети $ (G(V, E), c), c(e) \in 
\mathbb N: w(f_{max}) = c(C_{min})$ 

(для простоты считаем, что пропускные способности --- целые)

$\begin{matrix}
	\xymatrix{
		& a  \ar@{->}[r]_6 \ar@{->}[rdd]_3 & b \ar@{->}[rd]_8 \\
		s \ar@{->}[rd]_4 \ar@{->}[ru]_5 &  & & t \\
		& c \ar@{->}[r]_5 \ar@{<-}[ruu]_7 & d \ar@{->}[ru]_4 \ar@{->}[uu]_2 \\
	}
\end{matrix}$ --- пропускная способность рёбер

$\begin{matrix}
	\xymatrix{
		& a  \ar@{->}[r]_3 \ar@{->}[rdd]_0 & b \ar@{->}[rd]_3 \\
		s \ar@{->}[rd]_4 \ar@{->}[ru]_3 &  & & t \\
		& c \ar@{->}[r]_5 \ar@{<-}[ruu]_1 & d \ar@{->}[ru]_4 \ar@{->}[uu]_1 \\
	}
\end{matrix}$ --- поток\\

\textbf{Определение} Дополнительный граф для потока

$ \bar G $ имеет $ \bar V = V $

$ \bar E: $

$ e = (u, v) $

Если $ f(e) < c (e) $

То есть $ \bar e = (\bar u, \bar v) $

$ g(\bar e) = cA(e) - f(e) $\\

Если $ 0 < f(e): e = (u, v) $

То есть $ \tilde e = (\bar v, \bar u) $

$ g(\tilde e) = f(e) $

$\begin{matrix}
	\xymatrix{
		& a  \ar@{->}[r]_3 \ar@{->}[rdd]_3 & b \ar@{->}[rd]_5 \\
		s  \ar@{->}[ru]_2 &  & & t \\
		& c \ar@{<-}[ruu]_6 & d \ar@{->}[uu]_1 \\
	}
\end{matrix}$ --- дополнительный граф с дополненем до max\\

$\begin{matrix}
	\xymatrix{
		& a  \ar@{<-}[r]_3 \ar@{->}[rdd]_3 & b \ar@{<-}[rd]_3 \\
		s  \ar@{<-}[ru]_3 \ar@{<-}[rd]_4 &  & & t \ar@{->}[ld]_4 \\
		& c \ar@{->}[ruu]_1 & d \ar@{<-}[uu]_1 \ar@{->}[l]_5 \\
	}
\end{matrix}$ --- дополнительный граф\\

\textbf{Доказательство} Начнём с $ 0 $-го потока и будем его постепенно 

увеличивать

Построим дополнительный граф $ \bar G $ и найдём в нём путь из $ s $ в $ t: $
$\begin{matrix}
	\xymatrix{
		& s  \ar@{->}[r] & . \ar@{->}[r] & \dots \ar@{->}[r] & . \ar@{->}[r] & t \\
	}
\end{matrix}$

Найдём $ min \text{	} g(e) $ на этом пути

$ \sqsupset $ это $ x $

Вычтем в дополнительном графе $ x $ на каждом ребре:

1)
$\begin{matrix}
	\xymatrix{
		.  \ar@{->}[r] & .  \ar@{->}[r] &  \ar@{->}[rrr]_{c(e)-f(e) \rightarrow c(e)-f(e)-x } & &  & \ar@{->}[r] . & . \\
	}
\end{matrix}$ $ \bar f(e) := f(e)+x $\\

2)
$\begin{matrix}
	\xymatrix{
		.  \ar@{->}[r] & .  \ar@{->}[r] &  \ar@{->}[rrr]_{f(e)-x}^{\text{был } f(e)} & &  & \ar@{->}[r] . & . \\
	}
\end{matrix}$ $ \bar f(e) := f(e)-x $\\

Поймём:

1) новый поток $ \bar f $ остался потоком

2) величина потока увеличилась на $ x $\\

Проверим, что это поток:

$ 0 \le \bar f(e) \le c(e) $

Уменьшаем по обратному, увеличиваем по прямому пути

$ c(e) - (f(e)+x) \ge 0 $

В вершинах верно: $  $
$ \sum\limits_{ \text{входящих} } = \sum\limits_{ \text{исходящих} } $

Итого, $ \bar f $ --- поток

\newpage

\underline{Пример:}

$\begin{matrix}
	\xymatrix{
		& a  \ar@{->}[r]_6 \ar@{->}[rdd]_3 & b \ar@{->}[rd]_8 \\
		s \ar@{->}[rd]_4 \ar@{->}[ru]_5 &  & & t \\
		& c \ar@{->}[r]_5 \ar@{<-}[ruu]_7 & d \ar@{->}[ru]_4 \ar@{->}[uu]_2 \\
	}
\end{matrix}$ --- Пропускные способности

Строим пути

Сначала поток $ =0 $

Дополнительный граф:
$\begin{matrix}
	\xymatrix{
		& a  \ar@{->}[r]_6 \ar@{->}[rdd]_3 & b \ar@{->}[rd]_8 \\
		s \ar@{->}[rd]_4 \ar@{->}[ru]_5 &  & & t \\
		& c \ar@{->}[r]_5 \ar@{<-}[ruu]_7 & d \ar@{->}[ru]_4 \ar@{->}[uu]_2 \\
	}
\end{matrix}$\\

Ищем путь из $ s $ в $ t :$
$\begin{matrix}
	\xymatrix{
		& s  \ar@{->}[r]_5 & a \ar@{->}[r]_3 & d \ar@{->}[r]_2 & b \ar@{->}[r]_8 & t \\
	}
\end{matrix}$ --- $ min $ 2

Добавляем к потоку $ +2 $ на каждое из этих рёбер\\



Поток

$\begin{matrix}
	\xymatrix{
		& a  \ar@{->}[r]_0 \ar@{->}[rdd]_2 & b \ar@{->}[rd]_2 \\
		s \ar@{->}[rd]_0 \ar@{->}[ru]_2 &  & & t \\
		& c \ar@{->}[r]_0 \ar@{<-}[ruu]_0 & d \ar@{->}[ru]_0 \ar@{->}[uu]_2 \\
	}
\end{matrix}$

Дополнительный граф

$\begin{matrix}
	\xymatrix{
		& a  \ar@{->}[r]_6 \ar@/^/[ld]^2 \ar@/^/[rdd]^1  & b \ar@{->}[rd]^6 \ar@/^/[ldd]^7 \ar@/^/[dd]^2 \\
		s \ar@{->}[rd]_4 \ar@{->}[ru]^3 &  & & t \ar@/^/[lu]^2\\
		& c \ar@{->}[r]_5  & d \ar@{->}[ru]_4 \ar@/^/[luu]_2 \\
	}
\end{matrix}$

Ищем путь из $ s $ в $ t :$

$\begin{matrix}
	\xymatrix{
		& s  \ar@{->}[r]_3 & a \ar@{->}[r]_6 & b \ar@{->}[r]_7 & c \ar@{->}[r]_5 & d \ar@{->}[r]_4 & t \\
	}
\end{matrix}$ --- $ min $ 3


\newpage


Поток

$\begin{matrix}
	\xymatrix{
		& a  \ar@{->}[r]_3 \ar@{->}[rdd]_2 & b \ar@{->}[rd]_2 \\
		s \ar@{->}[rd]_0 \ar@{->}[ru]_5 &  & & t \\
		& c \ar@{->}[r]_3 \ar@{<-}[ruu]_3 & d \ar@{->}[ru]_3 \ar@{->}[uu]_2 \\
	}
\end{matrix}$

Дополнительный граф

$\begin{matrix}
	\xymatrix{
		& a  \ar@{->}[r]^3 \ar@/^/[ld]^5 \ar@/^/[rdd]^1  & b \ar@/^/[l]^3 \ar@{->}[rd]^6 \ar@/^/[ldd]^4 \ar@/^/[dd]^2 \\
		s \ar@{->}[rd]_4  &  & & t \ar@/^/[lu]^2 \ar@/^/[ld]^3\\
		& c \ar@{->}[r]^2 \ar@/^/[ruu]^3  & d \ar@/^/[l]^3 \ar@{->}[ru]^1 \ar@/^/[luu]_2 \\
	}
\end{matrix}$

Ищем путь из $ s $ в $ t :$

$\begin{matrix}
	\xymatrix{
		& s  \ar@{->}[r]_4 & c \ar@{->}[r]_2 & d \ar@{->}[r]_1 & t \\
	}
\end{matrix}$ --- $ min $ 1

Поток

$\begin{matrix}
	\xymatrix{
		& a  \ar@{->}[r]_3 \ar@{->}[rdd]_2 & b \ar@{->}[rd]_2 \\
		s \ar@{->}[rd]_1 \ar@{->}[ru]_5 &  & & t \\
		& c \ar@{->}[r]_4 \ar@{<-}[ruu]_3 & d \ar@{->}[ru]_4 \ar@{->}[uu]_2 \\
	}
\end{matrix}$

Дополнительный граф

$\begin{matrix}
	\xymatrix{
		& a  \ar@{->}[r]^3 \ar@/^/[ld]^5 \ar@/^/[rdd]^1  & b \ar@/^/[l]^3 \ar@{->}[rd]^6 \ar@/^/[ldd]^4 \ar@/^/[dd]^2 \\
		s \ar@{->}[rd]^3   &  & & t \ar@/^/[lu]^2 \ar@/^/[ld]^4\\
		& c \ar@{->}[r]^1 \ar@/^/[lu]^1 \ar@/^/[ruu]^3  & d \ar@/^/[l]^4  \ar@/^/[luu]_2 \\
	}
\end{matrix}$

Ищем путь из $ s $ в $ t :$

$\begin{matrix}
	\xymatrix{
		& s  \ar@{->}[r]_3 & c \ar@{->}[r]_3 & b \ar@{->}[r]_6 & t \\
	}
\end{matrix}$ --- $ min $ 3

\newpage
Поток

$\begin{matrix}
	\xymatrix{
		& a  \ar@{->}[r]_3 \ar@{->}[rdd]_2 & b \ar@{->}[rd]_5 \\
		s \ar@{->}[rd]_4 \ar@{->}[ru]_5 &  & & t \\
		& c \ar@{->}[r]_4 \ar@{<-}[ruu]_0 & d \ar@{->}[ru]_4 \ar@{->}[uu]_2 \\
	}
\end{matrix}$

Дополнительный граф

$\begin{matrix}
	\xymatrix{
		& a  \ar@{->}[r]^3 \ar@/^/[ld]^5 \ar@/^/[rdd]^1  & b \ar@/^/[l]^3 \ar@{->}[rd]^3 \ar@/^/[ldd]^7 \ar@/^/[dd]^2 \\
		s    &  & & t \ar@/^/[lu]^5 \ar@/^/[ld]^4\\
		& c \ar@{->}[r]^1 \ar@/^/[lu]^4   & d \ar@/^/[l]^4  \ar@/^/[luu]_2 \\
	}
\end{matrix}$

Больше пути из $ s $ в $ t $ нет

\newpage

\textbf{Доказательство теоремы Форда — Фалкерсона} \textit{Продолжение}\\

Почему если пути нет, то поток оптимальный?

$ \sqsupset V_{1} $ --- вершины, достижимые из $ s $ по рёбрам дочернего графа, 

$ V_{2} = V \backslash V_{1} $

$\begin{matrix}
	\xymatrix{
		& a  \ar@{->}[r]^3 \ar@/^/[ld]^5 \ar@/^/[rdd]^1  & b \ar@/^/[l]^3 \ar@{->}[rd]^3 \ar@/^/[ldd]^7 \ar@/^/[dd]^2 \\
		s    &  & & t \ar@/^/[lu]^5 \ar@/^/[ld]^4\\
		& c \ar@{->}[r]^1 \ar@/^/[lu]^4   & d \ar@/^/[l]^4  \ar@/^/[luu]_2 \\
	}
\end{matrix}$

В данном случае $ V_{1} = \{s\}, V_{2} = \{a, b, c, d, t\} $

$ t \in V_{2} $, поскольку нет пути из $ s $ в $ t $

Получаем разрез $ C $

В исходном графе есть рёбра из $ V_{1} $ в $ V_{2} = E_{c}^{+}$

Этих рёбер нет в дополнительном графе

В дополнительном графе веса $ = c(e) - f(e) =0 $

Течёт ли что-нибудь в обратном направлении? Нет, иначе $ V_{1} $ неверен\\

$ c(C) = \sum\limits_{e \in E_{c}^{+}} c(e) = \sum\limits_{e \in E_{c}^{+}} f(e) = \sum\limits_{e \in E^{+}} f(e) - \underbrace{\sum\limits_{e \in E^{-}} f(e)}_{0} = c(f)$

Итого, мы нашли разрез $ C: c(C)=c(f)$ 

\hyperlink{link2}{Ранее} было показано, что $ c(c) \ge c(f) $ 
($ \forall C $ --- разре, $ \forall f$ --- поток)

Получается: $ C $ --- минимальный разре, $  f$ --- максимальный поток

\underline{Пример:}

$\begin{matrix}
	\xymatrix{
		& a \ar@{->}[rd]_2 \\
		s \ar@{->}[rd]_3 \ar@{->}[ru]_{10} & & b \ar@{->}[r]_{20} & t \\
			& c \ar@{->}[ru]_{40} \\
	}
\end{matrix}$ --- сеть

$\begin{matrix}
	\xymatrix{
		& a \ar@{->}[rd]_2 \\
		s \ar@{->}[rd]_3 \ar@{->}[ru]_{2} & & b \ar@{->}[r]_{5} & t \\
		& c \ar@{->}[ru]_{3} \\
	}
\end{matrix}$ --- поток

\newpage

При $ V_{1} = \{s, a\}: c(C) = 2+3=5 $ --- минимальный разрез

При $ V_{1} = \{a, b, t\}: c(C) = 10+40=50 $

Поток = 5 --- максимальный поток

\textit{Замечание} Метод Форда — Фалкерсона строит минимальный разрез и максимальный поток\\

\textbf{Утверждение} Если каждый раз искать путь с минимальным

 количеством рёбер, то время поиска максимального потока $ \sim V^{2}E$

\textbf{Без доказательства}\\

\textbf{Утверждение} Для плоской сети (без пересечения рёбер)

эффективно искать верхние пути\\

\subsection{Задача о паросочетаниях}

Дан двудольный граф $ G(U, V, E) $

$\xymatrix{
	. \ar@{-}[rr] \ar@{-}[rrd] \ar@{-}[rrdd] & & . \ar@{-}[ll] \ar@{-}[lld] \ar@{-}[lldd]\\
	. \ar@{-}[rr] \ar@{-}[rrd] \ar@{-}[rru] & & . \ar@{-}[ll] \ar@{-}[lld] \ar@{-}[llu]\\
	U \ar@{-}[rr] \ar@{-}[rru] \ar@{-}[rruu] & & V \ar@{-}[ll] \ar@{-}[llu] \ar@{-}[lluu]\\
}$

\textbf{Определение} Паросочетание в $ G $ --- это $ P \subset E $, где рёбра из $ P $ не имеют общих вершин

\underline{Пример:}

$\xymatrix{
	. \ar@{-}[rr] & & . \\
	.    & & . \\
	U \ar@{-}[rr] & & V \\
}$

\textbf{Определение} Максимальное паросочетание ---  $ P \subset E :$ 

 $ |P|$ --- максимальное из возможного 
 
\underline{ Пример:}
 
 $\xymatrix{
 	a \ar@{-}[rrd]  & & A \\
 	b  \ar@{-}[rr] \ar@{-}[rrd]  & & B \\
 	c \ar@{-}[rruu] & & C \\
 }$

 $ P = \{cA, bC, aB\} $
 
 \newpage
 
 \underline{ Пример:}
 
 $\xymatrix{
 	. \ar@{-}[rrddd] \ar@{-}[rr] \ar@{-}[rrd] \ar@{-}[rrdd] & & . \\
 	.   \ar@{-}[rr] & & . \\
 	.  \ar@{-}[rru]  & &  . \\
 	.  \ar@{-}[rru] \ar@{-}[rruu] & &  . \\
 }$

4 паросочетания построить невозможно, 3 можно, например:

 $\xymatrix{
	.  \ar@{-}[rr]  & & . \\
	.   \ar@{-}[rr] & & . \\
	.    & &  . \\
	.  \ar@{-}[rru]  & &  . \\
}$\\

Сводим задачу о максимальном паросочетании к задаче о потоке:

$\xymatrix{
& 	. \ar@{->}[rr]   & & . \\
s \ar@{->}[r] \ar@{->}[rd] \ar@{->}[ru] &	. \ar@{->}[rr]    & & . & t \ar@{<-}[l] \ar@{<-}[ld] \ar@{<-}[lu] \\
&	U \ar@{->}[rr]   & & V \\
}$

На рёбрах $ U $ --- $ V $ ставим направление слева направо

$ c(e) = 1 $

\textbf{Утверждение} 
Каждому потоку  (из $ f = 0, 1 $) соотвествует 

паросочетание

\textbf{Доказательство} Рёбра с $ f(e)=1 $ --- рёбра паросочетания

$ \rightarrow $ поток: только одно из рёбер $ = 1 $

$ \leftarrow $ паросочетанию соотвествувет поток, где $ f(e)=1 $, для рёбер 

паросочетания

\textit{Следствие} Размер максимального потока равен размеру

 максимального паросочетания

Строим паросочтение методом Форда — Фалкерсона

$\xymatrix{
	& 	a \ar@{-}[rr] \ar@{-}[rrdd]  & & A \\
	s \ar@{-}[r] \ar@{-}[rd] \ar@{-}[ru] &	b \ar@{-}[rru]    & & B & t \ar@{-}[l] \ar@{-}[ld] \ar@{-}[lu] \\
	&	c \ar@{-}[rru]   & & C \\
}$

Строим дополнительный граф, но без чисел (всегда 1)

$\xymatrix{
	& 	a \ar@{->}[rr] \ar@{->}[rrdd]  & & A \\
	s \ar@{->}[r] \ar@{->}[rd] \ar@{->}[ru] &	b \ar@{->}[rru]    & & B & t \ar@{<-}[l] \ar@{<-}[ld] \ar@{<-}[lu] \\
	&	c \ar@{->}[rru]   & & C \\
}$\\

$\begin{matrix}
	\xymatrix{
		& s  \ar@{->}[r] & a \ar@{->}[r] & A \ar@{->}[r] & t \\
	}
\end{matrix}$

$\xymatrix{
	& 	a \ar@{<-}[rr] \ar@{->}[rrdd]  & & A \\
	s \ar@{->}[r] \ar@{->}[rd] \ar@{<-}[ru] &	b \ar@{->}[rru]    & & B & t \ar@{<-}[l] \ar@{<-}[ld] \ar@{->}[lu] \\
	&	c \ar@{->}[rru]   & & C \\
}$\\

$\begin{matrix}
	\xymatrix{
		& s  \ar@{->}[r] & a \ar@{->}[r] & c \ar@{->}[r] & B \ar@{-}[r] & t \\
	}
\end{matrix}$

$\xymatrix{
	& 	a \ar@{<-}[rr] \ar@{->}[rrdd]  & & A \\
	s \ar@{->}[r] \ar@{<-}[rd] \ar@{<-}[ru] &	b \ar@{->}[rru]    & & B & t \ar@{->}[l] \ar@{<-}[ld] \ar@{->}[lu] \\
	&	c \ar@{<-}[rru]   & & C \\
}$\\

$\begin{matrix}
	\xymatrix{
		& s  \ar@{->}[r] & b \ar@{->}[r] & A \ar@{->}[r] & a \ar@{-}[r] & C \ar@{-}[r] & t \\
	}
\end{matrix}$

$\xymatrix{
	& 	a \ar@{->}[rr] \ar@{<-}[rrdd]  & & A \\
	s \ar@{<-}[r] \ar@{<-}[rd] \ar@{<-}[ru] &	b \ar@{<-}[rru]    & & B & t \ar@{->}[l] \ar@{->}[ld] \ar@{->}[lu] \\
	&	c \ar@{<-}[rru]   & & C \\
}$\\

\underline{Ответ:} $ aC, bA, cB $ (рёбра, ориентированные от $ t $ к $ s $)

\newpage

\underline{Пример:}

$\xymatrix{
	.  \ar@{-}[rr]  & & . \\
	.  \ar@{-}[rru]  & & . \\
	.  \ar@{-}[rr] \ar@{-}[rru]  & &  . \\
	.  \ar@{-}[rr] \ar@{-}[rru]  & &  . \\
}$\\

$\xymatrix{
	& 	.  \ar@{->}[rr]  & & . \\
	&	.  \ar@{<-}[rru]  & & . \\
	s  \ar@{<-}[r] \ar@{<-}[rd] \ar@{->}[ruu] \ar@{<-}[ru] & .  \ar@{->}[rr] \ar@{<-}[rru]  & &  . & t \ar@{->}[l] \ar@{<-}[ld] \ar@{->}[luu] \ar@{->}[lu] \\
\ar@{--}[rrruuu]	& 	.  \ar@{->}[rr] \ar@{<-}[rru]  & &  . \\
}$\\

Пунктирная линия (разрез) разделяет 2 группы вершин: $ V_{1} $ и $ V_{2} $

$\xymatrix{
	& 	V_{1}  \ar@{->}[rr]  & & V_{1} \\
	&	V_{1}  \ar@{<-}[rru]  & & V_{2} \\
	V_{1}  \ar@{<-}[r] \ar@{<-}[rd] \ar@{->}[ruu] \ar@{<-}[ru] & V_{2}  \ar@{->}[rr] \ar@{<-}[rru]  & &  V_{2} & V_{2} \ar@{->}[l] \ar@{<-}[ld] \ar@{->}[luu] \ar@{->}[lu] \\
	\ar@{--}[rrruuu]	& 	V_{2}  \ar@{->}[rr] \ar@{<-}[rru]  & &  V_{2} \\
}$\\

\subsection{Задача о максимальном контролирующем множестве}

\textbf{Определение} $ \sqsupset G=(V, E); C \subset V$ --- конролирующее множество, если $ \forall \underbrace{e}_{=(u, v)} \in E : u \in C $ или $ v \in C $


$\xymatrix{
	  & . \ar@{-}[d] &  \\
	. \ar@{-}[r]   & a \ar@{-}[d] \ar@{-}[r] & b \ar@{-}[d] \\
  & . &  . \\
} $

$ C = \{a, b\} $

$\xymatrix{
	& 	a \ar@{-}[rr]   & & . \\
	.  \ar@{-}[rd] \ar@{-}[ru] &	 & &  & b  \ar@{-}[ld] \ar@{-}[lu] \\
	&	c \ar@{-}[rr]   & & . \\
}$

$ C = \{a, b, c\} $

\textit{Замечание} $ C = V $ --- всегда контролирующее множество\\

\underline{Задача:} найти контролирующее множество минимальнеого размера

\textit{(будем решать для двудольного графа)}

$\xymatrix{
	.  \ar@{-}[rr]  & & a \\
	.  \ar@{-}[rru]  & & . \\
	b  \ar@{-}[rr] \ar@{-}[rru]  & &  . \\
	c  \ar@{-}[rr] \ar@{-}[rru]  & &  . \\
}$\\

Минимальное $ C = \{a, b, c\} $\\

\textbf{Утверждение} В двудольном графе $ G = (U \cup V, E) \sqsupset C$ 

--- контролирующее множество, $ \sqsupset P $ --- паросочетание: тогда $ |C| \ge |P| $

\textbf{Доказательство}

$\xymatrix{
	.  \ar@{-}[rr]  & & . \\
	.  \ar@{-}[rru]  & & . \\
	.   \ar@{-}[rru]  & &  . \\
	.   \ar@{-}[rru]  & &  . \\
}$\\

У каждого ребра $ e=(u, v) \in P $ есть вершина $: u \in C $ или $ v \in C $ $ \blacksquare $\\

\textbf{Утверждение} $ \sqsupset G = (U \cup V, E) $ --- двудольный граф

Размер максималного парасочетания равен размеру минимального

 контролирующего множества

\textbf{Доказательство} Построим максимальное парасочтение по

 алгоритму Форда — Фалкерсона и рассмотрим разрез

$\xymatrix{
	& 	V_{1}  \ar@{->}[rr]  & & V_{1} \\
	&	V_{1}  \ar@{<-}[rru]  & & V_{2} \\
	V_{1}  \ar@{<-}[r] \ar@{<-}[rd] \ar@{->}[ruu] \ar@{<-}[ru] & V_{2}  \ar@{->}[rr] \ar@{<-}[rru]  & &  V_{2} & V_{2} \ar@{->}[l] \ar@{<-}[ld] \ar@{->}[luu] \ar@{->}[lu] \\
	\ar@{--}[rrruuu]	& 	V_{2}  \ar@{->}[rr] \ar@{<-}[rru]  & &  V_{2} \\
}$\\

$ |u|=x; |v|=y; |U \cap V_{1}|=a; |V \cap V_{1}|=b $

$ c(\underbrace{V_{1}, V_{2}}_{\text{разрез}})= \sum\limits_{e = (u, v), u \in V_{1}, v \in V_{2}} 1 = \underbrace{(x-a)}_{\text{из } S}+\underbrace{b}_{\text{в } T}+\underbrace{n}_{\text{рёбра из } V_{2} \text{ в } V_{1}}$

Итого, $ m = x-a+b+n(\le m) \le x-a+b+m \Rightarrow x-a+b \ge 0$

Возьмём в качестве контролирующего множества 

$ C = (U \backslash V_{1}) \cup (V \cap V_{1}) = (U \cap V_{2}) \cup (V \cap V_{1}) $

Есть ли ребро из $ (V \cap V_{2}) $ в $ (U \cap V_{1})? $

$\xymatrix{
	& 	a \ar@{<-}[rr]   & & a_{1} \\
	a  \ar@{->}[rd] \ar@{<-}[ru] \ar@{<-}[r] &	b_{1} \ar@{->}[rrd] \ar@{->}[rru] \ar@{<-}[rr] & & b \ar@{<-}[r] & b  \ar@{<-}[ld] \ar@{->}[lu] \\
	&	a \ar@{->}[rruu]   & & b \\
}$

$ V_{1} = \{a\}, V_{2} = \{b\} $

Контролирующее множество $ = \{a_{1}, b_{1}\}$

Контролирующее множество --- это $ (U \cap V_{2}) $  $ (V \cap V_{1}) $

$ \sqsupset $ есть ребро $ e = (u, v): $

1) $\xymatrix{
	. \ar@{<-}[r] & . \\
} \Rightarrow v \in V_{1}$ \textbf{Противоречие}

1) $\xymatrix{
	u \ar@{->}[r] & v \\
}$

 Как попасть в $ u? $ 
 
 Только из $ v $

$\xymatrix{
. \ar@{->}[r] &	u \ar@{<-}[r] & . \\
}$ --- невозможно

Значит, в этом минимальном разрезе нет рёбер между $ (U \cap V_{2}) $ и $ (V \cap V_{1}) $\\

\underline{Вывод 1:} $ С $ --- контролирующее множество $ (U \cap V_{2}) $ и $ (V \cap V_{1}) $

\underline{Вывод 2:} 

$ |P|=C(\underbrace{V_{1}, V_{2}}_{\text{разрез}})= \sum\limits_{e = (u, v), u \in V_{1}, v \in V_{2}} 1 = \underbrace{(x-a)}_{\text{из } S}+\underbrace{b}_{\text{в } T}+\underbrace{0}_{\text{рёбра из } V_{2} \text{ в } V_{1}}=|C|$

$ \blacksquare $

\newpage

\subsection{Поиск в глубину, в ширину}

1) Структура данных для хранния вершин $ D $ --- стек или очередь

$ v \rightarrow D $ --- положить $ v $ в $ D $

$ v := D $ --- посмотреть

$ D \rightarrow v $ --- убираем элемент из структуры данных

Стек: первый вошёл, последний вышел

Очередь: первый вошёл, первый вышел

\begin{tabular}{ | l | l | l |}
	\hline
	& Стек & Очередь  \\ \hline
	$ a \rightarrow D $ & $ a $ & $ a $  \\ \hline
	$ b \rightarrow D $ & $ ba $ & $ ba $  \\ \hline
	$ c \rightarrow D $ & $ cba $ &  $ cba $  \\ \hline
	$ := D $ & $ c $ &  $ a $  \\ \hline
	$ D \rightarrow $ & $ ba $ &  $ cb $  \\ \hline
	$:= D $ & $ b $ &  $ b $  \\ \hline
\end{tabular}\\

Поиск в ширину ($ D $--- очередь)

Поиск в глубину ($ D $--- стек)

$ D \in v_{0} $ --- начальная вершина (были ($ v_{0} )=1$)

Пока $ D $ не пусто

$ u := D $

если есть ребро $ (u, v): $ мы ещё $ \overline{\text{были в } v } $, тогда $ v \rightarrow D, \text{были }(v)=1 $

Иначе $ D \rightarrow v $

$\xymatrix{
	 &  & a \ar@{-}[rd] \ar@{-}[ld] &  \\
	  & b \ar@{-}[d] \ar@{-}[ld] &  & c \ar@{-}[d] \ar@{-}[ld] \\
	d \ar@{-}[r] & e & f & g \\
}$
\newpage

Поиск в глубину: \hfill Поиск в ширину:

$ a $ \hfill $ a $

$ ba $ \hfill $ ba $

$ dba $ \hfill $ cba $

$ edba $ \hfill $ cb $

$ dba $ \hfill $ dcb $

$ ba $ \hfill $ edcb $

$ a $ \hfill $ edc $

$ ca $ \hfill $ fedc $

$ fca $ \hfill $ gfedc $

$ ca $ \hfill $ gfed $

$ gca $ \hfill $ gfe $

$ ca $ \hfill $ gf $

$ a $ \hfill $ g $

$ \sqsupset $ \hfill $ \sqsupset $
\\

\begin{center}
	Поиск в глубину ($ D $ --- стек) и в ширину ($ D $ --- очередь)
\end{center}

\underline{Алгоритм поиска}

Дана начальная вершина $ u $

$ Used \leftarrow \varnothing $ (обработанные вершины; те, кто был в $ D $)

Пока $ D \ne \varnothing $

$ v := peak D $

Если есть ребро $ v-w $, где $ w \bcancel \subset Used $ (ребро дерева поиска)

$ D \leftarrow w; Used=Used \cup \{w\} $

Иначе $  \leftarrow D $ (убираем вершину из $ D $)
\\
$ 
\begin{matrix}
	\xymatrix{
		& c \ar@{->}[r] & f \ar@/^/[l]\\
		a \ar@{->}[d] & b \ar@{->}[l] \ar@{->}[u] & & g \ar@{->}[d] \\
		d \ar@{->}[r] & e \ar@{->}[u] & h \ar@{->}[r] \ar@{->}[l] \ar@{->}[ru] & i
	}
\end{matrix} $

\newpage

Поиск в глубину из вершины $ h $

Стек

$ h $

$ he $

$ heb $

$ heba $

$ hebad $

$ hebad $

$ heba \bcancel d $

$ heb \bcancel a \bcancel d $

$ hebc $

$ hebcf $

$ hebc \bcancel f $

$ heb \bcancel c \bcancel f $

$ he \bcancel b \bcancel c \bcancel f $

$ h \bcancel e \bcancel b \bcancel c \bcancel f $

$ h $

$ hg $

$ hgi $

$ hg \bcancel i $

$ h \bcancel g \bcancel i $

$ \bcancel h \bcancel g \bcancel i $

$ \\ $

Дерево поиска

$ 
\begin{matrix}
	\xymatrix{
		& c \ar@{->}[r] & f\\
		a \ar@{->}[d] & b \ar@{->}[l] \ar@{->}[u] & & g \ar@{->}[d] \\
		d & e \ar@{->}[u] & h \ar@{->}[l] \ar@{->}[ru] & i
	}
\end{matrix} $

\newpage

Поиск в ширину из вершины $ h $

Очередь

h

he

heg

$ \bcancel h eg $

eg

egb

$ \bcancel e gb $

gb

gbi

$ \bcancel eg bi $

bi

bia

biac

$ \bcancel b iac $

$ \bcancel b \bcancel i ac $

ac

acd

$ \bcancel a cd $

cd

cdf

$ \bcancel c df $

$ \bcancel c \bcancel d  f $

$ \bcancel c \bcancel d \bcancel f $

$ \\ $

Дерево поиска

$ 
\begin{matrix}
	\xymatrix{
		& c \ar@{->}[r] & f \\
		a \ar@{->}[d] & b \ar@{->}[l] \ar@{->}[u] & & g \ar@{->}[d] \\
		d  & e \ar@{->}[u] & h \ar@{->}[l] \ar@{->}[ru] & i
	}
\end{matrix} $\\

Вверёдем нумерацию:

$ f(u) $ --- номер, какой по счёту вершина попала в $ D $

$ b(u) $ --- обратный номер (номер, какой по счёту вершина ушла из $ D $)

\newpage

Поиск в глубину

Прямая нумерация

$ 
\begin{matrix}
	\xymatrix{
		& c(6) \ar@{->}[r] & f(7)\\
		a(4) \ar@{->}[d] & b(3) \ar@{->}[l] \ar@{->}[u] & & g(8) \ar@{->}[d] \\
		d(5) & e(2) \ar@{->}[u] & h(1) \ar@{->}[l] \ar@{->}[ru] & i(9)
	}
\end{matrix} $

Обратная нумерация

$ 
\begin{matrix}
	\xymatrix{
		& c(4) \ar@{->}[r] & f(3)\\
		a(2) \ar@{->}[d] & b(5) \ar@{->}[l] \ar@{->}[u] & & g(8) \ar@{->}[d] \\
		d(ф) & e(6) \ar@{->}[u] & h(9) \ar@{->}[l] \ar@{->}[ru] & i(7)
	}
\end{matrix} $

$ \\ $

\textit{Замечание} В случае поиска в ширину $ f(u) = b(u) $

$ 
\begin{matrix}
	\xymatrix{
		& c(7) \ar@{->}[r] & f(9)\\
		a(6) \ar@{->}[d] & b(4) \ar@{->}[l] \ar@{->}[u] & & g(3) \ar@{->}[d] \\
		d(8) & e(2) \ar@{->}[u] & h(1) \ar@{->}[l] \ar@{->}[ru] & i(5)
	}
\end{matrix}$

$ \\ $

\textbf{Утверждение} Поиск в ширину перебирает вершины в том же 

порядке, что и Алгоритм Дейкстры (веса рёбер $ =1 $)

Действительно, добавление вершинны в $ D $ --- это релаксация рёбер $ v - w $

Удаление из $ D $ --- удаление вершины с минимальным расстоянием

$ \begin{matrix}
	\xymatrix{
		& h(0) \ar@{->}[rd] \ar@{->}[ld]\\
		g(1) \ar@{->}[d] & & e(1) \ar@{->}[d]\\
		i(2) & & b(2)\\
	}
\end{matrix}
 $
\newpage

\textbf{Полный поиск в глубину (Depth-first search, DFS)}

\fbox{Пока} есть непосещённая вершина $ u $

поиск в глубину ($ u $)

$ \\ $

\underline{Пример (прямая нумерация):}

$ 
\begin{matrix}
	\xymatrix{
		& c \ar@{->}[r] & f \ar@/^/[l]\\
		a \ar@{->}[d] & b \ar@{->}[l] \ar@{->}[u] & & g \ar@{->}[d] \\
		d \ar@{->}[r] & e \ar@{->}[u] & h \ar@{->}[r] \ar@{->}[l] \ar@{->}[ru] & i
	}
\end{matrix} $\\

$ \textbf{dfs(c)} $

$ 
\begin{matrix}
	\xymatrix{
		& c(1) \ar@{->}[r] & f(2)
	}
\end{matrix} $\\

$ \textbf{dfs(e)} $

$ 
\begin{matrix}
	\xymatrix{
		& c(1) \ar@{->}[r] & f(2) \\
		a(5) \ar@{->}[d]  & b(4) \ar@{->}[l] \\
		d(6)  & e(3) \ar@{->}[u]
	}
\end{matrix} $\\

$ \textbf{dfs(i)} $

$ 
\begin{matrix}
	\xymatrix{
		& c(1) \ar@{->}[r] & f(2) \\
		a(5) \ar@{->}[d]  & b(4) \ar@{->}[l]  & & g(9)  \\
		d(6)  & e(3) \ar@{->}[u] & h(8) \ar@{->}[ru] & i(7) \ar@{->}[l]
	}
\end{matrix} $\\

\newpage

\hypertarget{d7}{\textbf{Утверждение}} Пусть $ G $ --- ориентированный граф без циклов

Пусть есть путь $ u - v $ (Нет пути $ v - u $, так как нет циклов)

Тогда после полного обхода в глубину($ DFS $)

$ b(u) > b(v) $\\

\textbf{Доказательство}

Делаем $ DFS $: куда попали раньше?

\textbf{1)} сначала попали в $ u $

$ \begin{matrix}
	\xymatrix{
		u \ar@{->}[r] & \dots \ar@{->}[r] & v
	}
\end{matrix} $

В стеке будет $ u \dots v $

$ \Rightarrow $ сначала из стека уйдёт $ v $, потом $ u \Rightarrow b(u) > b(v)$ $ \blacksquare $ 

\textbf{2)} сначала $ v $

$ \begin{matrix}
	\xymatrix{
		u \ar@{->}[r] & \dots \ar@{->}[r] & v \ar@{->}[r] & \dots
	}
\end{matrix} $

Мы не можем попасть в $ u $, потому что нет циклов

$ \Rightarrow $ мы закончим просмотр, так и не попав в $ u $

$ \Rightarrow $ номер $ b(v) $ присвоится раньше, чем номер $ b(u) $ $ \blacksquare $\\

\textit{Следствие} \textbf{Алгоритм топологической сортировки}

Делаем полный DFS и линейный порядок задаём как $ b(u) $

\newpage

\underline{Пример:}

$ \begin{matrix}
	\xymatrix{
		& & b \ar@{->}[rd] \ar@{->}[ld]  \\
		& a \ar@{->}[d] \ar@{->}[ld] & & c \ar@{->}[rd] \ar@{->}[ld]  \\
		d & e & f \ar@{->}[rd]  & & g  \ar@{->}[ld] \\
		& & & h}
\end{matrix} $

$ \textbf{dfs(a)} $

$ \begin{matrix}
	\xymatrix{
		& a(3) \ar@{->}[d] \ar@{->}[ld] \\
		d(1) & e(2)
}\end{matrix} $\\

$ \textbf{dfs(g)} $

$ \begin{matrix}
	\xymatrix{
		& a(3) \ar@{->}[d] \ar@{->}[ld] \\
		d(1) & e(2) & & & g(5) \ar@{->}[dl] \\
		& & & h(4)}
\end{matrix} $ \\

$ \textbf{dfs(b)} $

$ \begin{matrix}
	\xymatrix{
		& & b(8) \ar@{->}[rd] \\
		& a(3) \ar@{->}[d] \ar@{->}[ld] & & c(7) \ar@{->}[ld] \\
		d(1) & e(2) & f(6) & & g(5) \ar@{->}[dl] \\
		& & & h(4)}
\end{matrix} $ \\

\underline{Ответ:} $ deahgfcb $

\newpage

\subsection{Компоненты сильной связности}

\textit{Напоминание}
$ G(V, E) $ --- ориентированный граф. 

Введём отношение $ \leftrightarrow $ на $ V $:

$ u \leftrightarrow v :=  $ есть путь $ u-v $ и  $ v-u $

$ 
\begin{matrix}
	\xymatrix{
		& c \ar@{->}[r] & f \ar@/^/[l]\\
		a \ar@{->}[d] & b \ar@{->}[l] \ar@{->}[u] & & g \ar@{->}[d] \\
		d \ar@{->}[r] & e \ar@{->}[u] & h \ar@{->}[r] \ar@{->}[l] \ar@{->}[ru] & i
	}
\end{matrix} $

Напомним, что $ \leftrightarrow $ --- отношение эквивалентности

Классы эквивалентности называются \underline{компонентами сильной связности}

В данном графе 3 компоненты сильной связности:

$ 
\begin{matrix}
	\xymatrix{
		& 2 \ar@{->}[r] & 2 \ar@/^/[l]\\
		1 \ar@{->}[d] & 1 \ar@{->}[l] \ar@{->}[u] & & 3 \ar@{->}[d] \\
		1 \ar@{->}[r] & 1 \ar@{->}[u] & 3 \ar@{->}[r] \ar@{->}[l] \ar@{->}[ru] & 3
	}
\end{matrix} $\\

\textbf{Определение} $ \sqsupset G(V, E)$ --- ориентированный граф, $  G^{0}(V^{0}, E^{0})$ --- граф конденсации, если

$ V^{0}=V /_{\leftrightarrow} $ (классы эквивалентности)

В примере: 
$ 
\begin{matrix}
	\xymatrix{
		\fbox{1}  & \fbox{2} \\
		\fbox{3} \\
	}
\end{matrix} $\\

$ E^{0}: u^{0} $ в $ v^{0} $ есть ребро, если 
$ \exists$  $ e=(u, v) $, где $ u \in U^{0}, v \in V^{0} $

$ 
\begin{matrix}
	\xymatrix{
		\fbox{1} \ar@{->}[r]  & \fbox{2} \\
		\fbox{3} \ar@{->}[u] \\
	}
\end{matrix} $\\

\textit{Замечание} Граф конденсации $ G^{0} $ не имеет циклов

$ 
\begin{matrix}
	\xymatrix{
		. \ar@{->}[r]  & . \ar@{->}[d] \\
		. \ar@{->}[u] & . \ar@{->}[l] \\
	}
\end{matrix} $ $ \Rightarrow \forall $ вершины $ \leftrightarrow $

\newpage

\textbf{Утверждение} $ \sqsupset G(V, E)$ --- ориентированный граф, $  G^{0}(V^{0}, E^{0})$ --- граф конденсации $ G $

Делаем полный $ DFS $ в $ G $

Тогда если в $ G^{0} $ есть путь из $ u^{0} $  в $ v^{0} $,
то $ \underbrace{max}_{u \in U^{0}} b(u) > \underbrace{max}_{v \in V^{0}} b(v) $

\textbf{Доказательство}

Аналогично \hyperlink{d7}{предыдущему утверждению} $ \blacksquare $\\

\textit{Следствие} \textbf{Поиск компонент сильной связности
}

\textbf{1.} Полный $ DFS $ в $ G $

\textbf{2.} Находим $ u $  $b(u) \rightarrow max $

Делаем $ DFS $ по обратным рёбрам $ G $\\

\begin{center}
	The End
\end{center}

\end{document}