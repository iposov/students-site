\documentclass{article}
\usepackage[utf8]{inputenc}

%Russian-specific packages
%--------------------------------------
\usepackage[T2A]{fontenc}
\usepackage[utf8]{inputenc}
\usepackage[russian]{babel}
\usepackage{amsfonts}
\usepackage[normalem]{ulem}
\usepackage{graphicx}
\usepackage{tikz}
\usepackage{amsmath}
\usetikzlibrary{graphs}
\usetikzlibrary {positioning}
%--------------------------------------

%Hyphenation rules
%--------------------------------------
\usepackage{hyphenat}
\hyphenation{ма-те-ма-ти-ка вос-ста-нав-ли-вать}
%--------------------------------------
\begin{document}
Лекция№1\\
\underline{Бинарные отношения}\\
\underline{Определение}\\
М-множество $\neq$ 0\\
R$\subset$ MxM- бинарное отношение\\
\underline{Пояснение}\\
MxM- множество пар из элементов R\\
Допустим M={a,b,c}\\
MxM={(a,a),(a,b),(a,c),(b,a),(b,b),(b,c),(c,a),(c,b),(c,c)}\\
или M$=\mathbb N$\\
Отношение R-это подмножество пар\\
\underline{Обозначение}\\
(x,y)$\in R$ пара (x,y) принадлежит отношению\\
мы будем писать xRy\\
вместо (x,y)$\not \in$R мы пишем \sout {xRy}\\
\underline{Пример}\\
1.M=R >=R={(x,y): x>y}\\
(3,2) $\in R$ 3R2  3>2\\
(3,4) $\not \in R$ \sout{3R4} \sout{3>4}\\
2.M=$\mathbb R$ отношение $\geq$ \qquad $7\geq 6$,$7\geq 7$\\
3.M=$\mathbb R$ отношение =\qquad
$
\left.
\begin{array}{rcll}
7=7\qquad\\
(7,7)\in =\\
\end{array}
\right.
\left.
\begin{array}{rcll}
7\neq 8\qquad\\
(7,8)\not\in =\\
\end{array}
\right.
$ \\
4. M=$\mathbb R$: $\approx $\qquad x$\approx$y<=>|x-y|<1\\
5. M=$\mathbb R$: \# \qquad x\#y <=> $x^2>y$\\
2\#2 <=> $2^2>2$\\
\sout {7\#8}\\
6. M=$\mathbb N$или $\mathbb Z$ $ \vdots$\qquad x$\vdots$y <=> $\exists k\in \mathbb Z$ :x=ky\\
$4\vdots 2\quad 10\vdots 5\quad $ \sout{$7\vdots 0$}\\
\sout{$2\vdots4\quad$} \sout{$10\vdots3$}\\
7. $\equiv_3$\qquad 0$\equiv_3$3 1$\equiv_3$4 1$\equiv$ 7\\
\sout{0$\equiv_3$ 2} \sout{1$\equiv_3$ 8}\\
8.M=$\mathbb N$ \qquad aЦв, если в числе а 'в' цифр\\
100Ц3 238Ц3 \sout{238Ц8}\\
9.М=прямые на R$^2$\\
||-l1 и l2 если l1 не пересекает l2 или l1=l2\\
10.$\perp$ l1$\perp$l2 перпендикулярно \\
\begin{tikzpicture}
\node (a)[scale=0.9,style={circle,fill=blue!20}][scale=0.9] at (2,3) {a};
\node (b)[scale=0.9,style={circle,fill=blue!20}][scale=0.9] at (3,3) {b};
\node (c)[scale=0.9,style={circle,fill=blue!20}][scale=0.9] at (4,2) {c};
\node (d)[scale=0.9,style={circle,fill=blue!20}][scale=0.9] at (4,1) {d};
\path (a) edge [-] (0,0);
\path (b) edge [-] (3,0);
\path (c) edge [-] (0,2);
\path (d) edge [-]  (0,1);
\end{tikzpicture}\\
$
\left.
\begin{array}{rcll}
b\perp c\\
 d||c\\
 a||a\\
\end{array}
\right.
$
\sout{b$\perp$a} \\
11.Студенты ЛЭТИ\\
x$\succ$y средний балл за последнюю сессию больше у x\\
12. М=пользователи одноклассники\\
x$\rightarrow$y, если 'y' в друзьях у 'x'\\
Иванов $\rightarrow$ Петров\\
\sout{Петров $\rightarrow$ Посов}\\
\underline{Свойства бинарных отношений}\\
1. Определение\\
Бинарное отошение R на M называют рефлексивным, если $\forall x\in M \quad xRx$\\
$(x,x)\in R$\\
\underline{Замечание}\\
Отношение не рефлеквсивно <=> $\exists x \qquad xRx-$контрпример\\
\underline{Примеры}\\
=-рефлексивно $\forall$ x: x=x\\
$\geq$\qquad $\forall x$: x$\geq$x\\
$\approx$\qquad$\forall x:\quad x\approx x$, т.к. |x-x|=0<1\\
$\vdots$\qquad $x\vdots x$\\
>-не рефлексивно \sout{2>2}\\
Ц- не рефлексивно 3Ц3\\
$\rightarrow$ \sout{Посов$\rightarrow$Посов}\\
$\perp$ не рефлексивно  \sout{a$\perp$ a}\\
\underline{Определение}\\
Б отношение R на множестве M называется антирефлексивным, если $\forall x$ \sout{xRx}\\
\underline{Замечание}\\
R-не антирефлексивно <=>\\
$\exists x:$ xRx-контрпример\\
\underline{Примеры}\\
>-антирефлексивно \sout{x>x}\\
$\perp$\qquad \sout{l$\perp$l}\\
$\rightarrow$\qquad нельзя быть в друзьях у себя\\
Ц- не антирефлексивно\qquad 1Ц1\\
\underline{Замечание}Ц - не антирефлексивно и не рефлексивно\\
2)не бывает R, которое и рефлексивно и антирефлексивно\\
(рассмотрим a $\subset M\quad \rightarrow aRa$=> aRa=> не ар\\
\sout{aRa} => не р\\
\underline{Симметричность}\\
\underline{Определение}\\
Бинарное отношение R на множестве M симметрично, если $\forall x,y$ xRy<=> yRx\\
Замечание\\
R- не симметрично, <=> $\exists x,y:$ xRy, \sout{yRx}- контрпример\\
\underline{Пример}\\
=- симметрично x=y <=> y=x\\
$\approx$ - симметрично x$\approx$y |x-y|<1 |y-x|<1\\
$\vdots$- не симметрично $4\vdots2$ \sout{2$\vdots$4}\\
||, $\perp$- симметрично\\
Ц- не симметричен\\
\underline{Определение}\\
Бинарное отношение R на множестве M антисимметрично, если $\exists x\neq y$ xRy, yRx- контрпример\\
\underline{Пример}\\
>:\qquad x$\neq$y, x>y => \sout{y>x}\\
Попробуем построить контрпример\\
x$\neq$y, x>y, y>x- невозможно\\
=> нет контрпримера=>антисимметрично\\
$\geq$\qquad x$\neq$y,$x\geq y$, y$\geq$x- нет контрпримера\\
=\qquad x$\neq$y,x=y, y=x- нет контрпримера\\
$\equiv_3$\qquad 1$\neq$ 4,$1\equiv_3 4$, 4$\equiv_3$1- контрпример\\
$\vdots$ над $\mathbb N$ x$\neq$y, $x\vdots y$, y$\vdots$x нет для $\mathbb N$\\
$\vdots$ над $\mathbb Z$ 4$\neq$-4, $4\vdots -4$, -4$\vdots$4 не антисимметрино\\
Лекция№2\\
\underline{Антисимметричность}\\
$\vdots$ на $\mathbb Z$- не антисимметрично\\
$
\left.
\begin{array}{rcll}
-2\vdots 2\\
2\vdots-2\\
2\neq -2\\
\end{array}
\right.\textrm{-контрпример}\\
$
$\vdots$ на $\mathbb N$- антисимметрично $x\neq y\quad x\vdots y\quad y\vdots x$=> такого не бывает\\
$x\neq y \quad x\vdots y\quad =>\quad y\not \vdots x$\\
\underline{Определение}\\
R- бинарное отношение на М- ассиметричность, если\\
$\forall x,y$ xRy=> \sout{yRx}\\
($x\neq y$- у антисимметричность)\\
контрпример- xRy, yRx\\
\underline{Утверждение}\\
R- симметрично <=> R-антисимметрично и антирефлексивно\\
\underline{Пример}\\
>- асиметрично  $\forall x,y$ x>y=> \sout{y>x}\\
$\square$(пустое)- асиметрично (пусто, когда R=0)\\
"выше"- асиметрично\\
"начальник" x нач y =>\sout{y нач x}\\
R-бинарное отношение транзитивно, если\\
$\forall x,y,z$ x xRy, yRz => xRz\\
Контрпример\\
> трназитивно x>y, y>z=> x>z\\
$\geq$ транзитивно\\
$\vdots$ транзитивно $x\vdots y$, y$\vdots z$=> x$\vdots$ z\\
$\perp$ не транзитивно $x\perp y$ $y \perp z$ \sout{$x \perp z$}\\
\begin{tikzpicture}
\node (x)[scale=0.7,style={circle,fill=blue!10}] at ( 2,4) {x};
\node (z)[scale=0.7,style={circle,fill=blue!10}]  at ( 4,4) {z};
\node (y)[scale=0.7,style={circle,fill=blue!10}] at  ( 6,1) {y};
\path (x)  edge[-] (2,0);
\path (z)  edge[-] (4,0);
\path (y)  edge[-] (0,1);
\end{tikzpicture}\\
Ц( кол-во цифр) 100Ц3 3Ц1\\
не транзитивно 100Ц1\\
\underline{Определение}\\
Отношение R называется отношением эквивалентности, если\\ R-рефлексивно, симметрично, транзитивно\\
\underline{Пример}\\
= на $\mathbb R$(или $\forall$ другом множестве)\\
$\forall$x x=x- рефлексивно\\
$\forall$x,y x=y => y=x- симметрично\\
$\forall$x,y,z x=y, y=z => x=z- транзитивно\\
= -это ОЭ\\
||- параллельность\\
$\equiv_3$-сравнение\\
$\geq$- не ОЭ\\
т.к. не симметрично\\
$x\geq y$=> y$\geq$x\\
z$\geq$1, 1$\geq$z\\
$\approx$- не ОЭ(по транзитивности)\\
отношение $\uparrow$ на $\mathbb N$ x$\uparrow$y, если у x и y поровну цифр\\
$2\uparrow 5$ $35\uparrow 100$ \\
$12\uparrow 42$\\
ОЭ $x\uparrow x$- рефлексивно\\
$x\uparrow y$ => $y\uparrow x$- симметричность\\
$x\uparrow y$, $y\uparrow z$=> $x\uparrow z$- транзитивность\\
=,||,$\equiv_3$,$\uparrow$- ОЭ\\
\underline{Определение}\\
R-ОЭ на множестве M\\
x$\in$ M, класс элемента х\\
M$_x$={y|xRy}\\
\underline{Пример}\\
= $M_5$={5}\\
$\equiv_3$ M$_2$={2,5,8,11...}\\
// $M_e$={////////...}\\
\underline{Утверждение}\\
R-ОЭ на M\\
$\forall x,y\in M$ $M_x=M_y$ или $M_X \cap M_y$=0\\
\underline{Докозательство}\\
$\supset$ $M_X \cap M_y \quad \neq 0$=> $\exists$z$\in M_x$, z$\in M_y$ =>xRz,\\ yRz=>(симм.)=>zRy=>(транз.)=>xRy\\
Теперь проверим, что класс $M_x=M_y$\\
Возьмем u$\in M_x$, проверим, что u$\in M_y$\\
u$\in M_x$ => xRu\\
xRy=>yRx=>yRu=>$u\in M_y$\\
Следствие R-ОЭ не М\\
тогда M разбито на несколько классов эквивалентности\\
M=$M_1u...uM_n$\\
$M_i \cap M_j$=0\\
= на $\mathbb N$={1} и {2} и {3}...\\
$\equiv_3$ на $\mathbb N$={0,3,6,9,...}\\
{1,4,7,10,...}\\
{2,5,8,11,...}\\
Замечание\\
Если есть M=0 разбитое на M$_i$=0\\
M=$M_1$u...u$M_n$ и $M_i \cap M_j$\\
Тогда можно ввести отношение R\\
xRy, если $\exists M_i:$ x,y$\in M_i$\\
a b c d e f g\\
\begin{tikzpicture}
\node (a)[scale=0.7,style={circle,fill=blue!10}] at ( 0.5,0.5) {a};
\node (b)[scale=0.7,style={circle,fill=blue!10}]  at ( 1,1) {b};
\node (c)[scale=0.7,style={circle,fill=blue!10}] at  ( 0.5,1.5) {c};
\node (d)[scale=0.7,style={circle,fill=blue!10}] at ( 3.5,0.5) {d};
\node (f)[scale=0.7,style={circle,fill=blue!10}]  at ( 4,0.5) {f};
\node (e)[scale=0.7,style={circle,fill=blue!10}] at  ( 3.5,1.5) {e};
\node (g)[scale=0.7,style={circle,fill=blue!10}] at  ( 6,1) {g};
\draw (1,1)circle (1);
\draw (3.5,1) circle (1);
\draw (6,1) circle (1);
\end{tikzpicture}\\
aRb bRc \sout{aRd} gRy \sout{gRa}\\
Для // классы эквивалентности\\
\begin{tikzpicture}
\node () at (1.5,0) {$\underbrace{}_{M2}$};
\node () at (2.5,2) {$\}M1$};
\path (0,3)  edge[-] (0,0.8);
\path (0.5,3)  edge[-] (0.5,0.8);
\path (1,3)  edge[-] (1,0.8);
\path (1.5,3)  edge[-] (1.5,0.8);
\path (2,3)  edge[-] (2,0.8);
\path (0,3)  edge[-] (0,3);
\path (2,2.5)  edge[-] (0,2.5);
\path (2,2)  edge[-] (0,2);
\path (2,1.5)  edge[-] (0,1.5);
\path (2,1)  edge[-] (0,1);
\end{tikzpicture}\\
Отношения порядка\\
(выше, лучше, сильнее, важнее)\\
\underline{Определение}\\
R-бинарное отношение\\
R- транзитивно, антисимметрично\\
1)рефлексивно-нестрогий порядок\\
2)антирефлексивно- строгий порядок\\
обозначения обычно $\succeq$ нестрогий $\succ$строгий\\ $a\succ b$ b$\succ c$ => a$\succ$ c\\
антисимметрично a$\succ$ b\\
b$\succ$ a\\
\underline{Примеры}\\
> на $\mathbb R$- строий порядок\\
$\geq$ $\mathbb R$- не строгий порядок\\
$\vdots$ на $\mathbb N$- не строгий порядок\\
нач\\
a нач b\\
a нач c\\
b нач f\\
c нач f\\
\begin{tikzpicture}
\node (a)[scale=0.8,style={circle,fill=blue!10}][scale=0.9] at ( 2,4) {a};
\node (c)[scale=0.8,style={circle,fill=blue!10}][scale=0.9] at ( 2,3) {c};
\node (d)[scale=0.8,style={circle,fill=blue!10}][scale=0.9] at ( 3,3) {d};
\node (b)[scale=0.8,style={circle,fill=blue!10}][scale=0.9] at ( 1,3) {b};
\node (e)[scale=0.8,style={circle,fill=blue!10}][scale=0.9] at ( 1,2) {e};
\node (f)[scale=0.8,style={circle,fill=blue!10}][scale=0.9] at ( 1.7,2) {f};
\node (g)[scale=0.8,style={circle,fill=blue!10}][scale=0.9] at ( 2.25,2) {g};
\node (h)[scale=0.8,style={circle,fill=blue!10}][scale=0.9] at ( 3,2) {h};
\node (i)[scale=0.8,style={circle,fill=blue!10}][scale=0.9] at ( 1,1) {i};
\node (j)[scale=0.8,style={circle,fill=blue!10}][scale=0.9] at ( 1.7,1) {j};
\node (k)[scale=0.8,style={circle,fill=blue!10}][scale=0.9] at ( 2.25,1) {k};
\foreach\from/\to in {a/b,a/c,a/d,b/e,e/i,c/f,c/g,d/h,f/j,f/k}
\draw (\from) -- (\to);
\end{tikzpicture}\\
\underline{Определение}\\
$\supset$ R-строгий или нестрогий порядок\\
R-линейный, если $\forall x\neq y$ xRy или yRx\\
R- частичный иначе ($\exists x\neq y$ \sout{xRy} \sout{yRx})\\
Примеры >, $\geq$-линейный порядок\\
$\vdots$- частичный \sout{2$\vdots$3} \sout{3$\vdots$2}\\
нач- частичный\\
\underline{Утверждение}\\
R- порядок( строй или нестрогий) на М- конечное |M|<$\infty$\\
Тогда $\exists x-$минимальный, т.е. $\forall y:$ $x\succ$ y\\
Лекция 3\\
Утверждение R-отношение порядка (строгое или нестрогое)\\
тогда $\exists$ x-минимальный, т.е. $\forall$ y$\neq$ x\\
\underline{Пример}\\
$\geq$ на {1, 2, 3, 4, 5}\\
1- min т.к. $\forall$y 1 $\geq$y\\
\underline{Пример}\\
$\vdots$ на {2, 3, 4, 5, 6}\\
2- min $\forall$y \sout{2$\vdots$ y}\\
3-min \sout{3$\vdots$ y}\\
5-min \sout{5$\vdots$ y}\\
4,6- не min\\
$4\vdots 2$ $6 \vdots 2$\\
\underline{Докозательство}\\
берем x$_1$- $\forall$ элемента множества\\
если он не min => $\exists x_2 \neq x_1$ $x_1 \succ x_2$($\succ$-это R)\\
если $x_2$ не min=> $\exists x_3 \neq x_2$ $x_2 \succ x_3$\\
если $x_3$ не min=> $\exists x_4 \neq x_3$ $x_3 \succ x_4$\\
если не можем найти min элемент\\
=> т.к. множество M конечно\\
в какой-то момент $x_i = x_j$(первый повтор)\\
$x_i \succ x_{i+1}\succ x_{i+2}\quad \succ x_{j-1}\succ x_j =x_i$\\
$\succ$- транзитивно => $x_i \succ x_{j-1} \quad x_{j-1} \succ x_i \neq x_i$ невозможно, см. антисимм\\
\underline{Определение}\\
Отношение R$_1$ на множестве M рассматривает$R_2$ на M, если $R_2 \subset R_1$\\
Замечание. $R_1$ доставляет пары где xRy\\
Замечание. x$R_2$y => x$R_1$y\\
\underline{Теорема}\\
о топологической сортировке\\
Если $\succ$ - отношение порядка на M( множество конечное)\\
то $\exists \succ \succ$- отношение лин порядка на M т.к. $\succ \succ$ расширяет $\succ$\\
\underline{Пример подчинения}\\
\begin{tikzpicture}
\node (a)[scale=0.7,style={circle,fill=blue!10}] at ( 1,2) {ГД};
\node (b)[scale=0.7,style={circle,fill=blue!10}]  at ( 0.5,1) {01};
\node (c)[scale=0.7,style={circle,fill=blue!10}] at  ( 2,1) {02};
\node (d)[scale=0.7,style={circle,fill=blue!10}] at ( 0,0) {С1};
\node (f)[scale=0.7,style={circle,fill=blue!10}]  at ( 1,0) {C2};
\path (a)  edge[->] (b);
\path (a) edge[->,bend right] (d);
\path (a) edge[->] (f);
\path (a)  edge[->] (c);
\path (a)  edge[->] (b);
\path (b)  edge[->] (d);
\path (b)  edge[->] (f);
\end{tikzpicture} не линейный порядок\\
ГД-01-02-С1-С2\\
ГД-01-С1-С2-02\\
ГД-01-С1-02-С2\\
\underline{Докозательство}\\
Нахождение min элемента отношения $\succ$\\
$\supset$ это x$\in $M\\
удалим $x_1$ из M теперь имеем $\succ$| M-{$x_1$}\\
очевидно, новое отношение тоже антирефлексивно, рефлексивно, антисимметрично, транзитивно\\
в нем тоже есть min элемент, это $x_2$\\
удаляем $x_2$ из М и продолджаем\\
итого, имеем последовательность $x_1,x_2,x_3,...$\\
Вводим новый порядок $x_i \prec x_j$ для i<j...\\
$x_1 \prec \prec x_2 \prec \prec... \prec \prec x_n$\\
почему $\prec \prec$ расширяет $\prec$\\
Если x$\prec$ y => x был удален раньше y\\
=> $x\prec \prec y$\\
\underline{Замечание} этот алгоритм(поиск min и удаления) не самый эффективный, лучше - сделать поиск в глубину с обратной нумерацией\\
\underline{Замечание} топологическая сортировка практически важная задача\\
порядок работы\\
\begin{tikzpicture}
\node (1)[scale=0.7,style={rectangle,fill=blue!10}]  at ( 0,1) {p1};
\node (2)[scale=0.7,style={rectangle,fill=blue!10}] at  ( 1,2) {p2};
\node (3)[scale=0.7,style={rectangle,fill=blue!10}] at ( 1,0) {p3};
\node (5)[scale=0.7,style={rectangle,fill=blue!10}]  at ( 2,1) {Отчет};
\path (1)  edge[<-] (2);
\path (1) edge[<-] (3);
\path (3)  edge[<-] (5);
\path (2)  edge[<-] (5);
\end{tikzpicture}\\
Транзитивное замыкание\\
был порядок- расширили до линейного (Топологическая сортировка)\\
было отношение- расшилири до транзитивного(Транзитивность замыкается)\\
Пример\\
подчиненное отношение\\
\underline{Пример подчинения}\\
\begin{tikzpicture}
\node (a)[scale=0.7,style={circle,fill=blue!10}] at ( 1,2) {ГД};
\node (b)[scale=0.7,style={circle,fill=blue!10}]  at ( 0.5,1) {01};
\node (c)[scale=0.7,style={circle,fill=blue!10}] at  ( 2,1) {02};
\node (d)[scale=0.7,style={circle,fill=blue!10}] at ( 0,0) {С1};
\node (f)[scale=0.7,style={circle,fill=blue!10}]  at ( 1,0) {C2};
\path (a)  edge[->] (b);
\path (a) edge[->,dashed,bend right] (d);
\path (a) edge[->,dashed] (f);
\path (a)  edge[<-] (c);
\path (a)  edge[->] (b);
\path (b)  edge[->] (d);
\path (b)  edge[->] (f);
\end{tikzpicture}\\
Н.R 01\\
01 R C1\\
но \sout{Н.R С1}\\
Если добавить в отношение, что Н R C1, Н R C2- станет транзитивным\\
\underline{Теорема}\\
$\supset$ R- бинарное отношение на М\\
$\exists \overline R$ отношение на М\\
1)$\overline R$ расширяет R( R$\subset \overline R$)\\
2)$\overline R$ транзитивно\\
3)$\overline R$ min транз. расширение\\
т.е. если $\overline R$- тр расширяет R, то $\overline {\overline R} \supset \overline R$\\
\underline{Докозательство}(не для алгоритма)\\
рассмотрим се транзитивные расширения\\
{$R_i$} возьмем $\overline R$= $\cap$ $\overline R_i$\\
т.е. ьеем только те пары, которые есть во всех транзитивных расширениях\\
\underline{Пример}\\
M={a,b,c,d}\\
\begin{tikzpicture}
\node ()[scale=0.7] at ( 4,2) {aRb bRc bRd};
\node (a)[scale=0.7,style={circle,fill=blue!10}]  at ( 2,2) {a};
\node (b)[scale=0.7,style={circle,fill=blue!10}] at  ( 2,1) {b};
\node (c)[scale=0.7,style={circle,fill=blue!10}] at ( 3,0) {c};
\node (d)[scale=0.7,style={circle,fill=blue!10}]  at ( 1,0) {d};
\path (a)  edge[-] (b);
\path (b) edge[-](c);
\path (b) edge[-] (d);
\end{tikzpicture}\\
$\overline{R_1}$\\
\begin{tikzpicture}
\node (a)[scale=0.7,style={circle,fill=blue!10}]  at ( 2,2) {a};
\node (b)[scale=0.7,style={circle,fill=blue!10}] at  ( 2,1) {b};
\node (c)[scale=0.7,style={circle,fill=blue!10}] at ( 3,0) {c};
\node (d)[scale=0.7,style={circle,fill=blue!10}]  at ( 1,0) {d};
\path (a)  edge[->] (b);
\path (b) edge[->](c);
\path (c) edge[->] (d);
\path (b) edge[->] (d);
\path (a) edge[->,bend left] (c);
\path (a) edge[->,bend right] (d);
\end{tikzpicture}\\
стало транзитивным\\
\begin{tikzpicture}
\node (a)[scale=0.7,style={circle,fill=blue!10}]  at ( 2,2) {a};
\node (b)[scale=0.7,style={circle,fill=blue!10}] at  ( 2,1) {b};
\node (c)[scale=0.7,style={circle,fill=blue!10}] at ( 3,0) {c};
\node (d)[scale=0.7,style={circle,fill=blue!10}]  at ( 1,0) {d};
\path (a)  edge[->] (b);
\path (b) edge[->](c);
\path (b) edge[->] (d);
\path (a) edge[->,bend left] (c);
\path (a) edge[->,bend right] (d);
\path (a) edge[->,loop] (a);
\path (b) edge[->,loop] (b);
\path (c) edge[->,loop] (c);
\path (d) edge[->,loop] (d);
\end{tikzpicture}\\
$\overline{R}$\\
\begin{tikzpicture}
\node (a)[scale=0.7,style={circle,fill=blue!10}]  at ( 2,2) {a};
\node (b)[scale=0.7,style={circle,fill=blue!10}] at  ( 2,1) {b};
\node (c)[scale=0.7,style={circle,fill=blue!10}] at ( 3,0) {c};
\node (d)[scale=0.7,style={circle,fill=blue!10}]  at ( 1,0) {d};
\path (a)  edge[->] (b);
\path (b) edge[->](c);
\path (b) edge[->] (d);
\path (a) edge[->,bend left] (c);
\path (a) edge[->,bend right] (d);
\end{tikzpicture} только те пары, которые есть везде\\
Проверим, что $\overline R$ подходит под улсовия транзитивности\\
0)$\overline R_i$ существует? $R_i$ полное отошение МхМ\\
1)$\overline R$ - расширяет\\
$\supset xRy => \forall \overline R$, x$\overline R_i$y => x$\overline R$y\\
2)$\overline R$-транзитивно? $\supset$ x$\overline R$y,  $\supset y\overline R z$=>\\
$\forall \overline R_i$ x$\overline R_i$y y$\overline R_i$z => x$\overline R_i$z- транзитивно\\
=> x$\overline R$z\\
3) $\overline{\overline R}$=$\overline R_i$ >R т.е. R=$\overline R_i \cap$...\\
\underline{Графы}!!!\\
\underline{Определение}\\
неориентированный граф\\
G=(V,E) где V- множество(вершины)\\
E$\subset$ {(u,v), где u,v $\in$ V}\\
Замечание как рисовать\\
веришны $\bullet$ или $\bigcirc{}$\\
ребра $\rightarrow$ линии между $\bigcirc$\\
важно, только что реберо соединеяет\\
Примеры\\
обычный граф\\
\begin{tikzpicture}
\node (a)[scale=0.7,style={circle,fill=blue!10}]  at ( 1,1) {a};
\node (b)[scale=0.7,style={circle,fill=blue!10}] at  ( 1,0) {b};
\node (c)[scale=0.7,style={circle,fill=blue!10}] at ( 0,1) {c};
\path (a)  edge[-,bend left] (b);
\path (b) edge[-,bend left](c);
\end{tikzpicture}\\
полный граф\\
\begin{tikzpicture}
\node (a)[scale=0.7,style={circle,fill=blue!10}]  at ( 1,1) {a};
\node (b)[scale=0.7,style={circle,fill=blue!10}] at  ( 1,0) {b};
\node (c)[scale=0.7,style={circle,fill=blue!10}] at ( 0,1) {c};
\node (d)[scale=0.7,style={circle,fill=blue!10}]  at ( 0,0) {d};
\path (a)  edge[-] (b);
\path (b) edge[-](c);
\path (d) edge[-](b);
\path (a) edge[-](c);
\path (c) edge[-] (d);
\path (d) edge[-] (a);
\end{tikzpicture}\\
пустой граф\\
\begin{tikzpicture}
\node (a)[scale=0.7,style={circle,fill=blue!10}]  at ( 1,1) {a};
\node (b)[scale=0.7,style={circle,fill=blue!10}] at  ( 1,0) {b};
\node (c)[scale=0.7,style={circle,fill=blue!10}] at ( 0,1) {c};
\node (d)[scale=0.7,style={circle,fill=blue!10}]  at ( 0,0) {d};
\end{tikzpicture}\\
\underline{Определение}\\
G- полный, если $\forall u,v\in V \quad (u,v)\in E$\\
Замечание. V=vertex E=edge\\
\underline{Определение}\\
|G|- размер (порядок) графа |V|- кол-во вершин\\
|V|=n(обычно)\\
|E|=m(обычно) количество ребер\\
G-это (n,m) граф\\
\begin{tikzpicture}
\node (a)[scale=0.7,style={circle,fill=blue!10}]  at ( 1,1) {a};
\node (b)[scale=0.7,style={circle,fill=blue!10}] at  ( 1,0) {b};
\node (c)[scale=0.7,style={circle,fill=blue!10}] at ( 0,1) {c};
\node (d)[scale=0.7,style={circle,fill=blue!10}]  at ( 0,0) {d};
\path (a) edge[-](b);
\path (a) edge[-](c);
\path (c) edge[-] (d);
\path (d) edge[-] (a);
\path (d) edge[-] (b);
\end{tikzpicture}- (4,5) граф\\
\underline{Определение}\\
Степень вершины V$\in$ V-это |{(v,u)|(u,u) $\in$ E}|  степ 3\\
(кол-во ребер с этой вершиной) степ 2\\
обозначаем deg V\\
\underline{Определение}\\
К-регулярный граф- это граф у которого $\forall V\in V \quad degV=K$\\
\begin{tikzpicture}
\node (a)[scale=0.7,style={circle,fill=blue!10}]  at ( 1,1) {a};
\node (b)[scale=0.7,style={circle,fill=blue!10}] at  ( 1,0) {b};
\node (c)[scale=0.7,style={circle,fill=blue!10}] at ( 0,1) {c};
\node (d)[scale=0.7,style={circle,fill=blue!10}]  at ( 0,0) {d};
\path (a)  edge[-] (b);
\path (b) edge[-](c);
\path (d) edge[-](b);
\path (a) edge[-](c);
\path (c) edge[-] (d);
\path (d) edge[-] (a);
\end{tikzpicture} 3 per граф\\
\begin{tikzpicture}
\node (a)[scale=0.7,style={circle,fill=blue!10}]  at ( 2,2) {};
\node (b)[scale=0.7,style={circle,fill=blue!10}] at  ( 2,0) {};
\node (c)[scale=0.7,style={circle,fill=blue!10}] at ( 0,2) {};
\node (d)[scale=0.7,style={circle,fill=blue!10}]  at ( 0,0) {};
\node (1)[scale=0.7,style={circle,fill=blue!10}] at  ( 1,1.5) {};
\node (2)[scale=0.7,style={circle,fill=blue!10}] at ( 1.5,1) {};
\node (3)[scale=0.7,style={circle,fill=blue!10}]  at ( 1,1) {};
\path (a)  edge[-] (b);
\path (d) edge[-](b);
\path (a) edge[-](c);
\path (c) edge[-] (d);
\path (1) edge[-] (2);
\path (1) edge[-] (3);
\path (3) edge[-] (2);
\end{tikzpicture} 2 per граф\\
Лекция 4\\
Напоминание\\
Графы- множество вершин и ребер\\
Степени(напротив названий)\\
\begin{tikzpicture}
[scale=.8,auto=left,every node/.style={circle,fill=green!10}]
\node (1) at (10,7) {a-1};
\node (2) at (8,7) {b-3};
\node (3) at (6,9) {с-3};
\node (4) at (6,5) {e-3};
\node (5) at (4,7) {d-2};
\foreach\from/\to in {1/2,2/3,2/4,3/4,3/5,4/5}
\draw (\from) -- (\to);
\end{tikzpicture}\\
\underline{Определение}\\
Пусть в графе G(V,E)- последовательность $v_1e_1 v_2e_2...v_{n-1}e_{n-1}v_n$\\
$v_i \in V \quad e_i\in E \quad e_i=(v_i,v_{i+1})$\\
Пример\\
\begin{tikzpicture}
[scale=.8,auto=left,every node/.style={circle,fill=blue!10}]
\node (1) at (10,7) {a};
\node (2) at (8,7) {b};
\node (3) at (6,9) {c};
\node (4) at (6,5) {e};
\node (5) at (4,7) {d};
\foreach\from/\to in {1/2,2/3,2/4,3/4,3/5,4/5}
\draw (\from) -- (\to);
\end{tikzpicture}\\
1) a(a,b)b c(c,d)d\\
2)ab\\
3)aba\\
4)abcdecdeba\\
\underline{Определение}\\
Путь замкнутый- если $V_1=V_n$. "4)"- замкнутый\\
не замкнутый- открытые($v_i\neq v_n$)\\
\underline{Определение}\\
Путь простой, если$e_i \neq e_j$ при $i\neq j$\\
"4)" непростой, т.к. 2 раза de\\
5)bedce- простой, но не замкнутый\\
\begin{tabular}{l|l|l}
Пути& все ребра разные(простые)& все вершины разные \\
\hline
Замкнутый & простой замкнутый путь & цикл\\
\hline
Открытый & простой открытый путь & цепь\\
\hline
\end{tabular}\\
цикл- путь из одной вершины в неё же\\
путь может проходить через себя же неограниченное количество раз, идёт из одной вершины в другую\\
простой путь- путь не проходящий по пройденному пути, но имеющий точки пересечения\\
цепь- путь безпересечений\\
\underline{Теорема}\\
Если $\exists$ путь между вершинами u,v => есть цепт от u до v\\
\underline{Докозательство}\\
$\supset$ путь u $e_1v_1e_2v_2...e_n$v\\
рассмторим все пути из этих ребер и выберем min, это будет цепь\\
иначе: $u...u_i...u_j...v \quad \supset v_i=v_j$\\
укоротим $u...v_i=v_j...v$??\\
\underline{Теорема}\\
Если есть простой замкнутый путь через ребро e => есть цепи через е\\
\underline{Докозательство} аналогично\\
Замечание\\
\begin{tikzpicture}
[scale=.8,auto=left,every node/.style={circle,fill=green!10}]
\node (2) at (1,5) {};
\node (1) at (1,10){};
\node (3) at (3,7) {};
\node (6) at (9,5) {};
\node (5) at (9,10){};
\node (4) at (6,7) {};
\foreach\from/\to in {1/2,1/3,2/3,3/4,4/6,4/5,5/6}
\draw (\from) -- (\to);
\end{tikzpicture}\\
dbacbd- не простой путь(е-повторяется)\\
цикла через е нет\\
\underline{Связность графа}\\
\underline{Определение}\\
Граф связан, если $\forall \quad u,v \in V \quad \exists $ цепь(путь) из u в v\\
пример\\
1)
\begin{tikzpicture}
[scale=.4,auto=left,every node/.style={circle,fill=blue!10}]
\node (1) at (5,5) {};
\node (2) at (5,10) {};
\node (3) at (10,5) {};
\node (4) at (10,10){};
\node (5) at (13,7) {};
\foreach\from/\to in {1/2,1/3,2/4,3/4,3/5,4/5}
\draw (\from) -- (\to);
\end{tikzpicture} 
2)
\begin{tikzpicture}
[scale=.4,auto=left,every node/.style={circle,fill=blue!10}]
\node (1) at (5,5) {};
\node (2) at (5,10) {};
\node (3) at (10,5) {};
\node (4) at (10,10){};
\node (5) at (8,7) {};
\node (6) at (6,7) {};
\foreach\from/\to in {1/2,1/3,2/4,3/4,6/5}
\draw (\from) -- (\to);
\end{tikzpicture}
3)
\begin{tikzpicture}
[scale=.4,auto=left,every node/.style={circle,fill=blue!10}]
\node (1) at (5,5) {};
\node (2) at (5,10) {};
\node (3) at (10,5) {};
\node (4) at (10,10){};
\node (5) at (12,2) {};
\node (6) at (12,9){};
\node (7) at (16,6) {};
\foreach\from/\to in {1/2,1/3,2/4,3/4,5/6,6/7,7/5}
\draw (\from) -- (\to);
\end{tikzpicture}\\
1-связан\\
2,3-не связаны\\
Введем отношение $\equiv$ на вершинах графа:\\
U$\equiv$ V, если $\exists$ путь из U в V\\
Пример\\
\begin{tikzpicture}
[scale=.4,auto=left,every node/.style={circle,fill=blue!10}]
\node (1) at (5,5) {b};
\node (2) at (5,10) {a};
\node (3) at (10,5) {c};
\node (4) at (10,10){d};
\node (5) at (12,5) {f};
\node (6) at (12,9){e};
\node (7) at (16,6) {g};
\foreach\from/\to in {1/2,1/3,2/4,3/4,5/6,6/7,7/5}
\draw (\from) -- (\to);
\end{tikzpicture} $a \equiv c \quad a\equiv d \quad e\equiv g$ \sout{a$\equiv$e}\\
Проверим, что $\equiv$- это отношение эквивалентности\\
1)рефлексивность $u\equiv u$-верно, путь u\\
2)симметричность $u \equiv v\quad =>\quad v\equiv u$ путь $ue_1v_1...v$ путь $v...v_1e_1u$\\
3)транзитивно $u\equiv v, v\equiv \omega$ путь $\underbrace{ \underline{ue_1...v}\overline{v...\omega}}$ v не повторяется,\\
а входит в 2 пути\\
\underline{Определение}\\
Классы эквивалентности $\equiv$-это компоненты связности\\
\begin{tikzpicture}
[scale=.2,auto=left,every node/.style={circle,fill=blue!10}]
\node (1) at (5,5) {};
\node (2) at (5,10) {};
\node (3) at (10,5) {};
\node (4) at (10,10){};
\node (5) at (12,5) {};
\node (6) at (12,9){};
\node (7) at (16,6) {};
\foreach\from/\to in {1/2,1/3,2/4,3/4,5/6,6/7,7/5}
\draw (\from) -- (\to);
\end{tikzpicture}
- 2 компонента связности\\
\underline{Определение}\\
$G_1=(V_1,E_1)$- подграф G, если $V_1 \subset$ V $E_1 \subset$ E\\
Пример\\
\begin{tikzpicture}
[scale=.2,auto=left,every node/.style={circle,fill=blue!10}]
\node (5) at (12,5) {$\bullet$};
\node (6) at (12,9){$\bullet$};
\node (7) at (16,6) {$\bullet$};
\node (4) at (19,6) {};
\foreach\from/\to in {5/6,6/7,7/5,4/7}
\draw (\from) -- (\to);
\end{tikzpicture}\\
Помеченные $\bullet$ являются подграфом\\
Замечание\\
G-свой подгаф\\
0-пустой граф-подграф чего угодно\\
\underline{Определение}\\
G ребро e называется мостом,\\
если комп связности G< кол-воткомп. связности (V,E{e})\\
Пример\\
\begin{tikzpicture}
[scale=.8,auto=left,every node/.style={circle,fill=green!10}]
\node (2) at (1,2) {1};
\node (1) at (1,6){4};
\node (3) at (3,3) {5};
\node (6) at (8,2) {2};
\node (5) at (8,6){3};
\node (4) at (6,3) {6}; 
\foreach\from/\to in {1/2,1/3,2/3,3/4,4/6,4/5,5/6}
\draw (\from) -- (\to);
\end{tikzpicture}\\
56 мост\\
\underline{Определение}\\
Степень связности графа G- это количество ребер,\\
которые надо выкинуть,чтобы G стал несвязным\\
\underline{Определение}\\
Двусвязный граф- надо выкинуть $\geq$2 ребер, чтобы он стал несвязным\\
Замечание: двусвязный <=> нет мостов и связи\\
\underline{Определение}\\
Вершина $v\in V$ называется точкой сочленения,\\
если количество компонента связности < количества компонента связности\\
G'=(V{v}, E'{(u,v)|(v,u)$\in$ E})\\
Пример
\begin{tikzpicture}
[scale=.4,auto=left,every node/.style={circle,fill=blue!10}]
\node (1) at (5,5) {};
\node (2) at (5,10) {};
\node (3) at (10,5) {};
\node (4) at (10,10){d};
\node (5) at (14,5) {f};
\node (6) at (14,10){};
\node (7) at (18,5) {};
\foreach\from/\to in {1/2,1/3,2/4,3/4,5/6,4/5,4/6,7/5}
\draw (\from) -- (\to);
\end{tikzpicture}\\
\underline{Теорема}\\
В графе G=(V,E)\\
если deg(u)- степень вершины u\\
|E|=$\frac{1}{2} \sum \limits_{v\in V}$deg(v)\\
Пример\\
\begin{tikzpicture}
[scale=.6,auto=left,every node/.style={circle,fill=blue!10}]
\node (1) at (10,7) {1};
\node (2) at (8,7) {4};
\node (3) at (6,9) {2};
\node (4) at (6,5) {2};
\node (5) at (4,7) {3};
\foreach\from/\to in {1/2,2/3,2/4,2/5,3/5,4/5}
\draw (\from) -- (\to);
\end{tikzpicture}\\
ребер 6=1/2(3+2+2+4+1)=6, верно\\
\underline{Докозательство}\\
deg(v)= количество ребер, выходящих из вершин $\sum \limits_{v\in V}$= все ребра\\
посчитали дважды=2|E|\\
Следствие:\\
1)сумма степеней вершин всегда четна\\
2)вершин нечетной степени- четно\\
Пример\\
15 инопланетян, по 3 руки у каждого,мгу ли они взяться за руки, \\
чтобы не было свободных рук?\\
Решение:\\
нет, эьо граф из 15 нечетных вершин степени 3(нечет.)\\
\underline{Определение}\\
Висячая вершина- это вершина степени 1\\
\begin{tikzpicture}
[scale=.4,auto=left,every node/.style={circle,fill=blue!10}]
\node (4) at (10,10){};
\node (5) at (14,5) {};
\node (6) at (14,10){};
\node (7) at (18,5) {$\bullet$};
\foreach\from/\to in {5/6,4/5,4/6,7/5}
\draw (\from) -- (\to);
\end{tikzpicture}\\
$\bullet$- висячая\\
\underline{Теорема}\\
Если в графе есть ребра, но нет висячих вершин, то $\exists$ цикл\\
\underline{Докозательство}\\
Берем ребро e=($u1U2)$\\
$u_1-^{e}- u_2-^{e_1}-u_3$\\
$u_2$- не висячая вершина=> из неё есть еще ребро($u_2u_3)$\\
$u_3$- не висячая вершина=> из неё есть еще ребро($u_3u_4)$\\
...\\
Продолжаем, пока очередность $u_n$ не будет равна $u_i$\\
путь $u_1 \quad u_{i+1}... u_n$-цикл(ребра разные, вершины разные)\\
\underline{Определение}\\
Дерево- связный граф без циклов\\
Пример\\
\begin{tikzpicture}
[scale=.4,auto=left,every node/.style={circle,fill=blue!10}]
\node (1) at (10,10){};
\node (2) at (8,8) {};
\node (3) at (12,8){};
\node (4) at (13,6) {};
\node (5) at (14,6) {};
\node (6) at (14,4) {};
\node (7) at (12,4) {};
\node (8) at (8,4) {};
\foreach\from/\to in {1/2,1/3,3/4,3/5,5/6,4/7,2/8}
\draw (\from) -- (\to);
\end{tikzpicture} \qquad
\begin{tikzpicture}
[scale=.4,auto=left,every node/.style={circle,fill=blue!10}]
\node (1) at (10,15){};
\node (2) at (12,15){};
\node (3) at (14,15){};
\node (4) at (16,15){};
\node (5) at (18,15){};
\node (6) at (20,15){};
\foreach\from/\to in {1/2,2/3,3/4,4/5,5/6}
\draw (\from) -- (\to);
\end{tikzpicture}\\
\underline{Теорема}\\
В любом дереве $\geq$ 2 висячих вершины\\
\underline{Докозательство}\\
Берем $\forall$ вершину, если она не висячая, идём по ребру,\\
если опять не висячая, есть ребро и т.д. циклов нет=>конец\\
Чтобы найти вторую, надо начать из первой\\
\underline{Теорема}\\
Если G- дерево, то |V|=1+|E|\\
Пример\\
\begin{tikzpicture}
[scale=.4,auto=left,every node/.style={circle,fill=blue!10}]
\node (1) at (10,10){};
\node (2) at (8,8) {};
\node (3) at (12,8){};
\node (4) at (11,6) {};
\node (5) at (13,6) {};
\node (7) at (14,10) {};
\node (8) at (8,6) {};
\foreach\from/\to in {1/2,1/3,3/4,3/5,2/8,1/7}
\draw (\from) -- (\to);
\end{tikzpicture}\\
7 вершин, 6 ребер\\
\underline{Докозательство} по индукции(кол-во вершин)\\
Б. |V|=1 0 |E|=0, сходится |V|=|E|+1\\
П. $\supset$ |V|=n+1,\begin{tikzpicture}
[scale=.3,auto=left,every node/.style={circle,fill=blue!10}]
\node (1) at (10,10){};
\node (2) at (8,8) {};
\node (3) at (12,8){};
\node (4) at (11,6) {};
\node (5) at (13,6) {};
\node (7) at (14,10) {};
\node (8) at (8,6) {.};
\foreach\from/\to in {1/2,1/3,3/4,3/5,2/8,1/7}
\draw (\from) -- (\to);
\end{tikzpicture} .- висячая\\
найдем висячую и удалим с её ребром\\
G'=(V'{v},E'{e}), тоже дерево, т.к. связан, нет циклов=> |V'|=1+|E'|,\\
но |V|=|V'|+1 |E|=|E'|+1\\
отсюда следует |V|=1+|E|\\
Лекция 5\\
Напоминание\\
Дерево- связной граф\\
\begin{tikzpicture}
[scale=.4,auto=left,every node/.style={circle,fill=blue!10}]
\node (1) at (10,10){};
\node (2) at (8,8) {};
\node (3) at (12,8){};
\node (4) at (12,6) {};
\node (5) at (14,6) {};
\node (6) at (14,8) {};
\node (7) at (9,6) {};
\node (8) at (11,4) {};
\node (9) at (13,4) {};
\foreach\from/\to in {3/1,3/2,3/4,3/5,3/6,3/7,4/8,4/9}
\draw (\from) -- (\to);
\end{tikzpicture} |E|+1=|V|\\
$\supset$ G- полный граф, $\forall$ u $\neq$ $\subset$ V соединены ребром\\
\begin{tikzpicture}
[scale=.4,auto=left,every node/.style={circle,fill=blue!10}]
\node (1) at (3,2){};
\node (2) at (2,4){};
\node (3) at (4,5.5){};
\node (4) at (6,4) {};
\node (5) at (5,2) {};
\foreach\from/\to in {2/4,2/1,2/5,1/4,3/1,3/2,3/4,3/5,5/1,5/4}
\draw (\from) -- (\to);
\end{tikzpicture}\\
Если n вершин( |V|=n), то ребер\\
1) C$_n ^2$ ребер, выбираем пары = $\frac{n(n-1)}{2}$\\
2)степени всех вершин n-1\\
$\sum \limits_{v\in V} deg(v)$=2|E|=>(n-1)n=2|E| Ответ: $\frac{n(n-1)}{2}$\\
\underline{Планарные графы}\\
\underline{Определение}\\
G-планарный граф, если его можно нарисовать а плоскости так, чтобы ребра не пересекались\\
\begin{tikzpicture}
[scale=.4,auto=left,every node/.style={circle,fill=blue!10}]\\
%\tikzsryle{every node}=[draw,shape=circle];
\node (A) at (5,5) {};
\node (B) at (5,10) {};
\node (C) at (10,5) {};
\node (D) at (10,10){};
\path (A) edge [bend right = -15] (C);
\path (B) edge [bend right = -15] (D);
\path (C) edge [bend left =-15]  (D);
\end{tikzpicture} ребро [0,1]->$R^2$\\
\begin{tikzpicture}
[scale=.4,auto=left,every node/.style={circle,fill=blue!10}]
\node (1) at (5,5) {};
\node (2) at (5,10) {};
\node (3) at (10,5) {};
\node (4) at (10,10){};
\foreach\from/\to in {1/2,1/3,2/4,3/4}
\draw (\from) -- (\to);
\end{tikzpicture} \qquad
\begin{tikzpicture}
[scale=.4,auto=left,every node/.style={circle,fill=blue!10}]
\node (1) at (5,5) {};
\node (2) at (5,10) {};
\node (3) at (10,5) {};
\node (4) at (10,10){};
\foreach\from/\to in {1/4,1/3,2/4,2/3}
\draw (\from) -- (\to);
\end{tikzpicture} - планарный, но неправильно нарисованный\\
\underline{Формула Эйлера} Если планарный граф G=(V,E) нарисован на плоскости у него можно посчитать грани $\supset$ их f, |V|=n |E|=m\\
1)\begin{tikzpicture}
[scale=.4,auto=left,every node/.style={circle,fill=blue!10}]
\node (1) at (2.5,2.5) {};
\node (2) at (2.5,5) {};
\node (3) at (5,2.5) {};
\node (4) at (5,5){};
\node (5) at (4,4) {1};
\path (1) edge node[left] {$2$} (2);
\foreach\from/\to in {1/2,1/3,2/4,3/4}
\draw (\from) -- (\to);
\end{tikzpicture} 
2)\begin{tikzpicture}
[scale=.4,auto=left,every node/.style={circle,fill=blue!10}]
\node (1) at (5,5) {};
\node (2) at (5,10) {};
\node (3) at (10,5) {};
\node (4) at (10,10){};
\node (5) at (13,7) {};
\node (6) at (7,7) {};
\node (7) at (8.5,8.5) {1};
\node (8) at (11,7) {2};
\node (9) at (15,7) {3};
\foreach\from/\to in {1/2,1/3,2/4,3/4,3/5,4/5,2/6}
\draw (\from) -- (\to);
\end{tikzpicture} \\
Тогда: n-m+v=2\\
проверим:\\
1) 4-4+2=2 2) 6-7+3=2\\
\underline{Докозательство}\\
Индуция по количеству ребер\\
\underline{База} G-дерево\\
\begin{tikzpicture}
[scale=.8,auto=left,every node/.style={circle,fill=green!10}]
\node (2) at (1,1) {2};
\node (1) at (1,5){1};
\node (3) at (3,3) {3};
\node (6) at (7,1) {6};
\node (5) at (7,5){5};
\node (4) at (5,3) {4};
\foreach\from/\to in {1/3,2/3,3/4,4/6,4/5}
\draw (\from) -- (\to);
\end{tikzpicture}\\-1 грань\\
\begin{tikzpicture}
[scale=.4,auto=left,every node/.style={circle,fill=blue!10}]
\node (1) at (3,2){1};
\node (2) at (2,4){2};
\node (3) at (4,5.5){3};
\node (4) at (6,4) {4};
\node (5) at (5,2) {5};
\foreach\from/\to in {2/1,3/2,3/4,5/4,1/5}
\draw (\from) -- (\to);
\end{tikzpicture} вокруг грани есть цикл, а дерево без циклов\\
n-m+f=n-(n-1)+1=2\\
переход G- не знаем, верно ли (G,G'-связная планарные графы), если G' имеет меньше ребер=> верно\\
G- не дерево => есть цикл, берем ребро цикла\\
\begin{tikzpicture}
[scale=.4,auto=left,every node/.style={circle,fill=blue!10}]\\
%\tikzsryle{every node}=[draw,shape=circle];
\node (A) at (2,2.5) {};
\node (B) at (2,5) {};
\node (C) at (10,2.5) {};
\node (D) at (10,5){};
\node (f) at (6,5) {};
\node (E) at (6,2) {};
\path (A) edge [] (E);
\path (A) edge [] (B);
\path (E) edge [] (C);
\path (B) edge [] (f);
\path (f) edge [] (D);
\path (C) edge [bend left =-55]  (D);
\end{tikzpicture} -вокруг него 2 грани, удалим ребро, получим G' тоже связен и планарен\\
n' m' f'- вершины, ребра, грани G'\\
n'=n\\
m'=m-1\\
f'=f-1\\
по индукции предпололжим n'=m'+f'=2=>n-(m-1)+(f-1)=2 => n-m+f=2\\
Следствия\\
1) неважно, как рисовать планарный граф, количество граней постоянно\\
2) про многогранники также\\
\begin{tikzpicture}
[scale=.4,auto=left,every node/.style={circle,fill=blue!10}]\\
%\tikzsryle{every node}=[draw,shape=circle];
\node (A) at (2,2) {};
\node (B) at (2,5) {};
\node (C) at (7,2.5) {};
\node (D) at (7,6){};
\node (f) at (6,5) {};
\node (E) at (6,2) {};
\node (G) at (3,6) {};
\path (B) edge [] (G);
\path (A) edge [] (E);
\path (f) edge [] (E);
\path (D) edge [] (C);
\path (A) edge [] (B);
\path (E) edge [] (C);
\path (B) edge [] (f);
\path (f) edge [] (D);
\path (C) edge [] (D);
\path (G) edge [] (D);
\end{tikzpicture} \qquad \begin{tikzpicture}
[scale=.4,auto=left,every node/.style={circle,fill=blue!10}]
\node (1) at (5,5) {};
\node (2) at (5,10) {};
\node (3) at (10,5) {};
\node (4) at (10,10){};
\node (5) at (4,4) {};
\node (6) at (4,11) {};
\node (7) at (11,4) {};
\node (8) at (11,11){};
\foreach\from/\to in {1/2,1/3,2/4,3/4, 5/6,5/7,6/8,7/8, 5/1,3/7,2/6/,1/5,4/8}
\draw (\from) -- (\to);
\end{tikzpicture}\\
 8-12+6-2\\
3) Если G планарный( не обяательно связный) то n-m+f=1+|комп. связности G|\\
Докозательство упражнение\\
4. $\supset$ У каждой рани вокруг $\geq$ 3 ребра\\
3f$\leq$ $\sum _{y\in grani}$  количество ребер вокруг у $\leq$ 2m каждое ребро посчитно 1 или 2 раза=> 3f$\geq$ 2m\\
но n-m+f=2, умножим на 3\\
3n-3m+3f=6 => 3n-3m+2m$\geq$6=> 3n-m$\geq$6=> m$\leq$3n-6\\
Итого m$\leq$ 3n-6 в связном планарном графе\\
Следствие\\
полный граф при n=5- не планарен\\
\underline{Докозательство}\\
n=5\\
$m=\frac{5\cdot 4}{2}=10$\\
10$\leq$ 3*5-6=9??\\
\underline{Замечание}\\
$K_5$- полный граф n=5\\
\underline{Утверждение} граф $K_{3,3}$ тоже не планарный\\
\begin{tikzpicture}
[scale=.4,auto=left,every node/.style={circle,fill=blue!10}]
\node (1) at (1,1) {};
\node (2) at (1,3) {};
\node (3) at (1,5) {};
\node (4) at (3,1){};
\node (5) at (3,3) {};
\node (6) at (3,5) {};
\foreach\from/\to in {1/4,1/5,1/6,2/4, 2/6, 5/2,3/6/,3/5,3/4}
\draw (\from) -- (\to);
\end{tikzpicture}\\
\underline{Докозательствро}\\
n=6 m=9 9$\leq$ 3*6-6 нет противоречия\\
Сколько граней, если полярный 6-9+f=2 =>f=5 граней\\
В $K_{3,3}$ все циклы цетные C ходим лево-право или прав-лево\\
=> у грани $\leq$ 4 ребра \begin{tikzpicture}
[scale=.4,auto=left,every node/.style={circle,fill=blue!10}]
\node (1) at (1,1) {};
\node (2) at (3,1) {};
\node (3) at (2,3) {};
\foreach\from/\to in {1/3,1/2,2/3}
\draw (\from) -- (\to);
\end{tikzpicture} невозможно\\
4f $\leq$ $\sum$ребер грани g$\leq$ 2m=> m$\geq$2f но \sout{9$\geq$ 2*5}\\
\underline{Теорема} Понтрягина-Курлотовского\\
Связный граф планарен <=> если не содержит полуграфов стягивающихся к K$_5$ и $K_{3,3}$\\
\begin{tikzpicture}
[scale=.4,auto=left,every node/.style={circle,fill=blue!10}]
\node (1) at (1,1) {};
\node (2) at (1,3) {};
\node (3) at (1,5) {};
\node (4) at (3,1){};
\node (5) at (3,3) {};
\node (6) at (3,5) {};
\node (7) at (2,0) {};
\foreach\from/\to in {1/5,1/6,2/4, 2/6, 7/4,1/7, 5/2,3/6/,3/5,3/4}
\draw (\from) -- (\to);
\end{tikzpicture} стягивается и $K_{3,3}$ не планарен\\
\begin{tikzpicture}
[scale=.4,auto=left,every node/.style={circle,fill=blue!10}]
\node (1) at (3,2){};
\node (2) at (2,4){};
\node (3) at (4,5.5){};
\node (4) at (6,4) {};
\node (5) at (5,2) {};
\node (6) at (1,1) {};
\node (7) at (0,5) {};
\node (8) at (4,8) {};
\node (9) at (8,5) {};
\node (10) at (7,1) {};
\foreach\from/\to in {2/4,2/5,1/4,3/1,3/5,6/7,7/8,8/9,9/10,10/6,6/1,10/5,7/2,8/3,9/4}
\draw (\from) -- (\to);
\end{tikzpicture} не планарен, в нем есть $K_5$\\
\underline{Хроматизм}\\
\underline{Определение}\\
$\supset$ G=G(V,E)- граф раскраска графа\\
G в к цветов это функция c: V->{1..K}\\
причем, если есть ребро (U,V), то C(U)$\neq$C(V)\\
\begin{tikzpicture}
[scale=.4,auto=left,every node/.style={circle,fill=blue!10}]
\node (1) at (5,5) {2};
\node (2) at (5,10) {1};
\node (3) at (10,5) {1};
\node (4) at (10,10){2};
\node (5) at (14,5) {2};
\node (6) at (14,10){3};
\foreach\from/\to in {1/2,1/3,2/4,5/6,4/6,3/4,3/5}
\draw (\from) -- (\to);
\end{tikzpicture}-раскраска\\
\begin{tikzpicture}
[scale=.4,auto=left,every node/.style={circle,fill=blue!10}]
\node (1) at (5,5) {1};
\node (2) at (5,10) {1};
\node (3) at (10,5) {2};
\node (4) at (10,10){2};
\node (5) at (14,5) {3};
\node (6) at (14,10){3};
\foreach\from/\to in {1/2,1/3,2/4,5/6,4/6,3/5}
\draw (\from) -- (\to);
\end{tikzpicture}-не раскраска\\
какие графы можно раскрасить в 1 цвет\\
\begin{tikzpicture}
[scale=.4,auto=left,every node/.style={circle,fill=blue!10}]
\node (1) at (5,4) {1};
\node (2) at (5,10) {1};
\node (3) at (7,3) {1};
\node (4) at (9,11){1};
\node (5) at (13,4) {1};
\foreach\from/\to in {}
\draw (\from) -- (\to);
\end{tikzpicture}-это граф без ребер\\
какие графы можно раскрасить в 2 цветах\\
\underline{Определение}\\
граф G-двудолен, еслли его можно раскарсить в 2 цвета?\\
\begin{tikzpicture}
[scale=.4,auto=left,every node/.style={circle,fill=blue!10}]
\node (1) at (5,5) {1};
\node (2) at (5,10) {2};
\node (3) at (10,5) {2};
\node (4) at (10,10){1};
\foreach\from/\to in {1/2,1/3,2/4,3/4}
\draw (\from) -- (\to);
\end{tikzpicture}\\
\begin{tikzpicture}
[scale=.4,auto=left,every node/.style={circle,fill=blue!10}]
\node (1) at (5,5) {1};
\node (2) at (5,10) {2};
\node (3) at (10,5) {2};
\path (1) edge [] (2);
\path (3) edge []  (1);
\path (2) edge [] node[] {$??$} (3);
\end{tikzpicture}\\
$K_{3,3}$-двудольный\\
\begin{tikzpicture}
[scale=.4,auto=left,every node/.style={circle,fill=blue!10}]
\node (1) at (1,1) {1};
\node (2) at (1,3) {1};
\node (3) at (1,5) {1};
\node (4) at (5,1){2};
\node (5) at (5,3) {2};
\node (6) at (5,5) {2};
\foreach\from/\to in {1/4,1/5,1/6,2/4, 2/6, 5/2,3/6/,3/5,3/4}
\draw (\from) -- (\to);
\end{tikzpicture}\\
Замечание\\
Двудольные графы, часть рисуют из 2ух частей(долей)\\
\begin{tikzpicture}
[scale=.4,auto=left,every node/.style={circle,fill=blue!10}]
\node (1) at (1,3) {};
\node (2) at (1.5,5) {};
\node (3) at (0,4) {};
\node (0) at (1,1) {1};
\node (4) at (7,3){};
\node (5) at (7.5,5) {};
\node (6) at (6,4) {};
\node (7) at (7,1) {2};
\foreach\from/\to in {1/4,2/5,3/6}
\draw (\from) -- (\to);
\end{tikzpicture}\\
\underline{Теорема}\\
Двудолен <=> вс циклы G имеют четную длину\\
\underline{Докозательство}\\
1)двудолен=> циклы четные\\
\begin{tikzpicture}
[scale=.4,auto=left,every node/.style={circle,fill=blue!10}]
\node (1) at (5,5) {1};
\node (2) at (5,10) {1};
\node (3) at (10,5) {2};
\node (4) at (10,10){2};
\node (5) at (13,7) {1};
\foreach\from/\to in {1/3,2/4,3/5,4/5}
\draw (\from) -- (\to);
\end{tikzpicture} цикл поровну цвета 1 и цвета 2\\
2)циклы четные ?=> двудолен\\
"подвесим граф за вершину" $\forall$  вершина \begin{tikzpicture}
[scale=.3,auto=left,every node/.style={circle,fill=blue!10}]
\node (1) at (10,10){1};
\node (2) at (8,7) {2};
\node (3) at (12,7){2};
\node (4) at (11,4) {1};
\node (5) at (13,4) {1};
\node (8) at (8,4) {1};
\foreach\from/\to in {1/2,1/3,3/4,3/5,2/8}
\draw (\from) -- (\to);
\end{tikzpicture} ребра назад не рисуем \\
назначаем цвета по уровням\\
\begin{tikzpicture}
[scale=.3,auto=left,every node/.style={circle,fill=blue!10}]
\node (1) at (10,12){1};
\node (2) at (9,9) {2};
\node (3) at (8,6){2};
\node (4) at (7,3) {1};
\node (5) at (6,0) {1};
\node (8) at (13,9) {1};
\foreach\from/\to in {1/2,2/3,3/4,4/5,1/8}
\draw (\from) -- (\to);
\draw [dashed, ->] (5)--(8);
\end{tikzpicture} -обратное\\
Почему обратное ребро не соединяет оинаковые цвета? Потому что иначе цикл нечетный\\
Напоминание\\
К-раскраска графа k цветов у вершин\\
\begin{tikzpicture}
[scale=.4,auto=left,every node/.style={circle,fill=blue!10}]
\node (1) at (1,1){i};
\node (2) at (5,1) {i};
\foreach\from/\to in {1/2}
\draw (\from) -- (\to);
\end{tikzpicture} нельзя ребром один цвет\\
\begin{tikzpicture}
[scale=.4,auto=left,every node/.style={circle,fill=blue!10}]
\node (1) at (1,1){3};
\node (2) at (1,4) {1};
\node (3) at (4,1){1};
\node (4) at (4,4) {2};
\foreach\from/\to in {1/2,1/3,1/4,3/4}
\draw (\from) -- (\to);
\end{tikzpicture} и \quad
\begin{tikzpicture}
[scale=.4,auto=left,every node/.style={circle,fill=blue!10}]
\node (1) at (1,1){2};
\node (2) at (1,4) {1};
\node (3) at (4,1){3};
\node (4) at (4,4) {1};
\foreach\from/\to in {1/2,1/3,1/4,3/4}
\draw (\from) -- (\to);
\end{tikzpicture} \quad - раскраски, но не раскраска: 
\begin{tikzpicture}
[scale=.4,auto=left,every node/.style={circle,fill=blue!10}]
\node (1) at (1,1){2};
\node (2) at (1,4) {1};
\node (3) at (4,1){1};
\node (4) at (4,4) {1};
\foreach\from/\to in {1/2,1/3,1/4,3/4}
\draw (\from) -- (\to);
\end{tikzpicture}\\
граф имеет 1 раскраску <=> граф бз ребер\\
граф имеет 2 раскраски- двудольный граф(по определению)\\
Замечание\\
\begin{tikzpicture}
[scale=.4,auto=left,every node/.style={circle,fill=blue!10}]
\node (1) at (1,1) {};
\node (2) at (1,3) {};
\node (3) at (1,5) {};
\node (4) at (3,1){};
\node (5) at (3,3) {};
\node (6) at (3,5) {};
\foreach\from/\to in {1/4,1/5,1/6,2/4, 2/6, 5/2,3/6/,3/5,3/4}
\draw (\from) -- (\to);
\end{tikzpicture} граф двдолен<=> все циклы четные\\
\underline{Определение}\\
G=(V,E)- граф\\
$\phi$(G)- хроматическое число графа\\
минимальное количество цветов, в котром его можно раскрасить\\
$\psi$(\begin{tikzpicture}
[scale=.4,auto=left,every node/.style={circle,fill=blue!10}]
\node (1) at (1,1){};
\node (2) at (1,4) {};
\node (3) at (4,1){};
\node (4) at (4,4) {};
\foreach\from/\to in {1/2,1/3,1/4,3/4}
\draw (\from) -- (\to);
\end{tikzpicture})=3 \qquad
$\psi$(\begin{tikzpicture}
[scale=.4,auto=left,every node/.style={circle,fill=blue!10}]
\node (1) at (1,1){};
\node (2) at (1,4) {};
\node (3) at (4,1){};
\node (4) at (4,4) {};
\foreach\from/\to in {1/2,2/4,3/4,1/3}
\draw (\from) -- (\to);
\end{tikzpicture})=2\\
$\psi$(
\begin{tikzpicture}
[scale=.4,auto=left,every node/.style={circle,fill=blue!10}]
\node (1) at (3,2){};
\node (2) at (2,4){};
\node (3) at (4,5.5){};
\node (4) at (6,4) {};
\node (5) at (5,2) {};
\foreach\from/\to in {2/4,2/1,2/5,1/4,3/1,3/2,3/4,3/5,5/1,5/4}
\draw (\from) -- (\to);
\end{tikzpicture})=5\\
все должны быть разные $\psi(K_n)=n$ полный граф n вершин\\
\underline{Замечание}\\
Если $k\geq \psi(G)$, то G можно покрасить в k цветов\\
\underline{Утверждение}\\
$\psi(G) \leq$max deg v+1\\
\begin{tikzpicture}
[scale=.4,auto=left,every node/.style={circle,fill=blue!10}]
\node (1) at (1,5){1};
\node (2) at (3,5){2};
\node (3) at (5,5){3};
\node (4) at (5,3) {4};
\node (5) at (2,1) {5};
\node (6) at (4,1) {6};
\node (7) at (3,3) {7};
\foreach\from/\to in {1/2,1/7,5/6,1/5,6/4,6/7,3/2,3/4,3/7,7/2,4/2}
\draw (\from) -- (\to);
\end{tikzpicture} это степени вершины $\psi(G)=5+1$ max deg=5\\
\underline{Докозательство}\\
Индуцкция по количеству вершин\\
$
\left.
\begin{array}{rcll}
n=0\\
n=1\\
\end{array}
\right.
$верно\\
max deg =0\\
$\psi(G)\leq 1$\\
П G v- вершина max deg, уберем её, получим G'=G deg v\\
max deg G'$\geq$ max deg G=$\triangle$\\
раскрасим G' в A+1 цвет\\
цвет запрещены $\leq \triangle$ цветов => $\geq$ 1 цвет можно\\
\underline{Утверждение}\\
G-планарный граф => $\psi$(G)$\leq$5\\
\begin{tikzpicture}
[scale=.4,auto=left,every node/.style={circle,fill=blue!10}]
\node (1) at (1,3){1};
\node (2) at (3,5){2};
\node (3) at (3,1){3};
\node (4) at (5,3) {4};
\node (5) at (7,1) {5};
\node (6) at (7,3) {6};
\node (7) at (7,5) {7};
\foreach\from/\to in {1/2,1/3,2/4,3/4,6/4,6/7,3/5,7/2}
\draw (\from) -- (\to);
\end{tikzpicture}\\
\underline{Докозательство}\\
1)в G есть вершина степени $\leq$ 5  |V|=n\\
Если нет => deg V $\geq$ 6 => $\sum$ deg v$\geq$ 6n\\
=> 2m$\geq$ 6n=> m$\geq$ 3nб но в планарном G m$\leq$ 3n-6\\
\underline{Утверждение}\\
Раскрасим в 5 цветов по индуцкции\\
Б графы 1,2,3,4,5 вершин- можно раскрасить\\
$\supset$ у нас n вершин (для n-1 вершин есть раскраска)\\
Берем v: deg v$\leq$ 5
%\path (C) edge [bend left =-15]  (D);
\begin{tikzpicture}
[scale=.3,auto=left,every node/.style={circle,fill=blue!10}]
\node (1) at (1,0){v1};
\node (2) at (1,6) {v2};
\node (3) at (5.5,6.5){v3};
\node (4) at (8,3) {v4};
\node (5) at (6,0) {v5};
\node (8) at (4,3) {};
\foreach\from/\to in {1/8,2/8,3/8,8/5,4/8}
\draw (\from) -- (\to);
\path (1) edge [dashed ,bend left =25]  (2);
\path (2) edge [dashed ,bend left =25]  (3);
\path (3) edge [dashed ,bend left =25]  (4);
\path (4) edge [dashed ,bend left =25]  (5);
\path (5) edge [dashed ,bend left =25]  (1);
%\draw [dashed, ->] (1)--(2);
\end{tikzpicture}\\
раскрасим G' без v\\
если соседи $v_i$ исчезнут $\leq$u цветов => для V есть цвет\\
осталось \\
\begin{tikzpicture}
[scale=.3,auto=left,every node/.style={circle,fill=blue!10}]
\node (1) at (1,0){1};
\node (2) at (1,6) {2};
\node (3) at (5.5,6.5){3};
\node (4) at (8,3) {4};
\node (5) at (6,0) {5};
\node (8) at (4,3) {};
\foreach\from/\to in {1/8,2/8,3/8,8/5,4/8}
\draw (\from) -- (\to);
\end{tikzpicture}\\
\begin{tikzpicture}
[scale=.3,auto=left,every node/.style={circle,fill=blue!10}]
\node (1) at (1,0){v1};
\node (2) at (1,6) {v2};
\node (3) at (5.5,6.5){v3};
\node (4) at (8,3) {v4};
\node (5) at (6,0) {v5};
\node (8) at (4,3) {};
\foreach\from/\to in {1/8,2/8,3/8,8/5,4/8}
\draw (\from) -- (\to);
\path (1) edge [bend left =25]  (2);
\path (2) edge [bend left =25]  (3);
\path (3) edge [bend left =25]  (4);
\path (4) edge [bend left =25]  (5);
\path (5) edge [bend left =25]  (1);
\path (2,2) edge [<->] [bend left=-25] (6,2);
%\draw [dashed, ->] (1)--(2);
\end{tikzpicture}
стягиваем 1 граньбез V сделаем $\overline G$\\
$\overline G $\\
\begin{tikzpicture}
[scale=.3,auto=left,every node/.style={circle,fill=blue!10}]
\node (2) at (1,6) {v2};
\node (3) at (5.5,6.5){v3};
\node (5) at (6,0) {v5};
\node (8) at (4,3) {14};
\path (2) edge [bend left =25]  (3);
\path (2) edge [bend left =-25]  (8);
\path (3) edge [bend left =25]  (8);
\path (8) edge [bend left =25]  (5);
\path (8) edge [bend left =-25]  (5);
%\draw [dashed, ->] (1)--(2);
\end{tikzpicture}\\
$\overline G$ -n-2 вершины => можно раскрасить\\
вернемся у G, a и d имеют один цвет => есть для V\\
если $v_i \quad v_j$ все соединены ребром => есть $K_5$=> не планарен\\
\underline{Утверждение}\\
$\psi(G)\leq 4$( проблема 4х красок)\\
\underline{Хроматические многочлены}\\
$\supset \quad \psi (G,K)$-это функция, сколько способов расрасить G в k цветов\\
$\psi(0-0,k)=$
$
\left\{
\begin{array}{rcll}
k=0 \quad 0\\
k=1 \quad 0\\
k=2 \quad 2\\
k=3 \quad 6\\
k=4 \quad 12\\
\end{array}
\right.
$\\
иначе k(k-1)\\
$\psi(0-0,k)$=k(k-1)\\
$\psi(0 \quad 0,k)=k^2$\\
\underline{Утверждение}\\
1)$\psi (0_n,k)=k^n$ граф из n вершин без ребер\\
2)$\psi(K_n$(полный), k)=$\underbrace{ k(k-2)(k-1)...(k-n+1)=k^n}_\textrm{множители}$\\
3)$\psi (T_n,k)=$ подвесим дерево $\forall$ вершину\\
$T_n$- дерево
\begin{tikzpicture}
[scale=.3,auto=left,every node/.style={circle,fill=blue!10}]
\node (1) at (5,5){};
\node (2) at (4,3.5) {};
\node (3) at (6,3.5){};
\node (4) at (6.5,2) {};
\node (5) at (6.5,2) {};
\node (8) at (4,2) {};
\foreach\from/\to in {1/2,1/3,3/4,3/5,2/8}
\draw (\from) -- (\to);
\end{tikzpicture} k-1 цвет выполнен\\
4) $\overline G$- граф\\
u,k вершины с ребром (u,v)\\
G=$\overline G$(u,v)\\ без ребра\\
G'=$\overline G$ где u,v стянуты в вершину\\
\begin{tikzpicture}
[scale=.3,auto=left,every node/.style={circle,fill=blue!10}]
\node (1) at (1,3){u};
\node (2) at (1,6) {v};
\node (3) at (5,3){u};
\node (4) at (5,6) {v};
\node (4) at (9,5) {uv};
\node (5) at (1,0) {$\overline G$};
\node (6) at (5,0) {G};
\node (6) at (9,0) {G'};
\foreach\from/\to in {1/2}
\draw (\from) -- (\to);
\end{tikzpicture}\\
\begin{tikzpicture}
[scale=.3,auto=left,every node/.style={circle,fill=blue!10}]
\node (1) at (1,4){};
\node (2) at (3,1) {u};
\node (3) at (5,4){v};
\node (4) at (7,1) {};
\node (5) at (9,4) {};
\node (6) at (5,0) {$\overline G$};
\foreach\from/\to in {1/2,1/3,3/4,3/5,2/4,4/5,2/3}
\draw (\from) -- (\to);
\end{tikzpicture}  
\begin{tikzpicture}
[scale=.3,auto=left,every node/.style={circle,fill=blue!10}]
\node (1) at (1,4){};
\node (2) at (1,1) {u};
\node (3) at (5,4){v};
\node (4) at (7,1) {};
\node (5) at (9,4) {};
\node (6) at (5,0) {G};
\foreach\from/\to in {1/2,1/3,3/4,3/5,2/4,4/5,2/3}
\draw (\from) -- (\to);
\end{tikzpicture} 
\begin{tikzpicture}
[scale=.3,auto=left,every node/.style={circle,fill=blue!10}]
\node (1) at (1,4){};
\node (3) at (5,3){uv};
\node (4) at (7,1) {};
\node (5) at (9,4) {};
\node (6) at (5,0) {G'};
\path (1) edge [bend left =25]  (3);
\path (4) edge [bend left =25]  (3);
\path (5) edge [bend left =25]  (3);
\path (3) edge [bend left =-50]  (4);
\path (4) edge [bend left =-25]  (5);
\end{tikzpicture}\\
$\psi(G,k)=\psi(\overline G, k)+\psi(G',k)$\\
способы раскрасить G, где u и v- разные цвета\\
способы\\
раскраска G, где u,v- один цвет\\
\underline{Следствие}\\
$\psi(\overline G, k)=\psi(G, k)-\psi(G',k)$\\
Примеры\\
$\psi($\begin{tikzpicture}
[scale=.3,auto=left,every node/.style={circle,fill=blue!10}]
\node (1) at (1,1){};
\node (3) at (4,1){};
\node (2) at (1,4) {};
\node (4) at (4,4) {};
\foreach\from/\to in {1/2,1/3,3/4,2/4,1/4}
\draw (\from) -- (\to);
\end{tikzpicture},k)=$\psi($\begin{tikzpicture}
[scale=.3,auto=left,every node/.style={circle,fill=blue!10}]
\node (1) at (1,1){};
\node (3) at (4,1){};
\node (2) at (1,4) {};
\node (4) at (4,4) {};
\foreach\from/\to in {1/2,1/3,3/4,2/4,1/4,2/3}
\draw (\from) -- (\to);
\end{tikzpicture},k)=$\psi($\begin{tikzpicture}
[scale=.3,auto=left,every node/.style={circle,fill=blue!10}]
\node (1) at (1,1){};
\node (3) at (4,1){};
\node (2) at (2.5,4) {};
\foreach\from/\to in {1/2,1/3,2/3}
\draw (\from) -- (\to);
\end{tikzpicture},k)=\\
=k(k-1)(k-2)(k-3)+k(k-1)(k-2)\\
5)\\
$\psi($\begin{tikzpicture}
[scale=.4,auto=left,every node/.style={circle,fill=blue!10}]
\node (1) at (3,2){};
\node (2) at (2,4){};
\node (3) at (4,5.5){};
\node (4) at (6,4) {};
\node (5) at (5,2) {};
\foreach\from/\to in {2/1,3/2,3/4,5/4,1/5}
\draw (\from) -- (\to);
\end{tikzpicture}(n вершин),k)=?\\
$\psi(C_n,k)=\psi($\begin{tikzpicture}
[scale=.3,auto=left,every node/.style={circle,fill=blue!10}]
\node (1) at (10,8){};
\node (2) at (9,6) {};
\node (3) at (8,4){};
\node (4) at (7,2) {};
\node (5) at (6,0) {};
\foreach\from/\to in {1/2,2/3,3/4,4/5}
\draw (\from) -- (\to);
\end{tikzpicture},k)-$\psi(C_{n-1},k)=k(k-1)^{n-2}-\psi(C_{n-k},k)$\\
=>$C_n$=k(k-1)$^{n-1}-C_{n-1}=k(k-1)^{n-1}-k(k-1)^{n-2}+C_{n-2}=k(k-1)^{n-1}-k(k-1)^{n-2}+k(k-1)^{n-3+}$\\
$\pm k(k-1)^1+(-1)^{n-1}C \qquad c=\psi(0,k)=k$\\
геометрическая прогрессия\\
$q=(-1)^n k(k-1) \frac{q^n-1}{q-1}+(-1)^{n-1}k=(-1)^nk(k-1)\frac{(-1)^n(k-1)^{n-1}-1}{k-1}+(-1)^{n-1}k\\$
$\approx (k-1)^n+(-1)^n$\\
\begin{tikzpicture}
[scale=.4,auto=left,every node/.style={circle,fill=blue!10}]
\node (1) at (0,2){};
\node (2) at (2,0) {};
\node (3) at (3,4){};
\node (4) at (4,0) {};
\node (5) at (6,2) {};
\foreach\from/\to in {1/2,1/3,2/4,4/5,3/5}
\draw (\from) -- (\to);
\end{tikzpicture}- С$_5$ цикл из 5 вершин\\
$\psi (C_n,K)$- несколько способов раскрасить $C_n$ в к цветов\\
$\psi (T_n$(дерево),k)=$k(k-1)^{n-1}$\\
$\psi (K_n$(полный),k)=$k^n=k(k-1)...(k-n+1)$\\
$\psi(C_n,k)=k(k-1)^{n-1}-k(k-1)^{n-2}+k(k-1)^{n-3}..(-1)^n k(k-1)+ (-1)^{n+1}k-C_1$\\
=геом прогрессия\\
начало:$(-1)^{n+1}k$\\
множитель:$-(k-1)$\\
членов:n штук\\
S=$(-1)^{n+1}k\frac{q^{n+1}-1}{q-1}\\=(-1)^{n+1}k\frac{(-1)^k (k-1)^n -1}{-k}=(-1)^n((-1)^n(k-1)^n-1) =(k-1)^n-(-1)^n$\\
\underline{Утверждение}\\
$\supset$G имеет висячую вершину и\\
1)\\ \begin{tikzpicture}
[scale=.4,auto=left,every node/.style={circle,fill=blue!10}]
\node (1) at (4,2){G'};
\node (2) at (4,4) {};
\node (4) at (4,-2) {$\underbrace{}_G$};
\node (3) at (10,4) {u};
\path (0,4) edge [bend left =45]  (4,8);
\path (4,8) edge [bend left =45]  (8,4);
\path (8,4) edge [bend left =45]  (4,0);
\path (0,4) edge [bend left =-45]  (4,0);
\foreach\from/\to in{2/3}
\draw(\from)--(\to);
\end{tikzpicture} $\psi(G,K)=\psi(G',k)x(k-1)$- расчитали\\
2)\\\begin{tikzpicture}
[scale=.4,auto=left,every node/.style={circle,fill=blue!10}]
\node (1) at (4,2){G'};
\node (2) at (4,4) {};
\node (5) at (5,6) {};
\node (4) at (4,-2) {$\underbrace{}_G$};
\node (3) at (10,4) {u};
\path (0,4) edge [bend left =45]  (4,8);
\path (4,8) edge [bend left =45]  (8,4);
\path (8,4) edge [bend left =45]  (4,0);
\path (0,4) edge [bend left =-45]  (4,0);
\foreach\from/\to in{2/3,2/5,5/3}
\draw(\from)--(\to);
\end{tikzpicture} u соединены c $v_1$ и $v_2$ $(v_1,v_2)\in G$ $\psi(G,k)$\\
$=\psi(G',k)x(k-2)$\\
\underline{Утверждение}\\
$\supset$ G=$G'_1$ v $G'_2$ нет ребер между G' и $G'_2$\\
\begin{tikzpicture}
[scale=.4,auto=left,every node/.style={circle,fill=blue!10}]
\node (1) at (0,1) {$G'_1$};
\node (1) at (3,1) {$G'_2$};
\node (4) at (1,-2) {$\underbrace{}_G$};
\end{tikzpicture} нет ребер $\psi(G,k)=\psi(G'_1,k) \psi(G'_2,k)$\\
Пример\\
$\psi($
\begin{tikzpicture}
[scale=.3,auto=left,every node/.style={circle,fill=blue!10}]
\node (6) at (2.5,6) {};
\node (1) at (2.5,4){};
\node (2) at (0.5,1) {};
\node (3) at (4.5,1){};
\node (4) at (8.5,1) {};
\node (5) at (6.5,4) {};
\foreach\from/\to in {1/2,1/3,3/4,3/5,4/5,1/3,5/3,2/3,1/5,1/6}
\draw (\from) -- (\to);
\end{tikzpicture} ,k)=(k-1)$\psi$(\begin{tikzpicture}
[scale=.3,auto=left,every node/.style={circle,fill=blue!10}]
\node (1) at (2.5,4){};
\node (2) at (0.5,1) {};
\node (3) at (4.5,1){};
\node (4) at (8.5,1) {};
\node (5) at (6.5,4) {};
\foreach\from/\to in {1/2,1/3,3/4,3/5,4/5,1/3,5/3,2/3,1/5}
\draw (\from) -- (\to);
\end{tikzpicture},k)=\\
=(k-1)(k-2)$\psi$\begin{tikzpicture}
[scale=.3,auto=left,every node/.style={circle,fill=blue!10}]
\node (1) at (2.5,4){};
\node (2) at (0.5,1) {};
\node (3) at (4.5,1){};
\node (5) at (6.5,4) {};
\foreach\from/\to in {1/5,1/2,1/3,3/5,2/3}
\draw (\from) -- (\to);
\end{tikzpicture},k)=\\
=(k-1)$(k-2)^2 \psi($\begin{tikzpicture}
[scale=.3,auto=left,every node/.style={circle,fill=blue!10}]
\node (1) at (0.5,1){};
\node (3) at (3.5,1){};
\node (2) at (2,4) {};
\foreach\from/\to in {1/2,1/3,2/3}
\draw (\from) -- (\to);
\end{tikzpicture})=(k-1)($k-2)^2 \cdot k(k-1)(k-2)$\\
Напоминание\\
$\psi$
\begin{tikzpicture}
[scale=.4,auto=left,every node/.style={circle,fill=blue!10}]
\node (1) at (2,5){$\overline G$};
\node (2) at (2,2) {};
\node (5) at (2.5,3) {};
\path (0,2) edge [bend left =45]  (2,4);
\path (2,4) edge [bend left =45]  (4,2);
\path (4,2) edge [bend left =45]  (2,0);
\path (0,2) edge [bend left =-45]  (2,0);
\foreach\from/\to in{2/5}
\draw(\from)--(\to);
\end{tikzpicture}=$\psi$
\begin{tikzpicture}
[scale=.4,auto=left,every node/.style={circle,fill=blue!10}]
\node (1) at (2,5){G};
\node (2) at (2,2) {};
\node (5) at (2.5,3) {};
\path (0,2) edge [bend left =45]  (2,4);
\path (2,4) edge [bend left =45]  (4,2);
\path (4,2) edge [bend left =45]  (2,0);
\path (0,2) edge [bend left =-45]  (2,0);
\foreach\from/\to in{}
\draw(\from)--(\to);
\end{tikzpicture}-$\psi$
\begin{tikzpicture}
[scale=.4,auto=left,every node/.style={circle,fill=blue!10}]
\node (1) at (2,5){G'};
\node (2) at (2,2) {uv};
\path (0,2) edge [bend left =45]  (2,4);
\path (2,4) edge [bend left =45]  (4,2);
\path (4,2) edge [bend left =45]  (2,0);
\path (0,2) edge [bend left =-45]  (2,0);
\foreach\from/\to in{}
\draw(\from)--(\to);
\end{tikzpicture}\\
\underline{Утверждение}\\
$\psi(G,k)$- это многочлен\\
1)старший коэффициент=1\\
2)степень= n(количество вершин)\\
3)знаки чередуются\\
4)младший коэффициент=0\\
5)коэффициент при $k^{n-1}=\pm m$(количество ребер)\\
\underline{Докозательство}\\
Индукция по количеству вершин при равном количестве вершин: количество ребер\\
База. пустой граф ищ n вершин $\psi (\vdots \vdots,k)=k^n\pm 0k^{n-1}$\\
переход $\overline G$ с ребером $\psi($\begin{tikzpicture}
[scale=.4,auto=left,every node/.style={circle,fill=blue!10}]
\node (1) at (0,0){};\node (1) at (0,1){};
\path (0,0) edge []  (0,1);
\end{tikzpicture},k)=$\psi($\begin{tikzpicture}
[scale=.4,auto=left,every node/.style={circle,fill=blue!10}]
\node (1) at (0,0){};\node (1) at (0,1){};
\end{tikzpicture},k)(*1)-$\psi($\begin{tikzpicture}
[scale=.4,auto=left,every node/.style={circle,fill=blue!10}]
\node (1) at (0,0){};
\end{tikzpicture},k)(*2)\\
*1-мало ребер, количество вершин\\
*2-мало вершин\\
работает индукционное предположение\\
1) ст. коэффициент $(1k^n)-(k^{n-1})$\\
2)степень=n\\
3)$(k^n-k^{n-1}+k^{n-2})-(k^{n-1}-k^{n-2}+k^{n-3})$\\
4)младший коэффициент\\
5)ребер G$k^{n-1}$=-(количество ребер G+1)$k^{n-1}$=-ребер G$k^{n-1}$\\
на практике\\
$\psi($\begin{tikzpicture}
[scale=.3,auto=left,every node/.style={circle,fill=blue!10}]
\node (1) at (0.5,1){};
\node (3) at (3.5,1){};
\node (2) at (2,4) {};
\node (4) at (2,-2) {};
\foreach\from/\to in {1/2,1/3,2/3,1/4}
\draw (\from) -- (\to);
\end{tikzpicture},k)=(k-1)$\psi($\begin{tikzpicture}
[scale=.3,auto=left,every node/.style={circle,fill=blue!10}]
\node (1) at (0.5,1){};
\node (3) at (3.5,1){};
\node (2) at (2,4) {};
\foreach\from/\to in {1/2,1/3,2/3}
\draw (\from) -- (\to);
\end{tikzpicture},k)=(k-1)k(k-1)(k-2)=k(k-1$)^2$)(k-2)\\
Раскроем скобки\\
$k^4-4k^3+5k^2-2k$\\
\underline{Утверждение}\\
$\psi(G)$-хроматическое число(минмальное число цветов для раскраски)\\
$\psi(G,k)$ k=0,1,2,...$\psi(G)-1$- корни многочлена\\
$\psi(G)$- не корень\\
\underline{Эйлеровы графы}\\
Нарисовать не проводя дважды по одному ребру\\
\begin{tikzpicture}
[scale=.4,auto=left,every node/.style={circle,fill=blue!10}]
\node (1) at (1,1){};
\node (2) at (1,4) {};
\node (3) at (4,1){};
\node (4) at (4,4) {};
\node (5) at (2.5,6) {};
\foreach\from/\to in {1/2,1/3,1/4,3/4,2/4,2/3,4/5,2/5}
\draw (\from) -- (\to);
\end{tikzpicture}\\
\begin{tikzpicture}
%\path (4,2) edge [bend right =45]  (2,0); 
\path (1,1) edge [bend  right=-5] (1,3.2);
\path (1,1) edge [bend left =-90] (1.4,0.8);
\path (1.4,0.8) edge [bend  right=5] (3.6,3.2);
\path (3.6,3.2) edge [bend left =90] (3.7,3.2);
\path (3.7,3.2) edge [bend  right=-5] (3.7,0.8);
\path (1.6,0.8) edge [bend right=15] (3.4,0.8);
\path (3.4,0.8) edge [bend right=90] (3.4,1.2);
\path (3.4,1.2) edge [bend right=15] (1.3,3.2);
\path (1.3,3.2) edge [bend right=-90] (1.4,3.5);
\path (1.4,3.5) edge [bend right=-15] (3.5,3.4);
\path (3.5,3.4) edge [bend right=90] (3.7,3.6);
\path (3.7,3.6) edge [bend right=15] (2.6,6.1);
\path (2.6,6.1) edge [bend right=90] (2.4,6.1);
\path  (2.4,6.1) edge [bend right=15] (1,3.2);
\end{tikzpicture}\\
\underline{Определение}\\
Эейлеров путь- цикл, содержащий все ребра не проходящие дважды по ребру\\
\underline{Утверждение}\\
G содержит Эйлеров цикл <=> G связан и все степени вершин четные deg v-чет  $\forall v\in V$\\
Пример\\
\begin{tikzpicture}
[scale=.4,auto=left,every node/.style={circle,fill=blue!10}]
\node (1) at (1,1){3};
\node (2) at (1,4) {4};
\node (3) at (4,1){3};
\node (4) at (4,4) {4};
\node (5) at (2.5,6) {2};
\foreach\from/\to in {1/2,1/3,1/4,3/4,2/4,2/3,4/5,2/5}
\draw (\from) -- (\to);
\end{tikzpicture}- нет цикла \begin{tikzpicture}
[scale=.4,auto=left,every node/.style={circle,fill=blue!10}]
\node (1) at (1,1){4};
\node (2) at (1,4) {4};
\node (3) at (4,1){4};
\node (4) at (4,4) {4};
\node (5) at (2.5,6) {2};
\node (6) at (2.5,-1) {2};
\foreach\from/\to in {1/2,1/3,1/4,3/4,2/4,2/3,4/5,2/5,1/6,3/6}
\draw (\from) -- (\to);
\end{tikzpicture}- есть Эйлеров цикл\\
\underline{Докозательство}\\
представим граф имеющий одну веришну с длинным ребром пересекющимся самого себя, но входящим и исходящим из одной вершины- такой граф связен. \\Количество входов= количество выходов=> deg четн.\\
В каждой вершине по пути\\
Использовано чет. ребер( к вход, к выходу)\\
+1 реберо, через которые вошли\\
=> использовали нечет ребер\\
=> есть еще одно, по нему можно уйти, кроме начальной, из неё вышли на 1 раз больше\\
=> мы закончим в начатой вершине\\
\begin{tikzpicture}
[scale=.4,auto=left,every node/.style={circle,fill=blue!10}]
\node (1) at (5,5) {4};
\node (2) at (5,10) {2};
\node (3) at (10,5) {4};
\node (4) at (10,10){4};
\node (5) at (14,5) {2};
\node (6) at (14,10){4};
\path (1) edge[bend right=45] (14,4);
\path (14,4) edge[ bend right=45](6);
\foreach\from/\to in {1/4,3/6,1/2,1/3,2/4,5/6,4/6,3/4,3/5}
\draw (\from) -- (\to);
\end{tikzpicture} 
\begin{tikzpicture}
[scale=.4,auto=left,every node/.style={circle,fill=blue!10}]
\node (1) at (5,5) {};
\node (2) at (5,10) {};
\node (3) at (10,5) {};
\node (4) at (10,10){};
\node (5) at (14,5) {};
\node (6) at (14,10){};
\path (1) edge[dashed, bend right=45] (14,4);
\path (14,4) edge[dashed, bend right=45](6);
\foreach\from/\to in {1/4,3/6,1/2,1/3,2/4,5/6,4/6,3/4,3/5}
\draw (\from) -- (\to);
\end{tikzpicture}\\
граф с 1 вершиной и ребром входящим и исходящим из неё\\
Построена часть в остатке, все степени четные, т.к. G связен из начальной вершины x можно попасть в $\forall$ вершину и ребро\\
Повторим процесс из v$\in$ 1 циклу, из которой ведет новое ребро\\
объединим 2 цикла\\
\begin{tikzpicture}
[scale=.4,auto=left,every node/.style={circle,fill=blue!10}]
\path (2,2) edge [->,bend left =40]  (3,3);
\path (3,3) edge [bend left =40]  (5,2);
\path (5,2) edge [bend left =40]  (3,1);
\path (3,1) edge [bend left =40]  (2,2);
\path (0,2) edge [->,bend left =45]  (2,4);
\path (2,4) edge [bend left =45]  (4,2);
\path (4,2) edge [bend left =45]  (2,0);
\path (0,2) edge [bend left =-45]  (2,0);
\end{tikzpicture} Продолжаем пока все ребране объединятся в 1 цикл\\
\underline{Теорема}\\
G содержит Эйлеров путь <=>\\
1)связен\\
2)$
\left\{
\begin{array}{rcll}
\textrm{Степени всеъ вершин четны}\\
\textrm{Степени всех вершин кроме 2ух- чет}\\
\end{array}
\right.
$\\
\begin{tikzpicture}
[scale=.4,auto=left,every node/.style={circle,fill=blue!10}]
\node (1) at (1,1){3};
\node (2) at (1,4) {4};
\node (3) at (4,1){3};
\node (4) at (4,4) {4};
\node (5) at (2.5,6) {2};
\foreach\from/\to in {1/2,1/3,1/4,3/4,2/4,2/3,4/5,2/5}
\draw (\from) -- (\to);
\end{tikzpicture}
граф с двумя вершинами и ребром выходящим из одной и входящей в другую\\
В этом случае нечет вершины- это начало и конец\\
\underline{Определение}\\
Гамильтонов цилы- полсьые цепи/циклыпо всем вершинам\\
\begin{tikzpicture}
[scale=.4,auto=left,every node/.style={circle,fill=blue!10}]
\node (1) at (1,1){};
\node (2) at (1,4) {};
\node (3) at (4,1){};
\node (4) at (4,4) {};
\node (5) at (2.5,6) {};
\node (11) at (0,0){};
\node (22) at (0,5) {};
\node (33) at (5,0){};
\node (44) at (5,5) {};
\node (55) at (2.5,7) {};
\foreach\from/\to in {1/2,1/3,1/4,3/4,2/4,2/3,4/5,2/5,11/22,11/33,33/44,44/55,22/55}
\draw (\from) -- (\to);
\end{tikzpicture}\\
\begin{tikzpicture}
[scale=.4,auto=left,every node/.style={circle,fill=blue!10}]
\node (1) at (0,0) {};
\node (2) at (0,5) {};
\node (3) at (5,0) {};
\node (4) at (5,5){};
\node (5) at (2.5,2.5) {};
\path (5) edge[bend right=-25] (-2,-2);
\path (-2,-2) edge[ bend right=-25](-2,5);
\path (5) edge[bend right=-25] (7,7);
\path (7,7) edge[ bend right=-25](7,0);
\foreach\from/\to in {1/2,1/5,2/5,3/5,5/4,3/4}
\draw (\from) -- (\to);
\end{tikzpicture} Г. путь\\
\begin{tikzpicture}
[scale=.4,auto=left,every node/.style={circle,fill=blue!10}]
\node (1) at (2,0) {};
\node (2) at (2,5) {};
\node (3) at (7,0) {};
\node (4) at (7,5){};
\node (5) at (4.5,2.5) {};
\node (6) at (0,0) {};
\node (7) at (0,5) {};
\foreach\from/\to in {1/2,1/5,2/5,3/5,5/4,3/4,6/7,7/2,6/1}
\draw (\from) -- (\to);
\end{tikzpicture} нет Г. пути\\
В прошлый раз( по всем вершинам) гамильлтоновы пути\\
(по всем ребрам) эйлеровы циклы\\
\underline{Длины путей в графе}\\
\underline{Определение}\\
Длина пути в графе- количество ребер в пути\\
Пример\\
\begin{tikzpicture}
[scale=0.5,auto=left,every node/.style={circle,fill=blue!10}]
\node (1) at (0.5,1){A};
\node (3) at (3.5,1){C};
\node (2) at (2,4) {B};
\node (4) at (6.5,4) {D};
\node (5) at (6.5,1) {E};
\node (6) at (9,2.5) {F};
\foreach\from/\to in {1/2,1/3,2/3,3/4,4/6,4/5,3/5}
\draw (\from) -- (\to);
\end{tikzpicture}\\
ABCDF- путь от A до F- длина 4(4 ребра)\\
ACEDF- длина 4\\
ACDF-длина 3\\
ABCEDF- длина 5\\
\underline{Определение}\\
Расстояние между вершинами- минимальная длина пути между\\
вершинами или $+\infty$, если пути нет\\
\underline{Обозначение}
d(x,y)- растояние от X до Y\\
\underline{Пример}
d(A,F)=3\\
\underline{Определение}\\
Диаметр графа- минимальное расстояние между вершинами графа\\
\underline{Пример}\\
В примере выше для него =3(достигается на др)\\
\begin{tikzpicture}
[scale=.4,auto=left,every node/.style={circle,fill=blue!10}]
\node (1) at (0,0) {};
\node (2) at (0,4) {};
\node (3) at (4,0) {};
\node (4) at (4,4) {\textcolor{blue}{3}};
\node (5) at (8,4) {\textcolor{blue}{4}};
\node (6) at (8,0){\textcolor{blue}{5}};
\node (7) at (0,8){\textcolor{blue}{1}};
\node (8) at (4,8){\textcolor{blue}{2}};
\node (9) at (8,8){};
\path (7) edge[blue](8);
\path (4) edge[blue](8);
\path (5) edge[blue](6);
\path (4) edge[blue](5);
\foreach\from/\to in {1/2,1/3,3/6,2/4,8/9,5/9,3/4,7/2}
\draw (\from) -- (\to);
\end{tikzpicture} -диаметр 4\\
Все другие расстояние $\leq$ 4\\
\underline{Определение} Для каждой вершины графа G=(V,E) можно посчитать\\
max расстояние до других вершин\\
r(v)=max{d(v,s)| s$\in$V}\\
\underline{Радиус}\\
r(G)=min{r(v)|v$\in$ V}\\
те вершины, на которых достигается min- это центр\\
\begin{tikzpicture}
[scale=.4,auto=left,every node/.style={circle,fill=blue!10}]
\node (1) at (0,0) {4};
\node (2) at (0,4) {3};
\node (3) at (4,0) {3};
\node (4)  [fill=gray!80!blue] at (4,4) {2};
\node (5) at (8,4) {3};
\node (6) at (8,0) {4};
\node (7) at (0,8) {4};
\node (8) at (4,8) {3};
\node (9) at (8,8) {4};
\foreach\from/\to in {1/2,1/3,3/6,2/4,8/9,5/9,3/4,7/2,6/5,5/4,8/4,7/8}
\draw (\from) -- (\to);
\end{tikzpicture}\\ 
2-центр\\
r(6)=2-радиус графа\\
Центров может быть много\\
\begin{tikzpicture}
[scale=.4,auto=left,every node/.style={circle,fill=blue!10}]
\node (1) at (0,0) {6};
\node (2) at (0,3) {5};
\node (3) at (3,0) {5};
\node (4) [fill=gray!80!blue] at (3,3) {4};
\node (5) [fill=gray!80!blue] at (6,3) {4};
\node (6) at (6,0) {5};
\node (7) at (0,6) {5};
\node (8) [fill=gray!80!blue] at (3,6) {4};
\node (9) [fill=gray!80!blue] at (6,6) {4};
\node (10) at (0,9) {6};
\node (11) at (3,9) {5};
\node (12) at (6,9) {5};
\node (13) at (9,9) {5};
\node (14) at (9,6) {6};
\node (15) at (9,3) {5};
\node (16) at (9,0) {6};
\foreach\from/\to in {1/2,1/3,3/6,2/4,8/9,5/9,3/4,7/2,6/5,5/4,8/4,7/8,16/15,15/5,15/14,9/14,9/12,7/10,8/11,9/12,13/14,10/11,11/12,12/13,6/16}
\draw (\from) -- (\to);
\end{tikzpicture} -4 центра r(6)=4\\
или\\
\begin{tikzpicture}
[scale=0.6,auto=left,every node/.style={circle,fill=blue!10}]
\node (1) at (0,0) {3};
\node (2) [fill=gray!80!blue] at (2,0) {2};
\node (3)  [fill=gray!80!blue] at (4,0) {2};
\node (4) at (6,0) {3};
\node (5) at (5,-1) {3};
\foreach\from/\to in {1/2,2/3,3/4,3/5}
\draw (\from) -- (\to);
\end{tikzpicture} r(G)=2 2 центра\\
\underline{Утверждение} в G=(V,E) d(G)$\leq$ 2 r(G)\\
\underline{Докозательство}\\
$\supset$ c-центр графа u,v$\in$ V\\
\begin{tikzpicture}
[scale=0.6,auto=left,every node/.style={circle,fill=blue!10}]
\node (1) at (0,0) {u};
\node (2) at (2,-1) {c};
\node (3) at (3,0) {v};
\path (1) edge[bend right=-30] (2);
\path (3) edge[bend right=30] (2);
\foreach\from/\to in {}
\draw (\from) -- (\to);
\end{tikzpicture}\\
d(c,u)$\leq$ r\\
d(c,v)$\leq$ r\\
=> d(u,v)$\leq$ 2r=> d(G)= max d(u,v)$\leq$ 2r\\
\underline{Утверждение} В дереве $\leq$ 2 центров\\
\begin{tikzpicture}
[scale=0.6,auto=left,every node/.style={circle,fill=blue!10}]
\node (1) at (0,0) {};
\node (2) at (2,2) {};
\node (6) at (2,0) {};
\node (3) at (2,-2) {};
\node (4) at (2,-4) {};
\node (5) at (4,0) {};
\path (1) edge[] (-1,1);
\path (2) edge[] (1.5,2.5);
\path (2) edge[] (2.5,2.5);
\path (4) edge[] (1.5,-4.5);
\path (4) edge[] (2.5,-4.5);
\path (5) edge[] (5.5,1);
\path (5) edge[] (5.5,-1);
\path (1) edge[] (-1,-1);
\path (6) edge[] (3,-2);
\foreach\from/\to in {1/6,6/2,6/3,5/6,3/4}
\draw (\from) -- (\to);
\end{tikzpicture}\\
$\supset$ их 3: \\
\begin{tikzpicture}
[scale=0.6,auto=left,every node/.style={circle,fill=blue!10}]
\node (1) at (0,0) {с1};
\node (2) at (3,0) {с2};
\node (3) at (5,0) {с3};
\path (1) edge[dashed,bend right=10] (2);
\path (2) edge[dashed,bend right=-10] (3);
\path (1) edge[] (1,1);
\path (2) edge[] (2,-1);
\path (1,1) edge[] (2,-1);
\path (2) edge[blue] (4,1);
\path (3) edge[blue] (5,-1);
\path (4,1) edge[blue] (5,-1);
\foreach\from/\to in {}
\draw (\from) -- (\to);
\end{tikzpicture}\\
Построим пути между $C_1 \qquad C_2$\\
потом $C_2 \qquad C_3$( в дереве ровно 1 путь между вершинами)\\
\begin{tikzpicture}
[scale=0.6,auto=left,every node/.style={circle,fill=blue!10}]
\node (1) at (0,0) {с1};
\node (2) at (3,0) {};
\node (3) at (5,0) {с2};
\node (4) at (3,2) {с3};
\path (1) edge[] (2){};
\path (2) edge[] (4){};
\path (2) edge[] (4,-1){};
\path (3) edge[] (5,1){};
\path (4,-1) edge[] (5,1){};
\foreach\from/\to in {}
\draw (\from) -- (\to);
\end{tikzpicture}\\
$r(c_0)<r(c_1)=r(c_2)=r(c_3)=r(G)=r$\\
\underline{Замечание}\\
Будем далье иногда использовать ориентированиые графы G=(V,E)\\
(ребра в ориентированном графе иногда называют дугами)\\
E $\subset${(u,v)-упорядоченная пара}\\
\underline{Пример}\\
\begin{tikzpicture}
[scale=0.6,auto=left,every node/.style={circle,fill=blue!10}]
\node (1) at (0,0) {};
\node (2) at (2,0) {};
\node (3) at (2,2) {};
\node (4) at (0,2) {};
\node (5) at (4,0) {};
\node (6) at (4,2) {};
\node (8) at (5,2) {};
\node (7) at (5,0) {};
\path (1) edge[->,bend left=20] (2);
\path (1) edge[<-,bend left=-20] (2);
\path (5) edge[->] (6);
\path (7) edge[<-] (8);
\path (1) edge[->] (3);
\path (1) edge[->] (4);
\path (4) edge[->] (3);
\path (2) edge[->] (3);
\foreach\from/\to in {}
\draw (\from) -> (\to);
\end{tikzpicture}\\
\underline{Замечание} У рёбер будут весы\\
G=(V,E) вес- это f:E $\rightarrow$ R\\
то есть у числа каждому ребру\\
\begin{tikzpicture}
%\qbezier(22,2)(120,20)(20,77);
\node [style={circle,fill=blue!10}](1) at (0.5,1){};
\node [style={circle,fill=blue!10}](3) at (3.5,1){};
\node [style={circle,fill=blue!10}](2) at (2,4) {};
\draw (0.5,1) -- node[left] {$2$} (2,4) -- node[right]{$5$} (3.5,1) -- node[below] {$1$} cycle;
\end{tikzpicture}\\
 \begin{tikzpicture}
%\qbezier(22,2)(120,20)(20,77);
\node (1) [style={circle,fill=blue!10}] at (0.5,1){};
\node (2) [style={circle,fill=blue!10}] at (2,4) {};
\node (3) [style={circle,fill=blue!10}] at (3.5,1){};
\path (2)  edge[->]node[right] {$2$} (3);
\path (1)  edge[->]node[left] {$1$} (2);
\path (1)  edge[->]node[below] {$3$} (3);
\path (3)  edge[->,bend left=30]node[below] {$4$} (1);
\end{tikzpicture}\\
Расстояние на графе с весами считается как \\
min $\sum$ весов по всем путям\\
\begin{tikzpicture}
\node (1) [style={circle,fill=blue!10}] at (0,1) {A};
\node (2) [style={circle,fill=blue!10}] at (2,0) {E};
\node (3) [style={circle,fill=blue!10}] at (2,2){C};
\node (4) [style={circle,fill=blue!10}] at (4,0) {F};
\node (5) [style={circle,fill=blue!10}] at (4,2) {D};
\node (6) [style={circle,fill=blue!10}] at (6,1){B};
\path (2)  edge[-]node[left] {$3$} (5);
\draw (1) --node[left]{$1$} (3)--node[above]{$4$} (5)--node[right]{$2$} (6)--node[right]{$5$} (4)--node[below]{$1$} (2)--node[left]{$2$} (1) circle;
\end{tikzpicture}\\
d(A,B)=?\\
d(ACDB)=1+4+2=7\\
d(ACDEFB)=1+4+3+1+5=14\\
d(AEFB)=2+1+5=8\\
d(AEDB)=2+3+2=7\\
min=7\\
=> d(A,B)=7\\
\underline{Замечание} расстояние во взвешенном графе не всегда существует\\
\begin{tikzpicture}
\node (1) [style={circle,fill=blue!10}] at (0,3) {A};
\node (2) [style={circle,fill=blue!10}] at (2,0) {F};
\node (3) [style={circle,fill=blue!10}] at (2,2) {D};
\node (4) [style={circle,fill=blue!10}] at (3,4) {C};
\node (5) [style={circle,fill=blue!10}] at (4,0) {G};
\node (6) [style={circle,fill=blue!10}] at (4,2) {E};
\node (7) [style={circle,fill=blue!10}] at (6,3) {B};
\path (1)  edge[->]node[left] {$3$} (2);
\path (1)  edge[->]node[above] {$1$} (3);
\path (2)  edge[->]node[below] {$4$} (5);
\path (5)  edge[->]node[right] {$7$} (7);
\path (3)  edge[->]node[left] {$-5$} (4);
\path (3)  edge[<-]node[below] {$2$} (6);
\path (4)  edge[->]node[right] {$2$} (6);
\path (6)  edge[->]node[above] {$1$} (7);
\end{tikzpicture}\\
d(F,G)=4\\
d(G,F)=+$\infty$\\
d(A,B)=?\\
d(ADCEB)=1-5+2+1=-1\\
d(ADCEDCEB)=1-5+2+2-5+2+1=-2\\
и т.д. min=-$\infty$\\
\underline{Утверждение} В графе есть все расстояния <=> в графе нет\\
цикла отрицательной длины\\
\underline{Докозательство}\\
Если есть цикл <0 => $\forall$ две вершины
\begin{tikzpicture}
\node (1) [style={circle,fill=blue!10}] at (0,0) {};
\node (2) [style={circle,fill=blue!10}] at (0.5,1) {};
\node (3) [style={circle,fill=blue!10}] at (1,0) {};
\path (1) edge[->,bend left=40] (2);\path (2) edge[->,bend left=40] (3);
\path (3) edge[->,bend left=40] (1);
\end{tikzpicture}этого цикла не имеют\\
расстояние( или $\rightarrow \infty$\\
Если нет расстояния, то есть для u,v есть пути сколь угодно маленикие\\
$\supset$ есть путь длинее n=|v| ребер => повтор ведущих в пути\\
\begin{tikzpicture}
\node (1) [style={circle,fill=blue!10}] at (0,1) {u};
\node (2) at (1.5,1) {};
\node (3) [style={circle,fill=blue!10}] at (3,1) {v};
\path (1) edge[bend left=20] (2);
\draw (1.5,0.75) circle  (0.25);
\path (2) edge[bend left=20] (3);
\end{tikzpicture}- это и будет отрицтальный цикл\\
Как хранить графы в компьютере (представление графа в компьютере)\\
1. Матрица смежности: таблица ведущих вершин\\
a(i,j)=$
\left\{
\begin{array}{rcll}
0 \textrm{если нет ребра}\\
1 \textrm{если есть ребро}\\
\end{array}
\right.
$\\
\begin{tabular}{l|l l l l|}
 & 1 & 2& 3 & 4\\
\hline
1 & 0 & 1 & 1 & 1\\
2 & 1 & 0& 1&0 \\
3& 1&1&0&0\\
3& 1&0&0&0\\
\hline
\end{tabular}\\
\begin{tikzpicture}
[scale=0.5,auto=left,every node/.style={circle,fill=blue!10}]
\node (1) at (0.5,1){1};
\node (3) at (3.5,1){3};
\node (2) at (2,4) {2};
\node (4) at (1.5,-1) {4};
\foreach\from/\to in {1/2,1/3,1/4,3/2}
\draw (\from) -- (\to);
\end{tikzpicture}\\
-симметичная для неориентированного графа\\
Для графов с весами a(i,j)=вес ребра ij или +$\infty$, если нет\\
\begin{tikzpicture}
\node (1) [style={circle,fill=blue!10}] at (0,0) {1};
\node (2) [style={circle,fill=blue!10}] at (1.5,1.5) {2};
\node (3) [style={circle,fill=blue!10}] at (3,1) {3};
\node (4) [style={circle,fill=blue!10}] at (3,-1) {4};
\path (1) edge[->] node{$15$} (2);
\path (1) edge[<-,bend left=30] node{$10$} (2);
\path (2) edge[->] node{$21$} (3);
\path (3) edge[->] node{$14$} (4);
\path (4) edge[->] node{$42$} (2);
\end{tikzpicture}\\
\begin{tabular}{l|l l l l|}
 & 1 & 2& 3 & 4\\
\hline
1 & $+\infty$ & 15 & $+\infty$ & $+\infty$\\
2 & 10 & $+\infty$& 21&$+\infty$ \\
3& $+\infty$&$+\infty$&$+\infty$&14\\
3& $+\infty$&42&$+\infty$&$+\infty$\\
\hline
\end{tabular}\\
объем памяти $n^2=|V|^2$\\
2. Списки смежности\\
- для каждой вершины хранит список соседей\\
Пример 1: 2(15) 2:1(10),3(21)  3:4(14) 4:2(42)\\
\underline{Пример}\\
\begin{tikzpicture}
[scale=0.5,auto=left,every node/.style={circle,fill=blue!10}]
\node (1) at (0,2){1};
\node (2) at (2,2){2};
\node (3) at (2,0) {3};
\node (4) at (0,0) {4};
\foreach\from/\to in {1/2,1/3,1/4,3/2,3/4}
\draw (\from) -- (\to);
\end{tikzpicture}\\
1:234\\
2:13\\
3:24\\
4:13\\
Память $\approx$ |E| количество ребер\\
3.Неявные способы умеем вычислять всех соседей $\forall$ вершины\\
\underline{Пример}\\
Задача обход конем шахматной доски\\
\begin{tikzpicture}
\draw (0,0) rectangle  (3,3);
\node (3) at (1.25,1.875) {x};
\node (2) at (1.125,1.625) {x};
\node (1) at (1.5,1.5) {$\bullet$};
\node (5) at (1.75,1.875) {x};
\node (6) at (1.875,1.625) {x};
\node (7) at (1.75,1.125) {x};
\node (8) at (1.875,1.375) {x};
\node (9) at (1.25,1.125) {x};
\node (10) at (1.125,1.375) {x};
\end{tikzpicture}
граф: вершины=клета 64 шт\\
ребра- вершины перел ходом коня\\
Можно для $\forall$ клетки(вершины) посчитать, куда можно попасть\\
Задача обхода конем= гамильтонов цикл в этом графе\\
\underline{Задача}\\
дано две вершины u,v, найти d(u,v) и путь, на котором достигается это\\
расстояние\\
\underline{Замечание}\\
оказывается, что найти путь от u до v это то де самое, что иметь путь\\
от r до всех вершин.\\
\underline{Алгоритмы} Форда-Беллмана\\
Дано G=(V,E)\\
u$\in$ V, найти расстояния d(u,v) для $\forall v\in V$,\\
Будем писать d(v)=d(u,v) т.к. u не меняется\\
Будем хранить в массиве d текущие найденные расстояния. В начале\\
d(u)=0 d(v)=+$\infty$, если v$\neq$ u\\
Релаксация ребра e=($v_1 v_2$)\\
\begin{tikzpicture}
\node (1)[style={circle,fill=blue!10}] at(0,0){}; 
\node (2)[style={circle,fill=blue!10}] at(2,0) {};
\path (1)node[above]{$d(v_1)$};
\path (2)node[above]{$d(v_2)$};
\path (1)node[below]{$v_1$};
\path (2)node[below]{$v_2$};
\foreach\from/\to in {1/2}
\draw (\from) -- (\to);
\end{tikzpicture}\\
Если d($v_1$)+f($v_1v_2$)<d($v_2$)=> d($v_2$)=d($v_1$)+f($v_1v_2$)\\
Алгоритм: Повторить n-1 раз: перебрать все ребра e и каждое\\
релексировать\\
(в неориентированом графе  \begin{tikzpicture}
\node (1)[style={circle,fill=blue!10}] at(0,0){}; 
\node (2)[style={circle,fill=blue!10}] at(1,0) {};
\foreach\from/\to in {1/2}
\draw (\from) -- (\to);
\end{tikzpicture}
= ,\begin{tikzpicture}
\node (1)[style={circle,fill=blue!10}] at(0,0){}; 
\node (2)[style={circle,fill=blue!10}] at(1,0) {};
\path (1) edge[->,bend left=20] (2);
\path (1) edge[<-,bend left=-20] (2);
\end{tikzpicture}
то есть две релаксации\\ 
на ребро)\\
Пример\\
n=4(4 вершины)\\
\begin{tikzpicture}
\node (1)[style={circle,fill=blue!10}]
 at (0.5,1) {A};
\node (3)[style={circle,fill=blue!10}] at (3.5,1){C};
\node (2)[style={circle,fill=blue!10}] at (2,2) {B};
\node (4)[style={circle,fill=blue!10}] at (2,0) {D};
\path (1) edge[->] node[below]{$10$} (4);
\path (3) edge[->] node[below]{$3$} (4);
\path (1) edge[->] node[above]{$1$} (2);
\path (1) edge[->] node[below]{$5$} (3);
\path (2) edge[->] node[above]{$2$} (3);
\node () [scale=5] at (5,0) {$,\psi$} ;
\end{tikzpicture}\\
ребра: A:B(1)C(5)d(10) B:C(2) C:D(3) D:\\
Шаг 1\\
\begin{tabular}{l l l l l}
  & A&B&C&D\\
  &0&+$\infty$&+$\infty$&+$\infty$\\
AB& 0&1&+$\infty$&+$\infty$\\
AC& 0&0&5&+$\infty$ \\
AD& 0&1&5&10\\
AB& 0&1&3&10\\
CD& 0&1&3&6\\
\end{tabular}\\
Шаг 2\\
AD—//— C->D 3+3<10\\
AC—//—\\
AD—//—\\
AB—//—\\
CD—//—\\
Шаг 3\\
AD—//— \\
AC—//—\\
AD—//—\\
AB—//—\\
CD—//—\\
ответ: d(A)=0 d(B)=1 d(C)=3 d(D)=6\\
Время работы $\approx$ |V|*|E|$\leq|V|^3$.\\
\underline{Алгоритм Форда-Беллмана}\\
\begin{tikzpicture}
\node (1)[scale=0.8,style={circle,fill=blue!10}] at ( 0,0) {A};
\node (3)[scale=0.8,style={circle,fill=blue!10}] at ( 2,0) {C};
\node (2)[scale=0.8,style={circle,fill=blue!10}] at ( 1,1) {B};
\path (1)  edge[->]node[left] {$3$} (2);
\path (2)  edge[->]node[right] {$4$} (3);
\path (3)  edge[->, bend right=40]node[right] {$2$} (2);
\path (1)  edge[->]node[above] {$1$} (3);
\end{tikzpicture}\\
A:3B, 1C\\
B:4C\\
C:2B\\
Путь из А\\
сначала d: 
$
\left.
\begin{array}{rcll}
A&B&C\\
0&\infty&\infty\\
0&3&\infty\\
0&3&1\\
0&2&1\\
\end{array}
\right.
$\\
Релаксируем\\
A $\xrightarrow[]{3}$B\\
0 \quad $\infty$\\
0+3<$\infty$\\
A $\xrightarrow[]{1}$С\\
0+1<$\infty$\\
B $\xrightarrow[]{4}$С\\
3 \quad 1\\
3+4<1\\
C $\xrightarrow[]{3}$B\\
1+1<3\\
n=1=>n-1=2 раза цикл релаксирующий\\
AB,AC,BC,CB= нет улушений ABC\\
Корректность алгоритма\\
\underline{Теорема}\\
В конце массив d содержит расстояния от A\\
\underline{Докозательство}\\
После i-го цикла релаксации всех ребер, d хранит числа d(v)$\leq$ min длин путей, в которых $\leq$ i ребер\\
Дейститвтельно, i=0( База индукции)\\
min (по пути из 0 ребер) только A-A d(A)=0 d(u)=$\infty$\\
$\supset$ есть оптимальный путь из i+1\\
ребра\\
\begin{tikzpicture}
[scale=0.7,auto=left,every node/.style={circle,fill=blue!10}]
\node (1)[scale=0.6] at (0,0){A};
\node (2) at (2,0){};
\node (3) at (4,0) {};
\node (4)[scale=0.6] at (6,0) {G};
\node (5)[scale=0.6] at (8,0) {B};
\foreach\from/\to in {1/2,2/3,3/4,4/5}
\draw (\from) -- (\to);
\end{tikzpicture}\\
от A до B i+1 ребер\\
от A до G i ребер\\
По предположении d(C)= dB + (A,C)\\
длина пути $\underbrace{A-C-B}_{d(c)=dist(c)}$=dB +(C)+вес(CB)\\
проверка\\
d(C)+вес(СВ)$\leq$ d(B)\\
-верно, так как путь A-C-B оптимален => d(B)=d(C)+вес (CB)\\
\begin{tikzpicture}
\node (1)[scale=0.8,style={circle,fill=blue!10}] at ( 0,0) {A};
\node (3)[scale=0.8,style={circle,fill=blue!10}] at ( 2,0) {C};
\node (2)[scale=0.8,style={circle,fill=blue!10}] at ( 1,1) {B};
\path (1)  edge[->, bend left=20]node[left] {$3$} (2);
\path (3)  edge[->, bend right=20]node[right] {$1$} (2);
\path (1)  edge[->, bend right=20]node[below] {$7$} (3);
\end{tikzpicture}\\
Почем путь n-1 этап?\\
отрицательный путь не содержит цикл\\
\begin{tikzpicture}
\node (1) [style={circle,fill=blue!10}] at (0,1) {A};
\node (2) [scale=0.5,style={circle,fill=blue!10}] at (1.5,1) {};
\node (3) [style={circle,fill=blue!10}] at (3,1) {U};
\path (1) edge[bend left=20] (2);
\draw (1.5,0.75) circle  (0.25);
\path (2) edge[bend left=20] (3);
\end{tikzpicture}=> -//- $\in$ n-1 ребро\\
\underline{Замечания}\\
Мы вычисляем только расстояния, но путь неизвестен\\
Как восстановить путь?\\
Будем сохранять информацию об успешных релаксациях\\
prev-массив вершин\\
Если релаксация u v успешна, то Prev[v]=u оптимальный путь в v лежит через u\\
\begin{tikzpicture}
\node (1)[scale=0.9, style={circle,fill=blue!10}] at (0.5,1){A};
\node (3)[scale=0.9, style={circle,fill=blue!10}] at (3.5,0){C};
\node (2)[scale=0.9, style={circle,fill=blue!10}] at (3.5,3) {B};
\path (1)  edge[->]node[below] {$3$} (3);
\path (1)  edge[->]node[below] {$7$} (2);
\path (3)  edge[->, bend right=20]node[right] {$1$} (2);
\end{tikzpicture} 
\begin{tabular}{l|l l l}
 d& A &B&C\\
\hline
AB& 0 & $\infty$ &$\infty$ \\
AC& 0 & 3/A& 1/A \\
CB& 0 & 2/C&1/A\\
\end{tabular}\\
восстановить путь в B
A$\rightarrow$C$\rightarrow$B\\
A=prev(C) C=prev(B)\\
\underline{Алгоритм Дейкстры}\\
В отличие от ФБ требует, чтобы веса w(e)$\geq$0\\
Алгоритм\\
Дан граф G=(V,E) A$\in$ V\\
Найти расстояния до всех вершин d(u)=dist(A,u)\\
Алгоритм p=0 d(A)=0 d(u$\neq$A)=$\infty$- обаботанные вершины\\
for n раз (n=v)\\
\quadвыбрать u$\in V$ / p,  где d(u) $\rightarrow$ min (из необработ. min d)\\
for e$\in$ ребра из U, e=(u,v)\\
рефлексируем ребро е\\
p=p $\cup$ u\\
Пример\\
\begin{tikzpicture}
\node (1)[scale=0.9, style={circle,fill=blue!10}] at (0.5,1){A};
\node (3)[scale=0.9, style={circle,fill=blue!10}] at (2,0){C};
\node (2)[scale=0.9, style={circle,fill=blue!10}] at (2,2) {B};
\node (4)[scale=0.9, style={circle,fill=blue!10}] at (4,1) {D};
\node (5)[scale=0.9, style={circle,fill=blue!10}] at (5,3) {E};
\path (1)  edge[->]node[below] {$3$} (3);
\path (1)  edge[->]node[below] {$1$} (2);
\path (3)  edge[->]node[below] {$4$} (4);
\path (2)  edge[->]node[below] {$8$} (4);
\path (4)  edge[->]node[below] {$2$} (5);
\end{tikzpicture}\\
\begin{tabular}{l l l l l l}
 d& A &B&C&D&E\\
\hline
& 0 & $\infty$ &$\infty$&$\infty$ &$\infty$ \\
&  & 1& 3&$\infty$ &$\infty$ \\
&  & &3&9&8\\
&  & & &7&$\infty$\\
&  & & & &9\\
\end{tabular}\\
U=A\\
A $\xrightarrow[1]{}$B\\
0 \quad $\infty$ \\
A $\xrightarrow[3]{}$C\\
0 \quad $\infty$ \\
U=B\\
B $\xrightarrow[8]{}$D\\
1 \quad $\infty$ \\
U=C\\
C $\xrightarrow[4]{}$D\\
3 \quad 9 \\
4=D\\
Эффективность |V|x|E|xlog|V|\\
log|V|- выбор мин\\
Корректность\\
идея- на каждом шаге d(u)=min путей\\
База шаг=0 d(A)=0 d(U)=$\infty$\\
переход\\
\begin{tikzpicture}
\node (1) at (0.5,1){A};
\draw (0,0) circle (1.5);
\end{tikzpicture} P-обр\\
Выбрали u=min вершин из V/\{ p\}\\
$\supset$ есть оптимальный путь в u $\underbrace{A---\overline{u}}_{\textrm{обр}}$---u\\
dist($\overline{u}$)=dist(u)-x\\
По предложению dist($\overline{u}$)=d($\overline{u}$)\\
d(u)>d($\overline{u}$)=>?? d(u) был min\\
\begin{tikzpicture}
\node (1)[scale=0.9, style={circle,fill=blue!10}] at (0.5,1){P};
\node (2)[scale=0.9, style={circle,fill=blue!10}] at (2,1){U};
\node (3)[scale=0.9, style={circle,fill=blue!10}] at (3,0) {V};
\path (1)  edge[->] (2);
\path (2)  edge[->] (3);
\path (1)  edge[->,bend right] (3);
\end{tikzpicture}\\
путь через P\\
$\supset$ оптимальный путь в v идет через u\\
dist(A,U)+w(u,v)=dist(A,v)\\
=> релаксация u $\rightarrow$v успешна и d(v) получит оптимальное расстояние\\
для восстановления пути нужен аналогичный prev\\
усиленная релаксация u v prev[v]=u\\
\begin{tikzpicture}
\node (1)[scale=0.9, style={circle,fill=blue!10}] at (0.5,1){A};
\node (3)[scale=0.9, style={circle,fill=blue!10}] at (2,0){C};
\node (2)[scale=0.9, style={circle,fill=blue!10}] at (2,2) {B};
\node (4)[scale=0.9, style={circle,fill=blue!10}] at (4,1) {D};
\node (5)[scale=0.9, style={circle,fill=blue!10}] at (5,3) {E};
\path (1)  edge[->]node[below] {$3$} (3);
\path (1)  edge[->]node[below] {$1$} (2);
\path (3)  edge[->]node[below] {$4$} (4);
\path (2)  edge[->]node[below] {$8$} (4);
\path (4)  edge[->]node[below] {$2$} (5);
\end{tikzpicture}\\
\begin{tabular}{l l l l l}
  A &B&C&D&E\\
\hline
0 & $\infty$ &$\infty$&$\infty$ &$\infty$ \\
& 1/A& 3/A&$\infty$ &$\infty$ \\
& &3/A&9/D&8\\
& & &7/C&$\infty$\\
& & & &9/D\\
\end{tabular}\\
A$\rightarrow$C$\rightarrow$D$\rightarrow$E\\
\underline{Утверждение}\\
В дереве 1$\leq$ центров $\leq$2\\
\underline{Напомним}\\
\begin{tikzpicture}
[scale=0.7,auto=left,every node/.style={circle,fill=blue!10}]
\node (1)[scale=0.9] at ( 2,0) {A};
\node (2) at (2,2){};
\node (3) at (0,1){};
\node (4) at (4,3){};
\node (5)[scale=0.9] at (5,2){B};
\node (6) at (5,3) {};
\foreach\from/\to in {1/2,1/3,2/3,2/4,4/5,4/6}
\draw (\from) -- (\to);
\end{tikzpicture}\\
Максимальное расстояние -3\\
\begin{tikzpicture}
[scale=0.5,auto=left,every node/.style={circle,fill=blue!10}]
\node (1) at (0,0) {4};
\node (2) at (0,2){4};
\node (3) at (2,1){3};
\node (4) at (4,1){2};
\node (5) at (6,1){3};
\node (6) at (8,0) {4};
\node (7) at (8,2) {4};
\foreach\from/\to in {1/3,2/3,3/4,4/5,5/6,5/7}
\draw (\from) -- (\to);
\end{tikzpicture}\\
Докозательство\\
Если убрать у дерева все висячие вершины, расстояние изменяется на 1\\
\begin{tikzpicture}
[scale=0.5,auto=left,every node/.style={circle,fill=blue!10}]
\node (3) at (2,1){2};
\node (4) at (4,1){1};
\node (5) at (6,1){2};
\foreach\from/\to in {3/4,4/5}
\draw (\from) -- (\to);
\end{tikzpicture}\\
Если повторять убирания висячих\\
\begin{tikzpicture}
[scale=0.5,auto=left,every node/.style={rectangle,fill=blue!10}]
\node (3) at (2,1){};
\end{tikzpicture}
 или \begin{tikzpicture}
[scale=0.5,auto=left,every node/.style={rectangle,fill=blue!10}]
\node (3) at (2,1){};
\node (4) at (4,1){};
\foreach\from/\to in {3/4}
\draw (\from) -- (\to);
\end{tikzpicture} центры\\
\underline{Алгоритм Флойда}\\
Дан граф G=(V,E)\\
Вернуть d(u,v)\\
\begin{tabular}{l|l l l}
 u$\setminus$ c& &v&\\
\hline
&  &&\\
u&  &d(u;v) &  \\
&  & &\\
\end{tabular}\\
\underline{Алгоритм}\\
составляем матрицу $d_0$=\\
$d_0$:$d_0$(u,v)=0\\
$d_0$(u,v)=$
\left.
\begin{array}{l}
\infty \textrm{, если чет ребра u-v }\\
\omega(u,v)\textrm{, если есть ребро u-v }\\
\end{array}
\right.
$\\
\begin{tikzpicture}
\node (1)[scale=0.8,style={circle,fill=blue!10}] at ( 0,0) {A};
\node (3)[scale=0.8,style={circle,fill=blue!10}] at ( 2,0) {B};
\node (2)[scale=0.8,style={circle,fill=blue!10}] at ( 1,1) {C};
\path (1)  edge[->, bend left=20]node[left] {$2$} (2);
\path (3)  edge[->, bend right=20]node[right] {$1$} (2);
\path (1)  edge[->, bend right=20]node[below] {$5$} (3);
\path (3)  edge[->, bend right=20]node[above] {$3$} (1);
\end{tikzpicture}\\
$d_0$\\
\begin{tabular}{l|l l l}
& A &B&C\\
\hline
A& 0 &5 &2 \\
B& 3 & 0& $\infty$ \\
C& $\infty$ & 1&0\\
\end{tabular}\\
for$k\in$ V\\
for $v\in V$\\
if d(u,v)> d(u,k)+d(k,v)\\
=> d(u,v)=-//-\\
\underline{Пример}\\
k=A\\
\begin{tabular}{l|l l l}
& A &B&C\\
\hline
A& 0 &5 &2 \\
B& 3 & 0& 5 \\
C& $\infty$ & 1&0\\
\end{tabular}\\
BC>BA+AB\\
\begin{tabular}{l|l l l}
& A &B&C\\
\hline
A& 0 &5 &2 \\
B& 3 & 0& 5 \\
C& 4& 1&0\\
\end{tabular}\\
CA>CB+BC\\
\begin{tabular}{l|l l l}
& A &B&C\\
\hline
A& 0 & 3 & 2\\
B& 3 & 0 & 5\\
C& 4 & 1 & 0\\
\end{tabular}\\
Корректность\\
\underline{Утверждается}\\
после шага k в d(u,v) min d(пути) u- от 1 до k -v\\
База k=0\\
d(u,v)=u- нет -v\\
действительно, в начале d содержит длины ребер, переход u- содержит от 1 до k+1 -v\\
$\supset$ есть оптимальный путь из u- k+1 -v $\rightarrow$\\
1)$\rightarrow$ в нем нет k+1 => его длина d(u,v)\\
2)$\rightarrow$ есть k+1 u- O-k+1-O -v\\
его длина d(u,k+1) +d(k+1,v)\\
это ровно проверка цикла, меньший вариант ззанимается в d, в конце\\
d(u,v)= min( u $\forall$ вершины v)= dist(u,v)\\
\underline{Замечание}\\
Чтобы восстанавливать путь, можно ввести масcив through\\
if d(u,v)>d(u,k)+d(k,v) => d(u,v)= -//- through (u,v)=k\\
Для восстановления пути\\
1) A... th(A,i)...I...th(I,B)...B\\
2)и так далее, если  th(x,y) нет, запишем => реберо xy это оптимальный путь\\
Замечание\\
В алгоритме Флойда идет транзитивное замыкание бинарных отношений\\
$\supset$ R-бинарное отношение на M\\
$\supset$ $\overline{R}$- транзитивное замыкание R, если\\
1) $\overline{R}$ $\supset$ R\\
2)  $\overline{R}$-транзитивно\\
3)$\forall \quad \overline{R} \qquad \overline{R} \supset \overline{\overline{R}}\supset R$ не транзитивно\\
\underline{Пример}\\
\begin{tikzpicture}
\node (1)[scale=0.8,style={circle,fill=blue!10}] at ( 0,0) {B};
\node (2)[scale=0.8,style={circle,fill=blue!10}] at ( 1,1) {D};
\node (3)[scale=0.8,style={circle,fill=blue!10}] at ( 2,0) {C};
\node (4)[scale=0.8,style={circle,fill=blue!10}] at ( 1,-1) {A};
\path (1)  edge[->, bend left=20](2);
\path (4)  edge[->, bend right=-20](1);
\path (4)  edge[->, bend right=20](3);
\path (3)  edge[->, bend right=20](4);
\end{tikzpicture}\\
не транзитивно\\
aRb, bRd, но \sout{aRd}\\
aRc, cRa, но \sout{aRd}\\
делаем транзитивным\\
\begin{tikzpicture}
\node (1)[scale=0.8,style={circle,fill=blue!10}] at ( 0,0) {B};
\node (2)[scale=0.8,style={circle,fill=blue!10}] at ( 1,1) {D};
\node (3)[scale=0.8,style={circle,fill=blue!10}] at ( 2,0) {C};
\node (4)[scale=0.8,style={circle,fill=blue!10}] at ( 1,-1) {A};
\path (1)  edge[->, bend left=20](2);
\path (4)  edge[->, bend right=-20](1);
\path (4)  edge[->, bend right=20](3);
\path (3)  edge[->, bend right=20](4);
\path (4)  edge [loop below] (4);
\path (3)  edge [loop right] (3);
\path (3) edge[->](1);
\path (3) edge[->](2);
\path (4) edge[->](2);
\end{tikzpicture}\\
$\supset$ R-бинарно отношение $\supset$ G=(M,R) граф отношения,\\
тогда $\overline{R}$- это $x\overline{R}y$ <=> есть путь x-y в G\\
\underline{Докозательство}\\
1)$\overline{R} \supset$R так как xRy=> есть путь из 1 ребра => $x\overline{R}y$\\
x-y 2)$\overline{R}$- транзитивно, т.к. x$\overline{R}$y, y$\overline{R}$z => x$\overline{R}$z\\
3) $\supset \quad \overline{\overline{R}}\supset R$, $\overline{\overline{R}}$- транзитивно\\
$\supset$ есть путь x в y $\rightarrow$ $x_1$ $\rightarrow$ $x_2$ $\rightarrow$ $x_3$ $\rightarrow$ y\\
xR$x_1$ => $x_1 \overline{\overline{R}} x_2$=> x$\overline{\overline{R}} x_2$\\
$x_2 \overline{\overline{R}} x_3$=> x$\overline{\overline{R}} x_3$=>...=> x$\overline{\overline{R}}$y=>  $\overline{\overline{R}} \supset \overline{R} \supset R$\\
Применим алгоритм Флойда к графу G=(M,R)\\
$d_0 (x,y)=1$, если xRy\\
$d_0 (x,y)=\infty$, если \sout{xRy}\\
После конца алгоритма\\
\begin{tabular}{l|l l l}
 & &y&\\
\hline
&  &&\\
x&  &$\infty$ &  \\
&7  & &\\
\end{tabular}\\
замыкание x$\overline{R}$y, если d(x,y)<$\infty$\\
Алгоритм транзитивного замыкания\\
$\overline{R}$=R\\
for k$\in$ M\\
for x$\in$ M\\
for y $\in$ M\\
if xRk \& kRy\\
$\overline{\overline{R}} \in (x,y)$, то есть сделать x$\overline{R}$y\\
\underline{Потоки в сетях}\\
\underline{Определение}\\
Сеть- граф G=(V,E)\\ ориентированный\\
$s \in V$ $\not \exists$e=(4,s) ничего не входит\\
$t \in V$ $\not \exists$e=(t,u) ничего не выходит\\
\begin{tikzpicture}
[scale=0.7,auto=left,every node/.style={circle,fill=blue!10}]
\node (1) at ( 0,1) {S};
\node (2) at ( 2,2) {};
\node (3)at ( 2,1) {};
\node (4) at ( 2,0) {};
\node (5) at ( 4,2) {};
\node (6)at ( 4,1) {};
\node (7) at ( 4,0) {};
\node (8) at ( 6,1) {T};
\path (1) edge[->] (2); \path (1) edge[->] (3);\path (1) edge[->] (4);
\path (2) edge[->] (5);\path (2) edge[->] (6);\path (6) edge[->] (4);\path (7) edge[->] (4); \path (3) edge[->] (7);\path (5) edge[->] (8);\path (6) edge[->] (8); \path (7) edge[->] (8);
\end{tikzpicture}\\
E$\rightarrow$ N пропускные способности ребер целые>0\\
\begin{tikzpicture}

\node (1)[scale=0.7,style={circle,fill=blue!10}] at ( 0,2) {S};
\node (2)[scale=0.7,style={circle,fill=blue!10}]  at ( 1,4) {a};
\node (3)[scale=0.7,style={circle,fill=blue!10}] at  ( 1,1) {b};
\node (4)[scale=0.7,style={circle,fill=blue!10}]  at ( 2,1) {c};
\node (5)[scale=0.7,style={circle,fill=blue!10}]  at ( 3,2) {T};
\path (1)  edge[->]node[left] {$5$} (2);
\path (1)  edge[->]node[below] {$1$} (3);
\path (3)  edge[->]node[left] {$7$} (2);
\path (4)  edge[->]node[below] {$5$} (3);
\path (4)  edge[->]node[below] {$2$} (5);
\path (2)  edge[->]node[right] {$2$} (5);
\path (2)  edge[->]node[right] {$6$} (4);
\end{tikzpicture}\\
\underline{Определение}\\
Поток f в сеть G-это\\
f: E$\rightarrow \mathbf{R}$ 1) 0$\leq$f(e)$\leq$ c(e)\\
2) $\forall  u \neq S,t$\\
$\sum \limits_{l=(v,u)\in E}$ f(e)\\
пример\\
\begin{tikzpicture}
\node (1)[scale=0.7,style={circle,fill=blue!10}] at ( 0,2) {S};
\node (2)[scale=0.7,style={circle,fill=blue!10}]  at ( 1,4) {a};
\node (3)[scale=0.7,style={circle,fill=blue!10}] at  ( 1,1) {b};
\node (4)[scale=0.7,style={circle,fill=blue!10}]  at ( 2,1) {c};
\node (5)[scale=0.7,style={circle,fill=blue!10}]  at ( 3,2) {T};
\path (1)  edge[->]node[left] {$5$} (2);
\path (1)  edge[->]node[below] {$1$} (3);
\path (3)  edge[->]node[left] {$7$} (2);
\path (4)  edge[->]node[below] {$5$} (3);
\path (4)  edge[->]node[below] {$2$} (5);
\path (2)  edge[->]node[right] {$2$} (5);
\path (2)  edge[->]node[right] {$6$} (4);
\end{tikzpicture}\\
Поток в сетях\\
Пример\\
\begin{tikzpicture}
\node (s)[scale=0.9,style={circle,fill=blue!10}][scale=0.9] at ( 0,2) {s};
\node (1) at ( 0,1.5) {исток};
\node (a)[scale=0.9,style={circle,fill=blue!10}][scale=0.9] at ( 2,3) {a};
\node (e)[scale=0.9,style={circle,fill=blue!10}][scale=0.9] at ( 2,1) {e};
\node (b)[scale=0.9,style={circle,fill=blue!10}][scale=0.9] at ( 4,4) {b};
\node (c)[scale=0.9,style={circle,fill=blue!10}][scale=0.9] at ( 4,2) {c};
\node (f)[scale=0.9,style={circle,fill=blue!10}][scale=0.9] at ( 4,0) {f};
\node (d)[scale=0.9,style={circle,fill=blue!10}][scale=0.9] at ( 6,3) {d};
\node (g)[scale=0.9,style={circle,fill=blue!10}][scale=0.9] at ( 6,1) {g};
\node (t)[scale=0.9,style={circle,fill=blue!10}][scale=0.9] at ( 8,2) {t};
\node (2) at ( 8,1.5) {сток};
\path (s) edge[->] node[above]{3} (a);
\path (s) edge [->] node[below, red] {1} (a);
\path (s) edge [->] node[above]{8} (e); 
\path (s) edge [->] node[below, red]{3} (e);
\path (a) edge [->] node[above]{7} (b); 
\path (a) edge [->] node[below, red]{1} (b);
\path (a) edge [->] node[above]{8} (c); 
\path (a) edge [->] node[below, red]{0} (c);
\path (e) edge [->] node[above]{5} (c); 
\path (e) edge [->] node[below, red]{3} (c);
\path (e) edge [->] node[above]{6} (f); 
\path (e) edge [->] node[below, red]{0} (f);
\path (c) edge [->, bend left] node[above]{1} (d); 
\path (c) edge [->, bend left] node[below, red]{1} (d);
\path (d) edge [->, bend left] node[above]{4} (c); 
\path (d) edge [->, bend left] node[below, red]{0} (c);
\path (c) edge [->] node[right]{2} (b); 
\path (c) edge [->] node[left, red]{0} (b);
\path (b) edge [->] node[above]{4} (d); 
\path (b) edge [->] node[below, red]{1} (d);
\path (c) edge [->] node[above]{2} (g); 
\path (c) edge [->] node[below, red]{2} (g);
\path (f) edge [->] node[above]{3} (g); 
\path (f) edge [->] node[below, red]{0} (g);
\path (g) edge [->, bend right] node[left]{6} (d); 
\path (g) edge [->, bend right] node[right, red]{0} (d);
\path (g) edge [->] node[above]{6} (t); 
\path (g) edge [->] node[below, red]{2} (t);
\path (d) edge [->] node[above]{5} (t); 
\path (d) edge [->] node[below, red]{2} (t);
\end{tikzpicture}\\
\textcolor{red}{$\leftarrow$ поток, красные f}\\
\textcolor{black}{$\leftarrow$ поток, черные c}\\
\begin{tikzpicture}
\node (1)[scale=0.9,style={circle,fill=red!30}][scale=0.9] at (1,1) {};
\path (0,0)  [red] edge[->]node[red,above]{1} (1);
\path (0,1)  [red] edge[->]node[red,above]{2} (1);
\path (0,2) [red]edge[->]node[red,above]{3} (1);
\path (1)  [red] edge[->]node[red,above]{6}(2,0);
\path (1)  [red] edge[->]node[red,above]{1}(2,1);
\path (1)  [red]edge[->]node[red,above]{6}(2,2);
\end{tikzpicture}\\
\textcolor{red}{0$\leq$f(e)$\leq$c(e)}\\
Пример\\
\begin{tikzpicture}
\node (s)[scale=0.9,style={circle,fill=blue!10}][scale=0.9] at ( 0,2) {s};
\node (a)[scale=0.9,style={circle,fill=blue!10}][scale=0.9] at ( 2,3) {a};
\node (e)[scale=0.9,style={circle,fill=blue!10}][scale=0.9] at ( 2,1) {e};
\node (b)[scale=0.9,style={circle,fill=blue!10}][scale=0.9] at ( 4,4) {b};
\node (t)[scale=0.9,style={circle,fill=blue!10}][scale=0.9] at ( 5,2) {t};
\path (s) edge [->] node[above]{1} (a); 
\path (s) edge [->] node[below, red]{1} (a);
\path (s) edge [->] node[above]{1} (e); 
\path (s) edge [->] node[below, red]{1} (e);
\path (e) edge [->,bend left] node[left]{$\frac{100}{\textcolor{red}{100}}$} (a); 
\path (a) edge [->,bend left] node[right]{$\frac{100}{\textcolor{red}{100}}$} (e); 
\path (a) edge [->] node[above]{1} (t); 
\path (a) edge [->] node[below, red]{1} (t);
\path (e) edge [->] node[above]{1} (t); 
\path (e) edge [->] node[below, red]{1} (t);
\end{tikzpicture}\\
-поток корректный\\
величина потока в примере 1: 4, в примере 2: 2\\
\underline{Теорема}\\
Дана сеть (G=(V,E),c),поток f на G\\
Тогда\\
$\sum \limits_{u:e=(s,u)} f(e)= \sum \limits_{u:e=(u,t)} f(e)$\\
Рассмтрим\\
$\sum \limits_{\in E} f(e)= 1)\sum \limits_{v\in V} f(e) \quad \sum \limits_{e:e=(u,v)} f(e) =$\\
$\sum \limits_{\underbrace{e:e=(u,s)}_{0}} f(e) +\sum \limits_{e:e=(u,t)} f(e) + \sum \limits_{v\in V\{s,\}}  \sum \limits_{ e:e=(u,v)} f(e)$=\\
вытекает +$ \sum \limits_{v\in V\{s,t\}}  \sum \limits_{ e:e=(v,u)} f(e)$=\\
вытекает +$ \sum \limits_{v\in V(s,t)}  \sum \limits_{ e:e=(v,u)} f(e)-  \sum \limits_{ e:e=(s,u)} f(e)-  \sum \limits_{ e:e=(t,u)} f(e)=$\\ 
вытекает +$\sum \limits_{v\in V\{s,\}}  \sum \limits_{ e:e=(u,v)} f(e)$-втекает- O=\\
вытекает-втекает+$\sum \limits_{ e\in E} f(e)=$\\
$\Rightarrow$ вытекает- втекает $\Rightarrow$ вытекает=втекает\\
$\sum \limits_{ e:e=(u,t)} f(e)$ = $\sum \limits_{ e:e=(s,u)} f(e)$\\
\underline{Определение}\\
w(f)- эта величина называется вершиной потока\\
\underline{Определение}\\
Разрез в сети (G=(V,E),c) Разрез G=($V_1,V_2)$\\
$s\in V_1 \quad t\in V_2\quad V_1\cap V_2=\not{0} \quad V_1 \cup V_2=V$ \\
\underline{Пример}\\
\begin{tikzpicture}
\node (s)[scale=0.9,style={circle,fill=blue!10}][scale=0.9] at ( 0,2) {s};
\node (a)[scale=0.9,style={circle,fill=blue!10}][scale=0.9] at ( 2,3) {};
\node (e)[scale=0.9,style={circle,fill=blue!10}][scale=0.9] at ( 2,1) {};
\node (b)[scale=0.9,style={circle,fill=blue!10}][scale=0.9] at ( 4,1) {};
\node (d)[scale=0.9,style={circle,fill=blue!10}][scale=0.9] at ( 4,3) {};
\node (t)[scale=0.9,style={circle,fill=blue!10}][scale=0.9] at ( 6,2) {t};
\node at (0,0) {$V_1$};
\node at (6,0) {$V_2$};
\draw (1,2) circle (2);
\draw (5,2) circle (2);
\path (s) edge [->]  (a); 
\path (s) edge [->]  (e); 
\path (a) edge [->]  (b);
\path (a) edge [->]  (d);
\path (e) edge [->]  (b);
\path (e) edge [->]  (d);
\path (b) edge [->]  (t);
\path (d) edge [->]  (t);
\end{tikzpicture}\\
или\\
\begin{tikzpicture}
\node (s)[scale=0.9,style={circle,fill=blue!10}][scale=0.9] at ( 0,2) {s};
\node (a)[scale=0.9,style={circle,fill=blue!10}][scale=0.9] at ( 2,3) {};
\node (e)[scale=0.9,style={circle,fill=blue!10}][scale=0.9] at ( 2,1) {};
\node (b)[scale=0.9,style={circle,fill=blue!10}][scale=0.9] at ( 4,1) {};
\node (d)[scale=0.9,style={circle,fill=blue!10}][scale=0.9] at ( 4,3) {};
\node (t)[scale=0.9,style={circle,fill=blue!10}][scale=0.9] at ( 6,2) {t};
\node at (0,0) {$V_1$};
\node at (6,0) {$V_2$};
%and (4,4)and (5,0) ..(-1,0);
\path (-1,2) edge[-,bend left](2,4);
\path (2,4) edge[-,bend left](4.5,1);
\path (4.5,1) edge[-,bend left](-1,2);
\path (7,1) [red] edge[-,bend left](3,4);
\path (7,1) [red] edge[-,bend right](3,4);
\path (s) edge [->]  (a); 
\path (s) edge [->]  (e); 
\path (a) edge [->]  (b);
\path (a) edge [->]  (d);
\path (e) edge [->]  (b);
\path (e) edge [->]  (d);
\path (b) edge [->]  (t);
\path (d) edge [->]  (t);
\end{tikzpicture}\\
\underline{Определение}\\
$E_c$-ребра разреза это все рёбра, которые идут из $V_1$ в $V_2$ или наоборот\\
$E^+_c$-прямые ребра разреха (из $V_1$ в $V_2$)\\
$E^+_c$-обратные ребра разреха (из $V_2$ в $V_1$)\\
\underline{Пример}\\
\begin{tikzpicture}
\node (s)[scale=0.9,style={circle,fill=blue!20}][scale=0.9] at ( 0,2) {};
\node (a)[scale=0.9,style={circle,fill=blue!20}][scale=0.9] at ( 2,3) {};
\node (e)[scale=0.9,style={circle,fill=blue!20}][scale=0.9] at ( 2,1) {};
\node (b)[scale=0.9,style={circle,fill=blue!20}][scale=0.9] at ( 4,1) {};
\node (d)[scale=0.9,style={circle,fill=blue!20}][scale=0.9] at ( 4,3) {};
\node (t)[scale=0.9,style={circle,fill=blue!20}][scale=0.9] at ( 6,2) {};
\node at (0,1) {$V_1$};
\node at (6,1) {$V_2$};
\path (s) edge [->]  (a); 
\path (s) [red]edge [->]  (e); 
\path (a) edge [->]  (b);
\path (a) [red]edge [->]  (e);
\path (a) edge [->]  (d);
\path (e) edge [->]  (b);
\path (e) [green]edge [->]  (d);
\path (d) [red]edge [->]  (t);
\path (d) [red]edge [->]  (b);
\path (t) edge [->]  (b);
\path (-0.2,1.8)edge [-,bend right] (4.5,3.2); 
\path (1.5,0.8)edge [-,bend left] (6.5,2.2); 
\end{tikzpicture}\\
\textcolor{green}{ообратное $E^-_c$}\\
\textcolor{red}{прямое $E^+_c$}\\
$E_c=E_c^-\cap E_c^+$\\
\underline{Определение}\\
Величина разреза=$\sum \limits_{e\in E^+_c} c(e)$\\
\underline{Обозначение} c(G)\\
Например\\
\begin{tikzpicture}
\node (s)[scale=0.9,style={circle,fill=blue!20}][scale=0.9] at ( 0,2) {};
\node (a)[scale=0.9,style={circle,fill=blue!20}][scale=0.9] at ( 2,3) {};
\node (e)[scale=0.9,style={circle,fill=blue!20}][scale=0.9] at ( 2,1) {};
\node (b)[scale=0.9,style={circle,fill=blue!20}][scale=0.9] at ( 4,1) {};
\node (d)[scale=0.9,style={circle,fill=blue!20}][scale=0.9] at ( 4,3) {};
\node (t)[scale=0.9,style={circle,fill=blue!20}][scale=0.9] at ( 6,2) {};
\node at (0,1) {$V_1$};
\node at (6,1) {$V_2$};
\path (s) edge [->]  (a); 
\path (s) edge [->]  (e); 
\path (a) edge [->]  (b);
\path (a) edge [->]  (d);
\path (e) edge [->]  (b);
\path (e) [red]edge [->]  (d);
\path (b) [green]edge[->] (a);
\path (d) edge [->]  (t);
\path (t) edge [->]  (b);
\path (3,0) edge[dashed,-] (3,4);
\end{tikzpicture}\\
C=($V_1,V_2)$\\
c(C)=8+3+4=15\\
\underline{Утверждение}\\
Пусть есть сеть (G=(V,E),c), поток f, разрез C=($V_1,V_2$)\\
тогда w(f)=$\sum \limits_{e\in E^+_c}f(e)-\sum  \limits_{e\in E^-_c}f(e)$\\
\underline{Пример}\\
\begin{tikzpicture}
\node (s)[scale=0.9,style={circle,fill=blue!10}][scale=0.9] at ( 0,2) {s};
\node (1) at ( 0,1.5) {исток};
\node (a)[scale=0.9,style={circle,fill=blue!10}][scale=0.9] at ( 2,3) {a};
\node (e)[scale=0.9,style={circle,fill=blue!10}][scale=0.9] at ( 2,1) {e};
\node (b)[scale=0.9,style={circle,fill=blue!10}][scale=0.9] at ( 4,4) {b};
\node (c)[scale=0.9,style={circle,fill=blue!10}][scale=0.9] at ( 4,2) {c};
\node (f)[scale=0.9,style={circle,fill=blue!10}][scale=0.9] at ( 4,0) {f};
\node (d)[scale=0.9,style={circle,fill=blue!10}][scale=0.9] at ( 6,3) {d};
\node (g)[scale=0.9,style={circle,fill=blue!10}][scale=0.9] at ( 6,1) {g};
\node (t)[scale=0.9,style={circle,fill=blue!10}][scale=0.9] at ( 8,2) {t};
\node (2) at ( 8,1.5) {сток};
\path (s) edge[->] node[above]{3} (a);
\path (s) edge [->] node[below, red] {1} (a);
\path (s) [red]edge [->] node[above]{8} (e); 
\path (s) [red]edge [->] node[above]{8} (e); 
\path (s) edge [->] node[below, red]{3} (e);
\path (a) edge [->] node[above]{7} (b); 
\path (a) edge [->] node[below, red]{1} (b);
\path (a) edge [->] node[above]{8} (c); 
\path (a) edge [->] node[below, red]{0} (c);
\path (e) [red]edge [->] node[above]{5} (c); \path (e) [red]edge [->] node[above]{5} (c); 
\path (e) edge [->] node[below, red]{3} (c);
\path (e) edge [->] node[above]{6} (f); 
\path (e) edge [->] node[below, red]{0} (f);
\path (c) edge [->, bend left] node[above]{1} (d); 
\path (c) edge [->, bend left] node[below, red]{1} (d);
\path (d) edge [->, bend left] node[above]{4} (c); 
\path (d) edge [->, bend left] node[below, red]{0} (c);
\path (c) edge [->] node[right]{2} (b); 
\path (c) edge [->] node[left, red]{0} (b);
\path (b) edge [->] node[above]{4} (d); 
\path (b) edge [->] node[below, red]{1} (d);
\path (c) edge [->] node[above]{2} (g); 
\path (c) [green]edge [->] node[below, red]{2} (g);
\path (f) edge [->] node[above]{3} (g); 
\path (f) edge [->] node[below, red]{0} (g);
\path (g) [red]edge [->, bend right] node[left]{6} (d); \path (g) [red]edge [->, bend right] node[left]{6} (d); 
\path (g) edge [->, bend right] node[right, red]{0} (d);
\path (g) edge [->] node[above]{6} (t); 
\path (g) edge [->] node[below, red]{2} (t);
\path (d) edge [->] node[above]{5} (t); 
\path (d) edge [->] node[below, red]{2} (t);
\draw[rotate=-10,blue] (3,1.5) ellipse (5 and 1);
\draw[rotate=-10,blue] (4.5,4.2) ellipse (4 and 2);
\node at(3,-0.5){$V_1$};
\node at(4.5,4.5){$V_2$};
\end{tikzpicture}\\
\textcolor{red}{w(f)=1+3=2+2=4}\\
\textcolor{red}{$\sum \limits_{e\in E^+_c}f(e)=1+3+0+2=6$}\\
\textcolor{green}{$\sum \limits_{e\in E_c^-}f(e)=2$}\\
Да! 4=6-2\\
посчитаем сумму\\
$\sum \limits_{v \in V_1}(- \sum \limits_{e:e=(u,v)}f(e)+\sum \limits_{e:e=(v),u}f(e)$\\
1) для $\forall v\in V_1\setminus \{s\} $внутреняя $\sum - \sum =0$\\
для v=s получается w(f)=$\sum \limits_{e:e=(v),u}f(e)$\\
2)$\sum \limits_{e=(v),u}(f(e)-f(e))+\underbrace{\sum \limits_{e\in E^+_c}f(e)-\sum \limits_{e\in E^-_c}+[-f(e)]}_{\textrm{смотри условие}}$=0+величина их условия\\
\underline{Обозначение}\\
w(c,f)-выличина потока через разрез\\
$\sum \limits_{e\in E^+_c}f(e)-\sum \limits_{e\in E^-_c}f(e)$\\
\underline{Замечание}\\
$\forall C$ w(f)=w(C,f)- по Теореме\\
\underline{Замечание}\\
Будем решать задачу о максимальном потоке в сети то есть найти f:\\
w(f)$\rightarrow$max\\
\underline{Утверждение}\\
Дано G,c-сеть, C-разрез\\
Тогда w(f) $\geq$c(C)\\
\underline{Докозательсто}\\
\begin{tikzpicture}
\node (s)[scale=0.9,style={circle,fill=blue!20}][scale=0.9] at ( 0,2) {};
\node (t)[scale=0.9,style={circle,fill=blue!20}][scale=0.9] at ( 6,2) {};
\node at (0,0) {$V_1$};
\node at (6,0) {$V_2$};
\draw (1,2) circle (1.5);
\draw (5,2) circle (1.5);
\path (s) edge[] (2,2.7);
\path (s) edge[] (2,2.2);
\path (s) edge[] (2,1.7);
\path (s) edge[] (2,1.2);
\path (t) edge[] (4,2.7);
\path (t) edge[] (4,2.2);
\path (t) edge[] (4,1.7);
\path (t) edge[] (4,1.2);
\path (2.5,2) edge[line width=1,->,red] (3.5,2);
\path (2.2,2.9) edge[line width=1,->,red] (3.8,2.9);
\path (2.2,1.1) edge[line width=1,->,red] (3.8,1.1);
\path (3,3.5) edge[line width=1,-,dashed] (3,0.5);
\path (3.8,1.1) edge[line width=1,->,green] (2.2,2.9);
\path (3.8,2.9) edge[line width=1,->,green] (2.2,1.1);
\end{tikzpicture}\\
w(f)=w(C,f)=$\sum \limits_{e\in E^+_c}f(e)-\sum \limits_{e\in E^-_c}f(e)\leq$\\
$\sum \limits_{e\in E^+_c}f(e) \leq \sum \limits_{e\in E^+_c} c(e)=c(C)\Rightarrow$ w(f)$\leq $c(C)\\
\underline{Следствие}\\
В сети G w(f$_{max}$)$\leq$c(C$_{min}$)\\
где w(f$_{max}$)=max w(f) f-поток c($C_{min}$)=min c(C) С-разрез\\
\underline{Теорема Форда-Фалкерсона}\\
w(f$_{max}$)=c(C$_{min}$)\\
в сети (G,c) c(e)$\in$N\\
- для простоты считаем, что пропускные способности целые\\
\underline{Определение}\\
дополнительный граф для потока\\
$\overline{G}$имеет $\overline{V}$=V \\
$\overline{E}$:\\
\begin{tikzpicture}
\node (s)[scale=0.9,style={circle,fill=blue!20}][scale=0.9] at ( 0,2) {s};
\node (a)[scale=0.9,style={circle,fill=blue!20}][scale=0.9] at ( 2,3) {a};
\node (e)[scale=0.9,style={circle,fill=blue!20}][scale=0.9] at ( 2,1) {e};
\node (b)[scale=0.9,style={circle,fill=blue!20}][scale=0.9] at ( 4,1) {b};
\node (d)[scale=0.9,style={circle,fill=blue!20}][scale=0.9] at ( 4,3) {d};
\node (t)[scale=0.9,style={circle,fill=blue!20}][scale=0.9] at ( 6,2) {t};
\path (s) edge[->] node[above]{5}node[below, red] {3} (a);
\path (s) edge[->] node[above]{4}node[below, red] {4} (e);
\path (a) edge[->] node[above]{3} node[below, red] {1} (b);
\path (a) edge[->] node[above]{6} node[below, red] {3} (d);
\path (e) edge[->] node[above]{5} node[below, red] {5} (b);
\path (d) edge[->,bend left] node[above]{7} node[below, red] {1} (e);
\path (b) edge[->] node[left]{2} node[right, red] {1} (d);
\path (b) edge[->] node[above]{4} node[below, red] {4} (t);
\path (d) edge[->] node[above]{8} node[below, red] {3} (t);
\end{tikzpicture}\\
если f(e)<c(e)\\
e=(u,v)\\
то есть e'=(e',v')\\
g(e)=c(e)-f(e)\\
\begin{tikzpicture}
\node (s)[scale=0.9,style={circle,fill=blue!20}][scale=0.9] at ( 0,2) {s};
\node (a)[scale=0.9,style={circle,fill=blue!20}][scale=0.9] at ( 2,3) {a};
\node (e)[scale=0.9,style={circle,fill=blue!20}][scale=0.9] at ( 2,1) {e};
\node (b)[scale=0.9,style={circle,fill=blue!20}][scale=0.9] at ( 4,1) {b};
\node (d)[scale=0.9,style={circle,fill=blue!20}][scale=0.9] at ( 4,3) {d};
\node (t)[scale=0.9,style={circle,fill=blue!20}][scale=0.9] at ( 6,2) {t};
\path (s) edge[->] node[]{2}    (a);
\path (a) edge[->] node[]{3}    (b);
\path (a) edge[->] node[]{3}    (d);
\path (d) edge[->,bend left] node[]{6}    (e);
\path (b) edge[->] node[]{1}    (d);
\path (d) edge[->] node[]{5}    (t);
\path (a) edge[->,red,bend left] node[]{2}    (s);
\path (b) edge[->,red,bend left] node[]{3}    (a);
\path (d) edge[->,red,bend right] node[]{3}    (a);
\path (e) edge[->,red,bend left] node[]{6}    (d);
\path (d) edge[->,red,bend left] node[]{1}    (b);
\path (t) edge[->,red,bend left] node[]{5}    (d);
\end{tikzpicture}\\
если 0<f(e) e=(u,v)\\
то есть e''=(v',u')\\
g(e'')-f(e)\\
\underline{Докозательство}\\
Начнем с нулевого потока и будем его постепенно увеличивать\\
Построим дополнительный граф $\overline{G}$ и найдем в нём путь из s в t\\
s-O-t\\
Найдем min g(e) на этом пути $\supset$ это х\\
Вычтем в доп графе- х на каждом ребре\\
\begin{tikzpicture}
\node (s)[scale=0.9,style={circle,fill=blue!20}][scale=0.9] at (0,0) {};
\node (a)[scale=0.9,style={circle,fill=blue!20}][scale=0.9] at (2,0) {};
\node (e)[scale=0.9,style={circle,fill=blue!20}][scale=0.9] at (4,0) {};
\node (b)[scale=0.9,style={circle,fill=blue!20}][scale=0.9] at (6,0) {};
\path (s) edge[->] (a);
\path (a) edge[->] (e);
\path (e) edge[->] (b);
\end{tikzpicture}\\
1)c(e)- f(e) $\rightarrow$ c(e)- f(e)-x\\
f(e):=f(e)+x\\
\begin{tikzpicture}
\node (s)[scale=0.9,style={circle,fill=blue!20}][scale=0.9] at (0,0) {};
\node (a)[scale=0.9,style={circle,fill=blue!20}][scale=0.9] at (2,0) {};
\node (e)[scale=0.9,style={circle,fill=blue!20}][scale=0.9] at (4,0) {};
\node (b)[scale=0.9,style={circle,fill=blue!20}][scale=0.9] at (6,0) {};
\path (s) edge[->] (a);
\path (a) edge[->,green]node[below]{f(e)} (e);
\path (e) edge[->] (b);
\end{tikzpicture}\\
2)f(e)-x f'(e):=f(e)-x\\
Поймем, что 1) новый поток f' остался потое\\
2) величина потока увеличилась на х\\
Проверяем, что это поток 0$\leq$f'(e)$\leq$c(e) уменьшаем по обратному,\\
увеличиваем по прямому c(e)-(f(e)+x)$\geq$0\\
В вершинах верно $\sum$вход=$\sum$исх\\
1)\\
\begin{tikzpicture}
\node (s)[scale=0.9,style={circle,fill=blue!20}][scale=0.9] at (2,1) {};
\path (1,0) edge[->] (s);
\path (1,1) edge[->] (s);
\path (1,2) edge[->] node[right]{t+x} (s);
\path (3,0) edge[<-] (s);
\path (3,1) edge[<-] node[above]{f+x} (s);
\path (3,2) edge[<-] (s);
\end{tikzpicture}\\
2)\\
\begin{tikzpicture}
\node (s)[scale=0.9,style={circle,fill=blue!20}][scale=0.9] at (2,1) {};
\path (1,2) edge[->] node[right]{t+x} (s);
\path (3,1) edge[<-,green] (s);
\path (3,1) edge[->,dashed, bend left] node[below]{f-x} (s);
\end{tikzpicture}\\
3)\\
\begin{tikzpicture}
\node (s)[scale=0.9,style={circle,fill=blue!20}][scale=0.9] at (2,1) {};
\path (1,2) edge[->,green]  (s);
\path (s) edge[->,dashed,bend right] node[right]{f-x} (1,2);
\path (3,1) edge[<-]node[below]{t+x}(s);
\end{tikzpicture}\\
4)\\
\begin{tikzpicture}
\node (s)[scale=0.9,style={circle,fill=blue!20}][scale=0.9] at (2,1) {};
\path (1,2) edge[->,green]  (s);
\path (s) edge[->,dashed,bend right] node[right]{f-x} (1,2);
\path (3,1) edge[<-,green] (s);
\path (3,1) edge[->,dashed, bend left] node[below]{f-x} (s);
\end{tikzpicture}\\
Итого f'-поток\\
\begin{tikzpicture}
\node (s)[scale=0.9,style={circle,fill=blue!10}][scale=0.9] at ( 0,2) {s};
\node (a)[scale=0.9,style={circle,fill=blue!10}][scale=0.9] at ( 2,3) {};
\node (e)[scale=0.9,style={circle,fill=blue!10}][scale=0.9] at ( 2,1) {};
\node (b)[scale=0.9,style={circle,fill=blue!10}][scale=0.9] at ( 4,1) {};
\node (d)[scale=0.9,style={circle,fill=blue!10}][scale=0.9] at ( 4,3) {};
\node (t)[scale=0.9,style={circle,fill=blue!10}][scale=0.9] at ( 6,2) {t};
\path (s) edge [->]node[left]{5}  (a); 
\path (s) edge [->]node[left]{4}  (e); 
\path (a) edge [->]node[right]{3}  (b);
\path (a) edge [->]node[left]{6}  (d);
\path (e) edge [->]node[left]{5}  (b);
\path (e) edge [<-]node[left]{7}  (d);
\path (b) edge [->]node[left]{4}  (t);
\path (b) edge [->]node[right]{2}  (d);
\path (d) edge [->]node[left]{8}  (t);
\end{tikzpicture}\\
можно построить дополнительный граф\\
Пример. строим пути\\
сначала поток=0\\
Дополнительный граф не отличается от исходного\\
\begin{tikzpicture}
\node (s)[scale=0.9,style={circle,fill=blue!10}][scale=0.9] at ( 0,2) {s};
\node (a)[scale=0.9,style={circle,fill=blue!10}][scale=0.9] at ( 2,3) {};
\node (e)[scale=0.9,style={circle,fill=blue!10}][scale=0.9] at ( 2,1) {};
\node (b)[scale=0.9,style={circle,fill=blue!10}][scale=0.9] at ( 4,1) {};
\node (d)[scale=0.9,style={circle,fill=blue!10}][scale=0.9] at ( 4,3) {};
\node (t)[scale=0.9,style={circle,fill=blue!10}][scale=0.9] at ( 6,2) {t};
\path (s) edge [->]node[left]{5}  (a); 
\path (s) edge [->]node[left]{4}  (e); 
\path (a) edge [->]node[right]{3}  (b);
\path (a) edge [->]node[left]{6}  (d);
\path (e) edge [->]node[left]{5}  (b);
\path (e) edge [<-]node[left]{7}  (d);
\path (b) edge [->]node[left]{4}  (t);
\path (b) edge [->]node[right]{2}  (d);
\path (d) edge [->]node[left]{8}  (t);
\end{tikzpicture}\\
путь из s в t\\
S-(5)-a-(3)-d-(2)-b-(8)-t\\
добавим к поток +2 на каждое из этих ребер\\
поток\\
\begin{tikzpicture}
\node (s)[scale=0.9,style={circle,fill=blue!10}][scale=0.9] at ( 0,2) {s};
\node (a)[scale=0.9,style={circle,fill=blue!10}][scale=0.9] at ( 2,3) {a};
\node (e)[scale=0.9,style={circle,fill=blue!10}][scale=0.9] at ( 2,1) {e};
\node (b)[scale=0.9,style={circle,fill=blue!10}][scale=0.9] at ( 4,1) {b};
\node (d)[scale=0.9,style={circle,fill=blue!10}][scale=0.9] at ( 4,3) {d};
\node (t)[scale=0.9,style={circle,fill=blue!10}][scale=0.9] at ( 6,2) {t};
\path (s) edge [->]node[left]{2}  (a); 
\path (s) edge [->]node[left]{0}  (e); 
\path (a) edge [->]node[right]{2}  (b);
\path (a) edge [->]node[left]{0}  (d);
\path (e) edge [->]node[left]{0}  (b);
\path (e) edge [<-]node[left]{0}  (d);
\path (b) edge [->]node[left]{0}  (t);
\path (b) edge [->]node[right]{2}  (d);
\path (d) edge [->]node[left]{2}  (t);
\end{tikzpicture}\quad
\begin{tikzpicture}
\node (s)[scale=0.9,style={circle,fill=blue!10}][scale=0.9] at ( 0,2) {s};
\node (a)[scale=0.9,style={circle,fill=blue!10}][scale=0.9] at ( 2,3) {a};
\node (e)[scale=0.9,style={circle,fill=blue!10}][scale=0.9] at ( 2,1) {e};
\node (b)[scale=0.9,style={circle,fill=blue!10}][scale=0.9] at ( 4,1) {b};
\node (d)[scale=0.9,style={circle,fill=blue!10}][scale=0.9] at ( 4,3) {d};
\node (t)[scale=0.9,style={circle,fill=blue!10}][scale=0.9] at ( 6,2) {t};
\path (s) edge [->]node[left]{3}  (a); 
\path (s) [green]edge [<-,dashed,bend left]node[left,green]{2}  (a); 
\path (s) edge [->]node[left]{4}  (e); 
\path (a) edge [->]node[right]{1}  (b);
\path (a) [green]edge [<-,dashed,bend left]node[right,green]{2}  (b);
\path (a) edge [->]node[left]{6}  (d);
\path (e) edge [->]node[left]{5}  (b);
\path (e) edge [<-]node[left]{7}  (d);
\path (b) edge [->]node[left]{4}  (t);
\path (b) [green]edge [<-,dashed,bend right]node[right,green]{2}  (d);
\path (d) edge [->]node[left]{6}  (t);
\path (d) [green]edge [<-,dashed,bend left]node[right,green]{2}  (t);
\end{tikzpicture}\\
S-(3)-a-(6)-b-(7)-c-(5)-d-(4)-t\\
min=3\\
\begin{tikzpicture}
\node (s)[scale=0.9,style={circle,fill=blue!10}][scale=0.9] at ( 0,2) {s};
\node (a)[scale=0.9,style={circle,fill=blue!10}][scale=0.9] at ( 2,3) {a};
\node (e)[scale=0.9,style={circle,fill=blue!10}][scale=0.9] at ( 2,1) {e};
\node (b)[scale=0.9,style={circle,fill=blue!10}][scale=0.9] at ( 4,1) {b};
\node (d)[scale=0.9,style={circle,fill=blue!10}][scale=0.9] at ( 4,3) {d};
\node (t)[scale=0.9,style={circle,fill=blue!10}][scale=0.9] at ( 6,2) {t}; 
\path (s) edge [->]node[left]{0}  (e); 
\path (s) edge [->]node[left]{5}  (a);
\path (a) edge [->]node[right]{2}  (b);
\path (a) edge [->]node[left]{3}  (d);
\path (e) edge [->]node[left]{3}  (b);
\path (e) edge [<-]node[left]{3}  (d);
\path (b) edge [->]node[left]{3}  (t);
\path (b) edge [->]node[left]{2}  (d);
\path (d) edge [->]node[left]{2}  (t);
\end{tikzpicture}\quad
\begin{tikzpicture}
\node (s)[scale=0.9,style={circle,fill=blue!10}][scale=0.9] at ( 0,2) {s};
\node (a)[scale=0.9,style={circle,fill=blue!10}][scale=0.9] at ( 2,3) {a};
\node (e)[scale=0.9,style={circle,fill=blue!10}][scale=0.9] at ( 2,1) {e};
\node (b)[scale=0.9,style={circle,fill=blue!10}][scale=0.9] at ( 4,1) {b};
\node (d)[scale=0.9,style={circle,fill=blue!10}][scale=0.9] at ( 4,3) {d};
\node (t)[scale=0.9,style={circle,fill=blue!10}][scale=0.9] at ( 6,2) {t}; 
\path (s) [green]edge [<-,dashed,bend left ]node[left,green]{5}  (a); 
\path (s) edge [->]node[left]{4}  (e); 
\path (a) edge [->]node[right]{1}  (b);
\path (a) [green]edge [<-,dashed,bend left]node[right,green]{2}  (b);
\path (a) edge [->]node[left]{3}  (d);
\path (a) [green]edge [<-,dashed,bend left]node[left,green]{3}  (d);
\path (e) edge [->]node[left]{5}  (b);
\path (e) edge [<-]node[left]{4}  (d);
\path (e) [green]edge[->,dashed,bend left]node[left]{4}  (d);
\path (b) edge [->]node[left]{1}  (t);
\path (b) [green]edge [<-,dashed,bend right]node[left]{1}  (t);
\path (b) [green]edge [<-,dashed,bend right]node[right,green]{2}  (d);
\path (d) edge [->]node[left]{6}  (t);
\path (d) [green]edge [<-,dashed,bend left]node[right,green]{2}  (t);
\end{tikzpicture}\\
s-(4)-c-(2)-d-(1)-t\\
min=1\\
\begin{tikzpicture}
\node (s)[scale=0.9,style={circle,fill=blue!10}][scale=0.9] at ( 0,2) {s};
\node (a)[scale=0.9,style={circle,fill=blue!10}][scale=0.9] at ( 2,3) {a};
\node (e)[scale=0.9,style={circle,fill=blue!10}][scale=0.9] at ( 2,1) {e};
\node (b)[scale=0.9,style={circle,fill=blue!10}][scale=0.9] at ( 4,1) {b};
\node (d)[scale=0.9,style={circle,fill=blue!10}][scale=0.9] at ( 4,3) {d};
\node (t)[scale=0.9,style={circle,fill=blue!10}][scale=0.9] at ( 6,2) {t}; 
\path (s) edge [->]node[left]{1}  (e); 
\path (s) edge [->]node[left]{5}  (a);
\path (a) edge [->]node[right]{2}  (b);
\path (a) edge [->]node[left]{3}  (d);
\path (e) edge [->]node[left]{4}  (b);
\path (e) edge [<-]node[left]{3}  (d);
\path (b) edge [->]node[left]{4}  (t);
\path (b) edge [->]node[left]{2}  (d);
\path (d) edge [->]node[left]{2}  (t);
\end{tikzpicture}\quad
\begin{tikzpicture}
\node (s)[scale=0.9,style={circle,fill=blue!10}][scale=0.9] at ( 0,2) {s};
\node (a)[scale=0.9,style={circle,fill=blue!10}][scale=0.9] at ( 2,3) {a};
\node (e)[scale=0.9,style={circle,fill=blue!10}][scale=0.9] at ( 2,1) {e};
\node (b)[scale=0.9,style={circle,fill=blue!10}][scale=0.9] at ( 4,1) {b};
\node (d)[scale=0.9,style={circle,fill=blue!10}][scale=0.9] at ( 4,3) {d};
\node (t)[scale=0.9,style={circle,fill=blue!10}][scale=0.9] at ( 6,2) {t}; 
\path (s) [green]edge [<-,dashed,bend left ]node[left,green]{5}  (a); 
\path (s) edge [->]node[left]{3}  (e); 
\path (s) [green] edge [<-,dashed,bend left]node[right]{1}  (e); 
\path (a) edge [->]node[right]{1}  (b);
\path (a) [green]edge [<-,dashed,bend left]node[right,green]{2}  (b);
\path (a) edge [->]node[left]{3}  (d);
\path (a) [green]edge [<-,dashed,bend left]node[left,green]{3}  (d);
\path (e) edge [->]node[left]{1}  (b);
\path (e) [green] edge [<-,dashed, bend right]node[left]{3}  (b);
\path (e) edge [<-]node[left]{4}  (d);
\path (e) [green]edge[->,dashed,bend left]node[left]{4}  (d);
\path (b) [green]edge [<-,dashed,bend right]node[left]{4}  (t);
\path (b) [green]edge [<-,dashed,bend right]node[right,green]{2}  (d);
\path (d) edge [->]node[left]{0}  (t);
\path (d) [green]edge [<-,dashed,bend left]node[right,green]{2}  (t);
\end{tikzpicture}\\
S-(3)-c-(3)-b-(6)-t\\
\begin{tikzpicture}
\node (s)[scale=0.9,style={circle,fill=blue!10}][scale=0.9] at ( 0,2) {s};
\node (a)[scale=0.9,style={circle,fill=blue!10}][scale=0.9] at ( 2,3) {a};
\node (e)[scale=0.9,style={circle,fill=blue!10}][scale=0.9] at ( 2,1) {e};
\node (b)[scale=0.9,style={circle,fill=blue!10}][scale=0.9] at ( 4,1) {b};
\node (d)[scale=0.9,style={circle,fill=blue!10}][scale=0.9] at ( 4,3) {d};
\node (t)[scale=0.9,style={circle,fill=blue!10}][scale=0.9] at ( 6,2) {t}; 
\path (s) edge [->]node[left]{4}  (e); 
\path (s) edge [->]node[left]{5}  (a);
\path (a) edge [->]node[right]{2}  (b);
\path (a) edge [->]node[left]{3}  (d);
\path (e) edge [->]node[left]{4}  (b);
\path (e) edge [<-]node[left]{0}  (d);
\path (b) edge [->]node[left]{4}  (t);
\path (b) edge [->]node[left]{2}  (d);
\path (d) edge [->]node[left]{5}  (t);
\end{tikzpicture}\quad
\begin{tikzpicture}
\node (s)[scale=0.9,style={circle,fill=blue!10}][scale=0.9] at ( 0,2) {s};
\node (a)[scale=0.9,style={circle,fill=blue!10}][scale=0.9] at ( 2,3) {a};
\node (e)[scale=0.9,style={circle,fill=blue!10}][scale=0.9] at ( 2,1) {e};
\path (s) edge [<-]node[left]{5}  (a); 
\path (s) edge [<-]node[left]{4}  (e); 
\end{tikzpicture}\\
Продолжаем докозательство теорема Форда-Фалкерсона\\
Если пути нет, то поток оптимальный?\\
$\supset V_1$- вершины, достигшие из S по ребрам до t графа $V_2$=V $\setminus$ V$_1$\\
\begin{tikzpicture}
\node (s)[scale=0.9,style={circle,fill=blue!10}][scale=0.9] at ( 0,2) {};
\node (a)[scale=0.9,style={circle,fill=blue!10}][scale=0.9] at ( 2,3) {};
\node (e)[scale=0.9,style={circle,fill=blue!10}][scale=0.9] at ( 2,1) {};
\node (b)[scale=0.9,style={circle,fill=blue!10}][scale=0.9] at ( 4,1) {};
\node (d)[scale=0.9,style={circle,fill=blue!10}][scale=0.9] at ( 4,3) {};
\node (t)[scale=0.9,style={circle,fill=blue!10}][scale=0.9] at ( 6,2) {};
\node () at (0,0) {$V_1$};
\node () at (6,0) {$V_2$};
\path (s) edge [->]node[left]{}  (a); 
\path (s) edge [->]node[left]{}  (e); 
\path (a) edge [->]node[right]{}  (b);
\path (a) edge [->]node[left]{}  (d);
\path (e) edge [->]node[left]{}  (b);
\path (e) edge [<-]node[left]{}  (d);
\path (b) edge [->]node[left]{}  (t);
\path (b) edge [->]node[right]{}  (d);
\path (d) edge [->]node[left]{}  (t);
\draw (s)[green] circle (1);
\draw (4,2)[red] ellipse (3 and 2);
\end{tikzpicture}\\
$t\in V_2$, так как нет пути S$\rightarrow$t получаем разрез исходной сети ребра $E^+_G$\\
Ребер нет в дополнительном графе, в дополнительном графе веса=0\\
c(e)-f(e)=0\\
c(C)=$\sum \limits_{e\in E^+_C}c(e)=\sum \limits_{e\in E^+_C}f(e)$=c(f)\\
Течет ли что-то по нечетным? Нет, иначе $V_1$ неверен\\
В прошлый раз мы напоминали Vc- разрез $\forall f$-поток\\
c(C)$\geq$с(f)\\
получается c-min разрз f-max поток\\
\begin{tikzpicture}
\node (s)[scale=0.9,style={circle,fill=blue!10}][scale=0.9] at ( 0,2) {s};
\node (a)[scale=0.9,style={circle,fill=blue!10}][scale=0.9] at ( 2,4) {a};
\node (c)[scale=0.9,style={circle,fill=blue!10}][scale=0.9] at ( 2,0) {c};
\node (b)[scale=0.9,style={circle,fill=blue!10}][scale=0.9] at ( 4,2) {b};
\node (t)[scale=0.9,style={circle,fill=blue!10}][scale=0.9] at ( 6,2){t};
\path (s) edge [->]node[left]{10}  (a); 
\path (c) edge [->]node[left]{40}  (b); 
\path (a) edge [->]node[left]{2}  (b); 
\path (s) edge [->]node[left]{3}  (c); 
\path (b) edge [->]node[below]{20}  (t); 
\end{tikzpicture}\\
c(C)=2+3=5\\
min разрез\\
\underline{Утверждение}\\
Если каждый раз искать путь с min количеством ребер,\\
то время max поток $\sim$ $V^2E$\\
Без докозательства\\
\underline{Утверждение}\\
Для плоской сети(без пересечений ребер)=c(f) эффективно искать верхние\\
пути\\
\underline{Задачи о паросочетаниях}\\
Дан двудольный граф G=(u,v)\\
\begin{tikzpicture}
\node (1)[scale=0.9,style={circle,fill=blue!10}][scale=0.9] at ( 0,0) {};
\node (2)[scale=0.9,style={circle,fill=blue!10}][scale=0.9] at ( 0,1) {};
\node (3)[scale=0.9,style={circle,fill=blue!10}][scale=0.9] at ( 0,2) {};
\node (4)[scale=0.9,style={circle,fill=blue!10}][scale=0.9] at ( 0,3) {};
\node (5)[scale=0.9,style={circle,fill=blue!10}][scale=0.9] at ( 0,4){};
\node (6)[scale=0.9,style={circle,fill=blue!10}][scale=0.9] at ( 3,0) {};
\node (7)[scale=0.9,style={circle,fill=blue!10}][scale=0.9] at ( 3,1) {};
\node (8)[scale=0.9,style={circle,fill=blue!10}][scale=0.9] at ( 3,2) {};
\node (9)[scale=0.9,style={circle,fill=blue!10}][scale=0.9] at ( 3,3) {};
\node (10)[scale=0.9,style={circle,fill=blue!10}][scale=0.9] at (3,4){};
\path (2) edge [-](10); 
\path (3) edge [-](10); 
\path (5) edge [-](10); 
\path (4) edge [-](9); 
\path (1) edge [-](9); 
\end{tikzpicture}\\
\underline{Определение}\\
Паросочетания в G-это P$\in$ E, где ребра из P не имеют общие вершины\\
\underline{Определение}\\
Максимальное паросочетание, это PCE, |P|$\rightarrow$ max из возможных\\
\underline{Пример}\\
\begin{tikzpicture}
\node (1)[scale=0.9,style={circle,fill=blue!10}][scale=0.9] at ( 0,0) {c};
\node (2)[scale=0.9,style={circle,fill=blue!10}][scale=0.9] at ( 0,1) {b};
\node (3)[scale=0.9,style={circle,fill=blue!10}][scale=0.9] at ( 0,2) {a};
\node (4)[scale=0.9,style={circle,fill=blue!10}][scale=0.9] at ( 3,0) {C};
\node (5)[scale=0.9,style={circle,fill=blue!10}][scale=0.9] at ( 3,1){B};
\node (6)[scale=0.9,style={circle,fill=blue!10}][scale=0.9] at ( 3,2) {A};
\path (1) edge [-](6); 
\path (2) edge [-](4); 
\path (3) edge [-](5); 
\end{tikzpicture}\\
D$\{$cA,bC,aB $\}$\\
\underline{Пример}\\
\begin{tikzpicture}
\node (1)[scale=0.9,style={circle,fill=blue!10}][scale=0.9] at ( 0,0) {};
\node (2)[scale=0.9,style={circle,fill=blue!10}][scale=0.9] at ( 0,1) {};
\node (3)[scale=0.9,style={circle,fill=blue!10}][scale=0.9] at ( 0,2) {};
\node (4)[scale=0.9,style={circle,fill=blue!10}][scale=0.9] at ( 0,3) {};
\node (5)[scale=0.9,style={circle,fill=blue!10}][scale=0.9] at ( 3,0){};
\node (6)[scale=0.9,style={circle,fill=blue!10}][scale=0.9] at ( 3,1) {};
\node (7)[scale=0.9,style={circle,fill=blue!10}][scale=0.9] at ( 3,2){};
\node (8)[scale=0.9,style={circle,fill=blue!10}][scale=0.9] at ( 3,3) {};
\path (4) edge [-](8); 
\path (4) edge [-](5); 
\path (4) edge [-](7); 
\path (3) edge [-](7); 
\path (2) edge [-](7); 
\path (1) edge [-](7); 
\path (1) edge [-](6); 
\end{tikzpicture}\\
4-нельзя\\
Сводим к задаче о потоке\\
Ребра из S$\rightarrow$u\\
v$\rightarrow$t\\
u$\rightarrow$v\\
слева направо\\
c(e)=1\\
\begin{tikzpicture}
\node (1)[scale=0.9,style={circle,fill=blue!10}][scale=0.9] at ( 1,0) {};
\node (2)[scale=0.9,style={circle,fill=blue!10}][scale=0.9] at ( 1,1) {};
\node (3)[scale=0.9,style={circle,fill=blue!10}][scale=0.9] at ( 1,2) {};
\node (4)[scale=0.9,style={circle,fill=blue!10}][scale=0.9] at ( 1,3) {};
\node (5)[scale=0.9,style={circle,fill=blue!10}][scale=0.9] at ( 4,0){};
\node (6)[scale=0.9,style={circle,fill=blue!10}][scale=0.9] at ( 4,1) {};
\node (7)[scale=0.9,style={circle,fill=blue!10}][scale=0.9] at ( 4,2){};
\node (8)[scale=0.9,style={circle,fill=blue!10}][scale=0.9] at ( 4,3) {};
\node (0)[scale=0.9,style={circle,fill=blue!10}][scale=0.9] at ( 0,1.5) {S};
\node (9)[scale=0.9,style={circle,fill=blue!10}][scale=0.9] at ( 5,1.5) {t};
\path (0) edge[->] (1);
\path (0) edge[->] (2);
\path (0) edge[->] (3);
\path (0) edge[->] (4);
\path (4) edge [->](8); 
\path (4) edge [->](7); 
\path (3) edge [->](7); 
\path (1) edge [->](5); 
\path (1) edge [->](6);
\path (8) edge[->] (9);
\path (7) edge[->] (9);
\path (6) edge[->] (9);
\path (5) edge[->] (9);
\end{tikzpicture}\\
\underline{Утверждение}\\
каждому потоку (из f=0,1) соотвествует паросочетание\\
ребра с f(c)=1 это ребра паросочетания$\Rightarrow$ поток\\
\begin{tikzpicture}
\node (1)[scale=0.9,style={circle,fill=blue!10}][scale=0.9] at ( 1,0) {};
\node (2)[scale=0.9,style={circle,fill=blue!10}][scale=0.9] at ( 1,1) {};
\node (3)[scale=0.9,style={circle,fill=blue!10}][scale=0.9] at ( 1,2) {};
\node (4)[scale=0.9,style={circle,fill=blue!10}][scale=0.9] at ( 1,3) {};
\node (0)[scale=0.9,style={circle,fill=blue!10}][scale=0.9] at ( 0,1.5) {S};
\path (0) edge[->]node[]{0} (1);
\path (0) edge[->]node[]{0} (2);
\path (0) edge[->]node[]{1}(3);
\path (0) edge[->]node[]{1} (4);
\end{tikzpicture}\\
$\Leftarrow$ паросочетание соответвует потоку, где f(e)=1 для ребер паросочетания\\
\underline{Следствие} Размер max паросочетаний=размер max потока\\
Строим паросочетание методом ФФ\\
\begin{tikzpicture}
\node (0)[scale=0.9,style={circle,fill=blue!10}][scale=0.9] at ( 0,1) {};
\node (1)[scale=0.9,style={circle,fill=blue!10}][scale=0.9] at ( 1,0) {};
\node (2)[scale=0.9,style={circle,fill=blue!10}][scale=0.9] at ( 1,1) {};
\node (3)[scale=0.9,style={circle,fill=blue!10}][scale=0.9] at ( 1,2) {};
\node (4)[scale=0.9,style={circle,fill=blue!10}][scale=0.9] at ( 3,0) {};
\node (5)[scale=0.9,style={circle,fill=blue!10}][scale=0.9] at ( 3,1){};
\node (6)[scale=0.9,style={circle,fill=blue!10}][scale=0.9] at ( 3,2) {};
\node (7)[scale=0.9,style={circle,fill=blue!10}][scale=0.9] at ( 4,1) {};
\path (0) edge[-] (1);
\path (0) edge[-] (2);
\path (0) edge[-] (3);
\path (1) edge [-](6); 
\path (2) edge [-](4); 
\path (3) edge [-](5); 
\path (4) edge[-] (7);
\path (6) edge[-] (7);
\path (5) edge[-] (7);
\end{tikzpicture}\\
Строим дополнительный граф, но без чисел\\
\begin{tikzpicture}
\node (0)[scale=0.9,style={circle,fill=blue!10}][scale=0.9] at ( 0,1) {};
\node (1)[scale=0.9,style={circle,fill=blue!10}][scale=0.9] at ( 1,0) {};
\node (2)[scale=0.9,style={circle,fill=blue!10}][scale=0.9] at ( 1,1) {};
\node (3)[scale=0.9,style={circle,fill=blue!10}][scale=0.9] at ( 1,2) {};
\node (4)[scale=0.9,style={circle,fill=blue!10}][scale=0.9] at ( 3,0) {};
\node (5)[scale=0.9,style={circle,fill=blue!10}][scale=0.9] at ( 3,1){};
\node (6)[scale=0.9,style={circle,fill=blue!10}][scale=0.9] at ( 3,2) {};
\node (7)[scale=0.9,style={circle,fill=blue!10}][scale=0.9] at ( 4,1) {};
\path (0) edge[->] (1);
\path (0) edge[->] (2);
\path (0) edge[->] (3);
\path (1) edge [->](6); 
\path (2) edge [->](4); 
\path (3) edge [->](5); 
\path (3) edge [->](6);
\path (4) edge[->] (7);
\path (6) edge[->] (7);
\path (5) edge[->] (7);
\end{tikzpicture}\quad
\begin{tikzpicture}
\node (0)[scale=0.9,style={circle,fill=blue!10}][scale=0.9] at ( 0,1) {};
\node (1)[scale=0.9,style={circle,fill=blue!10}][scale=0.9] at ( 1,0) {};
\node (2)[scale=0.9,style={circle,fill=blue!10}][scale=0.9] at ( 1,1) {};
\node (3)[scale=0.9,style={circle,fill=blue!10}][scale=0.9] at ( 1,2) {};
\node (4)[scale=0.9,style={circle,fill=blue!10}][scale=0.9] at ( 3,0) {};
\node (5)[scale=0.9,style={circle,fill=blue!10}][scale=0.9] at ( 3,1){};
\node (6)[scale=0.9,style={circle,fill=blue!10}][scale=0.9] at ( 3,2) {};
\node (7)[scale=0.9,style={circle,fill=blue!10}][scale=0.9] at ( 4,1) {};
\path (0) edge[->] (1);
\path (0) edge[->] (2);
\path (0) edge[<-] (3);
\path (1) edge [->](6); 
\path (2) edge [->](4); 
\path (3) edge [->](5); 
\path (3) edge [<-](6);
\path (4) edge[->] (7);
\path (6) edge[<-] (7);
\path (5) edge[->] (7);
\end{tikzpicture}\\
\begin{tikzpicture}
\node (0)[scale=0.9,style={circle,fill=blue!10}][scale=0.9] at ( 0,1) {};
\node (1)[scale=0.9,style={circle,fill=blue!10}][scale=0.9] at ( 1,0) {};
\node (2)[scale=0.9,style={circle,fill=blue!10}][scale=0.9] at ( 1,1) {};
\node (3)[scale=0.9,style={circle,fill=blue!10}][scale=0.9] at ( 1,2) {};
\node (4)[scale=0.9,style={circle,fill=blue!10}][scale=0.9] at ( 3,0) {};
\node (5)[scale=0.9,style={circle,fill=blue!10}][scale=0.9] at ( 3,1){};
\node (6)[scale=0.9,style={circle,fill=blue!10}][scale=0.9] at ( 3,2) {};
\node (7)[scale=0.9,style={circle,fill=blue!10}][scale=0.9] at ( 4,1) {};
\path (0) edge[<-] (1);
\path (0) edge[->] (2);
\path (0) edge[->] (3);
\path (1) edge [<-](6); 
\path (2) edge [->](4); 
\path (3) edge [->](5); 
\path (3) edge [->](6);
\path (4) edge[->] (7);
\path (6) edge[<-] (7);
\path (5) edge[->] (7);
\end{tikzpicture}\quad
\begin{tikzpicture}
\node (0)[scale=0.9,style={circle,fill=blue!10}][scale=0.9] at ( 0,1) {};
\node (1)[scale=0.9,style={circle,fill=blue!10}][scale=0.9] at ( 1,0) {};
\node (2)[scale=0.9,style={circle,fill=blue!10}][scale=0.9] at ( 1,1) {};
\node (3)[scale=0.9,style={circle,fill=blue!10}][scale=0.9] at ( 1,2) {};
\node (4)[scale=0.9,style={circle,fill=blue!10}][scale=0.9] at ( 3,0) {};
\node (5)[scale=0.9,style={circle,fill=blue!10}][scale=0.9] at ( 3,1){};
\node (6)[scale=0.9,style={circle,fill=blue!10}][scale=0.9] at ( 3,2) {};
\node (7)[scale=0.9,style={circle,fill=blue!10}][scale=0.9] at ( 4,1) {};
\path (0) edge[<-] (1);
\path (0) edge[<-] (2);
\path (0) edge[<-] (3);
\path (1) edge [<-](6); 
\path (2) edge [<-](4); 
\path (3) edge [<-](5); 
\path (3) edge [->](6);
\path (4) edge[<-] (7);
\path (6) edge[<-] (7);
\path (5) edge[<-] (7);
\end{tikzpicture}\\
Поток в сетях\\
\begin{tikzpicture}
\node (1)[scale=0.9,style={circle,fill=blue!10}][scale=0.9] at ( 0,0) {};
\node (2)[scale=0.9,style={circle,fill=blue!10}][scale=0.9] at ( 0,1) {};
\node (3)[scale=0.9,style={circle,fill=blue!10}][scale=0.9] at ( 0,2) {};
\node (4)[scale=0.9,style={circle,fill=blue!10}][scale=0.9] at ( 0,3) {};
\node (5)[scale=0.9,style={circle,fill=blue!10}][scale=0.9] at ( 3,0){};
\node (6)[scale=0.9,style={circle,fill=blue!10}][scale=0.9] at ( 3,1) {};
\node (7)[scale=0.9,style={circle,fill=blue!10}][scale=0.9] at ( 3,2){};
\node (8)[scale=0.9,style={circle,fill=blue!10}][scale=0.9] at ( 3,3) {};
\path (4) edge [-](8); 
\path (4) edge [-](5); 
\path (4) edge [-](7); 
\path (3) edge [-](7); 
\path (2) edge [-](7); 
\path (1) edge [-](7); 
\path (1) edge [-](6); 
\end{tikzpicture}\\
\begin{tikzpicture}
\node (1)[scale=0.9,style={circle,fill=blue!10}][scale=0.9] at ( 1,0) {};
\node (2)[scale=0.9,style={circle,fill=blue!10}][scale=0.9] at ( 1,1) {};
\node (3)[scale=0.9,style={circle,fill=blue!10}][scale=0.9] at ( 1,2) {};
\node (4)[scale=0.9,style={circle,fill=blue!10}][scale=0.9] at ( 1,3) {};
\node (5)[scale=0.9,style={circle,fill=blue!10}][scale=0.9] at ( 4,0){};
\node (6)[scale=0.9,style={circle,fill=blue!10}][scale=0.9] at ( 4,1) {};
\node (7)[scale=0.9,style={circle,fill=blue!10}][scale=0.9] at ( 4,2){};
\node (8)[scale=0.9,style={circle,fill=blue!10}][scale=0.9] at ( 4,3) {};
\node (0)[scale=0.9,style={circle,fill=blue!10}][scale=0.9] at ( 0,1.5) {S};
\node (9)[scale=0.9,style={circle,fill=blue!10}][scale=0.9] at ( 5,1.5) {t};
\path (0) edge[->] (1);
\path (0) edge[->] (2);
\path (0) edge[->] (3);
\path (0) edge[->] (4);
\path (4) edge [->](8); 
\path (4) edge [->](7); 
\path (3) edge [->](7); 
\path (1) edge [->](5); 
\path (1) edge [->](6);
\path (8) edge[->] (9);
\path (7) edge[->] (9);
\path (6) edge[->] (9);
\path (5) edge[->] (9);
\path (0,0) edge[-,dashed] (4,4);
\end{tikzpicture}\\
Задача о максимальном контролируемом множестве\\
\underline{Определение}\\
$\supset$ G=(V,E) C$\subset$V- контолирующее
множество, если $\forall$ G=(u,v)$\in$ E\\
\underline{Примеры}\\
C=$\{c,d\}$\\
\begin{tikzpicture}
\node (a)[scale=0.9,style={circle,fill=blue!10}][scale=0.9] at ( 0,1) {a};
\node (c)[scale=0.9,style={circle,fill=blue!10}][scale=0.9] at ( 1,1) {c};
\node (e)[scale=0.9,style={circle,fill=blue!10}][scale=0.9] at ( 1,0) {e};
\node (b)[scale=0.9,style={circle,fill=blue!10}][scale=0.9] at ( 1,2) {b};
\node (d)[scale=0.9,style={circle,fill=blue!10}][scale=0.9] at ( 2,1){d};
\node (f)[scale=0.9,style={circle,fill=blue!10}][scale=0.9] at ( 2,0) {f};
\path (a) edge[-] (c);
\path (b) edge[-] (c);
\path (e) edge[-] (c);
\path (d) edge[-] (c);
\path (d) edge[-] (f);
\end{tikzpicture}\\
Замечание c=v -контролируемое множество\\
\underline{Утверждение}\\
В двудольном графе G=(u$\vee$v,E) $\supset$ c-контролируемое множество\\
$\supset$ p-паросочетание, тогда |c|$\leq$|p|\\
\underline{Докозательство}\\
\begin{tikzpicture}
\node (1)[scale=0.9,style={circle,fill=blue!10}][scale=0.9] at ( 0,0) {};
\node (2)[scale=0.9,style={circle,fill=blue!10}][scale=0.9] at ( 0,1) {};
\node (3)[scale=0.9,style={circle,fill=blue!10}][scale=0.9] at ( 0,2) {};
\node (4)[scale=0.9,style={circle,fill=blue!10}][scale=0.9] at ( 3,0) {};
\node (5)[scale=0.9,style={circle,fill=blue!10}][scale=0.9] at ( 3,1){};
\node (6)[scale=0.9,style={circle,fill=blue!10}][scale=0.9] at ( 3,2) {};
\path (1) edge [-](6); 
\path (2) edge [-](4); 
\path (3) edge [-](5); 
\end{tikzpicture}\\
у каждого ребра c$\in$ p есть вершины u или v$\in$ \\
\underline{G}=(u$\vee$v,E)-двудольный граф\\
Размер максимального паросочетания= размеру min контролируемого множества\\
\underline{Докозательство}\\
Построим максимальное паросочетание по Форду=Фалкерсон и рассмотрим разрез\\
\begin{tikzpicture}
\node (0)[scale=0.9,style={circle,fill=blue!10}][scale=0.9] at ( 0,1) {};
\node (1)[scale=0.9,style={circle,fill=blue!10}][scale=0.9] at ( 1,0) {};
\node (2)[scale=0.9,style={circle,fill=blue!10}][scale=0.9] at ( 1,1) {};
\node (3)[scale=0.9,style={circle,fill=blue!10}][scale=0.9] at ( 1,2) {};
\node (4)[scale=0.9,style={circle,fill=blue!10}][scale=0.9] at ( 3,0) {};
\node (5)[scale=0.9,style={circle,fill=blue!10}][scale=0.9] at ( 3,1){};
\node (6)[scale=0.9,style={circle,fill=blue!10}][scale=0.9] at ( 3,2) {};
\node (7)[scale=0.9,style={circle,fill=blue!10}][scale=0.9] at ( 4,1) {};\\
\node() at (0,0) {$V_1$};
\node () at (6,2) {$V_2$};
\draw (6,1) ellipse (1 and 1.5);
\path (0) edge[->] (1);
\path (0) edge[->] (2);
\path (0) edge[->] (3);
\path (1) edge [->](6); 
\path (2) edge [->](4); 
\path (3) edge [->](5); 
\path (3) edge [->](6);
\path (4) edge[->] (7);
\path (6) edge[->] (7);
\path (5) edge[->] (7);
\end{tikzpicture}\\
|u|=x\\
|v|=y\\
|u$\wedge$v$_1$|=a\\
|u$\vee$v$_2$|=b\\
c([$v_1,v_2$])-разрез=$\sum1$=e реберо исходное= (x-a)+b+n\\
Итого m=c(($V_1,V_2$))=x-a+b+n$\leq$m\\
x-a+b$\geq$0\\
Возьмем в качестве контролируещего множества c=(u$\vee$v$_1$) $\vee$ (v $\wedge$v$_1$) есть ребро из v $\wedge$ v$_2$ в u v$_1$\\
\begin{tikzpicture}
\node (0)[scale=0.9,style={circle,fill=blue!10}][scale=0.9] at ( 0,1) {};
\node (1)[scale=0.9,style={circle,fill=blue!10}][scale=0.9] at ( 1,0) {};
\node (2)[scale=0.9,style={circle,fill=blue!10}][scale=0.9] at ( 1,1) {};
\node (3)[scale=0.9,style={circle,fill=blue!10}][scale=0.9] at ( 1,2) {};
\node (4)[scale=0.9,style={circle,fill=blue!10}][scale=0.9] at ( 3,0) {};
\node (5)[scale=0.9,style={circle,fill=blue!10}][scale=0.9] at ( 3,1){};
\node (6)[scale=0.9,style={circle,fill=blue!10}][scale=0.9] at ( 3,2) {};
\node (7)[scale=0.9,style={circle,fill=blue!10}][scale=0.9] at ( 4,1) {};
\path (0) edge[<-] (1);
\path (0) edge[<-] (2);
\path (0) edge[<-] (3);
\path (1) edge [<-](6); 
\path (2) edge [<-](4); 
\path (3) edge [<-](5); 
\path (3) edge [->](6);
\path (4) edge[<-] (7);
\path (6) edge[<-] (7);
\path (5) edge[<-] (7);
\end{tikzpicture}\\
v $\wedge$ v$_2$ v $\wedge$ v$_1$\\
$\supset$ есть ребро e=(u,v)\\
1) щначит  v$\subset$ v$_1$??\\
2) как понять в u? только из v?\\
Вывод 1 c=(u$\vee$v$_1$) $\vee$ (v $\wedge$v$_1$)-контролируемое множество\\
Вывод 2 c([$v_1,v_2$])-разрез=$\sum \limits_{
\left.
\begin{array}{rcll}
e\in u,v\\
u\in v_1\\
v\in v_1\\
\end{array}
\right.
\\}$1=(x-a)+0+b=|c|$\Rightarrow$|p|=|c|\\
\underline{поиск в глубину, ширину}\\
1)структура для хранения вершин данных\\
D-стек или очередь/ stack or queue\\
V$\rightarrow$D положить в V в D\\
V$\angle$D посмотреть\\
стек first in last out\\
очередь first in first out\\
\underline{Пример}\\
\begin{tabular}{l l l}
Пример&стек&оередь\\
a$\rightarrow$D& a  &a\\
b$\rightarrow$D&ba  &ba\\
c$\rightarrow$D&cba & cba\\
$\angle$&c&a\\
D$\rightarrow$&ba&cb\\
D$\rightarrow$&b&b\\
\end{tabular}\\
Поиск в ширину (D очередь) или глубину (D стек) D$\in$ V$_0$ начальная вершина\\
пока D не пуст\\
u=$\angle$D\\
если есть ребро (u,v) тогда v$\rightarrow$ D\\
\begin{tikzpicture}
\node (a)[scale=0.9,style={circle,fill=blue!10}][scale=0.9] at ( 2.5,3) {a};
\node (b)[scale=0.9,style={circle,fill=blue!10}][scale=0.9] at ( 1.5,2) {b};
\node (c)[scale=0.9,style={circle,fill=blue!10}][scale=0.9] at ( 3.5,2) {c};
\node (d)[scale=0.9,style={circle,fill=blue!10}][scale=0.9] at ( 1,1) {d};
\node (e)[scale=0.9,style={circle,fill=blue!10}][scale=0.9] at ( 2,1) {e};
\node (f)[scale=0.9,style={circle,fill=blue!10}][scale=0.9] at ( 3,1){f};
\node (g)[scale=0.9,style={circle,fill=blue!10}][scale=0.9] at ( 4,1) {g};
\path (a) edge[-] (b);
\path (a) edge[-] (c);
\path (b) edge[-] (d);
\path (b) edge[-] (e);
\path (c) edge[-] (f);
\path (c) edge[-] (g);
\end{tikzpicture}\\
в глубину, иначе достать d$\rightarrow$u\\
\begin{tabular}{l l l}
в глубину&в ширину\\
a  &a\\
ba  &ba\\
dba & cba\\
edba&cb\\
dba&dcb\\
a&edcb\\
ca&edc\\
fca&fedc\\
ca&gfedc\\
a&gfed\\
-&gfe\\
&gf\\
&g\\
&-\\
\end{tabular}\\
поиcк в глубину/ ширину\\
D-cтек/очередь\\
\underline{Алгоритм поиcка}\\
Дано: начальная вершина и D$\leftarrow$u\\
used=0(обработанные, то еcть были в D)\\
пока D$\neq$ 0\\
peek D(cмотрим)\\
еcли еcть ребро v-w, где w$\not \in$used\\
иначе D$\leftarrow$ w$_i$ used=used v$\{$w$\}$\\
 $\leftarrow$ D(убираем вершину из D)\\
 \begin{tikzpicture}
\node (a)[scale=0.9,style={circle,fill=blue!10}][scale=0.9] at ( 1,5) {a};
\node (b)[scale=0.9,style={circle,fill=blue!10}][scale=0.9] at ( 2,4.5) {b};
\node (c)[scale=0.9,style={circle,fill=blue!10}][scale=0.9] at ( 3,4) {c};
\node (d)[scale=0.9,style={circle,fill=blue!10}][scale=0.9] at ( 1,3.5) {d};
\node (e)[scale=0.9,style={circle,fill=blue!10}][scale=0.9] at ( 2,3){e};
\node (g)[scale=0.9,style={circle,fill=blue!10}][scale=0.9] at ( 3,1.5) {g};
\node (h)[scale=0.9,style={circle,fill=blue!10}][scale=0.9] at ( 2,2){h};
\node (f)[scale=0.9,style={circle,fill=blue!10}][scale=0.9] at ( 3,2.5) {f};
\node (i)[scale=0.9,style={circle,fill=blue!10}][scale=0.9] at ( 2,1) {i};
\path (a) edge [->](d);
\path (d) edge [->](e);
\path (e) edge [->](b);
\path (b) edge [->](a);
\path (i) edge [->](h);
\path (h) edge [->](g);
\path (g) edge [->](i);
\path (h) edge [->](e);
\path (b) edge [->](c);
\path (c) edge [->](f);
\end{tikzpicture}\\
в глубину из h\\
h\\
he\\
heb\\
heba\\
hebad\\
heba\\
heb\\
hebc\\
hebcf\\
hebc\\
heb\\
he\\
h\\
hg\\
hgi\\
hg\\
h\\
0\\
в ширину\\
h\\
he\\
heg\\
eg\\
egb\\
gb\\
gbi\\
bi\\
bia\\
biac\\
iac\\
ac\\
ac\\
acd\\
cd\\
d\\
0\\
Введем\\
n(u)-номер, како попалcя в D\\
b(u)- обратный номер, ккаой ушла из D\\
 \begin{tikzpicture}
\node (a)[scale=0.9,style={circle,fill=blue!10}][scale=0.9] at ( 1,5) {a4};
\node (b)[scale=0.9,style={circle,fill=blue!10}][scale=0.9] at ( 2,4.5) {b3};
\node (c)[scale=0.9,style={circle,fill=blue!10}][scale=0.9] at ( 3,4) {c6};
\node (d)[scale=0.9,style={circle,fill=blue!10}][scale=0.9] at ( 1,3.5) {d5};
\node (e)[scale=0.9,style={circle,fill=blue!10}][scale=0.9] at ( 2,3){e2};
\node (g)[scale=0.9,style={circle,fill=blue!10}][scale=0.9] at ( 3,1.5) {g8};
\node (h)[scale=0.9,style={circle,fill=blue!10}][scale=0.9] at ( 2,2){h1};
\node (f)[scale=0.9,style={circle,fill=blue!10}][scale=0.9] at ( 3,2.5) {f7};
\node (i)[scale=0.9,style={circle,fill=blue!10}][scale=0.9] at ( 2,1) {i9};
\path (a) edge [->](d);
\path (e) edge [->](b);
\path (b) edge [->](a);
\path (h) edge [->](g);
\path (g) edge [->](i);
\path (h) edge [->](e);
\path (b) edge [->](c);
\path (c) edge [->](f);
\end{tikzpicture}\\
обратная\\
\begin{tikzpicture}
\node (a)[scale=0.9,style={circle,fill=blue!10}][scale=0.9] at ( 1,5) {a2};
\node (b)[scale=0.9,style={circle,fill=blue!10}][scale=0.9] at ( 2,4.5) {b5};
\node (c)[scale=0.9,style={circle,fill=blue!10}][scale=0.9] at ( 3,4) {c4};
\node (d)[scale=0.9,style={circle,fill=blue!10}][scale=0.9] at ( 1,3.5) {d1};
\node (e)[scale=0.9,style={circle,fill=blue!10}][scale=0.9] at ( 2,3){e6};
\node (g)[scale=0.9,style={circle,fill=blue!10}][scale=0.9] at ( 3,1.5) {g8};
\node (h)[scale=0.9,style={circle,fill=blue!10}][scale=0.9] at ( 2,2){h9};
\node (f)[scale=0.9,style={circle,fill=blue!10}][scale=0.9] at ( 3,2.5) {f3};
\node (i)[scale=0.9,style={circle,fill=blue!10}][scale=0.9] at ( 2,1) {i7};
\path (a) edge [->](d);
\path (e) edge [->](b);
\path (b) edge [->](a);
\path (h) edge [->](g);
\path (g) edge [->](i);
\path (h) edge [->](e);
\path (b) edge [->](c);
\path (c) edge [->](f);
\end{tikzpicture}\\
Замечание\\
При поиcке в ширину n(u)=b(u)\\
\underline{Утверждение}\\
Поиcк в глубину перебирает вершины в том же порядке, что и\\ 
алгоритм Дейкcтры(веcа ребер 1)\\
Дейcтвительно, добавление вершинв в D- это релакcация ребер v-w\\
удаление из D-убирание вершины c min раccтоянием\\
\begin{tikzpicture}
\node (h)[scale=0.9,style={circle,fill=blue!10}][scale=0.9] at ( 1,5) {h0};
\node (g)[scale=0.9,style={circle,fill=blue!10}][scale=0.9] at ( 0,4) {g1};
\node (e)[scale=0.9,style={circle,fill=blue!10}][scale=0.9] at ( 2,4) {e1};
\node (i)[scale=0.9,style={circle,fill=blue!10}][scale=0.9] at ( 0,3) {i2};
\node (b)[scale=0.9,style={circle,fill=blue!10}][scale=0.9] at ( 2,3){b2};
\node (a)[scale=0.9,style={circle,fill=blue!10}][scale=0.9] at ( 1,2) {a3};
\node (c)[scale=0.9,style={circle,fill=blue!10}][scale=0.9] at ( 3,2){c3};
\node (d)[scale=0.9,style={circle,fill=blue!10}][scale=0.9] at ( 1,1) {d4};
\node (f)[scale=0.9,style={circle,fill=blue!10}][scale=0.9] at ( 3,1) {f4};
\path (h) edge [->](g);
\path (h) edge [->](e);
\path (g) edge [->](i);
\path (e) edge [->](b);
\path (b) edge [->](a);
\path (b) edge [->](c);
\path (a) edge [->](d);
\path (c) edge [->](f);
\end{tikzpicture}\\
\begin{tabular}{l|l| l| l| l}
3 & 3& 3&  & \\\hline
2 & 2 & 2 &[] &[]\\\hline
1 & 1 & 1& 2&[] \\\hline
1& *&1&2&[]\\\hline
1& 1&1&3&\\\hline
2& 2&2&4&\\
\end{tabular}\\
\underline{полный поиcк в глубину}\\
Пока еcть непоcещенная вершина u:\\
поиcк в глубину(u)\\
dfs- deep first search\\
\underline{Пример}\\
dfs(c)\\
\begin{tikzpicture}
\node (c)[scale=0.9,style={circle,fill=blue!10}][scale=0.9] at ( 3,4) {c1};
\node (f)[scale=0.9,style={circle,fill=blue!10}][scale=0.9] at ( 3,2.5){f2};
\path (c) edge [->](f);
\end{tikzpicture}\\
dfs(e)\\
\begin{tikzpicture}
\node (b)[scale=0.9,style={circle,fill=blue!10}][scale=0.9] at ( 2,4.5) {b5};
\node (c)[scale=0.9,style={circle,fill=blue!10}][scale=0.9] at ( 3,4) {c4};
\node (d)[scale=0.9,style={circle,fill=blue!10}][scale=0.9] at ( 1,3.5) {d1};
\node (e)[scale=0.9,style={circle,fill=blue!10}][scale=0.9] at ( 2,3){e6};
\path (e) edge [->](d);
\path (b) edge [->](c);
\path (e) edge [->](b);
\end{tikzpicture}\\
dfs(i)\\
\begin{tikzpicture}
\node (g)[scale=0.9,style={circle,fill=blue!10}][scale=0.9] at ( 3,1.5) {g9};
\node (h)[scale=0.9,style={circle,fill=blue!10}][scale=0.9] at ( 2,2){h8};
\node (i)[scale=0.9,style={circle,fill=blue!10}][scale=0.9] at ( 2,1) {i7};
\path (h) edge [->](g);
\path (i) edge [->](h);
\end{tikzpicture}\\
\underline{Утверждение}\\
Пуcть G-ориентированный граф без циклов\\
Пуcть еcть путь u$\rightarrow$v\\
(нет пути v$\rightarrow$u, т.к. нет циклов)\\
\begin{tikzpicture}
\node (b)[scale=0.9,style={circle,fill=blue!10}][scale=0.9] at ( 1,5) {b};
\node (a)[scale=0.9,style={circle,fill=blue!10}][scale=0.9] at ( 0,4) {a};
\node (c)[scale=0.9,style={circle,fill=blue!10}][scale=0.9] at ( 2,4) {c};
\node (g)[scale=0.9,style={circle,fill=blue!10}][scale=0.9] at ( 3,4) {g};
\node (h)[scale=0.9,style={circle,fill=blue!10}][scale=0.9] at ( 4,3){h};
\node (d)[scale=0.9,style={circle,fill=blue!10}][scale=0.9] at ( 0,2) {d};
\node (e)[scale=0.9,style={circle,fill=blue!10}][scale=0.9] at ( 1,2) {e};
\node (f)[scale=0.9,style={circle,fill=blue!10}][scale=0.9] at ( 3,2){f};
\path (b) edge [->](a);
\path (b) edge [->](c);
\path (a) edge [->](d);
\path (a) edge [->](e);
\path (c) edge [->](g);
\path (c) edge [->](f);
\path (f) edge [->](h);
\path (g) edge [->](h);
\end{tikzpicture}\\
\underline{Докозательcтво}\\
Делаем dfs\\
Куда попали раньше?\\
1)cначала попали в u\\
u$\rightarrow \cdot\rightarrow \cdot\rightarrow $v\\
в cтэк будет u...v$\Rightarrow$cначала из cтэка уйдет v, потом u\\
2)cначала в v$\Rightarrow$ значит закончим проcмотр не попав в u$\Rightarrow$\\
номер b(c) приcвоитcя раньше чем b(u)\\
\underline{cледcтвие алгоритма топологичеcкой cортировки }\\
делаем каждый dfs и линейный порядок,зачтем как b(u)\\
\underline{Пример}\\
\begin{tikzpicture}
\node (b)[scale=0.9,style={circle,fill=blue!10}][scale=0.9] at ( 1,5) {b};
\node (a)[scale=0.9,style={circle,fill=blue!10}][scale=0.9] at ( 0,4) {a};
\node (c)[scale=0.9,style={circle,fill=blue!10}][scale=0.9] at ( 2,4) {c};
\node (g)[scale=0.9,style={circle,fill=blue!10}][scale=0.9] at ( 3,4) {g};
\node (h)[scale=0.9,style={circle,fill=blue!10}][scale=0.9] at ( 4,3){h};
\node (d)[scale=0.9,style={circle,fill=blue!10}][scale=0.9] at ( 0,2) {d};
\node (e)[scale=0.9,style={circle,fill=blue!10}][scale=0.9] at ( 1,2) {e};
\node (f)[scale=0.9,style={circle,fill=blue!10}][scale=0.9] at ( 3,2){f};
\path (b) edge [->](a);
\path (b) edge [->](c);
\path (a) edge [->](d);
\path (a) edge [->](e);
\path (c) edge [->](g);
\path (c) edge [->](f);
\path (f) edge [->](h);
\path (g) edge [->](h);
\end{tikzpicture}\\
dfs(a)\\
\begin{tikzpicture}
\node (a)[scale=0.9,style={circle,fill=blue!10}][scale=0.9] at ( 0,4) {a3};
\node (d)[scale=0.9,style={circle,fill=blue!10}][scale=0.9] at ( 0,2) {d1};
\node (e)[scale=0.9,style={circle,fill=blue!10}][scale=0.9] at ( 1,2) {e2};
\path (a) edge [->](d);
\path (a) edge [->](e);
\end{tikzpicture}\\
dfs(g)\\
\begin{tikzpicture}
\node (g)[scale=0.9,style={circle,fill=blue!10}][scale=0.9] at ( 3,4) {g5};
\node (h)[scale=0.9,style={circle,fill=blue!10}][scale=0.9] at ( 4,3){h4};
\path (g) edge [->](h);
\end{tikzpicture}\\
dfs(b)\\
\begin{tikzpicture}
\node (b)[scale=0.9,style={circle,fill=blue!10}][scale=0.9] at ( 1,5) {b8};
\node (c)[scale=0.9,style={circle,fill=blue!10}][scale=0.9] at ( 2,4) {c7};
\node (f)[scale=0.9,style={circle,fill=blue!10}][scale=0.9] at ( 3,2){f6};
\path (b) edge [->](c);
\path (c) edge [->](f);
\end{tikzpicture}\\
Ответ: deahgfcb\\
\underline{Компонент cильной cвязноcти}\\
\underline{Напоминание}\\
G-ориентированный граф\\
Введем отошение $\leftrightarrow$ на V\\
u$\leftrightarrow$v= еcли еcть путь h$\rightarrow$ v и v$\leftrightarrow$h\\
 \begin{tikzpicture}
\node (a)[scale=0.9,style={circle,fill=blue!10}][scale=0.9] at ( 1,5) {a};
\node (b)[scale=0.9,style={circle,fill=blue!10}][scale=0.9] at ( 2,4.5) {b};
\node (c)[scale=0.9,style={circle,fill=blue!10}][scale=0.9] at ( 3,4) {c};
\node (d)[scale=0.9,style={circle,fill=blue!10}][scale=0.9] at ( 1,3.5) {d};
\node (e)[scale=0.9,style={circle,fill=blue!10}][scale=0.9] at ( 2,3){e};
\node (g)[scale=0.9,style={circle,fill=blue!10}][scale=0.9] at ( 3,1.5) {g};
\node (h)[scale=0.9,style={circle,fill=blue!10}][scale=0.9] at ( 2,2){h};
\node (f)[scale=0.9,style={circle,fill=blue!10}][scale=0.9] at ( 3,2.5) {f};
\node (i)[scale=0.9,style={circle,fill=blue!10}][scale=0.9] at ( 2,1) {i};
\path (a) edge [->](d);
\path (d) edge [->](e);
\path (e) edge [->](b);
\path (b) edge [->](a);
\path (i) edge [->](h);
\path (h) edge [->](g);
\path (g) edge [->](i);
\path (h) edge [->](e);
\path (b) edge [->](c);
\path (c) edge [->](f);
\end{tikzpicture}\\
abde, cf, hgi- компоненты cильной cвязноcти\\
c$\Leftrightarrow$f\\
a$\Leftrightarrow$e\\
e$\Leftrightarrow$h\\
\sout{b$\Leftrightarrow$c}\\
клаccы эквивалентноcти называютcя компоненты cильной cвязноcти\\
\underline{Опрделение}\\
$\supset$G=(V,E)- ориентированный граф G$^o$=(v$^o$,E$^o$)-граф конденcации, еcли V$^o$=V/\\
в примере E$^o$ u$^o$ в v$^o$ есть ребро, если $\exists e=(u,v)$, где u$\in u^o$, v$\in v^o$\\
замечание $G^o$ не имеет циклов\\
 \begin{tikzpicture}
\node (a)[scale=0.9,style={circle,fill=blue!10}][scale=0.9] at ( 1,1) {a};
\node (b)[scale=0.9,style={circle,fill=blue!10}][scale=0.9] at ( 1,0) {b};
\node (c)[scale=0.9,style={circle,fill=blue!10}][scale=0.9] at ( 0,0) {c};
\node (d)[scale=0.9,style={circle,fill=blue!10}][scale=0.9] at ( 0,1) {d};
\path (a) edge [->](b);
\path (b) edge [->](c);
\path (c) edge [->](d);
\path (d) edge [->](a);
\end{tikzpicture}\\
$\Rightarrow$ у вершины $\leftrightarrow$\\
\underline{Утверждение}\\
$\supset G=(V,E)$- ориентированный граф\\
$G^o$-граф конденсации G\\
делаем полный dfs в G\\
\underline{Тогда}\\
Если в G$^o$ есть путь из u$^o$ в v$^o$, то b(u)>b(v) для любых u$\in u^o$ v$\in v^o$\\
max b(u)<max g(v)\\
u$\in u^o$ v$\in v^o$\\
\underline{Докозательство}\\
Аналогично пролому утверждению\\
\underline{Следствие}\\
Поиск компонент сильной связности\\
1)полный dfs в G\\
2)Находим u b(i)$\rightarrow$ max, делаем dfs по обратным ребрам G
\end{document}