\documentclass{article}

\usepackage[utf8]{inputenc}
\usepackage[russian]{babel}
\usepackage{amsmath}
\usepackage{framed}

\newenvironment{exercise}{%
\begin{framed}\par\noindent\slshape%
}%
{\end{framed}}

\title{Цепные дроби квадратичных иррациональностей}
\author{}
\date{}

\begin{document}
\maketitle

\section{Разложение квадратичной иррациональности в периодическую цепную дробь}
Наша задача~--- научиться раскладывать в цепные дроби квадратичные иррациональности, это числа вида $\frac{a + b\sqrt{c}}{d}$, где $a$, $b$, $c$ и $d$~--- целые числа. В частности, мы сможем раскладывать числа вида $\sqrt c$. Примеры квадратичных иррациональностей: $\sqrt{2}$, $3 + \sqrt{10}$, $\frac{1 + \sqrt5}2$.

Сначала вспомним процесс разложения в цепную дробь на примере числа $58/21$. В традиционной <<школьной>> записи он выглядит так:
$$\dfrac{58}{21}=
2 + \dfrac{16}{21}=
2 + \dfrac1{\dfrac{21}{16}}=
2 + \dfrac1{1 + \dfrac{5}{16}}=
2 + \dfrac1{1 + \dfrac{1}{\dfrac{16}{5}}}=
\mathbf2 + \dfrac1{\mathbf1 + \dfrac{1}{\mathbf3 + \dfrac{1}{\mathbf5}}}$$
Сокращенно такая дробь записывается $[2;1,3,5]$.

Более удобно записывать этот процесс следующим образом:

\begin{align*}
\alpha=\frac{58}{21}&=\mathbf{2}+\frac{16}{21}=2+\frac1\eta_1\\
\eta_1=\frac{21}{16}&=\mathbf{1}+\frac5{16}=1+\frac1\eta_2\\
\eta_2=\frac{16}5&=\mathbf{3}+\frac15=3+\frac1\eta_3\\
\eta_3=\frac51&=\mathbf{5}+0
\end{align*}

Жирным выделены целые части дробей, эти целые части образуют цепную дробь.

Вычисление разложения для квадратичной иррациональности происходит по этой же схеме. Практически, разложение иррациональности отличается тем, что требуется больше вычислений, при этом значительно усложняется выделение целой части числа. Фактически, самое сложное во всей задаче~--- это научиться правильно выделять целые части.

Разложим, например, число $\alpha=\frac{10+\sqrt3}{3}$:

\begin{align*}
\alpha=\frac{10+\sqrt3}{3}&=
\end{align*}

\noindent И здесь необходимо вычислить целую часть. Если бы можно было пользоваться калькулятором, мы бы узнали, что $\frac{10+\sqrt3}{3}=3.910683\ldots$, соответственно, целая часть равна 3.

Выделим целую часть без калькулятора. Для этого можно заменить $\sqrt3$ на его целую часть, она равна $1$. Это можно понять из того,что $1^2<3$, а $2^2>3$. После замены мы получим $\frac{10 + 1}{3}=\frac{11}3$, и целая часть этой дроби равна $3$. Соответственно, в исходной дроби тоже целая часть равна трем. Запишем эти же рассуждения еще раз:
$$\left[\frac{10+\sqrt3}{3}\right]=\left[\frac{10+[\sqrt3]}{3}\right]=\left[\frac{10+1}{3}\right]=
\left[\frac{11}{3}\right]=3$$

\begin{exercise}%
Докажите для целых $a$, $c>0$, $d>0$, что
$$
\left[\frac{a + \sqrt{c}}{d}\right]=
\left[\frac{a + [\sqrt{c}]}{d}\right]
$$
\end{exercise}

\begin{exercise}%
Приведите пример, который демонстрирует, что если перед корнем есть коэффициент, больший единицы, то указанный прием поиска целой части может не работать. Точнее:
$$
\left[\frac{a + b\sqrt{c}}{d}\right]\neq
\left[\frac{a + b[\sqrt{c}]}{d}\right]
$$
\end{exercise}

Теперь целая часть числа $\alpha$ известна, и можно продолжить вычисления:

\begin{align*}
\alpha=\frac{10+\sqrt3}{3}&=\mathbf3 + \frac{1+\sqrt3}{3}=3+\frac1\eta_1\\
\eta_1=\frac3{1+\sqrt3}&=
\end{align*}

Здесь необходимо опять выделить целую часть, но делать это неудобно, потому что корень находится в знаменателе.
Воспользуемся школьным приемом домножения на сопряженное:

\begin{align*}
\alpha=\frac{10+\sqrt3}{3}&=\mathbf3 + \frac{1+\sqrt3}{3}=3+\frac1\eta_1\\
\eta_1=\frac3{1+\sqrt3}&=\frac{3(\sqrt3-1)}{(1+\sqrt3)(\sqrt3-1)}=
\frac{3\sqrt3-3}{3-1}=\frac{3\sqrt3-3}{2}=
\end{align*}

Теперь корень в числителе, и мы можем воспользоваться рассмотренным выше приемом выделения целой части. Требуется только, чтобы у корня не было множителя. Поэтому заменим $3\sqrt3$ на $\sqrt{27}$. Здесь целая часть получилась 5, потому что $5^2<27$, а $6^2>27$.
Делаем замену, чтобы узнать целую часть: $\left[\frac{3\sqrt3-3}{2}\right]=\left[\frac{5-3}2\right]=1$. Теперь, когда целая часть известна, можно продолжить вычисления:

\begin{align*}
\alpha=\frac{10+\sqrt3}{3}&=\mathbf3 + \frac{1+\sqrt3}{3}=3+\frac1\eta_1\\
\eta_1=\frac3{1+\sqrt3}&=\frac{3(\sqrt3-1)}{(1+\sqrt3)(\sqrt3-1)}=
\frac{3\sqrt3-3}{3-1}=\frac{3\sqrt3-3}{2}=\\
&=\mathbf1+\frac{3\sqrt3-5}{2}=1+\frac1\eta_2\\
\eta_2=\frac2{3\sqrt3-5}&=\frac{2(3\sqrt3+5)}{(3\sqrt3-5)(3\sqrt3+5)}=\frac{6\sqrt3+10}{27 - 25}=3\sqrt3+5=
\end{align*}

Опять надо выделять целую часть. $[3\sqrt3+5]=[\sqrt{27}+5]=[[\sqrt{27}]+5]=[5+5]=10$. Продолжим, и доведем вычисления до конца:

\begin{align*}
\alpha=\frac{10+\sqrt3}{3}&=\mathbf3 + \frac{1+\sqrt3}{3}=3+\frac1\eta_1\\
\eta_1=\frac3{1+\sqrt3}&=\frac{3(\sqrt3-1)}{(1+\sqrt3)(\sqrt3-1)}=
\frac{3\sqrt3-3}{3-1}=\frac{3\sqrt3-3}{2}=\\
&=\mathbf1+\frac{3\sqrt3-5}{2}=1+\frac1\eta_2\\
\eta_2=\frac2{3\sqrt3-5}&=\frac{2(3\sqrt3+5)}{(3\sqrt3-5)(3\sqrt3+5)}=\frac{6\sqrt3+10}{27 - 25}=3\sqrt3+5=\\
&=\mathbf{10}+3\sqrt3-5=10+\frac1\eta_3\\
\eta_3=\frac1{3\sqrt3-5}&=\frac{3\sqrt3+5}{(3\sqrt3-5)(3\sqrt3-5)}=\frac{3\sqrt3+5}2=\mathbf5+\frac{3\sqrt3-5}2=\frac1\eta_4\\
\eta_4=\frac2{3\sqrt3-5}&=\eta_2
\end{align*}

\begin{exercise}%
Проверьте, что $\left[\frac{3\sqrt3+5}2\right]=5$.
\end{exercise}

Мы остановились, когда заметили, что очередной остаток $\eta$ повторился. Это значит, что все вычисления с этого момента начнут повторяться, соответственно, и целые части тоже будут повторяться. Пока что мы выписали следующие целые части:
$\alpha=[3;1,10,5,\ldots]$. Так как дальше мы получили $\eta_4=\eta_2$, целые части начнут повторяться с 10, и цепная дробь получилась $\alpha=[3;1,10,5,10,5,10,5,10,5,10,5,\ldots]$. Такую дробь с периодом принято записывать как $\alpha=[3;1,(10,5)]$, это и есть ответ задачи:
$$\frac{10+\sqrt3}{3}=[3;1,(10,5)]$$

Обратите внимание, дроби $[3;1,(10,5)]$ и $[3;1,10,5]$ это две совершенно разные дроби, вторая равна $\frac{219}{56}$.

Теорема Лагранжа утверждает, что у квадратичных иррациональностей цепная дробь всегда будет иметь период, т.е. при разложении мы обязательно получим остаток $\eta$, который совпадёт остатком, получавшимся раньше. Другие иррациональные числа не будут иметь периода в разложении.

\begin{exercise}%
Разложите в цепную дробь $\sqrt{42}$. Проверьте, что у вас получилось $[6; (2,12)]$.
\end{exercise}

\begin{exercise}%
Разложите в цепную дробь $\frac{1+\sqrt{3}}2$. Проверьте, что у вас получилось $[(1,2)]$.
\end{exercise}

\begin{exercise}%
Разложите в цепную дробь $1+\sqrt{3}$. Проверьте, что у вас получилось $[(2,1)]$.
\end{exercise}

\begin{exercise}%
Разложите в цепную дробь $\sqrt{23}$. Проверьте, что у вас получилось $[4; (1,3,1,8)]$.
\end{exercise}

Чтобы проверить вычисления, иногда можно пользоваться следующим утверждением. Числа вида $\sqrt a + [\sqrt a]$ имеют в цепной дроби чистый период. Например, $\sqrt {42} + [\sqrt {42}]=\sqrt{42}+6=[6; (2,12)] + 6 = [12; (2,12)]=[(12,2)]$. Аналогично, $\sqrt{23}+[\sqrt{23}]=\sqrt{23}+4=[4; (1,3,1,8)]+4=[8; (1,3,1,8)]=[(8,1,3,1)]$. В обоих случаях получен чистый период, т.е. числа в цепной дроби повторяются с самого начала.

\begin{exercise}
Какая квадратичная иррациональность соответствует цепной дроби $[(1)]$?
\end{exercise}

\section{Подбор приближения заданной точности}

В прошлом разделе мы определили, что $\alpha=\frac{10+\sqrt3}{3}=[3;1,(10,5)]$. Если оставить из цепной дроби только первые $k + 1$ чисел, получится подходящая дробь $\delta_k$. Например, $\delta_2=[3;1,10]$, а $\delta_4=[3;1,10,5,10]$. Вычислим подходящую дробь, которая приближает $\alpha$ с точностью как минимум $\frac{1}{1000}$.

Воспользуемся алгоритмом вычисления подходящих дробей, который мы обсуждали на паре:

\begin{center}
\begin{tabular}{c|c|c|c|c|c|c|c}
$i$&&0&1&2&3&4\\
\hline
\rule{0pt}{12pt}$a_i$&&3&1&10&\raisebox{1pt}{\fbox5}&10\\
\hline
\rule{0pt}{18pt} $\delta_i=\dfrac{p_i}{q_i}$&$\dfrac10$&$\dfrac31$&$\dfrac{\fbox4}1$&$\dfrac{\fbox{43}}{11}$&$\dfrac{219}{56}$&$\dfrac{2233}{571}$
\end{tabular}
\end{center}

Обведенные числа подсказывают, как проводились вычисления: $219 = 5\times43 + 4$.

Вспомним, что точность считается по одной из двух формул:
$$\left|\alpha-\dfrac{p_i}{q_i}\right|<\dfrac{1}{q_iq_{i+1}},\hskip 1em
\left|\alpha-\dfrac{p_i}{q_i}\right|<\dfrac{1}{q_i^2}$$

По этим формулам точность приближения дробью $\frac{43}{11}$ не превосходит $\frac{1}{11\times56}$ или $\frac{1}{11^2}$, в обоих случаях эта точность хуже, чем $\frac{1}{1000}$. Следующая дробь $\frac{219}{56}$ приближает с точностью $\frac1{56^2}<\frac1{1000}$. Соответственно, дробь $\frac{219}{56}$ является ответом, она приближаешь исходное число $\frac{10+\sqrt3}{3}$ с нужной точностью.

Проверим точность на калькуляторе:

$$\left|\frac{10+\sqrt3}{3}-\frac{219}{56}\right|\approx
|3,910683603-3,910714286|\approx0,000030683<0,001.$$

Заметим, что дробь $\frac{2233}{571}$ оказалась ненужной, ее можно было не вычислять. Но, если бы $56^2$ оказалось меньше $1000$, можно было бы вычислить эту дробь и дополнительно проверить $56\times571$. Раз это произведение соответствует требуемой точности, нужно сделать вывод, что дробь $\frac{219}{56}$ подходит.

\end{document}