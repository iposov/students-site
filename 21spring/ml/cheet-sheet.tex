\documentclass{article}

\usepackage[utf8]{inputenc}
\usepackage[russian]{babel}
\usepackage{amsmath}
\usepackage{hyperref}
\usepackage{framed}
\usepackage{xcolor}
\renewcommand{\c}[2]{$C_{#1}^{#2}$}
\newcommand{\cc}[2]{C_{#1}^{#2}}
\numberwithin{equation}{subsection}
\renewcommand{\[}{\begin{equation}}
\renewcommand{\]}{\end{equation}}
\newcommand{\bx}[1]{\fcolorbox{white}{gray!20}{#1}}

\newenvironment{exercise}{%
\begin{framed}
\par\noindent\slshape%
}%
{
\end{framed}}

\title{Шпаргалка по логике исчисления предикатов}
\author{}
\date{}

\begin{document}

\section{Формулы с одной переменной}

См. прошлое задание

\section{Формулы с двумя переменными}

\subsection{Симметричность}
\[X \vee Y = Y \vee X\]
\[XY = YX\]
\[X + Y = Y + X\]
\[X \Leftrightarrow Y = Y \Leftrightarrow X\]
\[X \Rightarrow Y \ne Y \Rightarrow X\]

\subsection{Отрицание}
\[\overline{X\vee Y} = \overline{X}\cdot\overline{Y}\]
\[\overline{X\cdot Y} = \overline{X}\vee\overline{Y}\]
\[\overline{X\Leftrightarrow Y} = \overline{X}+\overline{Y}\]
\[\overline{X + Y} = \overline{X}\Leftrightarrow\overline{Y}\]

\subsection{Связки}
\[X\Rightarrow Y = \overline{X} \vee Y \]
\[X\Leftrightarrow Y = (X\Rightarrow Y)(Y\Rightarrow X)\]

\subsection{Выражение одних связок через другие}

Ниже должны быть примеры вывода этих формул из других (пока нет)

\[X + Y = (\overline{X} \vee \overline{Y})(X \vee Y)\]
\[X \Leftrightarrow Y = (\overline{X} \vee Y)(X \vee \overline{Y})\]
\[X + Y = \overline{X}Y + X\overline{Y}\]
\[X \Leftrightarrow Y = \overline{X}\cdot\overline{Y} \vee XY\]
\[X \vee Y = X + Y + XY\]
\[X \Rightarrow Y = 1 + X + XY\]

\section{Формулы с тремя переменными}
\subsection{Ассоциативность операций}
\[(X \vee Y) \vee Z = X \vee (Y \vee Z)\]
\[(X \cdot Y) \cdot Z = X \cdot (Y \cdot Z)\]
\[(X + Y) + Z = X + (Y + Z)\]
\[(X \Leftrightarrow Y) \Leftrightarrow Z = X \Leftrightarrow (Y \Leftrightarrow Z)\]
\[(X \Rightarrow Y) \Rightarrow Z \ne X \Rightarrow (Y \Rightarrow Z)\]

\subsection{Дистрибутивность}
\[(X\vee Y)Z = XZ \vee YZ\]
\[XY \vee Z = (X \vee Z)(Y \vee Z)\]
\[(X+ Y)Z = XZ + YZ\]

\section{Примеры упрощений}
\subsection{Пример 1.
$(\overline {a\Leftrightarrow (bc+ c)} + c)\vee \overline {\overline b+\overline a}$
}

Сначала упростим $\overline {\overline b+\overline a}$ . Здесь мы будем пользоваться в первую очередь формулой $ X + 1 = \overline X$ . Заменим внутренние отрицания: $\overline {(b + 1) +(a + 1)}$, заменим внешнее отрицание: $ 1 + (b + 1) + (a + 1)$, дальше раскроем скобки, пользуясь симметричность и ассоциативностью сложения: $ a + b + 1 + 1 + 1 = a + b + 1 $ .

Итого, $\overline {\overline b+\overline a} = a + b + 1 $

Раскроем часть $\overline {a\Leftrightarrow (bc+ c)} + c $. Воспользуемся тем, что $ X\Leftrightarrow Y=\overline{X + Y} = 1 + X + Y $ :
\begin{align}
\overline {a\Leftrightarrow (bc+ c)} + c = \\
\overline {\bx{1 + a + (bc + c)}} + c = \\
\bx{1 + 1 + a + bc + c} + c = \\
\bx{a + bc + c} + c = \\
\bx{a + bc}
\end{align}

Объединяем обе части, получается:

\begin{align}
(\overline {a\Leftrightarrow (bc+ c)} + c)\vee \overline {\overline b+\overline a} = \\
\bx{(a + bc)} \vee \bx{(a + b + 1)}
\end{align}

Теперь используем, что $X + Y = \overline{X}Y \vee X\overline{Y}$,
$X \Leftrightarrow Y = \overline{X}\cdot\overline{Y} \vee XY$

Продолжение следует
%и после этого раскрываем скобки. Это обычное, привычное нам раскрытие скобок, оно работает, из-за того, что здесь выполнены все законы дистрибутивности и симметричности.
%\begin{align}
%{} = \bx{\overline{}bc \vee a\overline{}} \vee \overline{a + b} =
%\overline{a}bc \vee a(\overline b \vee \overline c) \vee (a \Leftrightarrow b) = \\
%\overline{a}bc \vee a\overline b \vee a\overline c \vee (\overline{a}\overline{b} \vee ab) =
%\end{align}
\end{document}
