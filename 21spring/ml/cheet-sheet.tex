\documentclass{article}

\usepackage[utf8]{inputenc}
\usepackage[russian]{babel}
\usepackage{amsmath}
\usepackage{hyperref}
\usepackage{framed}
\renewcommand{\c}[2]{$C_{#1}^{#2}$}
\newcommand{\cc}[2]{C_{#1}^{#2}}
\numberwithin{equation}{subsection}
\renewcommand{\[}{\begin{equation}}
\renewcommand{\]}{\end{equation}}

\newenvironment{exercise}{%
    \begin{framed}
        \par\noindent\slshape%
        }%
        {
    \end{framed}}

\title{Шпаргалка по логике исчисления предикатов}
\author{}
\date{}

\begin{document}

    \section{Формулы с одной переменной}

    См. прошлое задание

    \section{Формулы с двумя переменными}

    \subsection{Отрицание}
    \[\overline{X\vee Y} = \overline{X}\cdot\overline{Y}\]
    \[\overline{X\cdot Y} = \overline{X}\vee\overline{Y}\]
    \[\overline{X\Leftrightarrow Y} = \overline{X}+\overline{Y}\]
    \[\overline{X + Y} = \overline{X}\Leftrightarrow\overline{Y}\]

    \subsection{Связки}
    \[X\Rightarrow Y = \overline{X} \vee Y \]
    \[X\Leftrightarrow Y = (X\Rightarrow Y)(Y\Rightarrow X)\]

    \subsection{ДНФ и КНФ}
    Здесь пока специально не указан ответ
    \[X + Y = (? \vee ?)(? \vee ?)\]
    \[X \Leftrightarrow Y = (? \vee ?)(? \vee ?)\]
    \[X + Y = ?\cdot? \vee ?\cdot?\]
    \[X \Leftrightarrow Y = ?\cdot? \vee ?\cdot?\]

    \section{Формулы с тремя переменными}
    \subsection{Ассоциативность операций}
    \[(X \vee Y) \vee Z = X \vee (Y \vee Z)\]
    \[(X \cdot Y) \cdot Z = X \cdot (Y \cdot Z)\]
    \[(X + Y) + Z = X + (Y + Z)\]
    \[(X \Leftrightarrow Y) \Leftrightarrow Z = X \Leftrightarrow (Y \Leftrightarrow Z)\]
    \[(X \Rightarrow Y) \Rightarrow Z \ne X \Rightarrow (Y \Rightarrow Z)\]

    \subsection{Дистрибутивность}
    \[(X\vee Y)Z = XZ \vee YZ\]
    \[XY \vee Z = (X \vee Z)(Y \vee Z)\]


\end{document}
