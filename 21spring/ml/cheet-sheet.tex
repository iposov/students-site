\documentclass{article}

\usepackage[utf8]{inputenc}
\usepackage[russian]{babel}
\usepackage{amsmath}
\usepackage{hyperref}
\usepackage{framed}
\usepackage{xcolor}
\renewcommand{\c}[2]{$C_{#1}^{#2}$}
\newcommand{\cc}[2]{C_{#1}^{#2}}
\numberwithin{equation}{subsection}
\renewcommand{\[}{\begin{equation}}
\renewcommand{\]}{\end{equation}}
\newcommand{\bx}[1]{\fcolorbox{white}{gray!20}{#1}}
\newcommand{\NE}[1]{\overline{\vphantom{|} #1}}
\newcommand{\IMPL}{\Rightarrow}
\newcommand{\EQ}{\Leftrightarrow}
\newcommand{\OR}{\vee}

\newenvironment{exercise}{%
\begin{framed}
\par\noindent\slshape%
}%
{
\end{framed}}

\title{Шпаргалка по логике исчисления предикатов}
\author{}
\date{}

\begin{document}

\section{Формулы с одной переменной}

См. прошлое задание

\section{Формулы с двумя переменными}

\subsection{Симметричность}
\[X \vee Y = Y \vee X\]
\[XY = YX\]
\[X + Y = Y + X\]
\[X \Leftrightarrow Y = Y \Leftrightarrow X\]
\[X \Rightarrow Y \ne Y \Rightarrow X\]

\subsection{Отрицание}
\[\NE{X\vee Y} = \NE{X}\cdot\NE{Y}\]
\[\NE{X\cdot Y} = \NE{X}\vee\NE{Y}\]
\[\NE{X\Leftrightarrow Y} = \NE{X}+\NE{Y}\]
\[\NE{X + Y} = \NE{X}\Leftrightarrow\NE{Y}\]

\subsection{Выражение одних связок через другие}

\[X\Rightarrow Y = \NE{X} \vee Y \]
\[X\Leftrightarrow Y = (X\Rightarrow Y)(Y\Rightarrow X)\]

Все следующие формулы в этом подразделе могут быть выведены из других формул этого документа. Примеры вывода приведены в разделе~\ref{sec:examples}.

\[X + Y = (\NE{X} \vee \NE{Y})(X \vee Y)\]
\[X \Leftrightarrow Y = (\NE{X} \vee Y)(X \vee \NE{Y})\]
\[X + Y = \NE{X}Y + X\NE{Y}\]
\[X \Leftrightarrow Y = \NE{X}\cdot\NE{Y} \vee XY\]
\[X \vee Y = X + Y + XY\]
\[X \Rightarrow Y = 1 + X + XY\]

\section{Формулы с тремя переменными}
\subsection{Ассоциативность операций}
\[(X \vee Y) \vee Z = X \vee (Y \vee Z)\]
\[(X \cdot Y) \cdot Z = X \cdot (Y \cdot Z)\]
\[(X + Y) + Z = X + (Y + Z)\]
\[(X \Leftrightarrow Y) \Leftrightarrow Z = X \Leftrightarrow (Y \Leftrightarrow Z)\]
\[(X \Rightarrow Y) \Rightarrow Z \ne X \Rightarrow (Y \Rightarrow Z)\]

\subsection{Дистрибутивность}
\[(X\vee Y)Z = XZ \vee YZ\]
\[XY \vee Z = (X \vee Z)(Y \vee Z)\]
\[(X+ Y)Z = XZ + YZ\]

\section{Примеры упрощений}
\label{sec:examples}
\subsection{Пример 1.
$(\NE {a\Leftrightarrow (bc+ c)} + c)\vee \NE {\NE b+\NE a}$
}

Сначала упростим $\NE {\NE b+\NE a}$ . Здесь мы будем пользоваться в первую очередь формулой $ X + 1 = \NE X$ . Заменим внутренние отрицания: $\NE {(b + 1) +(a + 1)}$, заменим внешнее отрицание: $ 1 + (b + 1) + (a + 1)$, дальше раскроем скобки, пользуясь симметричность и ассоциативностью сложения: $ a + b + 1 + 1 + 1 = a + b + 1 $ .

Итого, $\NE {\NE b+\NE a} = a + b + 1 $

Раскроем часть $\NE {a\Leftrightarrow (bc+ c)} + c $. Воспользуемся тем, что $ X\Leftrightarrow Y=\NE{X + Y} = 1 + X + Y $ :
\begin{align}
\NE {a\Leftrightarrow (bc+ c)} + c = \\
\NE {\bx{1 + a + (bc + c)}} + c = \\
\bx{1 + 1 + a + bc + c} + c = \\
\bx{a + bc + c} + c = \\
\bx{a + bc}
\end{align}

Объединяем обе части, получается:

\begin{align}
(\NE {a\Leftrightarrow (bc+ c)} + c)\vee \NE {\NE b+\NE a} = \\
\bx{(a + bc)} \vee \bx{(a + b + 1)}
\end{align}

Далее есть разные пути. Можно свести все сложения к дизъюнкциям. Или дизъюнкцию к сложению.
Сначала пойдем по первому пути. Чтобы превратить сложения в дизъюнкции можно воспользоваться следующими формулами:

Теперь используем, что $X + Y = \NE{X}Y \vee X\NE{Y}$,
$X \Leftrightarrow Y = \NE{X}\cdot\NE{Y} \vee XY$.

\begin{align}
\bx{(a + bc)} \vee \bx{(a + b + 1)} = \\
\NE{a}bc \OR a\NE{bc} \vee a\EQ b =
%\NE{a}bc \vee a(\NE b \vee \NE c) \vee (a \Leftrightarrow b) = \\
%\NE{a}bc \vee a\NE b \vee a\NE c \vee (\NE{a}\NE{b} \vee ab) =
\end{align}
\end{document}
