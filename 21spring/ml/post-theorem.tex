\documentclass{article}

\usepackage[T2A]{fontenc}
\usepackage[utf8]{inputenc}
\usepackage[russian]{babel}
\usepackage{amsmath}
\usepackage{hyperref}
\usepackage{framed}
\usepackage{wasysym}
\usepackage{graphicx}
\usepackage{svg}

\newcommand{\impl}{\Rightarrow}
\renewcommand{\lor}{\vee}
\renewcommand{\land}{\cdot}
\newcommand{\xor}{+}
\newcommand{\eqiv}{\Leftrightarrow}
\newcommand{\lne}[1]{\overline{#1}}

\newenvironment{exercise}{%
    \begin{framed}\par\noindent\slshape%
    }%
    {\end{framed}}

\title{Теорема Поста}
\author{}
\date{}

\begin{document}
    \maketitle

    \section{Логическая функция}
    Логическая функция — это любая функция, аргументы которой это нули и единицы, и результат — 0 или 1.
    Например, \[f(x, y, z) = \max(x, y, z),\] или
    \[g(x, y, z, t) = \begin{cases}
        1, & x + y + z + t \ge 2\\
        0, & \text{иначе}
    \end{cases}\]
    или
    \[h(x, y, z) = (x \impl y) \impl z\]
    или
    \[k(x, y) = \begin{cases}
                    0, & \text{если } x=0, y=0\\
                    1, & \text{если } x=0, y=1\\
                    0, & \text{если } x=1, y=0\\
                    0, & \text{если } x=1, y=1
    \end{cases}\]
    Попробуем подставить разные аргументы в функции: $f(0, 0, 0) = 0$, $f(0, 1, 0) = 1$, $g(0, 1, 0, 1) = 1$,
    $g(0, 0, 0, 1) = 1$, $h(0, 0, 0) = 0$, $h(1, 1, 1) = 1$, $k(0, 0)=0$, $k(0, 1)=1$.

    Чаще всего логические функции мы будем описывать логическими связками $\lor$ $\land$ $\xor$ $\impl$ $\eqiv$, как была задана функция $h$, или таблицей истинности, как была задана функция $k$.

    \section{Композиция функций}
    Как и из любых других функций, из логических функций можно составлять композиции.
    Т.е. подставлять результаты вычисления одних функций в качестве аргументов других функций.
    Например, мы можем написать функцию $m(x, y) = f(g(x, x, x, x), h(x, x, y), y)$.
    Давайте разберемся, что это за функция:
    \begin{enumerate}
        \item \[g(x, x, x, x)=\begin{cases}
                                  g(0, 0, 0, 0)=0, & \text{если } x = 0 \\
                                  g(1, 1, 1, 1)=1, & \text{если } x = 1
        \end{cases} = x\]
        \item \[h(x, x, y)=(x \impl x) \impl y = 1 \impl y = y\]
        \item \[f(g(x, x, x, x), h(x, x, y), y) = f(x, y, y) = \max(x, y, y) = \max(x, y) = x \lor y\]
    \end{enumerate}

    \section{Классы и классы замкнутости}
    Все логические функции образуют множество.
    Класс функций — это любое подмножество этого множества.
    Например, рассмотрим класс $K_1$ функций, которые имеют две переменные, и такие, что $f(x, x) = x$.
    Проверим разные функции на принадлежность классу:
    \[f_1(x, y) = x \lor y \qquad f_1(x,x) = x \qquad f_1\in K_1\]
    \[f_2(x, y) = x \impl y \qquad f_2(x,x) = 1 \qquad f_2 \notin K_1\]
    \[f_3(x, y) = \lne x y \qquad f_3(x,x) = \lne xx = 0 \qquad f_3 \notin K_1\]
    \[f_4(x, y, z) = xyz \qquad f_4 \notin K_1 \qquad \text{т.к. три переменных}\]

    Или рассмотрим класс $K_2$ функций, от любого количества переменных, которые не изменяются при любой перестановке аргументов.
    \[f_1(x, y) = x \lor y \qquad f_1(x, y)=f_1(y, x) \qquad f_1\in K_2\]
    \[f_2(x, y) = x \impl y \qquad f_2(x, y)\ne f_2(y, x) \qquad f_2\notin K_2\]
    \[f_3(x, y, z) = xy+xz+yz \qquad \begin{split}&f_3(x, y, z)=f_3(x, z, y)=\\=&f_3(y, x, z)=f_3(y, z, x)=\\=&f_3(z, x, y)=f_3(z, y, x)\end{split} \qquad f_3\in K_2\]
    \[f_4(x, y, z) = x + yz\qquad \text{например, } f_4(1, 0, 0)\ne f_4(0, 1, 0)\qquad f_4\notin K_2\].

    \subsection{Классы замкнутости}
    Классы замкнутости — это классы функций, такие, что любая композиция функций класса тоже принадлежит классу.
    Понятие композиции рассматривали на практике.

    Классов замкнутости много, нас интересуют только пять:
    \subsubsection{Класс $T_0$}
    Функции, «сохраняющие ноль». Это множество функций, которые при подстановке всех нулей выдают 0. Рассматривали на практике. Например,
    $f(x, y, z)=(x \impl y) \impl z$. Подставим нули: $f(0, 0, 0)=(0 \impl 0) \impl 0 = 1 \impl 0 = 0$, значит
    $f(0, 0, 0)=0$, значит $f\in T_0$. Там же на практике мы видели, что композиция функций, сохраняющих 0, тоже
    сохраняет 0.
    \subsubsection{Класс $T_1$}
    Класс аналогичен $T_0$, но сохраняет единицу
    \subsubsection{Класс $S$}
    Класс «самодвойственных функций». Это функции, которые совпадают со своей двойственной. Универсальный способ проверки — построить таблицу истинности исходной функции, двойственной, и сравнить их.
    \subsubsection{Класс $L$}
    Класс «линейных функций». Это функции, у которых в многочлене Жегалкина нет умножений. Например,
    \[f_5(x, y) = x + y \in L\]
    \[f_6(x, y, z) = 1 + x + z \in L\]
    \[f_7(x, y, z) = 1 + xyz \notin L\]
    \[f_1(x, y) = x \lor y = x + y + xy \notin L\]
    \subsubsection{Класс $M$}
    Класс «монотонных функций».
    Проверка, является ли функция монотонной, самая сложная по сравнению с четырьмя предыдущими классами, поэтому рассмотрим ее подробно.

    Функция $f(a_1, a_2, \ldots, a_n)$ называется монотонной, если при увеличении любых аргументов с 0 до 1 значение функции не уменьшается. Т.е. если $a_1\le b_1$, $a_2\le b_2$, \ldots  $a_n\le b_n$, то
    $f(a_1, a_2, \ldots a_n) <= f(b_1, b_2, \ldots b_n)$.

    Другими словами запрещена ситуация, что при увеличении некоторых аргументов с 0 до 1 значение функции уменьшается с 1 до 0.

    Например, рассмотрим функцию $f_8(x, y) = \ne{xy}$. Подставим сначала $x_1=0$, $y_1=0$, получится $f(0, 0)=1$.
    Подставим теперь $x_2=1$, $y_2=1$. Получится $f_8(1, 1)=0$. Т.е. значение функции уменьшилось, хотя аргументы увеличились: $f_8 \notin M$.

    При проверке функции на монотонность обычно легко объяснить, что она \emph{не} монотонна.
    Достаточно привести один пример, когда увеличение аргументов приводит к уменьшению значения.
    Именно такой пример мы привели в прошлом абзаце. Рассмотрим другие примеры проверки на монотонность.

    $f_9(x, y, z)=(xyz \lor \lne x\land\lne y\land\lne z) \xor \lne z$. Получается ли найти место, где нарушается
    монотонность? Пробуем искать, начнем со значения функции, когда все переменные 0: $f(0, 0, 0)=0$.
    Попробуем увеличить, например, $y$: $f(0, 1, 0)=1$ Уменьшилось ли значение функции? Нет.
    Т.е. пока мы не нашли нарушение монотонности.

    Попробуем увеличить теперь $x$: $f(1, 1, 0)=1$. Значение опять не уменьшилось. А ведь если бы получился 0,
    мы бы сразу сказали, что монотонность нарушилась.

    Увеличиваем теперь $z$: $f(1, 1, 1)=1$, значение не уменьшилось, значит, мы опять не смогли найти нарушение монотонности. Значит ли это, что функция монотонна? Возможно, мы плохо искали, и монотонность все-таки нарушена.

    Как искать нарушение монотонности? Следующий метод позволяет наглядно расположить все значения функции так, чтобы нарушение монотонности было заметно.

    Сначала нарисуем кубик, аналогичный тому, который мы рисовали для поиска минимального ДНФ, это рисунок~\ref{fig:cube-truth-table}.

    \begin{figure}
        \begin{center}
            \includesvg[width=0.5\textwidth]{monotone_pattern.svg}
        \end{center}
        \caption{Шаблон таблицы истинности в форме куба}
        \label{fig:cube-truth-table}
    \end{figure}

\end{document}
