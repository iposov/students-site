\documentclass{article}

\usepackage[utf8]{inputenc}
\usepackage[russian]{babel}
\usepackage{amsmath}
\usepackage{hyperref}
\usepackage{framed}
\usepackage{algorithm}
\usepackage{algpseudocode}

\newenvironment{exercise}{%
    \begin{framed}
        \par\noindent\slshape%
        }%
        {
    \end{framed}}

\title{Задачи на динамическое программирование}
\author{}
\date{}

\begin{document}
    \maketitle

    \section{Динамическое программирование}
    В следующих задачах требуется реализовать решение с помощью динамического программирования.
    В общем случае, задачу нужно переформулировать через вычисление значения функции $F(n)$ так, чтобы
    \begin{enumerate}
        \item Для некоторого значения $n=N$, $F(N)$~--- это искомый ответ;
        \item Значение функции $F(n)$ при любом $n$ можно вычислить, зная значения при меньших аргументах.
        Другими словами, есть алгоритм вычисления $F(n)$ на основе значений $F(i)$, где $i < n$.
    \end{enumerate}
    Далее, алгоритм вычисления ответа следующий:

    \begin{algorithmic}[1]
        \State Завести массив $f$ для хранения значений функции.
        \State Задать значения функции, например:
        \State \verb|f[0] = |
        \State \verb|f[1] = |
        \State \verb|f[2] = |
        \For {$i=3,4,\ldots, N$}
            \State Вычислить \verb|f[i]| на основе предыдущих значений \verb|f[j]|, где $j < i$.
        \EndFor
        \State Ответ. Вернуть \verb|F(N)|
    \end{algorithmic}

    В этой схеме часто есть изменения, например, бывает, что достаточно хранить не весь массив целиком, а только
    несколько последних значений.
    Например, при вычислении чисел Фибоначчи ($F(n) = f(n - 1) + f(n)$) достаточно хранить
    только два последних вычисленных значения в массиве.
    Есть и другие модификации, которые не будут встречаться в следующих задачах, поэтому не будем их сейчас обсуждать.

    \section{Задача о наборе суммы монетками}
    Даны номиналы монет $1 \le a_i \le 10^9$ при $1 \le i \le 20$, и дана сумма $1 \le N \le 10^9$.
    Определить, какое минимальное число монет необходимо, чтобы набрать ими указанную сумму.
    Например, если даны номиналы монет в $a_1=2$, $a_2=3$, $a_3=10$, то набрать сумму $N=7$ можно тремя монетами: $7 = 3 + 2 + 2$, а сумму $N = 21$ оптимально можно набрать пятью: $21 = 10 + 3 + 3 + 3 + 2$.

    В задаче будет три варианта по возрастанию сложности. По заданному набору номиналов монет и сумме необходимо:
    \begin{enumerate}
        \item определить только, можно ли эту сумму набрать;
        \item определить минимальное необходимое количество монет;
        \item определить и минимальное необходимое количество монет, и перечислить сами монеты.
    \end{enumerate}
    \subsection{Формат входного и выходного файла}
    В первой строке файла дано число $k$~--- количество различных номиналов монет.
    В следующих $k$ строках даны числа $a_i$. В последней строке находится число $N$.
    Пример из условия будет указан так:
    \begin{verbatim}
        3
        2
        3
        10
        21
    \end{verbatim}
    В выходной файл необходимо вывести, в зависимости от уровня сложности:
    \begin{enumerate}
        \item слова \verb|YES| или \verb|NO|, в зависимости от того, можно ли набрать указанную сумму.
        \item одно число — минимальное необходимое количество монет для набора суммы, или \verb|-1|, если это невозможно.
        \item В первой строке число, как в прошлом пункте, после этого в каждой строке по количеству монет соответствующего номинала.
    \end{enumerate}

    Например, в примере из условия необходимо вывести ответ:
    \begin{verbatim}
        5
        1
        3
        1
    \end{verbatim}
    Он означает, что необходимо минимум пять монет для набора суммы, при этом монет номинала $a_1=2$ требуется взять 1 штуку, монет номинала $a_2=3$ нужно 3 штуки, монет номинала $a_3=10$~--- одна штука.

    \subsection{Решение}
    Зафиксируем номиналы монет $a_1, a_2, \ldots, a_k$ и введем следующую функцию $F(n)$:

    \[F(n) = \text{минимальное количество монет,\newline которое необходимо, чтобы набрать сумму $n$}\]

    В этой задаче функция повторяет условие задачи.
    Так бывает не всегда, и часто требуется придумать какую-то функцию, которая не очень похожа на условие задачи, но, во-первых, позволяет получить ответ, во-вторых, может быть вычислена постепенно, как указано выше в описании общей схемы динамического программирования.

    Процесс вычисления этой функции обсуждался на лекции: $F(n) = min_{1\le i \le k} F(n - a_i)$.

    Чтобы решить последнюю версию задачи, в которой требуется восстановить набор монеток, нужно модифицировать функцию $F$.
    Она должна возвращать и минимальное количество монет, и монетку, которая помогла получить эту сумму:

    \[F(n) = \quad\begin{aligned}
                      &\text{минимальное количество монет, которое необходимо, чтобы набрать}\\
                      &\text{сумму $n$, и величина одной монет, которую надо использовать.}
    \end{aligned}\]

    Не удивляйтесь тому, что функция возвращает два значения.
    Это значит, что решение задачи требует хранения большого количества информации, необходимо хранить не один массив чисел со значением функции, а два.
    Или один массив с двумя значениями.

    Теперь, для вычисления значения функции
\end{document}
