\documentclass{article}

\usepackage[utf8]{inputenc}
\usepackage[russian]{babel}
\usepackage{amsmath}
\usepackage{hyperref}
\usepackage{framed}
\renewcommand{\c}[2]{$C_{#1}^{#2}$}
\newcommand{\cc}[2]{C_{#1}^{#2}}

\newenvironment{exercise}{%
    \begin{framed}
        \par\noindent\slshape%
        }%
        {
    \end{framed}}

\title{Алгоритм Евклида}
\author{}
\date{}

\begin{document}
    \maketitle


    \section{Определение}
    Алгоритм Евклида позволяет найти наибольший общий делитель двух чисел (НОД).
    Подробно про НОД вы узнаете на лекциях, сейчас предлагается попробовать это понятие на примерах.
    Например, НОД 15 и 10 равен 5, обозначим это так: $(15, 10) = 5$.
    Действительно, и 15, и 10 делятся на 5, но оба они ни на какое большее число не делятся.

    Другие примеры:

    \[(6, 8) = 2\]
    \[(100, 60) = 20\]
    \[(123, 321) = 3\]
    \[(5, 7) = 1\]
    \[(42, 0) = 42\]
    \[(20, 30) = 10\]

    Заметим, что НОД всегда хотя бы 1, потому что любые числа делятся на 1.


    \section{Алгоритм Евклида}

    Оказывается, что если у вас есть два числа $a$ и $b$, для которых вы считаете НОД, можно уменьшить одно
    из чисел, не меняя НОД.
    Если точнее, то $(a, b) = (a, b \bmod a)$.
    Здесь $x \bmod y$ это остаток от деления.
    Остатки изучались в школе, на лекции мы их вспомним, если не помните до лекции, спросите в Discord.

    Например, $(6, 20) = (6, 20 \bmod 6) = (6, 2)$.
    Можете убедиться, что действительно, $(6, 20) = 2$ и $(6, 2) = 2$.

    Теперь мы можем писать алгоритм Евклида.
    Например, если нужно вычислить $(46, 20)$, можно написать следующий ряд чисел:

    \[46, 20, 6, 2, 0\]

    Здесь первые два числа взяты из условия.
    Каждое следующее число~--- это остаток от деления предпредпоследнего на предпоследнее.
    Здесь $6 = 46 \bmod 2$, $2 = 20 \bmod 6$, $0 = 6 \bmod 2$.

    Этот ряд чисел говорит, что $(46, 20) = (20, 6) = (6, 2) = (2, 0) = 2$.
    Т.е. НОД всегда равен предпоследнему числу, перед нулём. Ответ в задаче: $(46, 20) = 2$.

    Соответственно, чтобы выполнить алгоритм Евклида, нужно выписать ряд чисел, вычисляя остатки.


    \section{Расширенный алгоритм Евклида}
    Оказывается, что НОД двух чисел всегда «линейно выражается» через эти числа.
    Другими словами, если у вас есть два числа $a$ и $b$, их НОД $d = (a, b)$, то можно подобрать целые
    $x$ и $y$, такие, что $d = ax + by$.

    Вспомним, что $2 = (46, 20)$.
    Для этого примера подберем $2 = 46\cdot(-3) + 20\cdot7$, т.е. $x=-3$, $y=7$.

    Чтобы уметь находить такие $x$ и $y$, нужно расширить алгоритм Евклида, т.е. проделывать дополнительные вычисления. Сначала давайте добавим в вычисления не только остатки от деления, а еще неполные частные. Например, при делении 46 на 20 получается $\dfrac{46}{20} = 2,3$, т.е. неполное частное равно 2, это округление вниз результата деления, и остаток 6. Можно записать это так: $46 - 2\cdot20 = 6$. В этом выражении видно и неполное частное, и остаток.

    Получается вот такая запись вычисления НОД 46 и 20
    \[46 \quad {}^{\phantom{10-10\cdot10}}\]
    \[20 \quad {}^{\phantom{10-10\cdot10}}\]
    \[6 \quad {}^{46 - 2\cdot20}\]
    \[2 \quad {}^{20 - 3\cdot6}\]
    \[0 \quad {}^{6 - 3\cdot2}\]

    Повторю, что это повторение вычислений из обычной версии алгоритма Евклида, но записано больше информации о том, как именно получались остатки.

    Последнее, что остается сделать, чтобы получить расширенный алгоритм Евклида~--- это добавить вычисления линейных комбинаций на каждом шаге. Исходные числа 46 и 20 выражаются через себя всегда одним и тем же способом: $46 = 1\cdot46+0\cdot20$ и $46 = 0\cdot46+1\cdot20$. Вы во всех задачах будете писать эти 1, 0 и 0, 1.

    \pagebreak

    Каждое следующее число в списке получается вычитанием линейных комбинаций друг из друга:

    \[46 = 1\cdot46+0\cdot20 \quad {}^{\phantom{10-10\cdot10}}\]
    \[20 = 0\cdot46+1\cdot20 \quad {}^{\phantom{10-10\cdot10}}\]
    \[6 = 1\cdot46+(-2)\cdot20 \quad {}^{46 - 2\cdot20}\]
    \[2 = (-3)\cdot46+5\cdot20 \quad {}^{20 - 3\cdot6}\]
    \[0 = \mbox{неважно} \quad {}^{6 - 3\cdot2}\]

    Надо один раз понять, как делаются эти вычисления, больше ничего сложного в алгоритме нет.
    Имеется в виду, например, что раз $6 = 46 - 2\cdot20$, то, заменив 46 и 20 на их линейные комбинации, получится, что $6 = (1\cdot46+0\cdot20) - 2\cdot(0\cdot46+1\cdot20) = (1 - 2\cdot 0)\cdot46 + (0-2\cdot0)\cdot20=1\cdot46+(-2)\cdot20$.

    Дальше аналогично. Самая частая проблема при решении — это запутаться в знаках, часто приходится {\em вычитать отрицательные} числа.

    Ответ в задаче:
    \begin{enumerate}
        \item НОД $(46,20) = 2$
        \item линейное представление $2 = (-3)\cdot46+5\cdot20$.
    \end{enumerate}

    Если вы найдете этот алгоритм в другом месте, там будут те же вычисления, но записанные иначе. Посмотрите, например, учебник Рыбина и Позднякова, он выложен на сайте. Обычно многое из того, что написано в последнем решении, опускается. Нет смысла постоянно повторять 46 и 20, например.
\end{document}
