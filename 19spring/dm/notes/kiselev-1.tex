% Этот шаблон документа разработан в 2014 году
% Данилом Фёдоровых (danil@fedorovykh.ru) 
% для использования в курсе 
% <<Документы и презентации в \LaTeX>>, записанном НИУ ВШЭ
% для Coursera.org: http://coursera.org/course/latex .
% Вы можете изменять, использовать, распространять
% этот документ любым способом по своему усмотрению. 
% В качестве благодарности автору вы можете сохранить 
% в начале документа данный текст или просто ссылку на
% http://coursera.org/course/latex
% Исходная версия Шаблона --- 
% https://www.writelatex.com/coursera/latex/1.1


\documentclass[a4paper,12pt]{article}

\usepackage{cmap}					% поиск в PDF
\usepackage[T2A]{fontenc}			% кодировка
\usepackage[utf8]{inputenc}			% кодировка исходного текста
\usepackage[english,russian]{babel}	% локализация и переносы

\usepackage{amsmath,amsfonts,amssymb,amsthm,mathtools} %AMS
\usepackage{icomma} % умная запятая

\usepackage{euscript}
\usepackage{mathrsfs}


\author{Киселев Д.А 8371}
\title{Конспект по дискретной математике.}
\date{\today}

\begin{document} % Конец преамбулы, начало текста.

\maketitle


11.02.19

\section{Алгоритмы с целыми числами.}

Опр: Деление с остатком.

Поделить $a \in Z$ на $b \in Z \backslash \{0\}$ - это найти $q,r\in Z:$

1. $a=b\cdot q+r$ $\hspace{30pt}$2. $0\leqslant r \leqslant |b|$

$a$ -Делимое $b$ -Делитель $q$ -неполное частное $r$ -остаток 

Примеры:

$\pm7$ поделить на $3$ 

$7=3\cdot 2+1$  $\hspace{20pt}$ $a=b\cdot q+r$

$7=3\cdot4+(-5)$ Не является делением с остатком, так как $r<0$

$-7=3\cdot(-2)+(-1)$

Утверждение: деление с остатком единственно

Док-во:

Пусть существуют 
$a=b\cdot q_1 +r_1$ и $a=b\cdot q_2 +r_2$ при $0\leqslant r_1, r_2 \leqslant |b|$

Вычтем из первого второе и получим: $0=b(q_1-q_2)+(r_1-r_2)$

$b(q_2-q_1)=r_1-r_2$

b может принимать значения 0 или $\pm b,\pm 2b, \pm 3b ...$ 

$r_1-r_2$ принадлежит интервалу $(-|b|;|b|)$, значит b $\neq \pm b,\pm 2b, \pm 3b ...$ 

$b=0$, когда $q_1=q_2$ $\Rightarrow$ $0=r_1-r_2$ $\Rightarrow$ $r_1=r_2$

Доказано то, что и требовалось доказать.

Определиние: $a\in Z$, $b\in Z$ 

$a  \vdots  b$

a делится на b (a кратно b), если остаток от деления a на b равен нулю

Перефраз: $a  \vdots  b$, если существует $q\in Z$ и $a=b\cdot q$
Примеры:

$9 \vdots 3$ $12\vdots 4$ $0\vdots 5$

Свойства делимости:

1.$\forall x \in Z$ $ \hspace{30pt}$ $x\vdots 1$

2.$\forall x\in Z \backslash \{0\}$  $ \hspace{20pt}$ $x\vdots x$

3.$\forall x,y,z\in Z$ $x\vdots y $ $y\vdots z $ $ \Rightarrow $ $x\vdots z$ Транзитивность

4.Если $x\vdots y $ $\Rightarrow$ $\pm x\vdots \pm y$ 

5.Если $x_i\vdots y $, то $(\lambda_1x_1+\lambda_2x_2+\ldots +\lambda_n x_n)\vdots y$  при $\lambda_i \in Z$

План док-ва 5 свойства:

1. $x\vdots y \Rightarrow \lambda x\vdots y$

$x\vdots y\lambda \Rightarrow \lambda x=y(\lambda q)$

2. $x_1\vdots y$ и $x_2\vdots y \Rightarrow (x_1+x_2)\vdots y $

$x_1=yq_1 x_2=yq_2 \Rightarrow(x_1+x_2)=y(q_1+q_2)$

Обозначим:

$a mod b$ -остаток от деления a на b

Пример:
$7 mod 3=1$ $-7 mod 3=2$

Системы счисления:

Определение:

$P$ система счислений (пэичная система счислений)

Числа записываются цифрами, цифр ровно p штук - от 0 до p-1.

Пример цифр в различных с/сч:

в 3ной: 0,1,2

в 16ной: 0,1...9,A,B,C,D,E,F

Число $x\in N$ записываеься $x=(C_n,C_{n-1} \ldots C_0)_P$, где $C_i$ -цифры. $P$ -основание с/сч

$x=C_n\cdot P^n+C_{n-1} \cdot P^{n-1} +\ldots + C_1\cdot P+C_0$

Пример:

$57121_{10}=5\cdot10^4+7\cdot10^3+1\cdot10^2+2\cdot 10+1$

Утверждение: Для любого $N$ и $p\in N$

1. Существует представление в Pичной с/сч

2. Это представление единственно, если запретить нулевые цифры в начале

Док-во:

2) Пусть

$x=(C_n \ldots C_0) =C_n\cdot P^n \ldots +C_1\cdot P+C_0=P(Y_1)+C_0$

$x=(d_n \ldots d_0) =d_n\cdot P^n \ldots +d_1\cdot P+d_0=P(Y_2)+d_0$

Получисли деление с остатком x на p.  $0\leq  c_0,d_0<p$

Т.к деление с остатком единственное, значит представление единственное

$C_0=d_0 \Rightarrow Y_1=Y_2$

Аналогично $C_1=d_1$ и т.д $\Rightarrow C_i=d_i$


\end{document} % Конец текста.

