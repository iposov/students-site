\documentclass{article}
 
%Russian-specific packages
%--------------------------------------
\usepackage[T2A]{fontenc}
\usepackage[utf8]{inputenc}
\usepackage[russian]{babel}
\usepackage{cancel}
\usepackage{ dsfont }
\usepackage{ amssymb }
\usepackage{graphicx}
\graphicspath{ {images/} }
\usepackage{amsmath}
%--------------------------------------
 
%Hyphenation rules
%--------------------------------------
\usepackage{hyphenat}
\hyphenation{ма-те-ма-ти-ка вос-ста-нав-ли-вать}
%--------------------------------------
 
 \title{Лекция по дискретной математике}
 \date{15 апреля 2019}
 
\begin{document}
 
\maketitle

$\pi(n)\sim n/log(n)$ 

Сколько простых чисел от 1 до n?

log(n)-кол-во цифр

$\varphi(n)=\varphi(p)\varphi(q)=(p-1)(q-1)$

$e-\forall$ от 2 до $\varphi(N)-2 (e;\varphi(n))=1$

d-Находится, как решение ed$_{\varphi(n)}\equiv1$

Почему работает?

т.е. $m^{ed}_N\equiv m$

Проверка:

$ed\equiv 1 \Longrightarrow ed=1+k\varphi(N) \Longrightarrow
m^{ed} _{N}\equiv m^{1+k\varphi(N)}_{N}\equiv m(m^{\varphi(n)})^k_{N} \equiv m*1^k_{N} \equiv m$

Пример:

$p=5$ $q=7$ $N=35$ $\varphi(W)=4*6=24$

$e=5$  $d=19$ $5*5_{24}=1$

Задача:

1) $e=7$  $d=?$ $7d_{24}=1$

N,e-откр. ключ 

N,d-прив.кл.

$(m^e)^d_{N}\equiv m$

Шифрование: $\bar{m}=m^e mod N$

Расшифрование: $m=\bar{m}^d mod N$

в д-ве было $m^{\varphi(N)}_{N}\equiv1$ верно если $(N,m)=1$

\underline{Теор.Эйлера}

1) Можно д-ть иначе, идея $N=pq$

$\varphi(N)=(p-1)(q-1)$

2) Если $(m,N)\ne1 \Longrightarrow (m,N)=p$ или $q \Longrightarrow$ знач.
$\varphi(N)  \Longrightarrow$ знач. d


\underline{Электронная подпись}

m-сообщение, хочет д-ть, что это его сообщение

$(N,d)$-закр.ключ

Эл.Подпись: $m^d mod N=\bar{m}$ 

Проверка ЭП: $m=\bar{m}^e mod N$

$[hash(m)]^d \mod N$-ср-ния сжимает биты

Как убедиться, что никто не подлжолил открытый ключ? 

Откр. ключ тоже подписывают (центр сертификации)

\underline{Многочлены}

Выражение вида $a_{n}x^n+a_{n-1}x^{n-1}+\dots+a_{0}x+a_{0}$

$a_{i}$-эл-т НОТА $(\mathds{R},\mathds{C},\mathds{Q},\mathds{Z}_{p})$

x-формальная переменная 

$p(x)$-мн-н с переменной "x"

$\deg p(x)=$ степень x при первом ненулевом коэф.

если $a_{n}\ne0\Longrightarrow \deg(a_{n}x^n+\dots=n$

$deg0=-\infty$ (нулевой многочлен)

Пример:
$deg(x^2+5)=2$

$deg(o_{x}^4+o_{x}^2+x-5)=1$

$deg(7)=0$

Действия с многочленами:

\begin{itemize}
	\item $(x^3+x)+(x^3-x^2-x-1)=2x^3-x^2-1$
	
    \item $(x^2+2x+5)(x-1)=x^3+x^2+3x-5$
\end{itemize}   

В общем виде:

Умножение:

$(a_{n}x^p+...+a_{0})(b_{m}x^m+...+b_{0})=C_{n+m}x^{n+m}+...+C_{0}$

$C_{i}=a_{0}b_{i}+a_{i}b_{i+1}+a_{i}b_{i+2}+...+a_{i}b_{0}=\Sigma^i_{k=0}a_{k}b_{i-k}$

Деление:

$x^2+1$ не делится на X т.к. $x^2+1=p(x)*x$

Утв. deg $p(x)q(x)=$ $degp(x)+degq(x)$

Д-во очевидно 

Деление с остатком:

Делитель: p(X),q(x)

Деление $p(x)=q(x)\varphi(x)+\Psi(x^i)$ при этом $deg r(x)=deg q(x^i)$

Утв.: Деление с остатком единственно

Д-во: 

$\sigma p(x)=q(x)\varphi_{1}(x)+r_{1}(x)$

$p(x)=q(x)\varphi_{2}(x)+r_{2}(x)$

$0=q(x)(\varphi_{1}(x)-\varphi_{2}(x))+(r_{1}(x)-r_{2}(x)\ne0=x^n$

Пример:

Частное: $2x^2+5x+3$

Делитель: $3x+1$

Остаток: $14/9$

Итог: $2/3x+13/9$

$2x^2+5x+3=(3x+1)(1/3x+13/9)+14/9$

Замечание: если $p(x),q(x) \in Z[x]$ целый коэф.

если $q(x)-x^n+...$-унит пр.

Тогда при делении результата $(\varphi(x)+\psi(x)\in \mathds{Z}[x] $

Корни мн-на:

$x_{0}$-корень $p[x]$,если $p(x_{0})=0$

Деление на $x-\varphi$

$\supset p(x)=(x-\varphi)q(x)+r$

Подставим вместо x a: $\supset x=a $

Д-во:

$p(a)=(a-a)\varphi(x)+\psi\Longrightarrow p(a)=r$ т.Безу

Значение мн-на при $x=a$ равно остатку от деления его на $x-a$

Следствие: $p(x);x-a\Leftrightarrow p(a)=0$ a-корень 

Рассуждение:

$\sqsupset$ есть мн-н $p(x)\sqsupset$ x-корень $\Longrightarrow p(x)=(x-x_{1}p_{1}(x)$

$\sqsupset$ $x_{2}$-корень $p_{1}(x)\Longrightarrow p(x)=(x-x_{1})(x-x_{2})p_{2}(x)$ и т.д.

$\Longrightarrow p(x)=(x-x_{1})(x-x_{2}...(x-x_{n}p_{x}(x)$ нет корней 

$degp(x)\geqslant k$ $deg=0$

Корни  p:-это корень p

Итого корней у мн-на $\geqslant deg p(x)$ если $p(a)\ne0$

Опр.: В общем случае $p(x)=(x-x_{1})^k...(x-x_{m})^{km}q(x)$

$x_{i}$-корень $p_{i}$ кратности k

Утв:$\sqsupset degp\leqslant r \sqsupset p(a_{1})=q(a_{1})$ т.е совпадает в n+1 точек

$\sqsupset degp\leqslant n p(q_{n}+1) < (a_{n}+1)$

Тогда: $p(x)=q(x)$

Д-во: $\sqsupset p(x)=p(x)-q(x)$

$degr(x)\leqslant n$ $r(a_{1})=0 \Longrightarrow n+1$ корня $\Longrightarrow r(x)=0\Longrightarrow p(x)-q(x)$

\end{document}
