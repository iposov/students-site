\documentclass[a4paper,12pt]{article}

\usepackage{cmap}					% поиск в PDF
\usepackage[T2A]{fontenc}			% кодировка
\usepackage[utf8]{inputenc}			% кодировка исходного текста
\usepackage[english,russian]{babel}	% локализация и переносы

\usepackage{amsmath,amsfonts,amssymb,amsthm,mathtools} %AMS
\usepackage{icomma} % умная запятая

\usepackage{euscript}
\usepackage{mathrsfs}

\usepackage{graphicx}
\graphicspath{{pictures/}}
\DeclareGraphicsExtensions{.pdf,.png,.jpg}

% новая команда \RNumb для вывода римских цифр
\newcommand{\RNumb}[1]{\uppercase\expandafter{\romannumeral #1\relax}}

\author{Киселев Д.А 8371}
\title{Конспект по математической логике и теории алгоритмов.}
\date{04.09.2019}

\begin{document} % Конец преамбулы, начало текста.

    \maketitle

    %\tableofcontents

    \section{Исчисление высказываний.}

    Логическая формула - это выражение со значениями $(0,1)$, переменными $(x,y,z\ldots)$ и операциями $(\cdot,\vee,\Longrightarrow\ldots)$

    Пример арифметического выражения:
    $$\frac{(z+x)+y-10}{2}$$

    Логическое выражение:

    $$ (1+x)\Longrightarrow(xy\Longleftrightarrow\bar y \bar z)$$

    \begin{center}
        $1$ - Значение $\hspace{20pt}$ $x,xy,yz$ - Переменные $\hspace{20pt}$ $\Longrightarrow, \Longleftrightarrow,\bar y \bar z $ - Операции.
    \end{center}

    Значения: 0 - ложь, 1 - истина

    \textbf{Операции:}
    \begin{enumerate}
        \item \textbf{Унарная операция:}

        \begin{itemize}
            \item Отрицание $\bar \cdot, \lnot$
        \end{itemize}

        \begin{table}[ht]
            \centering
            \begin{tabular}{|c|c|}
                \hline
                $x$ &$\bar x$ \\ \hline
                0 & 1\\
                1 & 0\\ \hline
            \end{tabular}
            \caption{Отрицание}
        \end{table}

        \item \textbf{Бинарные операции:}

        \begin{table}[ht]
            \centering
            \begin{tabular}{|c|c|c|c|c|c|c|}
                \hline
                $x$ & $y$ & $xy$ & $x\vee y$ & $x\Longrightarrow y$ & $x\Longleftrightarrow y$ & $x \oplus y$  \\ \hline
                0 & 0 & 0 & 0 & 1 & 1 & 0\\ \hline
                0 & 1 & 0 & 1 & 1 & 1 & 0\\ \hline
                1 & 0 & 0 & 1 & 0 & 0 & 1\\ \hline
                1 & 1 & 1 & 1 & 1 & 1 & 0\\ \hline
            \end{tabular}
            \caption{Бинарные операции}
        \end{table}

        \begin{itemize}
            \item Конъюнкция $\wedge, \&$

            Логическое "И"

            \item Дизъюнкция $\vee$

            Логическое "Или"

            \item Импликация $\Longrightarrow$

            X влечет У, одно следует из другого

            \item Иквивалентность $\Longleftrightarrow$

            Проверка равносильности

            \item Исключающее "или"

            Сложение по mod 2
        \end{itemize}

        \item \textbf{Другие операции:}

        \begin{table}[ht]
            \centering
            \begin{tabular}{|c|c|c|c|c|c|c|c|c|c|c|c|c|}
                \hline
                $x$ & $y$ & $\boldsymbol{0}$ & $x\bigtriangleup y$ & $x\bigtriangledown y$ & $x$ & $y$ & $x \downarrow y$ & $\bar y$ & $y \Longrightarrow x$ & $\bar x$& $x|y$ & $\boldsymbol{1}$   \\ \hline
                0 & 0 & 0 & 0 & 0 & 0 & 0 & 1 & 1 & 1 & 1 & 1 & 1 \\ \hline
                0 & 1 & 0 & 0 & 1 & 0 & 1 & 0 & 0 & 0 & 1 & 1 & 1 \\ \hline
                1 & 0 & 0 & 1 & 0 & 1 & 0 & 0 & 1 & 1 & 0 & 1 & 1 \\ \hline
                1 & 1 & 0 & 0 & 0 & 1 & 1 & 0 & 0 & 1 & 0 & 0 & 1 \\ \hline
            \end{tabular}
            \caption{Другие операции}
        \end{table}

        \begin{minipage}{0.4\textwidth}
            \begin{itemize}
                \item Ноль $\boldsymbol{0}$
                \item Запрет по Y $x\bigtriangleup y$
                \item Запрет по X $x\bigtriangledown y$
                \item X
                \item Y
                \item Стрелка Пирса $x \downarrow y$
            \end{itemize}
        \end{minipage}
        \hfill
        \begin{minipage}{0.4\textwidth}
            \begin{itemize}
                \item Отрицание Y $\bar y$
                \item Импликация $y \Longrightarrow x$
                \item Отрицание X $\bar x$
                \item Штрих Шеффера $x|y$
                \item Единица $\boldsymbol{1}$
            \end{itemize}
        \end{minipage}

    \end{enumerate}

    \textbf{Свойства операций:}
    \begin{enumerate}
        \item \textbf{Коммутативность}

        $\&,\vee,\Longleftrightarrow,\oplus$ - коммутативны.

        $xy=yx$ $\hspace{20pt}$ $x\vee y=y\vee x$ $\hspace{20pt}$ $ x\Longleftrightarrow y = y\Longleftrightarrow x$ $\hspace{20pt}$ $ x+y=y+x $

        В данном случае, проверка может быть такой:

        $xy=yx$ - Умножение. $ x\Longleftrightarrow y = y\Longleftrightarrow x$ - Равенство. $ x+y=y+x $ - Сумма в $Z_{2}$

        Универсальный способ проверки: Проверить равенство двух логический выражений с помощью таблицы.

        \begin{table}[ht]
            \centering
            \begin{tabular}{|c|c|c|c|}
                \hline
                $x$ & $y$ & $x\vee y$ & $y\vee x$ \\ \hline
                0 & 0 & $0\vee 0=0$ & $0\vee 0=0$ \\ \hline
                0 & 1 & $0\vee 1=1$ & $1\vee 0=1$ \\ \hline
                1 & 0 & $1\vee 0=1$ & $0\vee 1=1$ \\ \hline
                1 & 1 & $1\vee 1=1$ & $1\vee 1=1$ \\ \hline
            \end{tabular}
            \caption{Проверка равенства выражений по таблице}
        \end{table}

        \item \textbf{Некоммутативность}

        $\Longrightarrow $ - некоммутативная операция.

        $x\Longrightarrow y \neq y \Longrightarrow x$

        Проверка:

        \begin{table}[ht]
            \centering
            \begin{tabular}{|c|c|c|c|c|}
                \hline
                $x$ & $y$ & $x\Longrightarrow y$ & $y \Longrightarrow x$ & Проверка \\ \hline
                0 & 0 & $0\vee 0=1$ & $0\vee 0=1$ & + \\ \hline
                0 & 1 & $0\vee 1=1$ & $1\vee 0=0$ & - \\ \hline
                1 & 0 & $1\vee 0=0$ & $0\vee 1=1$ & - \\ \hline
                1 & 1 & $1\vee 1=1$ & $1\vee 1=1$ & + \\ \hline
            \end{tabular}
            \caption{Проверка коммутативности импликации}
        \end{table}

        В таблице видно, что импликация - некоммутативная операция.

        \item \textbf{Ассоциативность}
        \begin{itemize}
            \item $(x\vee y)\vee z= x\vee (y\vee z)$ - Дизъюнкция.
            \item $(xy)z=x(yz)$ - Конъюнкция.
            \item $(x+y)+z=x+(y+z)$ - Исключающее "или".
            \item $(x \Longleftrightarrow y)\Longleftrightarrow z = x\Longleftrightarrow (y \Longleftrightarrow z) $ - Иквивалентность.
        \end{itemize}

        \begin{table}[ht]
            \centering
            \begin{tabular}{|c|c|c|c|c|c|c|}
                \hline
                $x$ & $y$ & $z$ & $x\vee y$ & $\boldsymbol{(x\vee y)\vee z}$& $y\vee z$ & $\boldsymbol{x\vee (y\vee z)}$  \\ \hline
                0 & 0 & 0 & 0 & 0 & 0 & 0 \\ \hline
                0 & 0 & 1 & 0 & 1 & 1 & 1 \\ \hline
                0 & 1 & 0 & 1 & 1 & 1 & 1 \\ \hline
                1 & 0 & 0 & 1 & 1 & 0 & 1 \\ \hline
                0 & 1 & 1 & 1 & 1 & 1 & 1 \\ \hline
                1 & 0 & 1 & 1 & 1 & 1 & 1 \\ \hline
                1 & 1 & 0 & 1 & 1 & 1 & 1 \\ \hline
                1 & 1 & 1 & 1 & 1 & 1 & 1 \\ \hline
            \end{tabular}
            \caption{Проверка ассоциативности дизъюнкции}
        \end{table}

        \textbf {Неассоциативность:} Импликация.

        $(x\Longrightarrow y) \Longrightarrow z \neq x\Longrightarrow (y \Longrightarrow z)$

        \item \textbf{Дистрибутивность}

        \begin{itemize}
            \item $x(y\vee z)= xy\vee xz$
            \item $x(y+z)=xy+xz$
            \item $x\vee (yz)=(x\vee y)(x\vee z)$
        \end{itemize}

        \textbf{Недистрибутивность}
        $x+y\cdot z \neq (x+y)(x+z)$

        \item \textbf{Приоритет операций}

        \begin{description}
            \item[1.] Отрицание.
            \item[2.] Умножение, конъюнкция
            \item[3.] Дизъюнкция, исключающее "или"
            \item[4.] Импликация, иквивалентность
        \end{description}

        \item \textbf{Правила де Моргана}

        \begin{itemize}
            \item $ \overline{x\vee y} = \bar x \cdot \bar y $
            \item $\overline{x\cdot y} = \bar x \vee \bar y$
        \end{itemize}

        \item \textbf{Другие свойства}

        \begin{minipage}{0.4\textwidth}
            \begin{itemize}
                \item $\overline{\overline{x}}=x$
                \item $0x=0$
                \item $1x=x$
            \end{itemize}
        \end{minipage}
        \hfill
        \begin{minipage}{0.4\textwidth}
            \begin{itemize}
                \item $0\vee x=x$
                \item $1\vee x=1$
                \item $x\Longrightarrow y = \bar x \vee y$
            \end{itemize}
        \end{minipage}

    \end{enumerate}

    \begin{center}
        \textbf{Дизъюнктивная нормальная форма}
    \end{center}

    Нормальная форма - один из вариантов записи логического выражения.

    $xy\vee z=(x\vee z)(y\vee z)=xy\vee z\vee 0$

    $xy\vee z=(x\vee z)(y\vee z)$ - Дизъюнктивная нормальная форма (ДНФ) выражения.

    \textbf{Определение:}

    Выражение имеет ДНФ, если оно является дизъюнкцией нескольких конъюнктов.

    Конъюнкт - это конъюнкция литералов.

    Литерал - переменная или отрицание переменной.

    Пример:
    \begin{itemize}
        \item $xy\vee z$ $\hspace{20pt}$ $xy,z$ - Конъюнкты $\hspace{20pt}$ $x,y,z$ - Литералы

        \item $x\bar y z\vee x\bar y z\vee \bar y z$ $\hspace{20pt}$ 3 конъюнкта, 3 литерала

        \item $x\vee \bar y \vee z$ $\hspace{20pt}$ 3 конъюнкта по 1 литералу

        \item $\bar x$ $\hspace{20pt}$ 1 конъюнкт, 1 литерал
    \end{itemize}

    \textbf{Не ДНФ:} $x\vee 1$,$\hspace{20pt}$$(x\vee y)\cdot z$,$\hspace{20pt}$$x\vee y \vee z \vee x \Longrightarrow xy$

\end{document} % Конец текста.

