\documentclass{article}
 
%Russian-specific packages
%--------------------------------------
\usepackage[T2A]{fontenc}
\usepackage[utf8]{inputenc}
\usepackage[russian]{babel}
\usepackage{graphicx}
\graphicspath{ {images/} }
\usepackage{amsmath}
\usepackage{amssymb}
\usepackage{cancel}
%--------------------------------------
 
%Hyphenation rules
%--------------------------------------
\usepackage{hyphenat}
\hyphenation{ма-те-ма-ти-ка вос-ста-нав-ли-вать}
%--------------------------------------
 
 \title{Лекция по дискретной математике}
 \date{4 сентября 2019}
 
\begin{document}
 
\maketitle

\textbf{Исчисление высказываний}

Логическая формула - выражение со значением (0, 1), переменными (х, y...) и операциями (*, v, $\Rightarrow$...)

Арифметические выражения. Пример: $(2+x) + y - 10$

Логические выражения. Пример: $(1 + \overline{x}) \Rightarrow (x y \Leftrightarrow \overline{y} \overline{z})$

\underline{Операции:}

Унарная операция отрицание

\begin{center}
\begin{tabular}{ |c|c| } 
 \hline
 x & $\overline{x}$  \\ 
 0 & 1 \\ 
 1 & 0 \\ 
 \hline
\end{tabular}
\end{center}

Бинарные операции
\begin{center}
\begin{tabular}{ |c|c|c|c|c|c|c|c|c|c| } 
 \hline
 x & y & x $\centerdot$ y & x v y & x $\Rightarrow$ y & x $\Leftrightarrow$ y & x+y & 0 & x $\bigtriangleup$ y & x $\bigtriangledown$ y \\ 
 0 & 0 & 0 & 0 & 1 & 1 & 0 & 0 & 0 & 0 \\ 
 0 & 1 & 0 & 1 & 1 & 0 & 1 & 0 & 0 & 1 \\ 
 1 & 0 & 0 & 1 & 0 & 0 & 1 & 0 & 1 & 0 \\
 1 & 1 & 1 & 1 & 1 & 1 & 0 & 0 & 0 & 0 \\
 \hline
\end{tabular}
\end{center}

\begin{center}
\begin{tabular}{ |c|c|c|c|c|c|c|c|c|c| } 
 \hline
 x & y & x & y & x $\downarrow$ y & $\overline{y}$ & y $\Rightarrow$ x & $\overline{x}$ & x $\mid$ y & 1  \\ 
 0 & 0 & 0 & 0 & 1 & 1 & 1 & 1 & 1 & 1 \\ 
 0 & 1 & 0 & 1 & 0 & 0 & 0 & 1 & 1 & 1 \\ 
 1 & 0 & 1 & 0 & 0 & 1 & 1 & 0 & 1 & 1 \\
 1 & 1 & 1 & 1 & 0 & 0 & 1 & 0 & 0 & 1 \\
 \hline
\end{tabular}
\end{center}

\textbf{Свойства операций}

$\&$, $ \centerdot$ , v, $\Leftrightarrow , + $ - коммутативны

x $ \centerdot$ y = y $ \centerdot$ x умножение

x $\Leftrightarrow$ y = y $\Leftrightarrow$ x равенство

x v y = y v x

x + y = y + x сумма в $Z_2$

Универсальный способ проверить равенство двух логических выражений - это сравнить значения выражений при всех возможных значениях переменных

x $\Rightarrow$ y не коммутативно. Проверка:

\begin{center}
\begin{tabular}{ |c|c|c|c| } 
 \hline
 x & y & x $\Rightarrow$ y & y $\Rightarrow$ x  \\ 
 0 & 0 & 0 $\Rightarrow$ 0 = 1 & 0 $\Rightarrow$ 0 = 1 \\ 
 0 & 1 & 0 $\Rightarrow$ 1 = 1 & 1 $\Rightarrow$ 0 = 0 \\
 \hline
\end{tabular}
\end{center}

 Дальше можно не продолжать, т.к. результаты не совпали
 
 \underline{Ассоциативность}
 
 (x v y) v z = x v (y v z)
 
 \begin{center}
\begin{tabular}{ |c|c|c|c|c|c|c| } 
 \hline
 x & y & z & x v y & (x v y) v z & y v z & x v (y v z)  \\ 
 0 & 0 & 0 & 0 & 0 & 0 & 0 \\
 0 & 0 & 1 & 0 & 1 & 1 & 1 \\
 0 & 1 & 0 & 1 & 1 & 1 & 1 \\
 0 & 1 & 1 & 1 & 1 & 1 & 1 \\
 1 & 0 & 0 & 1 & 1 & 0 & 1 \\
 1 & 0 & 1 & 1 & 1 & 1 & 1 \\
 1 & 1 & 0 & 1 & 1 & 1 & 1 \\
 1 & 1 & 1 & 1 & 1 & 1 & 1 \\
 \hline
\end{tabular}
\end{center}

( x $\centerdot$ y ) $\centerdot$ z = x $\centerdot$ (y $\centerdot$ z) умножение ассоциативно
(x + y) + z = x + (y + z) сложение по $Z_2$ ассоциативно

(x $\Leftrightarrow$ y) $\Leftrightarrow$ z = x $\Leftrightarrow$ (y $\Leftrightarrow$ z) 

(x $\Rightarrow$ y) $\Rightarrow$ z ? x $\Rightarrow$ (y $\Rightarrow$ z)

Проверка по таблице истинности:

\begin{center}
\begin{tabular}{ |c|c|c|c|c| } 
 \hline
 x & y & z & (x $\Rightarrow$ y) $\Rightarrow$ z & x $\Rightarrow$ (y $\Rightarrow$ z)  \\ 
 0 & 0 & 0 & 0 & 1 \\
 \hline
\end{tabular}
\end{center}

Несовпадение в первой же строке, дальше можно не проверять.

В записи логических выражений у ассоциативных операций ( ) не нужны

\textbf{Приоритет}

x v yz $\Rightarrow$ $\overline{xy}$

1. Отрицание

2. Умножение, конъюнкция

3. Дизъюнкция

4. v, +

5. $\Rightarrow , \Leftrightarrow$

\textbf{Правила де Моргана }

$\overline{x v y}$ = $\overline{x} \centerdot \overline{y}$

$\overline{x \centerdot y} = \overline{x}$ v $\overline{y}$ 

Проверим второе правило с помощью таблицы истинности

\begin{center}
\begin{tabular}{ |c|c|c|c| } 
 \hline
 x & y & $\overline{x \centerdot y}$ & $\overline{x}$ v $\overline{y}$  \\ 
 0 & 0 & 1 & 1 \\
 0 & 1 & 1 & 1 \\
 1 & 0 & 1 & 1 \\
 1 & 1 & 0 & 0 \\
 \hline
\end{tabular}
\end{center}

 \underline{Дистрибутивность}
 
 x $\centerdot$ (y v z) = xy v xz
 
 x $\centerdot$ (y + z) = xy + xz
 
 x v (yz) = (x v y) $\centerdot$ (x v z)
 
 x + (y $\centerdot$ z) \cancel = (x + y)(x + z)
 
 Еще набор свойств:
 
 $\overline{\overline{x}} = x$
 
 0x = 0
 
 |x = x
 
 0 v x = x
 
 1 v x = 1
 
 x $\Rightarrow$ y = $\overline{x}$ v y (см таблицу истинности)
 
 \textbf{Дизъюнктивно-нормальная форма (ДНФ)}
 
 Нормальная форма - один из вариантов записи логического выражения
 
 xy v z = (x v z)(y v z) = xy v z v 0 = xy + z + xyz

xy v z - ДНФ

\underline{Определение}: Выражение имеет ДНФ, если оно является дизъюнкцией нескольких конъюнкций

Конъюнкт - это конъюнкция литералов

Литерал - переменная или отрицание переменной

Пример: xy v z, где - xy и z конъюнкты, а x, y, z - литералы

 $x\overline{y}z$  v  $x\overline{y} \overline{z}$ v $\overline{y}z$ - 3 конъюнкта 

$\overline{x}$ - ДНФ, 1 конъюнкт из 1 литерала

Не являются ДНФ:

$\centerdot$ x v 1

$\centerdot$ (x v y) $\centerdot$ z

$\centerdot$ x v y v z v x $\Rightarrow$ y

$\centerdot$ x + y
 
\end{document}
