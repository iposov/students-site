\documentclass{article}
\usepackage[T2A]{fontenc}
\usepackage[utf8]{inputenc}
\usepackage[russian]{babel}

\begin{document}
	\title{Лекция №1}
	\author{Владислав Рожков}
	\date{\today}
	\maketitle
\section{Исчисление высказываний}
Логическая формула - это выражение с:

\begin{itemize}
	\item Переменными значениями ($0,1$)
	\item Перенными ($x, y, ...$)
	\item Операциями ($\cdot, \vee, \Rightarrow, ...$)
	
\end{itemize}

Пример арифметического выражения: 

\[(z+x)+y=10\]

Пример логического выражения:

\[(1+!x) \Rightarrow (xy \Leftrightarrow !y!x)\]

Где 1 - это значение; x и y - переменные; $!, \Rightarrow, \Leftrightarrow$ - операции.

Отрицание ($!$) - операция с одной переменной:

\begin{table}[h]
	\centering
	\begin{tabular}{ | l | l |}
		\hline
		$x$ & $!x$      \\ \hline
		0 & 1   \\ \hline
		1 & 0  \\ \hline
	\end{tabular}
\end{table}

Наиболее используемые логические операции с двумя переменными:

\begin{table}[h]
	\centering
	\begin{tabular}{ | l | l | l | l | l | l |}
		\hline
		$x$ & $y$ & $xy$  & $x \vee y$  & $x \Rightarrow y$  & $x\Leftrightarrow y$    \\ \hline
		0 & 0 & 0 & 0 & 1 & 1   \\ \hline
		0 & 1 & 0 & 1 & 1 & 0   \\ \hline
		1 & 0 & 0 & 1 & 0 & 0 \\
		\hline
		1 & 1 & 1 & 1 & 1 & 1 \\
		\hline
	\end{tabular}
\end{table}
Где:
\begin{itemize}
	\item $xy$ - конъюнкция ($x \cdot y$); 
	\item $x \vee y$ - дизъюнкция; 
	\item $x \Rightarrow y$ - импликация/следствие; 
	\item $x \Leftrightarrow y$ - эквивалентность.
\end{itemize}

Остальные логические операции с двумя переменными:

\begin{table}[h]
	\centering
	\begin{tabular}{ | l | l | l | l | l | l | l | l | l | l | l | l | l | l |}
		\hline
		$x$ & $y$ & $x + y$  & $0$  & $x \bigtriangleup y$  & $x \bigtriangledown y$ & $x$ & $y$ & $x \downarrow y$ & $!y$ & $y \Rightarrow x$ & $!x$ & $x | y$ & $1$  \\ \hline
		0 & 0 & 0 & 0 & 0 & 0 & 0 & 0 & 1 & 1 & 1 & 1 & 1 & 1 \\
		\hline
		0 & 1 & 1 & 0 & 0 & 1 & 0 & 1 & 0 & 0 & 0 & 1 & 1 & 1 \\
		\hline
		1 & 0 & 1 & 0 & 1 & 0 & 1 & 0 & 0 & 1 & 1 & 0 & 1 & 1 \\
		\hline
		1 & 1 & 0 & 0 & 0 & 0 & 1 & 1 & 0 & 0 & 1 & 0 & 0 & 1 \\
		\hline
	\end{tabular}
\end{table}

\section{Свойства операций}

\textbf {Коммутативность} ($\cdot, \vee, \Leftrightarrow, +$):
\[xy = yx\]
\[x \vee y = y \vee x\]
\[x \Leftrightarrow y = y \Leftrightarrow x\]
\[ x + y = y + x\]

Операция $\Rightarrow$ не коммутативна:
\[ x \Rightarrow y \neq y \Rightarrow x\]

\textbf {Ассоциативность} ($\cdot, \vee, \Leftrightarrow, +$)

Проверим ассоциативность дизъюнкции:
\[ (x \vee y) \vee z = x \vee (y \vee z)\]

Универсальный способ проверить равенство двух логических выражений - это сравнить их результаты при всех возможных случаях:

\begin{table}[h]
	\centering
	\begin{tabular}{ | l | l | l | l | l | l | l |}
		\hline
		$x$ & $y$ & $z$  & $x \vee y$  & $(x \vee y) \vee z$  & $y \vee z$ & $ x \vee (y \vee z)$   \\ \hline
		0 & 0 & 0 & 0 & 0 & 0 & 0   \\ \hline
		0 & 0 & 1 & 0 & 1 & 1 & 1   \\ \hline
		0 & 1 & 0 & 1 & 1 & 1 & 1 \\
		\hline
		0 & 1 & 1 & 1 & 1 & 1 & 1 \\
		\hline
		1 & 0 & 0 & 1 & 1 & 0 & 1 \\
		\hline
		1 & 0 & 1 & 1 & 1 & 1 & 1 \\
		\hline
		1 & 1 & 0 & 1 & 1 & 1 & 1 \\
		\hline
		1 & 1 & 1 & 1 & 1 & 1 & 1 \\
		\hline
	\end{tabular}
\end{table}

Операция $\Rightarrow$ не ассоциативна

\textbf {Приоритет операций} (От более важных к менее):

1) $!$ - отрицание

2) $\cdot$ - конъюнкция 

3) $\vee$ - дизъюнкция и $+$ - строгая дизъюнкция (исключающее или)

4) $\Rightarrow$ - импликация и $\Leftrightarrow$ - эквивалентность

\textbf {Правила Де Моргана:}
\[!(x \vee y) = !x \cdot !y\]
\[!(x \cdot y) = !x \vee !y\]

\textbf {Дистрибутивность:}
\[ x \cdot (y \vee z) = xy \vee xz\]
\[ x \cdot (y + z) = xy + xz\]
\[ x \vee (yz) = (x \vee y)\cdot(x \vee z)\]

\textbf {Другие свойства:}
\[ !x = x \]
\[ 0x = 0\]
\[ 1x = x\]
\[ 0 \vee x = x\]
\[ 1 \vee x = 1\]

\section{Дизъюнктивно-нормальная форма}

Нормальная форма - один из вариантов записи выражений:
\[ xy \vee z = (x \vee z)\cdot(y \vee z) = xy + z + xyz\]

Литерал - переменная или отрицание переменной 

Выражение имеет дизъюнктивно-нормальную форму если имеет дизъюнкцию множества конъюнкций:

 $x \cdot !y \cdot z \vee x \cdot !y \cdot !z \vee z \cdot !y$ - данная ДНФ имеет 3 конъюна

$x \cdot !y \cdot z$ - 1 конъюн

$x \vee y \vee !z$ - 3 конъюна 

$!x$ - 1 конъюн
\end{document}