\documentclass[a4paper, 12pt] {article}
\usepackage{cmap}
\usepackage[T2A]{fontenc}
\usepackage[utf8]{inputenc}
\usepackage[english, russian]{babel}
\usepackage{icomma}
\usepackage{amsmath,amsfonts,amssymb,amsthm,mathtools}
\usepackage{cancel}
\usepackage[normalem]{ulem}
\usepackage[all]{xy}
\usepackage{xcolor}
\usepackage{hyperref}
\usepackage{slashbox}


\title{Математическая логика\\
	 и \\
	 Теория алгоритмов}
\author{Посов И.А.}
\date{Весна 2022 г.}



\begin{document}
	
	
	\maketitle
	\begin{center}
		Запись конспекта: Спиридонов А.
	\end{center}

	\thispagestyle{empty}
	\newpage
	
	\begin{center}
		\tableofcontents
	\end{center}
	
	\newpage
	
	\begin{center}
		\section{Математическая логика}
	\end{center}

\subsection{Исчисление высказываний}

\textbf{ Определение} $ \wp = \{0, 1 \} $

0 -- ложь, false

1 -- истина, true

$ \wp $ ---  множество логических значений\\

\textbf{Определение} Логическая функция (от $ n $ переменных) $ f: \wp^{n} \rightarrow \wp$

\textit{Замечание} Часто логические функции вводят перечислением 
возможных  аргументов и значений для них\\

\underline{Пример}

$ \begin{tabular}{ | c | c | }
	\hline
	$ x $ $ y $ & $ f(x, y) $ \\ \hline
	0 0 & 0 \\
	0 1 & 1  \\
	1 0 & 1 \\
	1 1 & 1 \\
	\hline
\end{tabular} $

Ту же функцию можно задать формулой $ f(x, y) = max$  $ (x, y)$\\

\textbf{Утверждение}

Функций от $ n $ переменных может быть $ 2^{(2^{n})} = 2^{2^{n}} $


$ \begin{tabular}{ | c | c | c | }
	\hline
	$ x_{1}, x_{2} \dots x_{n} $ & & $ f(x_{1}, x_{2} \dots x_{n}) $  \\ \hline
	0 0 0 $ \dots $ 0 & $ 2^{n} $ & Для каждой строчки 0 или 1 \\
	$ \dots \dots \dots $ & разных & (2 варианта) \\
 	$ \dots \dots \dots $ & наборов & итого $ 2^{n} $ вариантов \\
	1 1 1 $ \dots 1 $ & аргументов &  \\
	\hline
\end{tabular} $\\

\textit{Следствие} 

$ \bullet $  $ n=1: 2^{2}=4$ функции $ f(x) $

$ \begin{tabular}{ | c | c | c |  c | c | }
	\hline
	$ x $ & $ f_{1}(x) $ & $ f_{2}(x) $ & $ f_{3}(x) $ & $ f_{4}(x) $  \\ \hline
	0 & 0 & 0 & 1  & 1 \\
	1 & 0 & 1 & 0 & 1 \\
	\hline
\end{tabular} $

$ f_{1}(x)= $ тождественный $ 0 $

$ f_{2}(x)=x $

$  f_{3}(x) -$ отрицание.

\underline{Обозначение:} $ \neg x, \bar x $

\underline{Обозначение в языках программирования:} $ !x, not$  $ x $

\underline{Примеры:} $ \neg 1 = 0, \neg 0 = 1, \neg \neg 0 = 0 $

$ f_{4}(x)= $ тождественная $ 1 $
\\

$ \bullet $ $ n=2: 2^{4}=16$ функций $ f(x, y) $

$ \begin{tabular}{ | c | c |c |c |c |c |c |c |c |c |c |c |c |c |c |c |c | }
	\hline
	$ x $ $ y $ & $ f_{1} $ & $ f_{2} $ & $ f_{3} $ & $ f_{4} $ & $ f_{5} $ & $ f_{6} $ & $ f_{7} $ & $ f_{8} $ & $ f_{9} $ & $ f_{10} $ & $ f_{11} $ & $ f_{12} $ &$  f_{13} $ & $ f_{14} $ & $ f_{15} $ & $ f_{16} $ \\ \hline
	0 0 & 0 & 0 & 0& 0& 0& 0& 0& 0& 1& 1& 1& 1& 1& 1& 1& 1  \\
	0 1 & 0 & 0 & 0& 0& 1& 1& 1& 1& 0& 0& 0& 0& 1& 1& 1& 1 \\
	1 0 & 0 & 0 & 1& 1& 0& 0& 1& 1& 0& 0& 1& 1& 0& 0& 1& 1 \\
	1 1 & 0 & 1 & 0& 1& 0& 1& 0& 1& 0& 1& 0& 1& 0& 1& 0& 1 \\
	\hline
\end{tabular} $\\

$ f_{1}(x, y)= $ тождественный $ 0 $\\

$ f_{2}(x, y)- $ логическое И, конъюнкция

\underline{Математическая запись:} $ f_{2}(x, y)=xy $

\underline{Обозначение:} $ x \cap y, x \wedge y, x*y, xy $

\underline{Обозначение в языках программирования:} $ x \& y $\\

$ f_{3}(x, y) $
\underline{Математическая запись:} $  f_{3}(x, y): x>y $

\underline{Обозначение:} $ x \rhd y$ (запрет по $ y $) $ = \overline{x \Rightarrow y} $\\

$ f_{4}(x, y)=x $\\

$ f_{5}(x, y) $
\underline{Математическая запись:} $  f_{5}(x, y): x<y $

\underline{Обозначение:} $ x \lhd y$ (запрет по $ x $) $ = \overline{y \Rightarrow x} $\\

$ f_{6}(x, y)=y $\\

$ f_{7}(x, y)- $ исключающее или (ровно один элемент истина)

\underline{Математическая запись:} $  f_{t}(x, y) : x + y$  $ mod$  $ 2$

\underline{Обозначение:} $ x + y, x \oplus y $

\underline{Обозначение в языках программирования:} $ x^\wedge y, x $  $xor  $ $ y  $\\

$ f_{8}(x, y)- $ логическое ИЛИ, дизъюнкция (если истина хотя бы одна)

\underline{Математическая запись:} $  f_{8}(x, y) = max$  $ (x, y)$

\underline{Обозначение:} $ x \cup y, x \vee y, $ редко $ x \mid y $\\

$ f_{9}(x, y) - $ стрелка Пирса

\underline{Обозначение:} $ x \downarrow y = \overline{x \cup y}$\\

$ f_{10}(x, y)- $ эквивалентность (оба истина или оба ложь)

\underline{Математическая запись:} $  f_{10}(x, y) : x=y$

\underline{Обозначение:} $ x \Leftrightarrow y, x \leftrightarrow y, x \equiv y $

\underline{Обозначение в языках программирования:} $ x==y $\\

$ f_{11}(x, y)=\bar y $\\

$ f_{12}(x, y)- $ импликация в обратную сторону

\underline{Обозначение:} $ x \Leftarrow y = y \Rightarrow x $\\

$ f_{13}(x, y)=\bar x $\\

$ f_{14}(x, y)- $ импликация (следование)

\underline{Математическая запись:} $ f_{14}(x, y): x \le y $

\underline{Обозначение:} $ x \Rightarrow y, x \rightarrow y $\\

$ \star $ импликация

$ - $ истина следует из чего угодно ($ f_{14}(x, 1)=1 $)

"Леших не существует" $ \Rightarrow $ "Русалок не существует" ($ 1\Rightarrow1=1 $)

"Все крокодилы оранжевые" $ \Rightarrow $ "Русалок не существует" ($ 0\Rightarrow1=1 $)

"из лжи следует истина"

$ - $ из лжи следует что угодно  ($ f_{14}(0, x)=1 $)

"Русалки существуют" $ \Rightarrow $ "Драконы существуют" ($ 0\Rightarrow0=1 $)

Например, если одновременно $ x=0 $ и $ x=1 \Rightarrow$ можно доказать, что $ x=5 $

\textit{Замечание}  $ x \Rightarrow y =0$ только в случае, если $  x=1, y =0 $\\

$ f_{15}(x, y) $ -- штрих Шеффера

\underline{Обозначение:} $ x \mid y = \overline{x y}$\\


$ f_{16}(x, y)= $ тождественная $ 1 $\\

\textit{Замечание} Используя исключительно стрелку Пирса или штрих 

Шеффера можно выразить любую другую функцию\\

$ \bullet $ $ n=3: 2^{8}=256$ функций $ f(x, y, z) $\\

\textbf{Определение} Логические выражения --- способ задания логических функций с помощью переменных и операций $ 0$  $ 1$  $ \neg$  $ *$  $ \vee$  $ \Rightarrow$  $ \Leftarrow$  $ +$  $ \equiv$  $ \mid$  $ \downarrow$  $ \lhd$  $ \rhd $\\

\underline{Примеры:}\\

$ (x \vee y)z $

$ (x \Rightarrow yz) \vee (y \equiv z) $

$ (0 \Rightarrow x) \vee (1 \Rightarrow y) $

\textbf{Определение} Значения логического выражения можно записать 

\underline{Таблицей истинности}\\

$ f(x, y, z) = (x \vee y)z $

$ \begin{tabular}{ | c | c | }
	\hline
	$ x $ $ y $ $ z $ & $ f(x, y, z) $ \\ \hline
	0 0 0 &0  \\
	0 0 1 &0   \\
	0 1 0 &0  \\
	0 1 1 &1  \\
	1 0 0 &0  \\
	1 0 1 &1   \\
	1 1 0 &0 \\
	1 1 1 &1  \\
	\hline
\end{tabular} $\\

\textit{Замечания}

$ - $ Порядок строчек в таблице истинности может быть любым, но мы возьмём порядок двоичных чисел$ : 000 $ $ 001 $ $ 010 $ $ 011 $ $ 100 $ $ 101 $ $ 110 $ $ 111 $

$ - $ Таблицы истинности часто считают постепенно

$ \begin{tabular}{ | c | c | c | }
	\hline
	$ x $ $ y $ $ z $ & $ (x \vee y) $ & $ (x \vee y)z $ \\ \hline
	0 0 0 &0 &0  \\
	0 0 1 &0 &0   \\
	0 1 0 &1 &0  \\
	0 1 1 &1 &1  \\
	1 0 0 &1 &0  \\
	1 0 1 &1 &1   \\
	1 1 0 &1 &0 \\
	1 1 1 &1 &1  \\
	\hline
\end{tabular} $\\

\textit{Замечание}

Приоритет операций в исчислении высказываний:

$ \neg $

$ * $

$ \vee $

$ + $ $ \equiv $

$ \Rightarrow$  $ \Leftarrow$

$ \mid$  $ \downarrow$  $ \lhd$  $ \rhd $\\

\underline{Примеры:}

$ \neg x \vee y = (\neg x) \vee y $

$ x \vee yz = x \vee (yz) \ne (x \vee y)z$

$ x \Rightarrow y \vee z = x \Rightarrow (y \vee z) $

$ \overline{x \vee y} = \neg (x \vee y) $

\begin{center}
	\textbf{Алгебраические преобразования логических выражений}
\end{center}

Алгебраические преобразования логических выражений --- 

изменение выражения по правилам, обычно в сторону упрощения

Например, $ \underbrace{(0 \Rightarrow x)}_{1} \vee (1 \Rightarrow y) = \underbrace{1\vee (1 \Rightarrow y)}_{1 \vee \alpha = 1}=1$\\

$ \bullet $ Отрицание

$ \neg \bar x = x $

\textbf{Доказательство}

$ \begin{tabular}{ | c | c | c | }
	\hline
	$ x $ & $ \bar x $ & $ \neg \bar x $ \\ \hline
	0  &1 &0  \\
	1 &0 &1   \\
	\hline
\end{tabular} $\\

$ \bullet $ $ \vee $

$ 1 \vee x = 1 $

$ 0 \vee x = x $

\textbf{Доказательство}

$ \begin{tabular}{ | c | c | }
	\hline
	$ x $ & $ 0 \vee x $ \\  \hline
	0  & $ 0 \vee 0 = 0 $  \\
	1 & $ 0 \vee 1 = 1  $  \\
	\hline
\end{tabular} $\\

$ x \vee y = y \vee x $ (Симметричность)

$ x \vee (y \vee z)= (x \vee y) \vee z $ (Ассоциативность)

$ x \vee \bar x = 1 $

\textbf{Доказательство}

$ \begin{tabular}{ | c | c | c | }
	\hline
	$ x $ & $ \bar x $ & $ x \vee \bar x $ \\  \hline
	0  & 1 & $ 0 \vee 1 = 1 $  \\
	1 & 0 & $ 1 \vee 0 = 1  $  \\
	\hline
\end{tabular} $\\

$ \bullet $ $ \wedge $

$ xy = yx $

$ x(yz)=(xy)z $

$ x*0 = 0 $

$ x*1 = x $

$ x*x=x $

$ x* \bar x =0 $

$ \bullet $ $ +, \oplus $

$ x+y=y+x $

$ x + (y + z)= (x + y) + z $

$ x+0=x $

$ x+1= \bar x $

$ x+x=0 $

$ x+\bar x=1 $

\textbf{Доказательство}

$ x+\bar x= x + (1+x)=x + 1+x=1+x+x=1+0=1 $

$ \bullet $ $ \Rightarrow $

$ x \Rightarrow y \ne y \Rightarrow x $

$ \begin{tabular}{ | c | c | c |}
	\hline
	$ x $ $ y $ & $ x \Rightarrow y $ & $ y \Rightarrow x $ \\ \hline
	0 0 & 1 & 1\\
	0 1 & 1 & 0\\
	1 0 & 0 & 1\\
	1 1 & 1 & 1\\
	\hline
\end{tabular} $\\

$ x \Rightarrow 0= \bar x $

\textbf{Доказательство}

$ \begin{tabular}{ | c | c | }
	\hline
	$ x $ & $ x \Rightarrow 0 $ \\  \hline
	0  & $ 0 \Rightarrow 0 = 1 $  \\
	1 & $ 1 \Rightarrow 0 = 0  $  \\
	\hline
\end{tabular} $\\

$ 0 \Rightarrow x= 1 $

$ x \Rightarrow 1= 1 $

$ 1 \Rightarrow x= x $

$ x \Rightarrow x= 1 $

$ x \Rightarrow \bar x= \bar x $

$ \bar x \Rightarrow x= x $

$ x \Rightarrow y \Rightarrow z: $ договоримся, что это $ x \Rightarrow (y \Rightarrow z) \ne (x \Rightarrow y) \Rightarrow z $

$ \begin{tabular}{ | c | c | c | c | c | }
	\hline
	$ x $ $ y $ $ z $ & $ x \Rightarrow y $ & $ y \Rightarrow z $ & $ x \Rightarrow (y \Rightarrow z) $ & $ (x \Rightarrow y) \Rightarrow z $ \\ \hline
	0 0 0 &1 &1 &1 &0  \\
	0 0 1 &1 &1 &1 &1  \\
	0 1 0 &1 &0 &1 &0  \\
	0 1 1 &1 &1 &1 &1 \\
	1 0 0 &0 &1 &1 &1 \\
	1 0 1 &0 &1 &1 &1  \\
	1 1 0 &1 &0 &0 &0\\
	1 1 1 &1 &1 &1 &1 \\
	\hline
\end{tabular} $\\

$ \bullet $ $ \Leftrightarrow $

$ x \Leftrightarrow y = y \Leftrightarrow x $

$ x \Leftrightarrow 0 = \bar x $

$ x \Leftrightarrow 1 = x $

$ x \Leftrightarrow x = 1 $

$ x \Leftrightarrow \bar x = 0 $

$ x \Leftrightarrow (y \Leftrightarrow z)= (x \Leftrightarrow y) \Leftrightarrow z $ (Ассоциативность)\\

\begin{center}
	\textbf{Логические законы}
\end{center}

\textbf{Дистрибутивность}

$ (x \vee y)\&z = x \& z \vee y \& z = y \& z \vee x \& z $

$ (x + y)z = xz + yz $

$ (x \& y) \vee z = (x \vee z) \& (y\vee z) = (y \vee z) \& (x \vee z) $\\

\textit{Замечание} $ (x_{1} \vee x_{2} \vee x_{3})(y_{1} \vee y_{2}) = (x_{1} \vee x_{2} \vee x_{3})y_{1} \vee (x_{1} \vee x_{2} \vee x_{3})y_{2}= $

$ = x_{1}y_{1} \vee x_{2}y_{1} \vee x_{3}y_{1} \vee x_{1}y_{2} \vee x_{2}y_{2} \vee x_{3}y_{2} $\\

\textit{Интересно $ \dots $} $ x y \vee z = (x \vee z)  (y\vee z) = xy \vee xz \vee zy \vee zz = xy \vee xz \vee zy \vee z =$

$ = xy \vee xz \vee zy \vee z*1 = xy \vee z(\underbrace{x \vee y \vee 1}_1 ) = x y \vee z*1 = x y \vee z $\\

$ x + y = \overline{x \Leftrightarrow y} $

$ (x \Rightarrow y)(y \Rightarrow x) = x \Leftrightarrow y $

\subsection{Многочлен Жегалкина}

\textit{Замечание} Одну и ту же функцию можно записать по-разному

\underline{В алгебре:} $ f(x) = x+1 = 1+x = cos(x-x)+x= \dots $

$ g(x)= x^{2}-1=(x-1)(x+1)= \dots $

\underline{В логике:} $ f(x, y) = x \vee y = x \vee y \vee 0 = (x \vee y)\underbrace{(y \vee \bar y)}_1 = x \bar y \vee y $\\

\textbf{Определение} Многочлен Жегалкина для логической функции 

$ f(x_{1}, \dots , x_{n}) $ --- это многочлен с переменными $ x_{i} $ , константами 0, 1 и со степенями переменных $ \le 1 $\\

\textbf{Альтернативное определение} Многочлен Жегалкина 

для логической функции $ f(x_{1}, \dots , x_{n}) $ --- это многочлены от $ x_{i} $ над $ \mathbb Z_{2} $\\

\underline{Примеры:}

$ f(x, y, x) = 1+x+yz+xyz $ (Коэффициенты при остальных слагаемых $ =0 $)

$ f(x, y, x) = 1+x $ 

$ f(x, y, x) = 1+xy $ 

$ f(x, y, x) = 1+xy+xyz $\\

\underline{\textbf{НЕ} многочлены Жегалкина:}

$ 1+x+ \bcancel{(y \vee z)} $

$ 1+x+ \bcancel{z^{2}} $

\textit{Замечание} В общем случае многочлен ($ a_{i} = 0 $ или $1 $):

от 1 переменной $ a_{0} +a_{1}x$

от 2 переменных $ a_{0} +a_{1}x+a_{2}y +a_{3}xy$

от 3 переменных $ a_{0} +a_{1}x+a_{2}y +a_{3}z + a_{4}xy+a_{5}xz+a_{6}yz+a_{7}xyz$

$ \dots $

В общем случае $ f(x_{1}, \dots , x_{n}):  a_{0}+ a_{1} x_{1}+\dots+a_{n} x_{n}+ax_{1}x_{2}+ax_{1}x_{3}+\dots+ax_{n-1}x_{n}+ax_{1}x_{2}x_{3} +\dots+ax_{n-2}x_{n-1}x_{n}+\dots+ax_{1}x_{2}\dots x_{n} $

(Рассматриваем все пары, тройки, $ \dots $ переменных)\\

\textbf{Утверждение} 

$ \forall f(x_{1}, \dots , x_{n}) $ --- логической функции $ \exists! $ многочлен Жегалкина $ g(x_{1}, \dots , x_{n}): f=g $\\

\underline{Пример:}

Всего существует 4 функции от 1 переменной:

$ f(x)= 0 $ $= \text{	} \text{	} \text{	} \text{	} \text{	} 0 \text{	} \text{	} \text{	} \text{	}  = 0+0x$

$ f(x)= 1 $ $ = \text{	} \text{	} \text{	} \text{	} \text{	} 1 \text{	} \text{	} \text{	} \text{	}  = 1+0x$

$ f_(x)=x $ $= \text{	} \text{	} \text{	} \text{	} \text{	} x \text{	} \text{	} \text{	} \text{	}   = 0+1x$

$f(x)= \bar x = \underbrace{1+x}_{\text{Многочлены}}= 1+1x$\\

\textbf{Доказательство}

\textbf{I.} Разные многочлены --- это разные логические функции. То есть 

$ f(x_{1}, \dots , x_{n})=a_{0}+\dots+ax_{1}x_{2}\dots x_{n} $

$ g(x_{1}, \dots , x_{n})=b_{0}+\dots+bx_{1}x_{2}\dots x_{n} $

$ \exists i : a_{i} \ne b_{i} $

Возьмём различающийся индекс с самым маленьким количеством 

переменных

\underline{Пример:}

$ f(x, y, x) = 1+x+xy+xyz = \dots + 1x + 0y +0z +1xy + \dots$

$ g(x, y, x) = 1+y+z+xyz \text{	} \text{	}= \dots + \underbrace{0x}_{min} + 1y +1z + 0xy + \dots$

Для переменных этого слагаемого подставим 1, 

для остальных слагаемых --- 0

$ \left[ \text{В примере } x=1, y=0, z=0: f(1,0,0), g(1,0,0) \right] $

И в $ f $, и в $ g $ все другие слагаемые $ =0 $, поскольку в них обязательно есть нулевая переменная

Тогда 

$ f(x_{1}, \dots , x_{n})$ $ (= a_{i}\underbrace{x_{.}x_{.}\dots x_{.}}_1=a_{i}) $ и $ g(x_{1}, \dots , x_{n})$ $ (= b_{i}\underbrace{x_{.}x_{.}\dots x_{.}}_1=b_{i}) $ 

$ a_{i} \ne b_{i} \Rightarrow f(x_{1}, \dots , x_{n}) \ne g(x_{1}, \dots , x_{n})$

Мы нашли точку, в которой они различаются,  $ \Rightarrow $ они различаются всегда

\textbf{II.} Проверим, что многочленов Жегалкина от $ n $ переменных столько же, сколько функций от $ n $ переменных

$ a_{0}+\dots+ax_{1}x_{2}\dots x_{n} $

Посчитаем количество слагаемых

\textbf{1)} 1 слагаемое без переменных, $ n $ слагаемых с одной переменной 

($ a_{0}+ a_{1} x_{1}+\dots+a_{n} x_{n} $)

$ C_{n}^{2} $ слагаемых с двумя переменными 

$ C_{n}^{3} $ слагаемых с тремя переменными 

$ \dots $

$ C_{n}^{n} $ слагаемых с $ n $ переменными 

Всего: $ \underbrace{C_{n}^{0}}_1 + C_{n}^{1}+\dots+C_{n}^{n}=[\text{С помощью комбинаторики}]=2^{n} $\\

\underline{Пример:}

от 1 переменной $ a_{0} +a_{1}x$ --- 2 слагаемых $ = 2^{1}=2 $

от 2 переменных $ a_{0} +a_{1}x+a_{2}y +a_{3}xy$ --- 4 слагаемых $ = 2^{2}=4 $

от 3 переменных $ a_{0} +a_{1}x+a_{2}y +a_{3}z + a_{4}xy+a_{5}xz+a_{6}yz+a_{7}xyz$ --- 8 слагаемых $ = 2^{3}=8 $\\

\textbf{2)} Все слагаемые имеют вид: $ \underbrace{x_{1}x_{2}\dots x_{n}}_{\text{Каждая переменная 0 или 1}} = 2^{n} $ слагаемых

Итого, многочлен Жегалкина от $ n $ переменных имеет $ 2^{n} $ слагаемых

$ a_{0}+\dots+a_{2^{n}-1}x_{1}x_{2}\dots x_{n} $\\

Сколько разных многочленов?

Каждое $ a_{i} $ --- это 0 или 1

Ответ: $ 2^{2^{n}} $, и это столько же, сколько логических функций

Итого, количество логических функций от $ n $ переменных

 и многочленов Жегалкина от $ n $ переменных совпадает

$ \blacksquare $\\

\textit{Следствие} Любая логическая функция может быть представлена в виде многочлена Жегалкина

\newpage
\underline{Примеры:}

$ f(x, y) = x * y $ --- уже многочлен Жегалкина

$ f(x, y) = x \vee y $ --- не многочлен Жегалкина

Подберём, воспользовавшись 

\hypertarget{a1}{\textit{Методом неопределённых коэффициентов}}: $ x \vee y = a_{0} +a_{1}x+a_{2}y +a_{3}xy $

$ f(0, 0) = 0=a_{0} +a_{1}0+a_{2}0 +a_{3}0=a_{0} \Rightarrow a_{0}=0 $

$ f(1, 0) = 1 \vee 0=1=a_{0} +a_{1}=a_{1} \Rightarrow a_{1}=1 $

$ f(0, 1) = 0 \vee 1=1=a_{0} +a_{2}=a_{2} \Rightarrow a_{2}=1 $

$  f(x, y)=1+1 +a_{3}xy$

$ f(1, 1) = 1 \vee 1=1=1+1 +a_{3}=0+a_{3}=a_{3} \Rightarrow a_{3}=1 $

Ответ: $ x \vee y = x+y +xy $\\

Другой способ получить многочлен Жегалкина из $ x \vee y $

Преобразуем $ x \vee y $

С учётом $ \overline{x \vee y} = \bar x \bar y $ и $ \bar x = 1+x $

$ x \vee y = \overline{ \bar x \bar y } = \overline{(1+x)(1+y)}=1+(1+x)(1+y)=\underbrace{1+1}_0+x+y+xy=x+y+xy$\\

Многочлен Жегалкина для $ x \Leftrightarrow y $

$ x \Leftrightarrow y = \overline{x+y} =1+x+y$

Многочлен Жегалкина для $ x \Rightarrow y $

$ x \Rightarrow y = \bar x \vee y = (1+x) \vee y= $

С учётом $ x \vee y = x+y+xy $

$ = (1+x)+y+(1+x)y=1+x+\underbrace{y+y}_0+xy=1+x+xy$

Итого, $ x \Rightarrow y = 1+x+xy $

\textit{Замечание} Если есть логическая формула, её можно привести к 

форме многочлена Жегалкина двумя способами:

$ \bullet $ $  $ \hyperlink{a1}{Методом неопределённых коэффициентов}

$ \bullet $ $  $ Методом алгебраических преобразований

Например: 

$ x \vee y = \overline{ \bar x \bar y } = \overline{(1+x)(1+y)}= \dots =x+y+xy$

$ x \Rightarrow y = \bar x \vee y = \dots =1+x+xy$\\

$ x \Rightarrow (y \vee \bar z) = $ (Также можно сказать, что $ (y \vee \bar z) = (z \Rightarrow y) $) $ \downarrow $

$ = x \Rightarrow (y + \bar z + y\bar z) = x \Rightarrow (y + (1+z) + y(1+z)) = x \Rightarrow (1+z + yz) =$

$ = 1+x+x(1+z + yz)=1+x+x+xz+xyz=1+xz+xyz$\\

$ x \Rightarrow (y \vee \bar z) = 1+xz+xyz$\\

$ x \Leftrightarrow y \Leftrightarrow z= (1+x+y) \Leftrightarrow z = 1+ (1+x+y)+z = \underbrace{1+ 1}_0+x+y+z = x+y+z $\\

\underline{\textit{Вывод}}
Заранее не ясно, сложно ли привести функцию к многочлену Жегалкина

\subsection{Дизъюнктивно-нормальная форма (ДНФ)}

\textbf{Определение} Литерал --- это переменная или отрицание переменной

Например, $ x, \bar x, y, \bar y, z, \bar z $\\

\textbf{Определение} Конъюнкт --- конъюнкция литералов

Например, $ x, x\bar y, xyz, xy\bar z, \underbrace{\bar y}_{\text{один литерал}}, \underbrace{\square}_{\text{пустой конъюнкт}} $

Не является конъюнктом $ \bcancel{\overline{xy}}, \bcancel{x \vee y} $\\

\textbf{Определение} Логическое выражение имеет дизъюнктивно-нормальную форму, если она является дизъюнкцией конъюнктов

Например, $ x \bar y \vee \bar x \bar z t \vee z \vee \bar x y$\\

Не ДНФ: $ \overline{  x y } $

Но если преобразовать $ \overline{  x y } = \bar x \vee \bar y$

ДНФ: $ \bar x \vee \bar y $

\textbf{\begin{center}
		Построение ДНФ по таблице истинности функции
\end{center}}

Алгоритм (на примере 3 переменных)

$ \begin{tabular}{ | c | c | }
	\hline
	$ x $ $ y $ $ z $ & $ f(x, y, z) $ \\ \hline
	0 0 0 &0  \\
	0 0 1 &0   \\
	0 1 0 &1  \\
	0 1 1 &1  \\
	1 0 0 &0  \\
	1 0 1 &0   \\
	1 1 0 &1 \\
	1 1 1 &0  \\
	\hline
\end{tabular} $

\newpage

Берём строки с $ f(x, y, z) = 1$

Допустим, есть строка

$ x = a_{1} $ (0 или 1)

$ y = a_{2} $ (0 или 1)

$ z = a_{3} $ (0 или 1)\\

В ответ добавляется конъюнкт трёх литералов: $ xyz $

Если значение переменной 0 --- литерал берётся с отрицанием, если значение переменной 1 --- литерал берётся без отрицания

$ \begin{tabular}{ | c | c | c | }
	\hline
	$ x $ $ y $ $ z $ & $ f(x, y, z) $ & \\ \hline
	0 0 0 &0 & \\
	0 0 1 &0 & \\
	0 1 0 &1 & $ \bar x y \bar z  $\\
	0 1 1 &1 & $ \bar x y z  $\\
	1 0 0 &0 & \\
	1 0 1 &0 & \\
	1 1 0 &1 & $ x y \bar z  $\\
	1 1 1 &0 & \\
	\hline
\end{tabular} $\\

Ответ: $  \bar x y \bar z \vee \bar x y z \vee x y \bar z $\\

\textbf{Доказательство корректности алгоритма}

Когда полученная ДНФ $ =1? $

Когда есть конъюнкт $ =1$\\

Если первый конъюнкт $ =1$ (В примере $ \bar x y \bar z=1 $) 

$ \Rightarrow $ все его литералы $ =1$ (В примере $ x = 0, y =1, z =0 $)\\

Если второй конъюнкт $ =1$ (В примере $ \bar x y z=1 $) 

$ \Rightarrow $ все его литералы $ =1$ (В примере $ x = 0, y =1, z =1 $)\\

Если третий конъюнкт $ =1$ (В примере $ x y \bar z=1 $) 

$ \Rightarrow $ все его литералы $ =1$ (В примере $ x = 1, y =1, z =0 $)\\

$ \begin{tabular}{ | c | c | c |c |c |c | }
	\hline
	$ x $ $ y $ $ z $ & $ \bar x y \bar z $ & $ \bar x y z $ & $ x y \bar z $ & Ответ = $ f(x, y, z) $ \\ \hline
	0 0 0 &0 &0 &0 &0 \\
	0 0 1 &0 &0 &0 &0 \\
	0 1 0 &1 &0 &0 &1 \\
	0 1 1 &0 &1 &0 &1 \\
	1 0 0 &0 &0 &0 &0 \\
	1 0 1 &0 &0 &0 &0 \\
	1 1 0 &0 &0 &1 &1 \\
	1 1 1 &0 &0 &0 &0 \\
	\hline
\end{tabular} $\\

\textit{Замечание}
У одной функции могут быть разные ДНФ\\

$  \underbrace{\bar x y \bar z \vee \bar x y z \vee x y \bar z}_\text{ДНФ} =$\\

$ =   \bar x y( \underbrace{\bar z \vee z}_1) \vee x y \bar z = 
\underbrace{\bar x y \vee x y \bar z}_\text{ДНФ}$\\

$ = (x \vee \bar x)y \bar z \vee \bar x yz =\underbrace{y \bar z \vee \bar x yz}_\text{ДНФ}= y \bar z \vee \bar x yz \vee \underbrace{x \bar x}_0 \dots = \infty$ способов\\

Как получать ДНФ для формулы/функции?

$ \bullet $$  $ По таблице истинности

$ \bullet $$  $ Алгебраическими преобразованиями

$ x = x $

$ \bar x = \bar x $

$ x \vee y = x \vee y $

$ xy = xy $

$ x \Rightarrow y = \bar x \vee y $

$ x \Leftrightarrow y = (x \Rightarrow y)(y \Rightarrow x) = (\bar x \vee y)(\bar y \vee x) 
= \underbrace{\bar x \bar y \vee \bar x x \vee y \bar y \vee y x}_\text{ДНФ} = $

$ = \underbrace{\bar x \bar y \vee y x}_\text{ДНФ} $

$ \begin{tabular}{ | c | c | c | }
	\hline
	$ x $ $ y $ & $ x \Leftrightarrow y $ &  \\ \hline
	0 0 & 1 & $ \bar x \bar y $\\
	0 1 & 0 & \\
	1 0 & 0 & \\
	1 1 & 1 & $ xy $\\
	\hline
\end{tabular} $\\

$ x \Leftrightarrow y = \bar x \bar y \vee y x $

\newpage

$ x+y = \overline{x \Leftrightarrow y} = \overline{(x \Rightarrow y)(y \Rightarrow x)} = 
 \overline{(\bar x \vee y)(\bar y \vee x)} = \overline{\bar x \vee y} \vee \overline{\bar y \vee x} = \neg \bar x * \bar y \vee \neg \bar y * \bar x = x \bar y \vee \bar x y$
 
 $ \begin{tabular}{ | c | c | c | }
 	\hline
 	$ x $ $ y $ & $ x + y $ &  \\ \hline
 	0 0 & 0 & \\
 	0 1 & 1 & $ \bar x y $ \\
 	1 0 & 1 & $ x \bar y $ \\
 	1 1 & 0 & \\
 	\hline
 \end{tabular} $\\
 
 $ x+y = x \bar y \vee \bar x y$\\

\underline{Пример:}

$ x \Rightarrow (y+z) = \bar x \vee (y+z) = \bar x \vee \bar y z \vee y \bar z$\\

\hypertarget{a2}{\textbf{Задача (не)выполнимости}}

Дана логическая формула в ДНФ

Проверить, бывает ли она равна 0?

$ \bar x \bar y \vee x \vee y $

Для этого и $ x $, и $ y $ должны быть $ =0 $, однако в этом случае $ \bar x \bar y = 1 $

$ \Rightarrow $ не бывает\\

Если знать значения переменных(ответ) для 0, то их можно быстро проверить

Подобрать значения переменных для 0 --- трудно

Не известно алгоритма, который "принципиально"$  $ быстрее полного перебора\\

Нерешённая проблема компьютерных наук:

 \href{https://en.wikipedia.org/wiki/P_versus_NP_problem}{сравнение классов $ P $ и $ NP $}

$ P $ --- задача, которую можно эффективно решить

$ NP $ --- задача, которую можно эффективно проверить

\hyperlink{a2}{Эта задача} из класса $ NP $: если бы для неё нашёлся эффективный алгоритм, классы $ P $ и $ NP $ совпали бы, то есть эти классы совпали бы для любой задачи такого рода

Та задача, к которой сводится задача выполнимости, --- тоже сложна:

$ \bullet $ $  $ упростить логическое выражение

$ \bullet $ $  $ поиск минимальной ДНФ\\

\newpage

\textbf{Запись таблицы истинности в виде графика}

$ f(x, y) = x+y $

$ 
\begin{matrix}
	\xymatrix{
		y\\
		1 \ar@{->}[u] & 0\\
		0 \ar@{-}[u] \ar@{-}[r] & 1 \ar@{->}[r] & x\\	
	}
\end{matrix}
 $\\
 
$ f(0, 0) = 0 $

$ f(0, 1) = 1 $
 
$ f(1, 0) = 1 $
  
$ f(1, 1) = 0 $\\

$ f(x, y, z) = x+y+z $

$ 
\begin{matrix}
	\xymatrix{
		z &   &  &   & & &\\
		&   & 0 \ar@{-}[lld] \ar@{--}[ddd] &   & & 1 \ar@{-}[lld] \ar@{-}[lll]&\\
		1 \ar@{->}[uu] \ar@{-}[rrr] &   &  & 0  & & &\\
		&   &  &   & y & &\\
		&   & 1 \ar@{--}[lld] \ar@{-->}[rru] \ar@{--}[rrr] &   & & 0 \ar@{-}[uuu]  &\\
		0  \ar@{-}[uuu] \ar@{-}[rrr] & & & 1 \ar@{->}[rrr] \ar@{-}[uuu] \ar@{-}[rru] & & & x \\
	}
\end{matrix}
$

\newpage

\textbf{Задача минимизации ДНФ}

\underline{Дано:} логическая функция (в виде ДНФ)

\underline{Найти:} самую короткую эквивалентную ДНФ 

(с минимальным количеством литералов и $ \vee $)

Например, $ \bar x \bar y \vee \bar z $ короче $ xy \vee yz $\\

\textit{Замечание}

Далее в этом разделе рассматриваем только $ f(x, y, z) $\\

\textit{Замечание}

$ \bullet $ Какова таблица истинности для $ \stackrel{a}{X} \stackrel{b}{Y} \stackrel{c}{Z} ? $

$ a = 0 $ или $ 1 $

$ b = 0 $ или $ 1 $

$ c = 0 $ или $ 1 $

0 --- отрицание; 1 --- без отрицания\\

Пример:

$ \bar X Y \bar Z $

$ a = 0 $

$ b = 1 $

$ c = 0 $

Если $ \bar X Y \bar Z =1$, то $ \bar X =1, Y =1, \bar Z =1 \Rightarrow X =0, Y =1, Z =0 $

Если $ \bar X Y \bar Z =1$, то $ X =a, Y =b, Z =c $

$ 
\begin{matrix}
	\xymatrix{
		z &   &  &   & & &\\
		&   & 0 \ar@{-}[lld] \ar@{--}[ddd] &   & & 0 \ar@{-}[lld] \ar@{-}[lll]&\\
		0 \ar@{->}[uu] \ar@{-}[rrr] &   &  & 0  & & &\\
		&   &  &   & y & &\\
		&   & 1 \ar@{--}[lld] \ar@{-->}[rru] \ar@{--}[rrr] &   & & 0 \ar@{-}[uuu]  &\\
		0  \ar@{-}[uuu] \ar@{-}[rrr] & & & 0 \ar@{->}[rrr] \ar@{-}[uuu] \ar@{-}[rru] & & & x \\
	}
\end{matrix}
$

\newpage

$ \bullet $ Какова таблица истинности для $ \stackrel{a}{X} \stackrel{b}{Y}? $

Если $ \stackrel{a}{X} \stackrel{b}{Y} =1$, то $ \stackrel{a}{X}=1, \stackrel{b}{Y}=1 \Leftrightarrow  X =a, Y =b $

Пример:

$ \bar X Y $

$ 
\begin{matrix}
	\xymatrix{
		z &   &  &   & & &\\
		&   & 1 \ar@{-}[lld] \ar@{=}[ddd] &   & & 0 \ar@{-}[lld] \ar@{-}[lll]&\\
		0 \ar@{->}[uu] \ar@{-}[rrr] &   &  & 0  & & &\\
		&   &  &   & y & &\\
		&   & 1 \ar@{--}[lld] \ar@{-->}[rru] \ar@{--}[rrr] &   & & 0 \ar@{-}[uuu]  &\\
		0  \ar@{-}[uuu] \ar@{-}[rrr] & & & 0 \ar@{->}[rrr] \ar@{-}[uuu] \ar@{-}[rru] & & & x \\
	}
\end{matrix}
$\\                                                                                                   

$ \bar X Y $ --- единицы на ребре $ x = 0, y = 1, z = ? $\\

Аналогично, $ \bar Y \bar Z $

$ 
\begin{matrix}
	\xymatrix{
		z &   &  &   & & &\\
		&   & 0 \ar@{-}[lld] \ar@{--}[ddd] &   & & 0 \ar@{-}[lld] \ar@{-}[lll]&\\
		0 \ar@{->}[uu] \ar@{-}[rrr] &   &  & 0  & & &\\
		&   &  &   & y & &\\
		&   & 0 \ar@{--}[lld] \ar@{-->}[rru] \ar@{--}[rrr] &   & & 0 \ar@{-}[uuu]  &\\
		1  \ar@{-}[uuu] \ar@{=}[rrr] & & & 1 \ar@{->}[rrr] \ar@{-}[uuu] \ar@{-}[rru] & & & x \\
	}
\end{matrix}
$\\

$ \bar X Y $ --- единицы на ребре $ y = 0, z = 1, x = ? $

\newpage

$ \bullet $ Какова таблица истинности для конъюнкта из одного литерала $ X, \bar X, Y, \bar Y, Z, \bar Z? $

Например, $ \bar Y $

$ \bar y = 1 \Rightarrow y = 0 $

$ 
\begin{matrix}
	\xymatrix{
		z &   &  &   & & &\\
		&   & 0 \ar@{-}[lld] \ar@{--}[ddd] &   & & 0 \ar@{-}[lld] \ar@{-}[lll]&\\
		1 \ar@{->}[uu] \ar@{=}[rrr] &   &  & 1  & & &\\
		&   &  &   & y & &\\
		&   & 0 \ar@{--}[lld] \ar@{-->}[rru] \ar@{--}[rrr] &   & & 0 \ar@{-}[uuu]  &\\
		1  \ar@{=}[uuu] \ar@{=}[rrr] & & & 1 \ar@{->}[rrr] \ar@{=}[uuu] \ar@{-}[rru] & & & x \\
	}
\end{matrix}
$\\

$ \bar Y $ --- грань $ y =0 $

Например, $ X $

$ x = 1 $

$ 
\begin{matrix}
	\xymatrix{
		z &   &  &   & & &\\
		&   & 0 \ar@{-}[lld] \ar@{--}[ddd] &   & & 1 \ar@{=}[lld] \ar@{-}[lll]&\\
		0 \ar@{->}[uu] \ar@{-}[rrr] &   &  & 1  & & &\\
		&   &  &   & y & &\\
		&   & 0 \ar@{--}[lld] \ar@{-->}[rru] \ar@{--}[rrr] &   & & 1 \ar@{=}[uuu]  &\\
		0  \ar@{-}[uuu] \ar@{-}[rrr] & & & 1 \ar@{->}[rrr] \ar@{=}[uuu] \ar@{=}[rru] & & & x \\
	}
\end{matrix}
$\\

$ X $ --- грань $ x =0 $\\

Итого, 

Таблица истинности для $ \stackrel{a}{X} \stackrel{b}{Y} \stackrel{c}{Z} $ --- это вершина: $ X =a, Y =b, Z =c $

Таблица истинности для $ \stackrel{a}{X} \stackrel{b}{Y} $ --- это ребро: $ X =a, Y =b$

Таблица истинности для $ \stackrel{a}{X} $ --- это грань: $ X =a$\\

\textbf{Попробуем минимизировать ДНФ}

\underline{Пример 1:}

\textit{Дано:} $ \bar x \bar y \bar z \vee  x \bar y \bar z \vee  x  y \bar z $

\textit{Найти:} самый короткий ДНФ

Рисуем таблицу истинности 

$ 
\begin{matrix}
	\xymatrix{
		z &   &  &   & & &\\
		&   & 0 \ar@{-}[lld] \ar@{--}[ddd] &   & & 0 \ar@{-}[lld] \ar@{-}[lll]&\\
		0 \ar@{->}[uu] \ar@{-}[rrr] &   &  & 0  & & &\\
		&   &  &   & y & &\\
		&   & 0 \ar@{--}[lld] \ar@{-->}[rru] \ar@{--}[rrr] &   & & 1 \ar@{-}[uuu]  &\\
		1  \ar@{-}[uuu] \ar@{=}[rrr] & & & 1 \ar@{->}[rrr] \ar@{-}[uuu] \ar@{=}[rru] & & & x \\
	}
\end{matrix}
$\\

$ \bar x \bar y \bar z $ --- вершина $ (0,0,0) $

$  x \bar y \bar z $ --- вершина $ (1,0,0) $

$  x  y \bar z $ --- вершина $ (1,1,0) $\\

Иначе это можно записать как $ \bar y \bar z \vee x \bar z $

$ 
\begin{matrix}
	\xymatrix{
		z &   &  &   & & &\\
		&   & 0 \ar@{-}[lld] \ar@{--}[ddd] &   & & 0 \ar@{-}[lld] \ar@{-}[lll]&\\
		0 \ar@{->}[uu] \ar@{-}[rrr] &   &  & 0  & & &\\
		&   &  &   & y & &\\
		&   & 0 \ar@{--}[lld] \ar@{-->}[rru] \ar@{--}[rrr] &   & & 1 \ar@{-}[uuu]  &\\
		1  \ar@{-}[uuu] \ar@{=}[rrr]_{\bar y \bar z} & & & 1 \ar@{->}[rrr] \ar@{-}[uuu] \ar@{=}[rru]_{x \bar z} & & & x \\
	}
\end{matrix}
$\\

\newpage

Иначе это можно записать как $ \bar x \bar y \bar z \vee xy \bar z $

$ 
\begin{matrix}
	\xymatrix{
		z &   &  &   & & &\\
		&   & 0 \ar@{-}[lld] \ar@{--}[ddd] &   & & 0 \ar@{-}[lld] \ar@{-}[lll]&\\
		0 \ar@{->}[uu] \ar@{-}[rrr] &   &  & 0  & & &\\
		&   &  &   & y & &\\
		&   & 0 \ar@{--}[lld] \ar@{-->}[rru] \ar@{--}[rrr] &   & & 1 \ar@{-}[uuu]  &\\
		1  \ar@{-}[uuu] \ar@{-}[rrr] & & & 1 \ar@{->}[rrr] \ar@{-}[uuu] \ar@{=}[rru]_{x \bar z} & & & x \\
	}
\end{matrix}
$\\

Иначе это можно записать как $ \bar y \bar z \vee xy \bar z $

$ 
\begin{matrix}
	\xymatrix{
		z &   &  &   & & &\\
		&   & 0 \ar@{-}[lld] \ar@{--}[ddd] &   & & 0 \ar@{-}[lld] \ar@{-}[lll]&\\
		0 \ar@{->}[uu] \ar@{-}[rrr] &   &  & 0  & & &\\
		&   &  &   & y & &\\
		&   & 0 \ar@{--}[lld] \ar@{-->}[rru] \ar@{--}[rrr] &   & & 1 \ar@{-}[uuu]  &\\
		1  \ar@{-}[uuu] \ar@{=}[rrr]_{\bar y \bar z} & & & 1 \ar@{->}[rrr] \ar@{-}[uuu] \ar@{-}[rru] & & & x \\
	}
\end{matrix}
$\\

То есть $ \bar x \bar y \bar z \vee  x \bar y \bar z \vee  x  y \bar z = $

$ = \bar y \bar z \vee x \bar z = $

$ = \bar x \bar y \bar z \vee xy \bar z = $

$ = \bar y \bar z \vee xy \bar z$

Самая короткая ДНФ, ответ: $ \bar y \bar z \vee x \bar z $

\newpage

\underline{Пример 2:}

$ \bar x \bar y \bar z \vee  x \bar y  \vee  x  y $

$ 
\begin{matrix}
	\xymatrix{
		z &   &  &   & & &\\
		&   & 0 \ar@{-}[lld] \ar@{--}[ddd] &   & & 1 \ar@{=}[lld] \ar@{-}[lll]&\\
		0 \ar@{->}[uu] \ar@{-}[rrr] &   &  & 1  & & &\\
		&   &  &   & y & &\\
		&   & 0 \ar@{--}[lld] \ar@{-->}[rru] \ar@{--}[rrr] &   & & 1 \ar@{=}[uuu]  &\\
		1  \ar@{-}[uuu] \ar@{=}[rrr] & & & 1 \ar@{->}[rrr] \ar@{=}[uuu] \ar@{=}[rru] & & & x \\
	}
\end{matrix}
$\\

Иначе это можно записать как $ x \vee  \bar x \bar y \bar z $

$ 
\begin{matrix}
	\xymatrix{
		z &   &  &   & & &\\
		&   & 0 \ar@{-}[lld] \ar@{--}[ddd] &   & & 1 \ar@{=}[lld] \ar@{-}[lll]&\\
		0 \ar@{->}[uu] \ar@{-}[rrr] &   &  & 1  & & &\\
		&   &  &   & y & &\\
		&   & 0 \ar@{--}[lld] \ar@{-->}[rru] \ar@{--}[rrr] &   & & 1 \ar@{=}[uuu]  &\\
		1  \ar@{-}[uuu] \ar@{-}[rrr] & & & 1 \ar@{->}[rrr] \ar@{=}[uuu] \ar@{=}[rru] & & & x \\
	}
\end{matrix}
$

\newpage

Иначе это можно записать как $ x \vee  \bar y \bar z $

$ 
\begin{matrix}
	\xymatrix{
		z &   &  &   & & &\\
		&   & 0 \ar@{-}[lld] \ar@{--}[ddd] &   & & 1 \ar@{=}[lld] \ar@{-}[lll]&\\
		0 \ar@{->}[uu] \ar@{-}[rrr] &   &  & 1  & & &\\
		&   &  &   & y & &\\
		&   & 0 \ar@{--}[lld] \ar@{-->}[rru] \ar@{--}[rrr] &   & & 1 \ar@{=}[uuu]  &\\
		1  \ar@{-}[uuu] \ar@{=}[rrr] & & & 1 \ar@{->}[rrr] \ar@{=}[uuu] \ar@{=}[rru] & & & x \\
	}
\end{matrix}
$\\

$ \bar x \bar y \bar z \vee  x \bar y  \vee  x  y =$

$ = x \vee  \bar x \bar y \bar z =$

$= x \vee  \bar y \bar z $

Самая короткая ДНФ, ответ: $ x \vee  \bar y \bar z $\\

\textit{Замечание}

Этот метод позволяет наглядно перебрать все ДНФ

 и выбрать минимальную

Метод алгебраических преобразований не позволит проверить, что ответ оптимальный\\

 $ \bar x \bar y \bar z \vee  x \bar y \bar z \vee  x  y \bar z = \bar x \bar y \bar z \vee  x \bar y \bar z \vee  x \bar y \bar z \vee  x  y \bar z = \underbrace{(\bar x \vee x)}_1 \bar y \bar z \vee x \bar z \underbrace{(\bar y \vee y)}_1=$
 
 $ = \bar y \bar z \vee x \bar z = \underbrace{\dots}_{\text{Вдруг можно короче?}} $ 
 
 \newpage
 
\textbf{Определение} Двойственная функция

$ \sqsupset $ есть $ f: \wp^{n} \rightarrow \wp = \{0, 1 \} $

Двойственная $ f^{*}: \wp^{n} \rightarrow \wp : f^{*}(x_{1}, \dots, x_{n}) = \overline{f(\overline{x_{1}}, \dots, \overline{x_{n}})} $\\

\textit{Замечание} 

Смысл двойственной функции заключается в переходе в мир замены лжи на истину ($ 0 \leftrightarrow 1 $)\\

$ \begin{tabular}{ | c | c | }
	\hline
	$ x $ $ y $ & $ x \vee y $ \\ \hline
	0 0 & 0 \\
	0 1 & 1  \\
	1 0 & 1 \\
	1 1 & 1 \\
	\hline
\end{tabular} $\\

"Новый мир"

$ \begin{tabular}{ | c | c | }
	\hline
	$ x $ $ y $ & $ f^{*}(x, y) = x \wedge y $ \\ \hline
	1 1 & 1 \\
	1 0 & 0  \\
	0 1 & 0 \\
	0 0 & 0 \\
	\hline
\end{tabular} $\\

Проверим, что $ ( x \vee y)^{*} = xy $ по определению:

$ ( x \vee y)^{*} = \overline{\overline{x}\vee\overline{y}}=\neg \bar x * \neg \bar y = xy$\\

$ ( x + y)^{*} = \overline{\overline{x}+\overline{y}} =1+(1+x)+(1+y)=1+x+y=x \Leftrightarrow  y$\\

\textit{Замечание}

$f^{**}(x_{1}, \dots, x_{n}) = \overline{f^{*}(\overline{x_{1}}, \dots, \overline{x_{n}})} = \neg \overline{f(\overline{x_{1}}, \dots, \overline{x_{n}})} = f(x_{1}, \dots, x_{n}) $

То есть $ f^{**} = f $\\

\underline{Следствие }

$ (xy)^{*} = x \vee y $

$ (x\Leftrightarrow y)^{*} = x + y $

\newpage

\textbf{Теорема о композиции}

$ f(x_{1}, \dots, x_{m}) = f_{0}(f_{1}(x_{1}, \dots, x_{n}), f_{2}(x_{1}, \dots, x_{n}), \dots, f_{m}(x_{1}, \dots, x_{n})) $

$ f_{i} = \wp^{n} \rightarrow \wp, i = 1..m $

$ f_{0} = \wp^{m} \rightarrow \wp $\\

Тогда $ f^{*}(x_{1}, \dots, x_{m}) = f_{0}^{*}(f_{1}^{*}(x_{1}, \dots, x_{n}), f_{2}^{*}(x_{1}, \dots, x_{n}), \dots, f_{m}^{*}(x_{1}, \dots, x_{n})) $\\

\textbf{Доказательство}

$ f^{*} =  \overline{f(\overline{x_{1}}, \dots, \overline{x_{m}})} = \overline{f_{0}(f_{1}(\overline{x_{1}}, \dots, \overline{x_{n}}), f_{2}(\overline{x_{1}}, \dots, \overline{x_{n}}), \dots, f_{m}(\overline{x_{1}}, \dots, \overline{x_{n}}))}=$

$ = f_{0}^{*}(\overline{f_{1}(\overline{x_{1}}, \dots, \overline{x_{n}})}, \overline{f_{2}(\overline{x_{1}}, \dots, \overline{x_{n}})}, \dots, \overline{f_{m}(\overline{x_{1}}, \dots, \overline{x_{n}})}) =$

$ = f_{0}^{*}(f_{1}^{*}(x_{1}, \dots, x_{n}), f_{2}^{*}(x_{1}, \dots, x_{n}), \dots, f_{m}^{*}(x_{1}, \dots, x_{n})) $

$ \blacksquare $\\

\underline{Следствие}

Если есть $ f(x_{1}, \dots, x_{n}) $ --- записано как логическое выражение с $ \vee, *, \neg, +, \Leftrightarrow  $,

то $ f^{*} $ --- такое же выражение, но связки заменены на двойственные:

$ \vee \leftrightarrow * $

$ + \leftrightarrow $  $\Leftrightarrow  $

$ \neg \leftrightarrow \neg $\\

\underline{Пример:}

$ f(x, y, z) = \overline{x \vee (\bar y z)} \Leftrightarrow  (x+y+z) $

$ f^{*}(x, y, z) = \overline{x * ( \bar y \vee  z)} +  (x\Leftrightarrow y\Leftrightarrow z) $\\ 

$ 1^{*}=0 $

$ 0^{*}=1 $\\

\subsection{Конъюнктивно-нормальная форма (КНФ)}

\textbf{Определение} Литерал --- это переменная или отрицание переменной

Например, $ x, \bar x, y, \bar y, z, \bar z $

\textbf{Определение} Дизъюнкт --- дизъюнкция литералов

Например, $ x \vee y, x \vee y \vee \bar z, x \vee \bar z, \bar x $

\textbf{Не} является дизъюнктом, например, $ \bcancel{\bar x y},$  $ \bcancel{x \vee yz} $\\

\textbf{Определение} Конъюнктивно-нормальная форма (КНФ)

 --- это конъюнкция нескольких дизъюнктов

Например, $ (x \vee y)(y \vee \bar z), (x \vee y)z, xyz, x \vee z $

\textbf{Не} является КНФ, например,  $ \bcancel{xy \vee z} $, но является $ (x \vee z)(y \vee z ) $

\textbf{Утверждение} У любой логической функции есть КНФ, её можно построить по таблице истинности

\textbf{Доказательство}

Заметим, что если вычислить $ ($КНФ$)^{*} $ (двойственную к КНФ), то 

получится ДНФ

\underline{Пример:}

$ [(x \vee y \vee z)(x \vee \bar y)(\bar y \vee \bar z)]^{*} = (x y z) \vee (x \bar y) \vee (\bar y \bar z) = x y z \vee x \bar y \vee \bar y \bar z $

и наоборот: $ ($ДНФ$)^{*} = $ КНФ\\

Итого, чтобы получить КНФ для функции $ f $, надо построить ДНФ для $ f^{*} $

ДНФ для $ f^{*} $ --- существует

$ \blacksquare $\\

\underline{Пример:}

$ f(x, y, z) = xy \Leftrightarrow z$

$ \begin{tabular}{ | c | c | c |c |c | }
	\hline
	$ x $ $ y $ $ z $ & $ xy $ & $ f $ & $ f^{*} $ & ДНФ для  $ f^{*} $ \\ \hline
	0 0 0 &0 &1 &0 &  \\
	0 0 1 &0 &0 &1 & $ \bar x \bar y z $  \\
	0 1 0 &0 &1 &1 &  $ \bar x  y \bar z $  \\
	0 1 1 &0 &0 &0 &  \\
	1 0 0 &0 &1 &1 &  $  x \bar y \bar z $ \\
	1 0 1 &0 &0 &0 &   \\
	1 1 0 &1 &0 &1 &  $  x  y \bar z $ \\
	1 1 1 &1 &1 &0 &  \\
	\hline
\end{tabular} $\\

$ f^{*}(x, y, z) = \overline{f(\overline{x},\overline{y}, \overline{z}}) $

$ f^{*}(0,0,0) = \overline{f(1,1,1)} $

$ f^{*}(0,0,1) = \overline{f(1,1,0)} $

$ f^{*}(0,1,0) = \overline{f(1,0,1)} $\\

Столбец  $ f^{*} $ --- отрицание перевёрнутого столбца  $ f $\\

Итого, $ f^{*} =  \bar x \bar y z  \vee \bar x  y \bar z \vee  x \bar y \bar z \vee  x  y \bar z  $

По теореме о композиции, $ f = (\bar x \vee \bar y \vee z)(\bar x \vee y \vee \bar z)(x \vee \bar y \vee \bar z)(x \vee y \vee \bar z) $\\

\newpage

Получение КНФ по таблице истинности без двойственной функции

$ f(x, y, z) = xy \Leftrightarrow z$

$ \begin{tabular}{ | c | c | c | }
	\hline
	$ x $ $ y $ $ z $ &  $ f $  & ДНФ для  $ f^{*} $ \\ \hline
	0 0 0  &1  &  \\
	0 0 1  &0  & $ x \vee y \vee \bar z $ \\
	0 1 0  &1  &   \\
	0 1 1  &0  & $ x \vee \bar y \vee \bar z $ \\
	1 0 0  &1  &   \\
	1 0 1  &0  & $ \bar x \vee y \vee \bar z $  \\
	1 1 0  &0  & $ \bar x \vee \bar y \vee z $ \\
	1 1 1  &1  &  \\
	\hline
\end{tabular} $\\

В столбце значений переменных:

--- 1: есть отрицание

--- 0: нет отрицания\\

Ответ: $ f = (\bar x \vee \bar y \vee z)(\bar x \vee y \vee \bar z)(x \vee \bar y \vee \bar z)(x \vee y \vee \bar z) $\\

ДНФ --- строки с 1 $\left\{
\begin{array}{ccc}
	0 \leftrightarrow \bar x \bar y \bar z\\
	1 \leftrightarrow xyz \\
\end{array}
\right. $\\

КНФ --- строки с 0 $\left\{
\begin{array}{ccc}
	0 \leftrightarrow xyz\\
	1 \leftrightarrow \bar x \bar y \bar z \\
\end{array}
\right. $\\

\underline{Пример 2:}

$ f(x, y) = x+y $ в КНФ

$ \begin{tabular}{ | c | c |c | }
	\hline
	$ x $ $ y $ & $ x + y $ & \\ \hline
	0 0 & 0 & $ x \vee y $ \\
	0 1 & 1 & \\
	1 0 & 1 & \\
	1 1 & 0 &1 $ \bar x \vee \bar y $ \\
	\hline
\end{tabular} $\\

Ответ: $ (x \vee y)( \bar x \vee \bar y) $\\

\textit{Замечание} 

Для функции, записанной в форме КНФ, можно поставить задачу "выполнимости".

Вопрос: может ли значение быть $ =1? $

--- не известно решений, принципиально эффективнее полного 

перебора значений переменных

\underline{Пример:}

$ (x \vee y \vee z)(x \vee \bar y)(y \vee \bar z)(\bar x \vee \bar z) $

При $ (x = 1, y = 1, z = 0) $ значение функции $ =1 $

Многие задачи, головоломки сводятся к задаче выполнимости 

\underline{Пример:}

\href{https://ru.wikipedia.org/wiki/%D0%9F%D1%80%D0%B8%D0%BD%D1%86%D0%B8%D0%BF_%D0%94%D0%B8%D1%80%D0%B8%D1%85%D0%BB%D0%B5_(%D0%BA%D0%BE%D0%BC%D0%B1%D0%B8%D0%BD%D0%B0%D1%82%D0%BE%D1%80%D0%B8%D0%BA%D0%B0)}{Принцип Дирихле} --- если есть $ n $ клеток и в них $ n +1$ заяц, то 

$ \exists $ клетка, в которой $ \ge 2 $ зайца

Докажем это утверждение при помощи сведения к задаче 

выполнимости при $ n =2$

$ X_{i, j} $ --- в клетке $ i $ сидит заяц $ j $
$\left\{
\begin{array}{ccc}
	i &=& 1 \text{ или } 2 \text{ (клетка) }\\
	j &=& 1 \text{ или } 2 \text{ или } 3 \text{ (заяц) }\\
\end{array}
\right. $\\

Попробуем записать, что в каждой клетке $ \le 1$ зайца:\\

$ I. $ Каждый заяц ровно в одной клетке 

$ X_{11} + X_{21} $ --- заяц 1

$ X_{12} + X_{22} $ --- заяц 2

$ X_{13} + X_{23} $ --- заяц 3

(Если клеток много: $ X_{1 i} \bar X_{2 i} \dots \bar X_{k i} \vee \bar X_{1 i}  X_{2 i} \dots \bar X_{k i} \vee \dots \vee \bar X_{1 i} \bar X_{2 i} \dots  X_{k i}$)\\

$ II. $ В каждой клетке $ \le 1 $ зайца

$ \begin{tabular}{ | c | c | c | c | }
	\hline
	\backslashbox{\text{клетка}}{\text{заяц}} & 1 & 2 & 3 \\ \hline
	1 & $ X_{1 1} $ & $ X_{1 2} $& $ X_{1 3} $ \\
	2 & $ X_{2 1} $ & $ X_{2 2} $& $ X_{2 3} $ \\
	
	\hline
\end{tabular} $\\

Если есть 2 зайца, то один из конъюнктов$  =1 $

В клетке 1 $ \le 1 $ зайца: $ \overline{X_{1 1}X_{1 2} \vee X_{1 1}X_{1 3} \vee X_{1 2}X_{1 3}}$

В клетке 2 $ \le 1 $ зайца: $ \overline{X_{2 1}X_{2 2} \vee X_{2 1}X_{2 3} \vee X_{2 2}X_{2 3}}$\\

Соединяем все утверждения:

$ (X_{11} + X_{21})(X_{12} + X_{22})(X_{13} + X_{23}) \overline{(X_{1 1}X_{1 2} \vee X_{1 1}X_{1 3} \vee X_{1 2}X_{1 3})} $  $ \overline{(X_{2 1}X_{2 2} \vee X_{2 1}X_{2 3} \vee X_{2 2}X_{2 3})} $

$ = $ тождественный 0\\

$ \overline{a \vee b} = \bar a * \bar b$

$ \overline{a * b} = \bar a \vee \bar b$\\

Преобразуем в КНФ:

$ (X_{11} + X_{21})(X_{12} + X_{22})(X_{13} + X_{23}) =$ 

$= (X_{11} \vee X_{21})(\bar X_{11} \vee \bar X_{21}) (X_{12} \vee X_{22})(\bar X_{12} \vee \bar X_{22}) (X_{13} \vee X_{23})(\bar X_{13} \vee \bar X_{23}) $\\
 
$ \overline{(X_{1 1}X_{1 2} \vee X_{1 1}X_{1 3} \vee X_{1 2}X_{1 3})} = 
(\bar X_{1 1} \vee \bar X_{1 2})(\bar X_{1 1} \vee \bar X_{1 3})(\bar X_{1 2} \vee \bar X_{1 3}) $\\

$ \overline{(X_{2 1}X_{2 2} \vee X_{2 1}X_{2 3} \vee X_{2 2}X_{2 3})} = 
 (\bar X_{2 1} \vee \bar X_{2 2})(\bar X_{2 1} \vee \bar X_{2 3})(\bar X_{2 2} \vee \bar X_{2 3})$
 
 \newpage
 
 КНФ: 
 
 $ (X_{11} \vee X_{21})(\bar X_{11} \vee \bar X_{21}) (X_{12} \vee X_{22})(\bar X_{12} \vee \bar X_{22}) (X_{13} \vee X_{23})(\bar X_{13} \vee \bar X_{23})$ 
 
 $(\bar X_{1 1} \vee \bar X_{1 2})(\bar X_{1 1} \vee \bar X_{1 3})(\bar X_{1 2} \vee \bar X_{1 3})
 (\bar X_{2 1} \vee \bar X_{2 2})(\bar X_{2 1} \vee \bar X_{2 3})(\bar X_{2 2} \vee \bar X_{2 3}) $\\
 
 --- берём программу, которая решает задачу выполнимости
 
 Она скажет: \underline{невыполнима}\\
 
 Основные причины, по которым удобна КНФ:
 
 1) КНФ в памяти компьютера представляет из себя список списков, что очень удобно в плане хранения и при работе
 
 2) Почему не ДНФ? Поскольку требуется выполнение всех условий, их необходимо перемножить --- однако при перемножении ДНФ не всегда получается ДНФ; в отличие от КНФ, при перемножении которых получается КНФ\\
 
 Ещё один пример головоломки, сводящейся к задаче выполнимости:  
 \href{https://ru.wikipedia.org/wiki/%D0%AF%D0%BF%D0%BE%D0%BD%D1%81%D0%BA%D0%B8%D0%B9_%D0%BA%D1%80%D0%BE%D1%81%D1%81%D0%B2%D0%BE%D1%80%D0%B4}{японский кроссворд}\\
 
 \subsection{Классы замкнутости}
 
 \textbf{Определение} Логическая функция $ f: \wp^{n} \rightarrow \wp$
 
 $ \wp = \{0, 1 \} $\\
 
 Класс --- это множество логических функций
 
\underline{ Примеры:}
 
$ K_{1}= $ класс функций от двух переменных
 
$ K_{2}= $ класс функций от двух переменных $ : f(x, y) = f(y, x) $ (симметричных)\\
 
 $ f(x, y) = x \vee y \in K_{1}, \in K_{2} $
 
 $ g(x, y) = x \Rightarrow y \in K_{1}, \notin K_{2} $\\
 
 $ K_{3}= $ класс функций $ : f(x, \dots) = f(\bar x, \dots) $\\
 
$  f(x, y, z) = y \Rightarrow z \in K_{3} $

$  f(x, y, z) = (x \Rightarrow y) \vee z \notin K_{3} $

$  f(x, y, z) = \underbrace{x \bar x}_0 \vee y \vee z \in K_{3} $\\

 $ K_{4}= \{ f(x, y) = x \vee y, g(x, y) = x \Rightarrow y \}$
 
 
 $ \begin{matrix}
 	\xymatrix{
 		x \ar@{->}[d] & y \ar@{->}[dl] \ar@{->}[dd] & z \ar@{->}[ddl] \ar@{->}[ddd] \\
 		f_{1} \ar@{->}[dr] \\
 		& f_{2} \ar@{->}[dr]\\
 		& & f_{3} \ar@{->}[d] \\
 		& & \text{\textit{g(x, y, z)}} \\
 	}
 \end{matrix} $\\

$ g(x, y, z) = \underbrace{f_{1}(f_{2}(f_{1}(x, y), y, z),z)}_{\text{композиция функций}} $\\

$ \sqsupset K = \{ f_{1 }, f_{2} , \dots\} -$ класс функций

$ K^{*} -$ замыкание класса --- это класс, состоящий из всех композиций функций $ K $\\

\underline{Примеры:}

$ K = \{ f() = 0, g(x) = \bar x\} $

$ K^{*} = \{ 0, g(x), \underbrace{g(f())}_1, \underbrace{g(g(f()))}_0, \underbrace{g(g(g(f())))}_1, \dots, \underbrace{g(g(x))}_x, \underbrace{g(g(g(x)))}_{\bar x}, \dots \} $

$ K^{*} = \{x, \bar x, 0, 1\} $\\

$ K = \{  g(x) = \bar x\} $

$ K^{*} = \{ g(x), \underbrace{g(g(x))}_x, \underbrace{g(g(g(x)))}_{\bar x}, \dots \} $

$ K^{*} = \{x, \bar x\} $\\

$ K = \{ \bar x, x \vee y, xy\} $

$ K^{*} = \{ \forall$ функция $\} $ (поскольку можно составить любую ДНФ/КНФ)\\

\textbf{ Определение} Если $ K -$ класс, $ K^{*}= \zeta $ (все логические функции), то K --- \underline{полный} класс\\

Вывод: $ K = \{ \bar x, x \vee y, xy\} $ --- полный класс

$ K = \{ f(x) = \bar x, f(x, y) = x \vee y\} $

$ xy = \overline{(\overline{xy})} = \overline{(\bar x \vee \bar y)} = f(g(f(x), f(y)))$

$ K = \{ \bar x, = x \vee y\} $ --- тоже полный класс

$ \begin{matrix}
	\xymatrix{
		x \ar@{->}[d] & & y \ar@{->}[d]  \\
		f \ar@{->}[dr] & & f \ar@{->}[dl] \\
		& g \ar@{->}[d] \\
		& f \ar@{->}[d] \\
		& \text{\textit{xy}} \\
	}
\end{matrix} $\\

$ K = \{ f(x, y) = xy, g(x, y) = x + y\} $

$ f(0, 0) = 0 $

$ g(0, 0) = 0 $

Следовательно, любая композиция функций, принимающая на вход только 0, будет $ =0 $

Но если $ h(0)=0 $, мы не сможем получить, к примеру, $ \bar x $

$ \bar x \notin K^{*}$

$ K = \{  xy, x + y\} $ --- не полный класс
 
\end{document}