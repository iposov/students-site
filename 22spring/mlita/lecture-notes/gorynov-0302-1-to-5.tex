\documentclass{article}
\usepackage[utf8]{inputenc}

%Russian-specific packages
%--------------------------------------
\usepackage[T2A]{fontenc}
\usepackage[utf8]{inputenc}
\usepackage[russian]{babel}
\usepackage{amsfonts}
\usepackage[normalem]{ulem}
\usepackage{graphicx}
\usepackage{tikz}
\usepackage{amsmath}
\usetikzlibrary{graphs}
\usetikzlibrary {positioning}
%--------------------------------------

%Hyphenation rules
%--------------------------------------
\usepackage{hyphenat}
\hyphenation{ма-те-ма-ти-ка вос-ста-нав-ли-вать}
%--------------------------------------
\begin{document}
\textbf{(Лекция№1 от 09.02)}\\
Темы:\\
$\bullet$ Математическая Логика\\
-исчисление высказываний\\
-исчесление предикатов\\
$\bullet$ Теория формальных языков\\
-конечные автоматы(регулярные выражения)\\
-Контекстно свободные грамматики\\
-машины Тьюринга
\section{Исчесление Высказываний}
\underline{Логические функции}\\
\underline{Определение}\\
$\beta$ =\{0,1\} $ \left.
\begin{array}{rcll}
\textrm{0-ложь/false }\\
\textrm{1-истина/true}\\
\end{array}
\right.
$ \\  
-множество логических значений\\
\underline{Определение}\\
Длштчнская функция(от n переменных)\\
f: $\beta^n\rightarrow \beta$\\
Замечание: Часто логические функции вводят перечислением возможных аргументов и значений для них\\
Примечание\\
f(x,y)\\
\begin{tabular}{ll|l}
 x&y&f(x,y)\\
\hline
0 & 0 &0\\
0 &  1 & 1\\
1 &  0 & 1\\
1&  1 & 1\\
\end{tabular}\\
Ту же функцию можно задать формулой\\
f(x,y)=max(x,y)\\
\underline{Утверждение}\\
Функций от n переменных может быть $2^{2^n}$\\
\begin{tabular}{llll}
 $x_1$&$x_2$&...&$x_n$\\
0 & 0& ...&0\\
0& 0 &...& 1\\
.& . &...& .\\
. &  . &...& .\\
.&  . &...& .\\
1&  1 &...& 1\\
\end{tabular} всего $2^n$ разных аргументов\\
f($x_1$...$x_n$) может принимать значения 0 или 1. Следовательно, $2^{2^n}$\\
Следствие\\
n=1: $2^{2^1}=4$ функций f(x)\\
n=2: $2^{2^2}=16$ функций f(x,y)\\
n=1: $2^{2^3}=256$ функций f(x,y,z)\\
\begin{tabular}{l|l|l|l|l}
x&$f_1(x)=0$&$f_2(x)=x$&$f_3(x)$&$f_4(x)=1$\\
\hline
0& 0        &0 &1&1\\
1& 0        &1 &0&1
\end{tabular}\\
$f_3$-отрицание, $\overline{x}$($\neg$x, !x)\\
\underline{Прмимер}\\
$\overline{0}$=1 \quad $\overline{1}$=0 \quad $\overline{0}{\overline{0}}$=0\\
\begin{tabular}{l|l|l|l|l|l|l|l|l|l|l|l|l|l|l}
x y&$f_1$&$f_2$&$f_3$&$f_4$&$f_5$&$f_6$&$f_7$&$f_8$\\
\hline
0 0 &0 &0 &0 &0 &0 &0 &0 &0\\
0 1 &0 &0 &0 &0 &1 &1 &1 &1\\
1 0 &0 &0 &1 &1 &0 &0 &1 &1\\
1 1 &0 &1 &0 &1 &0 &1 &0 &1\\
\end{tabular}\\
\begin{tabular}{l|l|l|l|l|l|l|l|l|l|l|l|l|l|l}
x y&$f_9$&$f_{10}$&$f_{11}$&$f_{12}$&$f_{13}$&$f_{14}$&$f_{15}$&$f_{16}$\\
\hline
0 0 &1 &1 &1 &1 &1 &1 &1 &1\\
0 1 &0 &0 &0 &0 &1 &1 &1 &1\\
1 0 &0 &0 &1 &1 &0 &0 &1 &1\\
1 1 &0 &1 &0 &1 &0 &1 &0 &1\\
\end{tabular}\\
$f_1$(x,y)= нулевая $\emptyset$\\
$f_2$(x,y)= логическое 'и'(коньюнкция) f(x,y)=x$\bullet$y x$\bullet$y x\&y x$\wedge$y\\
$f_3$(x,y)=x>y запрет по y = $\overline{x\Rightarrow y}$\\
$f_4$(x,y)=x\\
$f_5$(x,y)=x<y запрет по x =$\overline{x\Leftarrow y}$\\
$f_6$(x,y)=y\\
$f_7$(x,y)=исключающее или x+y= x xor y = (x+y) mod2\\
$f_8$(x,y)=логическое 'или' f(x,y)=max(x,y) x$\vee$y если истина зотя бы одна\\
$f_9$(x,y)=x$\downarrow$y=$\overline{x\vee y}$ стрелка Пирса\\
$f_{10}$(x,y)=эквивалентность x$\Leftrightarrow$y x$\equiv$y\\
$f_{11}$(x,y)=$\overline{y}$\\
$f_{12}$(x,y)=x$\Leftarrow$y=y$\Rightarrow$x обратная импликация\\
$f_{13}$(X,y)=$\overline{x}$\\
$f_{14}$(x,y)=$x\Rightarrow y$=x$\rightarrow y$\\
$f_{15}$(x,y)=x|y=$\overline{xy}$ штрих Шеффера\\
$f_{16}$(x,y)=единичная 1\\
\underline{Определение}\\
Логическое выражение- способ задания логических функций с помощью переменных и операций\\
$\bullet \quad \vee \quad \Rightarrow \quad \Leftarrow \quad + \quad \equiv \quad |\quad \downarrow\quad <\quad > $\\
Пример\\
(x $\vee$ y)z\\
(x$\Rightarrow$yz)$\vee$(y$\equiv$z)\\
(0$\rightarrow$x)$\vee$(1$\rightarrow$y)- всегда истина\\
\underline{Определение}\\
Значения логического выражения можно записать таблицей истинности\\
f(x,y,z)=(x$\vee$y)z\\
\begin{tabular}{l|l}
x y z&$(x\vee y)z$\\
\hline
0 0 0 &0\qquad (0$\vee$0)0=0\\
0 0 1 &0\qquad (0$\vee$0)1=0\\
0 1 0 &0\qquad (0$\vee$1)0=0\\
0 1 1 &1\qquad (0$\vee$1)1=1\\
1 0 0 &0\qquad (1$\vee$0)0=0\\
1 0 1 &1\qquad (1$\vee$0)1=1\\
1 1 0 &0\qquad (1$\vee$1)0=0\\
1 1 1 &1\qquad (1$\vee$1)1=1\\
\end{tabular}\\
\underline{Замечания}\\
порядок строчек в таблице истинности(ТИ)\\
могут быть любими, но мы возьмем 000 001 010 011 100 101 110 111\\
Таблицу истинности часто считают постепенно\\
\begin{tabular}{l|l|l}
x y z&x$\vee$ y&$(x\vee y)z$\\
\hline
0 0 0 &0&0\\
0 0 1 &0&0\\
0 1 0 &1&0\\
0 1 1 &1&1\\
\end{tabular}\\
\underline{Приоритет операций}\\
$\neg$\\
* $\bullet$\\
$\vee$\\
+ $\equiv$\\
$\Leftarrow$ $\Rightarrow$\\
| $\downarrow$ < >\\
Пример\\
$\neg$x$\vee$y=($\overline{x}\vee y$\\
x$\vee$yz=x$\vee$(yz)\\
$\overline{x\vee y}=\neg(x\vee y)$\\
Аонебраические преобразования логический выражений- изменяем выражения обычно в сторону упрощения\\
(0$\rightarrow x$)$\vee$(1$\rightarrow$y)=1$\vee$(1$\rightarrow y$)=1\\
\underline{Утверждение}\\
$\overline{\overline{x}}=x$\\
Докозательство: \begin{tabular}{l|l|l}
x&$\overline{x}$&$\overline{\overline{x}}$\\
\hline
0 &1&0\\
1 &0&1\\
\end{tabular}\\
про $\vee$\\
1$\vee$x=1\\
0$\vee$x=x\\
x$\vee$y=y$\vee$x-симметричность\\
\textbf{(Лекция№2 от 16.02)}\\
Напоминание\\
-логические функции\\
-все 16 f(x,y)\\
-таблицы истинности\\
\begin{tabular}{ll|l}
 x&y&f(x,y)\\
\hline
0 & 0 &0\\
0 &  1 & 1\\
1 &  0 & 1\\
1&  1 & 1\\
\end{tabular}\quad
\begin{tabular}{ll|l}
 x&y&x*y\\
\hline
0 & 0 &0\\
0 & 1 & 0\\
0 & 0.5 & 0\\
1&  0.2 & 0.2\\
\end{tabular} \quad
\begin{tabular}{l|l}
 x&sin(x)\\
\hline
 0 &0\\
 1 & 1\\
 0 & 1\\
\end{tabular}\\
-порядок строк фиксирован\\
Таблица эквивалентности логических выражений\\
$\overline{\overline{x}}=x$ \quad
Докозательство: \begin{tabular}{l|l|l}
x&$\overline{x}$&$\overline{\overline{x}}$\\
\hline
0 &1&0\\
1 &0&1\\
\end{tabular}\\
x$\vee$y=y$\vee$x-симметричность\\
1$\vee$x=1\\
0$\vee$x=x\\
x$\vee$x=x\\
x$\vee \overline{x}$=1\\
\begin{tabular}{l|l|l}
x&$\overline{x}$&x$\vee$ $\overline{x}$\\
\hline
0 & 1& 1\\
1 & 0& 1\\
\end{tabular}\\
xy=yx\\
x*0=0\\
x*1=x\\
x*x=x\\
x*$\overline{x}$=0\\
\begin{tabular}{l|l|l}
x&$\overline{x}$&x $\overline{x}$\\
\hline
0 & 1& 0\\
1 & 0& 0\\
\end{tabular}\\
x+y=y+x\\
x+0=x\\
$
\left.
\begin{array}{rcll}
x+1=\overline{x}\\
x+x=0\\
x+\overline{x}=1
\end{array}
\right\}{x+\overline{x}=x+1+x=1+0=1}\\
$
Ассоциацивность\\
x$\vee(y\vee z)=(x \vee y) \vee z$\\
x(yz)=(xy)z\\
x+(y+z)=(x+y)+z\\
x$\Rightarrow$y$\neg$y$\Rightarrow$x\\
\begin{tabular}{l|l|l}
x y&x$\Rightarrow$y&y$\Rightarrow$x\\
\hline
0 0& 1& 1\\
0 1& 1& 0\\
1 0& 0& 1\\
1 1& 1& 1\\
\end{tabular}\\
x$\Rightarrow$0=$\overline{x}$\\
0$\Rightarrow$x=1\\
x$\Rightarrow$1=x\\
1$\Rightarrow$x=x\\
x$\Rightarrow$x=1\\
x$\Rightarrow \overline{x}$=$\overline{x}$\\
$\overline{x} \Rightarrow$x=x\\
x$\Rightarrow$y$\Rightarrow$z= договоримся =, что это x$\Rightarrow$(y$\Rightarrow$z)$\neg$(x$\Rightarrow$y)$\Rightarrow$z\\
\begin{tabular}{l|l|l|l|l}
x y z&x$\Rightarrow$y&y$\Rightarrow$z&x$\Rightarrow$(y$\Rightarrow$z)&(x$\Rightarrow$)y$\Rightarrow$z\\
\hline
0 0 0& 1& 1&1&0\\
0 0 1& 1& 0&1&1\\
0 1 0& 1& 1&1&0\\
0 1 1& 1& 1&1&1\\
1 0 0& 0& 1&1&1\\
1 0 1& 0& 1&1&1\\
1 1 0& 1& 0&0&0\\
1 1 1& 1& 1&1&1\\
\end{tabular}\\
x$\Leftrightarrow$y=y$\Leftrightarrow$x\\
x$\Leftrightarrow$0=$\overline{x}$\\
x$\Leftrightarrow$1=x\\
x$\Leftrightarrow$x=1\\
x$\Leftrightarrow \overline{x}$=0\\
x$\Leftrightarrow$(y$\Leftrightarrow$z)=(x$\Leftrightarrow$)y$\Leftrightarrow$z -ассоциативно\\
\underline{Дистрибутивность}\\
(x$\vee$y)$\vee$z=xz$\vee$yz\\
\begin{tabular}{l|l|l}
x y z&(x$\vee$y)z&xz$\vee$yz\\
\hline
0 0 0& 0& 0\\
0 0 1& 0& 0\\
0 1 0& 0& 0\\
0 1 1& 1& 1\\
1 0 0& 0& 0\\
1 0 1& 1& 1\\
1 1 0& 0& 0\\
1 1 1& 1& 1\\
\end{tabular}\\
(x+y)z=xz+yz так как обычные $+_2 \quad \bullet$\\
(x\&y)$\vee$z=xy$\vee$z=(x$\vee$z)(y$\vee$z)\\
\underline{Замечание}\\
($x_1\vee x_2\vee x_3)(y_1\vee y_2)=(x_1\vee x_2 \vee x_3)y_1\vee(x_1\vee x_2\vee x_3)y_2$=\\
=$x_1y_1\vee x_2y_1\vee x_3y_1 \vee x_1y_2\vee x_2y_2\vee x_3y_2$\\
(xy$\vee$z)=(x$\vee$z)(y$\vee$z)=xy$\vee$xz$\vee$zy$\vee$zz=xy$\vee$xz$\vee$zy$\vee$zz=xy$\vee$xz$\vee$zy$\vee$z*1=\\
=xy$\vee$z(x$\vee$y$\vee$1)=xy$\vee$z(x$\vee$y$\vee$1)=xy$\vee$z*1=xy$\vee$z- сошлось\\
x+y=$\overline{x\Leftrightarrow y}$\\
\begin{tabular}{l|l|l}
x y&(x$\vee$y)z\\
\hline
0 0& 0\\
0 1& 1\\
1 0& 1\\
1 1& 0\\
\end{tabular}\\
(x$\Rightarrow$y)(y$\Rightarrow $x)=$\overline{x\Leftrightarrow y}$
\section{Многочлены Жегалкина}
\underline{Замечание}\\
одну и ту же функцию можно записать по разному\\
В алгебре
f(x)=1+x=x+1=x+5-4=cos(x-x)+x=...\\
g(x)=$x^2-1=(x-1)(x+1)=\dots$\\
В логике
f(x,y)=x$\vee$y=x$\vee$y$\vee$0=(x$\vee$y)($\overline{y} \vee $y)=x$\overline{y} \vee$y- дистрибутивно\\
\underline{Определение}\\
Многочлен Жегалкина для логическиой функции\\
f($x_1,x_2,\dots x_n$)- это многочлен с переменными $x_i$, константами 0,1 и со степенями перменных $\leq$1\\
\underline{или}\\
это многочлен от $x_i$ над $\mathbb Z_2$\\
\underline{Пример}\\
f(x,y,z)=1+x+yz+xyz\\
1+x\\
1+xy\\
x+xyz\\
Не многочлены:\\
1+x+(y$\vee$z)\\
1+x+$z^2$\\
\underline{Замечание}\\
В общем случае многочлен:\\
от 1 переменной ($a_i$=0 или 1) $a_0+a_1x$\\
от 2 переменных $a_0+a_1x+a_2y+a_3z+a_4xy+a_5zx+a_6yz+a_7xyz$\\
от 3 переменных $a_0+a_1x_1+\dots+a_nx_n+ax_1x_2+ax_1x_3+\dots+ax_1x_2x_3+ax_1x_2x_4+\dots+ax_1x_2x_3\dots x_n$\\
\underline{Утверждение}\\
$\forall$ f($x_1\dots x_n)$-логическая функция $\exists$! многочлен Жегалкина\\
$g(x_1\dots x_n):f=g$\\
Примеры\\
f(x)=0=0=0+0x\\
f(x)=1=1=1+0x\\
f(x)=x=x=0+1x\\
f(x)=$\overline{x}$=1+X=1+1x\\
\underline{Докозательство}\\
1.\\
разные многочлены- это разные логические функции, то есть f$(x_1 \dots x_n)=a_0+\dots+a_nx_1\dots x_n$\\
g$(x_1 \dots x_n)=b_0+\dots+b_nx_1\dots x_n$\\
$\exists i:a_i\neg b_i$\\
Возьмем различающийся индекс с самым min количеством переменных\\
Пример\\
f(x,y,z)=1+x+xy+xyz=$\dots$+1x+0y+0z+1xy\\
g(x,y,z)=1+y+z+xyz=$\dots$+0x+1y+1z+0xy\\
Для переменных этого слогаемого подставим 1\\
Для остальных переменных:0\\
(В примере x=1 y=0 z=0: f(1,0,0) и g(1,0,0))\\
И в f и в g все другие слогаемые равны 0 или совпадают\\
Теперь f($\dots$) и g($\dots$) $\Rightarrow$ f($x_1 \dots x_n$) $\neg$ g($y_1 \dots y_n$)\\
$a_ixxx \neg b_ixxx$, так как $a_i\neg b_i$\\
2.\\
Проверим, что многочленов столько же, сколько функций от n переменных\\
посчитает\\
$a_0+a_1x_1+\dots +a_nx_1x_2 \dots x_n$\\
сколько слагаемых\\
1) 1 слагаемое без переменных и слагаемой с 1ой переменной\\
$a_1x_1+\dots +a_nx_n$\\
$C_n^2$слагаемых с 2 переменными\\
$C_n^3$слагаемых с 3 переменными\\
$\dots$\\
$C_n^n$слагаемых с n переменными\\
Всего $C_n^0+C_n^1+C_n^2\dots +C_n^n=2^n$\\
Пример\\
$a_0+a_1x$- 2 слагаемых=2\\
$a_0+a_1x+a_2y+a_3xy- 2^2$= 4 слагаемых\\
2) все слогаемые имеют вид:  $x_1,x_2,x_3 \dots x_n -2^n$\\
Итого многочлен Жегалкина от n переменных имеет $2^n$ слагаемых\\
теперь сколько разных многочленов?\\
Каждое $a_i$=0 или 1\\
Ответ: $2^{2^n}$ столько же, сколько логических функций\\
Итог:\\
Логические функции от n переменных($2^{2^n}$ штук)$\equiv$ многочленов Жегалкина от n перменных ($2^{2^n}$ штук)\\
Следствие: любая логическая функция может быть представлена в виде многочлена Жегалкина\\
Примеры\\
f(x,y)=x$\vee$y не многочлен Жегалкина\\
f(x,y)=x*y- многочлен Жегалкина\\
подберем x$\vee$ y=$a_0+a_1x+a_2y+a_3xy$\\
f(0,0)= 1$\vee$0=1\\
$a_0+a_1=1$ $a_1=1$
f(0,1) аналогично\\
f(x,y)=x+y+$a_3$xy\\
f(1,1)=1$\vee$1=1\\
1+1+$a_3$=0+$a_3$=$a_3 \Rightarrow a_3=1$\\
Ответ: x$\vee$y=x+y+xy\\
\textbf{(Лекция№3 от 02.03)}\\
Все многочлены функции можно представить в виде многочлена\\
\begin{tabular}{lll}
f(x)&&многочлены \\
\hline
0&=&0\\
1&=&1\\
x&=&x\\
$\overline{x}$&=&1+x\\
\end{tabular} \begin{tabular}{lll}
f(x,y)&&многочлены \\
\hline
0&=&0\\
1&=&1\\
xy&=&xy\\
x+y&=&x+y\\
x$\vee$y&=&x+y+xy\\
\end{tabular}\\
x$\vee$y=x+y+xy- Многочлен Жегалкина для $\vee$(Другой способ получить многочлен из x$\vee$y)\\
Добавим формулы в список\\
$\overline{xy}=\neg(x,y)=\overline{x}\vee\overline{y}$\\
$\overline{x\vee y}=\overline{x}\overline{y}=\overline{y}\overline{x}$\\
Замечание $\overline{xy}\neg \overline{y}\overline{x}$\\
Докозательство в таблице истинности\\
\begin{tabular}{l|l|l|l|l|l}
x y&$\overline{x\vee y}$&$\overline{x}$&$\overline{y}$&$\overline{x} \bullet \overline{y}$ \\
\hline
0 0&0&1&1&1\\
0 1&1&1&0&0\\
1 0&1&0&1&0\\
1 1&1&0&0&0\\
\end{tabular}\\
x$\vee$y=$\overline{\overline{x}\bullet\overline{y}}$\\
$\overline{(1+x)(1+y)}=1+(1+x)(1+y)=\underbrace{1+1}_{=0}+x+y+xy=x+y+xy$\\
Многочлен Жегалкина для $\Leftrightarrow$?\\
x$\Leftrightarrow$y=$\overline{x+y}=1+x+y$\\
Многочлен Жегалкина для x$\Rightarrow$y\\
x$\Rightarrow$y=$\overline{x} \vee \overline{y}$\\
Если есть логическая функция, её можно привести к форме многочлена Жегалкина\\
-Метод неопределенных коэффициентов(смотри учебник Рыбина) $a_0+a_1x+a_2y+\dots+axyz$\\
-Метод алгебраических преобразований\\
\underline{Пример}\\
$x\vee y=\overline{\overline{x}\vee \overline{y}}=\dots=x+y+xy$\\
$x\Rightarrow y=\overline{x}\vee y=\dots=1+x+xy$
x$\Rightarrow(y\vee \overline{z})=x\Rightarrow(y+\overline{z}+y\overline{z})=x\Rightarrow(y+(1+z)+y(1+z))=x\Rightarrow(y+1+z+y+yz)=x\Rightarrow(1+z+yz)=1+x+x(1+z+yz)=1+xy+yz$-Ответ\\
$(x \Leftrightarrow y)\Leftrightarrow z=x\Leftrightarrow(y\Leftrightarrow z)$\\
$x\Leftrightarrow y\Leftrightarrow=(1+x+y)\Leftrightarrow z=1+ (1+x+y)+z=1+1+x+y+z=x+y+z$\\
Вывод: Заранее не ясно, сложно ли привести функцию к многочлену Жегалкина
\section{Дизъюнктивно нормальная форма (ДНФ)}
\underline{Определение}\\
Литерал- это переменная или отрицание переменной\\
Например: x,$\overline{x}$, y, $\overline{y}$, z,$\overline{z}$\\
\underline{Определение}\\
Конъюнкт- конъюнкция литералов\\
$x\overline{y},xyz,\overline{x}\overline{y}\overline{z},\overline{x}z,\overline{z},\emptyset$,\sout{$\overline{xy}$},\sout{$x\vee y$}\\
\underline{Определение}\\
Логическое выражение, уравнение, имеет дизъюнктивно нормальную\\
форму, если она является дизъюнкцией конъюнктов\\
\underline{Примеры}\\
$x\overline{y} \vee \overline{x}\overline{z}T\vee z\vee \overline{x}\overline{y}$\\
$xy\vee \overline{x}\overline{y}$\\
$x\vee y$\\
xy\\
НЕ ДНФ \sout{$\overline{xy}$} но $\overline{x}\vee\overline{y}$ ДНФ\\
\sout{$x\Rightarrow yz$}=$\overline{x}\vee yz$\\
\underline{Построение ДНФ по таблице истинности}\\
Алгоритм на примере 3х цифр\\
\begin{tabular}{l|l|l|l|l|l}
x y z&$\overline{x }y\overline{z}$&$\overline{x}yz$&$xy\overline{z}$&$\overline{x }y\overline{z}\vee \overline{x}yz\vee xy\overline{z}$\\
\hline
0 0 0&0&0&0&0\\
0 0 1&0&0&0&0\\
0 1 0&1&0&0&1\\
0 1 1&0&1&0&1\\
1 0 0&0&0&0&0\\
1 0 1&0&0&0&0\\
1 1 0&0&0&1&1\\
1 1 1&0&0&0&0\\
\end{tabular}\\
\underline{Замечание}\\
У одной функции могут быть разные ДНФ\\
$\overline{x}y\overline{z}\vee\overline{x}yz\vee xy\overline{z}$
$
\left.
\begin{array}{lcl}
\overset{1}{=}\overline{x}y(\overline{z}\vee z)\vee xy\overline{z}=\overline{x}y\vee xy\overline{z}\\
\overset{2}{=}(\overline{x}\vee x)y\overline{z}\vee \overline{x}yz-y\overline{z}\vee \overline{x}yz=y\overline{z}\vee \overline{x}yz\vee x\overline{x}\vee y\overline{y}\vee=\infty\\
\end{array}
\right.
$\\
Как получить ДНФ для формулы/функции\\
-по таблице истинности\\
-алгебраические преобразованиями\\
Примеры\\
$\overline{x}=\overline{x}$\\
$x\vee y=y\vee x$ 2 кон.\\
xy=yx 1 кон.\\
$x\Rightarrow \overline{x}\vee y$ 2кон.\\
$x\Leftrightarrow y=(x\Rightarrow y)(y\Rightarrow x)=(\overline{x}\vee y)(\overline{y}\vee x)=$раскроем скобки\\
=$\overline{x}\overline{y}\vee \overline{x}x\vee y\overline{y}\vee yx=\overline{x}\overline{y}\vee xy$\\
По Таблице Истинности\\
\begin{tabular}{l|l}
x y& $x\Leftrightarrow y$\\
\hline
0 0&1 ($\overline{x}\overline{y})$\\
0 1&0\\
1 0&0\\
1 1&1 (xy)
\end{tabular}\\
Пример, построить ДНФ\\
$x\Rightarrow(y+z)$\\
1)Таблица Истинности\\
2)преобразования $\overline{x}\vee(y+z)=\overline{x}\overline{y}z\vee y\overline{z}$-Ответ\\
Задача\\
Дана логическая формула в ДНФ, проверить, бывает ли она равна 0\\
$\overline{x}\overline{y}\vee x\vee y =0$\\
x=0,y=0$\Rightarrow \overline{x}\overline{y}=1$\\
эта задача\\
- Если знать значения перемнных(ответ) для 0, то их можно\\
быстро проверить\\
- Подобрать значения переменных для 0- трудно, не известно алгоритма,\\
который быстрее полного перебора\\
Задача в информатики P ?= NP\\
то, к чему сводится задача выполнимости- тоже слодна\\
-упростит логическое выражение\\
-поиск минимального ДНФ(следующий раз)
\section{Запись таблицы истинности в виде графика}
f(x,y)=x+y\\
\begin{tikzpicture}
\node (0)[scale=0.9] at ( 0,0) {0};
\node [scale=0.9] at ( 3,3) {y};
\node (fxy)[scale=0.9] at ( 0,3) {f(x,y)};
\node (x)[scale=0.9] at ( 3,0) {x};
\node (f10)[scale=0.9] at (1,0) {1 f(1,0)};
\node (f11)[scale=0.9] at (2.5,1){0 f(1,1)};
\node (f01)[scale=0.9] at ( 1,1) {1 f(0,1)};
\path (0) edge [->](fxy);
\path (0) edge [->](x);
\path (-1,-1) edge [->](3,3);
\path (f01) edge[dashed](f11);
\path (f10) edge[dashed](f11);
\end{tikzpicture}\\
f(x,y,z)=x+y+z\\
\begin{tikzpicture}
\node (0)[scale=0.9] at ( 0,0) {0};
\node (x)[scale=0.9] at ( 3,0) {x};
\node (z)[scale=0.9] at ( 0,3) {z};
\node (y)[scale=0.9] at ( 4.2,4) {y};
\node (f001)[scale=0.9] at (2,0) {1};
\node (f010)[scale=0.9] at (0.9,1){1};
\node (f100)[scale=0.9] at (-0.1,2) {1};
\node (f011)[scale=0.9] at ( 3,3) {};
\node ()[scale=0.9] at ( 3.1,2.9) {1};
\node (f101)[scale=0.9] at ( 2,2) {0};
\node (f110)[scale=0.9] at ( 3,1) {0};
\node (f111)[scale=0.9] at ( 0.9,3) {0};
\path (f010) edge[dashed](f110);
\path (0) edge [->,red](x);
\path (-1,-1) edge [->,red](y);
\path (0) edge[dashed] (f010);
\path (0) edge[->,red](z);
\path (f010) edge[dashed] (f111);
\path (f101) edge[] (f100);
\path (f101) edge[] (f001);
\path (f101) edge[] (f011);
\path (f011) edge[] (f110);
\path (f011) edge[] (f111);
\path (f111) edge[] (f100);
\path (f110) edge[] (f001);
\end{tikzpicture}\\
\textbf{(Лекция№1 от 09.03)}
\section{Задача минимизации ДНФ}
Дана: логическая функция(в виде ДНФ), найти самую короткую эквивалентную ДНФ(min количество литералов и дизъюнкция $\vee$)\\
$\overline{x}\overline{y}\vee \overline{z}$
$
\left.
\begin{array}{lcl}
\textrm{короче} xy\vee yz\\
\textrm{короче} x\overline{y}\overline{z}TU\\
\end{array}
\right.
$\\
Напоминание про куб\\
\underline{Замечание}\\
Далее рассматриваем только f(x,y,z)- 3 переменных\\
\underline{Замечание}\\
Какова таблица истинности $\overset{abc}{xyz}$, где a=0/1 b=0/1 c=0/1\\
0- отрицание 1-без отрицния\\
Пример: $\overline{x}y\overline{z}$\\
\begin{tikzpicture}
\node (0)[scale=0.9] at ( 0,0) {0};
\node (x)[scale=0.9] at ( 3,0) {x};
\node (z)[scale=0.9] at ( 0,3) {z};
\node (y)[scale=0.9] at ( 4.2,4) {y};
\node (f001)[scale=0.9] at (2,0) {0};
\node (f010)[scale=0.9] at (0.9,1){1};
\node (f100)[scale=0.9] at (-0.1,2){0};
\node (f011)[scale=0.9] at ( 3,3) {};
\node ()[scale=0.9] at ( 3.1,2.9) {0};
\node (f101)[scale=0.9] at ( 2,2) {0};
\node (f110)[scale=0.9] at ( 3,1) {0};
\node (f111)[scale=0.9] at ( 0.9,3) {0};
\path (f010) edge[dashed](f110);
\path (0) edge [->,red](x);
\path (-1,-1) edge [->,red](y);
\path (0) edge[dashed] (f010);
\path (0) edge[->,red](z);
\path (f010) edge[dashed] (f111);
\path (f101) edge[] (f100);
\path (f101) edge[] (f001);
\path (f101) edge[] (f011);
\path (f011) edge[] (f110);
\path (f011) edge[] (f111);
\path (f111) edge[] (f100);
\path (f110) edge[] (f001);
\end{tikzpicture}\\
Если $\overline{x}y\overline{z}=1$, то $\overline{x}=1$ x=0
\begin{tabular}{llll}
если $\overline{x}y\overline{z}=1$, то&$\overline{x}=1$&y=1&z=1\\
&x=0&y=1&z=0\\
если xyz=1, то&x=a&y=b&z=c\\
\end{tabular}\\
Какова таблица истинности из двух $\overset{ab}{xy}$?\\
xy=1 $\Leftrightarrow$ \begin{tabular}{ll}
x=1&y=1\\
x=a&y=b\\
\end{tabular}\\
\begin{tikzpicture}
\node (0)[scale=0.9] at ( 0,0) {0};
\node (x)[scale=0.9] at ( 3,0) {x};
\node (z)[scale=0.9] at ( 0,3) {z};
\node (y)[scale=0.9] at ( 4.2,4) {y};
\node (f001)[scale=0.9] at (2,0) {0};
\node (f010)[scale=0.9] at (0.9,1){1};
\node (f100)[scale=0.9] at (-0.1,2){0};
\node (f011)[scale=0.9] at ( 3,3) {};
\node ()[scale=0.9] at ( 3.1,2.9) {0};
\node (f101)[scale=0.9] at ( 2,2) {0};
\node (f110)[scale=0.9] at ( 3,1) {0};
\node (f111)[scale=0.9] at ( 0.9,3) {1};
\path (f010) edge[dashed](f110);
\path (0) edge [->,red](x);
\path (-1,-1) edge [->,red](y);
\path (0) edge[dashed] (f010);
\path (0) edge[->,red](z);
\path (f010) edge[dashed] (f111);
\path (f101) edge[] (f100);
\path (f101) edge[] (f001);
\path (f101) edge[] (f011);
\path (f011) edge[] (f110);
\path (f011) edge[] (f111);
\path (f111) edge[] (f100);
\path (f110) edge[] (f001);
\end{tikzpicture}\\
$\overline{x}y$-единцы, на ребре x=0 y=1 z=?\\
аналогично $\overline{y}\overline{z}$рисуем ТИ\\
ребро y=0 z=0\\
\begin{tikzpicture}
\node (0)[scale=0.9] at ( 0,0) {1};
\node (x)[scale=0.9] at ( 3,0) {x};
\node (z)[scale=0.9] at ( 0,3) {z};
\node (y)[scale=0.9] at ( 4.2,4) {y};
\node (f001)[scale=0.9] at (2,0) {1};
\node (f010)[scale=0.9] at (0.9,1){0};
\node (f100)[scale=0.9] at (-0.1,2){0};
\node (f011)[scale=0.9] at ( 3,3) {};
\node ()[scale=0.9] at ( 3.1,2.9) {0};
\node (f101)[scale=0.9] at ( 2,2) {0};
\node (f110)[scale=0.9] at ( 3,1) {0};
\node (f111)[scale=0.9] at ( 0.9,3) {0};
\path (f010) edge[dashed](f110);
\path (0) edge [->,red](x);
\path (-1,-1) edge [->,red](y);
\path (0) edge[dashed] (f010);
\path (0) edge[->,red](z);
\path (f010) edge[dashed] (f111);
\path (f101) edge[] (f100);
\path (f101) edge[] (f001);
\path (f101) edge[] (f011);
\path (f011) edge[] (f110);
\path (f011) edge[] (f111);
\path (f111) edge[] (f100);
\path (f110) edge[] (f001);
\end{tikzpicture}\\
Последнее, конъюнкт из 1 литерала\\
x,y,z,$\overline{x},\overline{z},\overline{y}$\\
Например, $\overline{y}$, какая ТИ?\\
\begin{tikzpicture}
\fill[fill=blue!7] (0,0) rectangle (2,2);
\node (0)[scale=0.9] at ( 0,0) {1};
\node (x)[scale=0.9] at ( 3,0) {x};
\node (z)[scale=0.9] at ( 0,3) {z};
\node (y)[scale=0.9] at ( 4.2,4) {y};
\node (f001)[scale=0.9] at (2,0) {1};
\node (f010)[scale=0.9] at (0.9,1){0};
\node (f100)[scale=0.9] at (-0.1,2){1};
\node (f011)[scale=0.9] at ( 3,3) {};
\node ()[scale=0.9] at ( 3.1,2.9) {0};
\node (f101)[scale=0.9] at ( 2,2) {1};
\node (f110)[scale=0.9] at ( 3,1) {0};
\node (f111)[scale=0.9] at ( 0.9,3) {0};
\path (f010) edge[dashed](f110);
\path (0) edge [->,red](x);
\path (-1,-1) edge [->,red](y);
\path (0) edge[dashed] (f010);
\path (0) edge[->,red](z);
\path (f010) edge[dashed] (f111);
\path (f101) edge[] (f100);
\path (f101) edge[] (f001);
\path (f101) edge[] (f011);
\path (f011) edge[] (f110);
\path (f011) edge[] (f111);
\path (f111) edge[] (f100);
\path (f110) edge[] (f001);

\end{tikzpicture}\\
y=0\\
или конъюнкт x:грань x=1\\
\begin{tikzpicture}
\fill[fill=blue!7]  (2,2) -- (3,1) -- (3,3) -- cycle;
\fill[fill=blue!7] (2,2) -- (3,1) -- (2,0) -- cycle;
\node (0)[scale=0.9] at ( 0,0) {0};
\node (x)[scale=0.9] at ( 3,0) {x};
\node (z)[scale=0.9] at ( 0,3) {z};
\node (y)[scale=0.9] at ( 4.2,4) {y};
\node (f001)[scale=0.9] at (2,0) {1};
\node (f010)[scale=0.9] at (0.9,1){0};
\node (f100)[scale=0.9] at (-0.1,2){0};
\node (f011)[scale=0.9] at ( 3,3) {1};
\node ()[scale=0.9] at ( 3.1,2.9) {};
\node (f101)[scale=0.9] at ( 2,2) {1};
\node (f110)[scale=0.9] at ( 3,1) {1};
\node (f111)[scale=0.9] at ( 0.9,3) {0};
\path (f010) edge[dashed](f110);
\path (0) edge [->,red](x);
\path (-1,-1) edge [->,red](y);
\path (0) edge[dashed] (f010);
\path (0) edge[->,red](z);
\path (f010) edge[dashed] (f111);
\path (f101) edge[] (f100);
\path (f101) edge[] (f001);
\path (f101) edge[] (f011);
\path (f011) edge[] (f110);
\path (f011) edge[] (f111);
\path (f111) edge[] (f100);
\path (f110) edge[] (f001);
\end{tikzpicture}\\
Итого ТИ:\\
$\overset{abc}{xyz}$- это вершина x=a y=b z=c\\
$\overset{ab}{xy}$- это ребро x=a y=b\\
$\overset{a}{x}$- это грани x=a\\
Попробуем минимизировать ДНФ\\
Пример\\
$\overline{x}\overline{y}\overline{z}\vee x\overline{y}\overline{z}\vee xy\overline{z}$\\
Найти самый короткий ДНФ для неё\\
$\bullet$Шаг 1. Рисуем ТИ\\
\begin{tikzpicture}
\node (0)[scale=0.9] at ( 0,0) {1};
\node (x)[scale=0.9] at ( 3,0) {x};
\node (z)[scale=0.9] at ( 0,3) {z};
\node (y)[scale=0.9] at ( 4.2,4) {y};
\node (f001)[scale=0.9] at (2,0) {1};
\node (f010)[scale=0.9] at (0.9,1){0};
\node (f100)[scale=0.9] at (-0.1,2){0};
\node (f011)[scale=0.9] at ( 3,3) {0};
\node ()[scale=0.9] at ( 3.1,2.9) {};
\node (f101)[scale=0.9] at ( 2,2) {0};
\node (f110)[scale=0.9] at ( 3,1) {1};
\node (f111)[scale=0.9] at ( 0.9,3) {0};
\path (f010) edge[dashed](f110);

\path (0) edge [->,red](x);
\path (-1,-1) edge [->,red](y);
\path (0) edge[dashed] (f010);
\path (0) edge[->,red](z);
\path (0) edge[blue!60]node[below]{$\overline{y}$ $\overline{z}$} (f001);
\path (f010) edge[dashed] (f111);
\path (f101) edge[](f100);
\path (f101) edge[](f001);
\path (f101) edge[] (f011);
\path (f011) edge[] (f110);
\path (f011) edge[] (f111);
\path (f111) edge[] (f100);
\path (f110) edge[blue!60]node[right]{x$\overline{z}$}  (f001);
\end{tikzpicture}\\
$\overline{x}\overline{y}\overline{z}$-вершина 0,0,0\\
$x\overline{y}\overline{z}$-вершина 1,0,0\\
$xy\overline{z}$-вершина 1,1,0\\
Другие ДНФ\\
$\overline{y}\overline{z}\vee x\overline{z}$\\
или\\
$\overline{x}y\overline{z}\vee x\overline{z}$\\
\begin{tikzpicture}
\path (-1,-1) edge [->,red](y);
\node (0)[scale=0.9,blue!60] at ( 0,0) {1};
\node (x)[scale=0.9] at ( 3,0) {x};
\node (z)[scale=0.9] at ( 0,3) {z};
\node (y)[scale=0.9] at ( 4.2,4) {y};
\node (f001)[scale=0.9] at (2,0) {1};
\node (f010)[scale=0.9] at (0.9,1){0};
\node (f100)[scale=0.9] at (-0.1,2){0};
\node (f011)[scale=0.9] at ( 3,3) {0};
\node ()[scale=0.9] at ( 3.1,2.9) {};
\node (f101)[scale=0.9] at ( 2,2) {0};
\node (f110)[scale=0.9] at ( 3,1) {1};
\node (f111)[scale=0.9] at ( 0.9,3) {0};
\path (f010) edge[dashed](f110);
\path (0) edge[] (f001);
\path (0) edge [->,red](x);
\path (0) edge[dashed] (f010);
\path (0) edge[->,red](z);
\path (f010) edge[dashed] (f111);
\path (f101) edge[](f100);
\path (f101) edge[](f001);
\path (f101) edge[] (f011);
\path (f011) edge[] (f110);
\path (f011) edge[] (f111);
\path (f111) edge[] (f100);
\path (f110) edge[blue!60]node[right]{x$\overline{z}$}  (f001);
\end{tikzpicture}\\
$\overline{y}\overline{z}\vee xy\overline{z}$\\
\begin{tikzpicture}
\node (0)[scale=0.9] at ( 0,0) {1};
\node (x)[scale=0.9] at ( 3,0) {x};
\node (z)[scale=0.9] at ( 0,3) {z};
\node (y)[scale=0.9] at ( 4.2,4) {y};
\node (f001)[scale=0.9] at (2,0) {1};
\node (f010)[scale=0.9] at (0.9,1){0};
\node (f100)[scale=0.9] at (-0.1,2){0};
\node (f011)[scale=0.9] at ( 3,3) {0};
\node ()[scale=0.9] at ( 3.1,2.9) {};
\node (f101)[scale=0.9] at ( 2,2) {0};
\node (f110)[scale=0.9,blue!60] at ( 3,1) {1};
\node (f111)[scale=0.9] at ( 0.9,3) {0};
\path (f010) edge[dashed](f110);
\path (0) edge [->,red](x);
\path (-1,-1) edge [->,red](y);
\path (0) edge[dashed] (f010);
\path (0) edge[->,red](z);
\path (0) edge[blue!60]node[below]{$\overline{y}$ $\overline{z}$} (f001);
\path (f010) edge[dashed] (f111);
\path (f101) edge[](f100);
\path (f101) edge[](f001);
\path (f101) edge[] (f011);
\path (f011) edge[] (f110);
\path (f011) edge[] (f111);
\path (f111) edge[] (f100);
\path (f110) edge[]  (f001);
\end{tikzpicture}\\
то есть $\overline{x}\overline{y}\overline{z}\vee x\overline{y}\overline{z}=\overline{y}\overline{z}\vee x\overline{z}=\overline{x}y\overline{z}\vee x\overline{z}=\overline{y}\overline{z}\vee xy\overline{z}$
$\overline{y}\overline{z}\vee x\overline{z}$- самый короткий ответ\\
Пример 2\\
$\overline{x}\overline{y}\overline{z}\vee x\overline{y}\vee xy$\\
\begin{tikzpicture}
\path (-1,-1) edge [->,red](y);
\node (0)[scale=0.9,blue!60] at ( 0,0) {1};
\node (x)[scale=0.9] at ( 3,0) {x};
\node (z)[scale=0.9] at ( 0,3) {z};
\node (y)[scale=0.9] at ( 4.2,4) {y};
\node (f001)[scale=0.9] at (2,0) {1};
\node (f010)[scale=0.9] at (0.9,1){0};
\node (f100)[scale=0.9] at (-0.1,2){0};
\node (f011)[scale=0.9] at ( 3,3) {0};
\node ()[scale=0.9] at ( 3.1,2.9) {};
\node (f101)[scale=0.9] at ( 2,2) {0};
\node (f110)[scale=0.9] at ( 3,1) {1};
\node (f111)[scale=0.9] at ( 0.9,3) {0};
\path (f010) edge[dashed](f110);
\path (0) edge [->,red](x);
\path (0) edge[dashed] (f010);
\path (0) edge[->,red](z);
\path (f010) edge[dashed] (f111);
\path (f101) edge[](f100);
\path (f101) edge[blue!60]node[right]{x$\overline{y}$}(f001);
\path (f101) edge[] (f011);
\path (f011) edge[] (f110);
\path (f011) edge[] (f111);
\path (f111) edge[] (f100);
\path (f110) edge[blue!60]node[right]{xy}  (f001);
\end{tikzpicture}\\
x=1$\Rightarrow$ $x\vee \overline{x}\overline{y}\overline{z}=x\vee \overline{y}\overline{z}$\\
\begin{tikzpicture}
\path (-1,-1) edge [->,red](y);
\fill[fill=blue!7]  (2,2) -- (3,1) -- (3,3) -- cycle;
\fill[fill=blue!7] (2,2) -- (3,1) -- (2,0) -- cycle;
\node (0)[scale=0.9,blue!60] at ( 0,0) {0};
\node (x)[scale=0.9] at ( 3,0) {x};
\node (z)[scale=0.9] at ( 0,3) {z};
\node (y)[scale=0.9] at ( 4.2,4) {y};
\node (f001)[scale=0.9] at (2,0) {1};
\node (f010)[scale=0.9] at (0.9,1){0};
\node (f100)[scale=0.9] at (-0.1,2){0};
\node (f011)[scale=0.9] at ( 3,3) {1};
\node ()[scale=0.9] at ( 3.1,2.9) {};
\node (f101)[scale=0.9] at ( 2,2) {1};
\node (f110)[scale=0.9] at ( 3,1) {1};
\node (f111)[scale=0.9] at ( 0.9,3) {0};
\path (f010) edge[dashed](f110);
\path (0) edge [->,red](x);
\path (0) edge[dashed] (f010);
\path (0) edge[->,red](z);
\path (f010) edge[dashed] (f111);
\path (f101) edge[] (f100);
\path (f101) edge[] (f001);
\path (f101) edge[] (f011);
\path (f011) edge[] (f110);
\path (f011) edge[] (f111);
\path (f111) edge[] (f100);
\path (f110) edge[] (f001);
\end{tikzpicture}\\
Замечание:\\
этот метод позволяет нагдядно перебрать все ДНФ и выбрать min\\
Преобразования не позволяют проверить оптимальность\\
Пример преобразований\\
$\overline{x}\overline{y}\overline{z}\vee x\overline{y}\overline{z}\vee x y\overline{z}=(\overline{x}\vee x)\overline{y}z\vee x\overline{y}\overline{z}\vee xy\overline{z}=\overline{y}z\vee x\overline{z}(\overline{y}\vee y)=x\overline{z}\vee \overline{y}z=$возможно короче
\section{Двойственная функция}
Пусть есть f: $B^n\rightarrow B={0,1}$\\
Двойственная $f^*$:$B^n\rightarrow B: f^*(x_1,\dots,x_n)=\overline{f(\overline{x_1},\dots,x_n)}$\\
Замечание: мир замены лжи на истину\\
f(x,y)=$x\vee y$\\
\begin{tabular}{ll|l}
 x&y&f(x,y)\\
\hline
0 & 0 &0\\
0 &  1 & 1\\
1 &  0 & 1\\
1&  1 & 1\\
\end{tabular}
0$\leftrightarrow$1 новый мир\begin{tabular}{ll|l}
 x&y&f(x,y)\\
\hline
0 & 0 &1\\
0 &  1 & 0\\
1 &  0 & 0\\
1&  1 & 0\\
\end{tabular}\\
проверим, что $(x\vee y)^*=xy$\\
по определению: $(x\vee y)^*=(\overline{(\overline{x}\vee \overline{y}}=\overline{\overline{x}}$ $\overline{\overline{y}}=xy$\\
Пример 2\\
$(x+y)^*=\overline{\overline{x}+\overline{y}}=1+(1+x)+(1+y)=1+x+y=x\Leftrightarrow y$\\
Замечание\\
$f^{**}(x_1,\dots,x_n)=\overline{f^*(\overline{x_1},\dots,\overline{x_n})}=\overline{\overline{f^*(\overline{\overline{x_1}},\dots\overline{\overline{x_n}}}}=f(x_1,\dots,x_n)$\\
то есть $f^{**}=f$\\
Следствие\\
$(x\cdot y)^*=x\vee y$\\
$(x\Leftrightarrow y)^*=x+y$\\
\underline{Теорема о конъюнкции}\\
$f_0(f_1(x_1,\dots,x_n),(f_2(x_1,\dots,x_n),\dots,f_m(x_1,\dots,x_n))$\\
$f_i:B^n\rightarrow B$\\
i=1$\dots$m\\
$f_0:B^m\rightarrow B$\\
тогда\\
$f^*(x_1,\dots,x_n)=f_0^*(f^*_1(x_1,\dots,x_n)(f^*_2(x_1,\dots,x_n)\dots f_m^*(x_1,\dots,x_n))$\\
\underline{Докозательство}\\
$f^*=\overline{f(\overline{x_1}\dots \overline{x_n}}\overset{\textrm{оп определению}}{=}\overline{f_0(f_1(\overline{x_1},\dots \overline{x_n})f_2(\overline{x_1},\dots \overline{x_n})\dots f_m(\overline{x_1},\dots \overline{x_n}}=$\\
$=f^*_0(\overline{f_1(\overline{x_1},\dots \overline{x_n)}},\overline{f_2(\overline{x_1},\dots \overline{x_n})}\dots \overline{f_m(\overline{x_1},\dots \overline{x_n})})=$\\
$f_0^*(f_1^*(x_1,\dots x_n),f_2^*(x_1,\dots x_n),\dots,f_m^*(x_1,\dots x_n)$\\
Следствие\\
Если есть $f(x_1,\dots x_n)$или логическое выражение с $\vee,\cdot,\neg,+,\equiv$, то $f^*$ такое же выражение, но связки заменены на двойственные\\
$\vee \leftrightarrow \cdot$\\
$+\leftrightarrow \Leftrightarrow$\\
$\neg \leftrightarrow \neg$\\
Пример\\
$f(x,y,z)=\overline{x\vee \overline{y}z}\Leftrightarrow(x+y+z)$\\
$f^*(x,y,z)=\overline{x\cdot(\overline{y}\vee z}+(x\Leftrightarrow y\Leftrightarrow z$\\
$1^*=0$\\
$0^*=1$\\
Конъюктивно- нормальная форма- еще одна нормальная форма, похожа на ДНФ\\
\underline{Определение}\\
Литерал- как раньше, переменная или отрицательная переменная: x,y,z,$\overline{x}$\\
Дизъюнкт- дизъюнкция литералов $x\vee y$\\
КНФ- это конъюнкция нескольких дизъюнктов$(x\vee y)(y\vee z)$\\
(Лекция за 17.03)\\
\underline{Утвеждение}\\
у$\forall$ логической функции есть КНФ, можно построить по таблице истинности\\
\underline{Докозательство}\\
Заметим, что если вычислить (КНФ)$^*$ (двойственную КНФ), то получится ДНФ\\
Пример\\
$((x\vee y\vee z)(x\vee \overline{y})(\overline{y}\vee \overline{z})^*=(xyz)\vee(x\overline{y})\vee(\overline{y}\overline{z})$\\
и наоборот (ДНФ)$^*$=КНФ\\
итого, чтобы получить КНФ для функции f, надо построить ДНФ для $f^*$, ДНФ для $f^*$- существует\\
Пример\\
$f(x,y,z)=xy\Leftrightarrow z$\\
\begin{tabular}{l|l|l|l}
 x y z&xy&f&$f^*$\\
\hline
0 0 0&  0 & 1 & 0\\
0 0 1&  0 & 0 & 1\\
0 1 0&  0 & 1 & 1\\
0 1 1&  0 & 0 & 0\\
1 0 0&  0 & 1 & 1\\
1 0 1&  0 & 0 & 0\\
1 1 0&  1 & 0 & 1\\
1 1 1&  1 & 1 & 0\\
\end{tabular}\\
$f^*$отрицание перевертнутого f\\
Вспомним определение\\
$f^*(x,y,z)=\overline{f(\overline{x} \overline{y} \overline{z}}$\\
$f^*(0,0,0)=\overline{f(1,1,1)}$\\
$f^*(0,0,1)=\overline{f(1,1,0)}$\\
$f^*(0,1,0)=\overline{f(1,0,1)}$\\
ДНФ для $f^*$\\
$\overline{x} \overline{y} z$\\
$\overline{x} y \overline{z}$\\
$x \overline{y} \overline{z}$\\
$x y \overline{z}$\\
$f^*=\overline{x} \overline{y} z \vee \overline{x} y \overline{z}\vee x \overline{y} \overline{z}\vee  x y \overline{z}$\\
по теореме о композиции\\
$f=(\overline{x} \vee\overline{y}\vee z)( \overline{x}\vee y \vee\overline{z})( x\vee \overline{y}\vee \overline{z})( x\vee y\vee \overline{z})$\\
Получение КНФ по таблице истинности без двойственной функции\\
$f(x,y,z)=xy\Leftrightarrow z$\\
\begin{tabular}{l|ll}
 x y z&f&\\
\hline
0 0 0&  1 &  \\
0 0 1&  0 & $x\vee y\vee\overline{z}$  \\
0 1 0&  1 &  \\
0 1 1&  0 & $x\vee\overline{y}\vee\overline{z}$ \\
1 0 0&  1 &  \\
1 0 1&  0 & $\overline{x}\vee y\vee \overline{z}$\\
1 1 0&  0 & $\overline{x}\vee\overline{y}\vee z$ \\
1 1 1&  1 &  \\
\end{tabular} 1-отрицание 0- не отрицание\\
Ответ:$f=(x\vee y\vee\overline{z})(x\vee\overline{y}\vee\overline{z})(\overline{x}\vee y\vee \overline{z})(\overline{x}\vee\overline{y}\vee z)$\\
Итого, чтобы построить ДНФ- строки с 1 $
\left.
\begin{array}{lcl}
0\leftrightarrow \overline{x}\quad \overline{y} \quad\overline{z}\\
1\leftrightarrow x \quad y\quad  z
\end{array}
\right.
$\\
Чтобы получить КНФ строки с 0, $
\left.
\begin{array}{lcl}
0\leftrightarrow x \quad y\quad  z\\
1\leftrightarrow \overline{x}\quad \overline{y} \quad\overline{z}\\
\end{array}
\right.
$\\
Пример 2\\
\begin{tabular}{l|ll}
 x y&$x + y$&\\
\hline
0 0 &  0 & $x\vee y$ \\
0 1 &  1 &   \\
1 0 &  1 &  \\
1 1 &  0 & $\overline{x}\vee\overline{y}$ \\
\end{tabular}\\
Ответ:$(x\vee y)(\overline{x}\vee \overline{y})$\\
Замечание\\
Для функции записанной в форме КНФ, можно поставить задачу "выполнимости"\\
Вопрос: может ли значение быть=1?\\
-не известно решений, принципиально эффективней полного перебора значений переменных\\
Пример\\
$(x\vee y\vee z)(x\vee \overline{y})(y\vee \overline{z})(\overline{x}\vee \overline{z})=1$\\
x=1 y=1 z=0\\
Многие задачи головоломки сводятся к задаче выполнимости\\
Пример\\
\underline{Принцип Дирихле}\\
Если есть n клеток и в них n+1 заяц, то $\exists$ клетка, где зайцев $\geq$2\\
при n=2 $x_{ij}$-в клетке i=1 или 2, j=1 или 2 или 3 заяц\\
Попробуем записать, что в каждой клетке$\leq 1$ заяц\\
1. каждый заяц ровно в одной клетке\\
$x_{11}+x_{21}$-заяц 1\\
$x_{12}+x_{22}$-заяц 2\\
$x_{13}+x_{23}$...\\
Если клеток $x_{1j}\overline{x_{2j}}\dots \overline{x_{kj}}\vee\overline{x_{1j}}x_{2j}x_{3j}\dots\overline{x_{ki}}\vee \overline{x_{1j}}\dots x_{kj}$\\
2. в каждой клетке не больше 1 зайца\\
\begin{tabular}{l|lll}
 кл$\backslash$з&1&2&3\\
\hline
0  &  $x_{11}$ &$x_{12}$&$x_{13}$ \\
1  &  $x_{21}$ &$x_{22}$&$x_{23}$ \\
\end{tabular}\\
Если есть 2 зайца, то один из конъюнктов\\
$\overline{x_{11}x_{12}\vee x_{11}x_{13}\vee x_{12}x_{13}}=1$ в клетке 1$\leq$1 зайца\\
$\overline{x_{11}x_{22}\vee x_{21}x_{23}\vee x_{22}x_{23}}$ в клетке 2$\leq$1 зайца\\
соединяем все утверждения\\
$(x_{11}+x_{21})(x_{12}+x_{22})(x_{13}+x_{23})(\overline{x_{11}x_{12}\vee x_{11}x_{13}\vee x_{12}x_{13}})*\\
*(\overline{x_{11}x_{22}\vee x_{21}x_{23}\vee x_{22}x_{23}})=0$ всегда\\
$(x_{11}\vee x_{21})(\overline{x_{11}}\vee\overline{x_{21}})(x_{12}\vee x_{22})(\overline{x_{12}}\vee\overline{x_{22}})*$\\
$(x_{13}\vee x_{23})(\overline{x_{13}}\vee \overline{x_{23}})(\overline{x_{11}}\vee\overline{x_{12}})(\overline{x_{11}}\vee \overline{x_{13}})*$\\
$(\overline{x_{12}}\vee \overline{x_{13}})(\overline{x_{21}}\vee \overline{x_{22}})(\overline{x_{21}}\vee\overline{x_{23}})(\overline{x_{22}}\vee \overline{x_{23}})*$- КНФ\\
Берем программу, которая решает задачу выполнимости: она скажет- невыполнима
\section{Классы замкнутости}
\underline{Определение}\\
Класс- это множество логических функций\\
Пример: $K_2$- класс функций от двух переменных: $f(x,y)=f(y,x)$\\
$f(x,y)=x\vee y \in K_1 \in K_2$\\
$g(x,y)=x\Rightarrow y \in K_1 \not\in K_2$\\
$K_3$- класс функций $f(x_1,\dots)-f(0,\dots)$ функции, которые не зависят от первой переменной\\
$f(x,y,z)=y\Rightarrow z \in K_3$\\
$f(x,y,z)=\overline{x}\vee x\vee y \vee z \in K_3$\\
$K_4:\{ f(x,y)=x\vee y\quad g(x,y)=x\Rightarrow y\}$\\
\underline{Определение}\\
$\bullet$ Замыкание класса\\
$K=\{f_1,f_2,\dots \}$\\
$K^*$-замыкание класса- это класс, состоящий из всех композиций функций из K\\
g(x,y,z)=
\begin{tikzpicture}
\node (x)[scale=0.9,style={rectangle,fill=black!10}][scale=0.9] at ( 0,5) {x};
\node (y)[scale=0.9,style={rectangle,fill=black!10}][scale=0.9] at (2,5) {y};
\node (z)[scale=0.9,style={rectangle,fill=black!10}][scale=0.9] at ( 4,5) {z};
\node (f1)[scale=0.9,style={rectangle,fill=black!10}][scale=0.9] at (1,4) {f1};
\node (f2)[scale=0.9,style={rectangle,fill=black!10}][scale=0.9] at (2,3){f2};
\node (f11)[scale=0.9,style={rectangle,fill=black!10}][scale=0.9] at (3,2){f1};
\node (g)[scale=0.9,style={rectangle,fill=black!10}][scale=0.9] at ( 3,1) {g(x,y,z)};
\path (x) edge [->,bend right] (f1);
\path (y) edge [->,bend right] (f1);
\path (f1) edge [->,bend right] (f2);
\path (y) edge [->,bend right] (f2);
\path (z) edge [->,bend right] (f2);
\path (f2) edge [->,bend right] (f11);
\path (z) edge [->,bend right] (f11);
\path (f11) edge [->] (g);
\end{tikzpicture}\\
$f_1(f_2(f_1(x,y),y,z),z)$-композиция функций\\
Пример\\
1)$k=\{0,\overline{x}\}$\\
$k^*=\{f(0),g(f(0)),g(g(f(0))),\dots \}$\\
$k^*=\{\overline{x},0,1,x \}$\\
2)$k=\{\overline{x}\}$\\
$k^*=\{\overline{x},x\}$\\
$k^*=\{g(x),g(g(x)),g(g(g(x))),\dots\}$\\
3)$k=\{\overline{x},x\vee y,xy \}$\\
$k^*=\{ x,\dots \forall $функция$\}$\\
\underline{Определение}\\
Если К-класс\\
$K^*=\alpha$(все логические функции), то К-полный\\
Вывод:$K=\{\overline{x},x\vee y,xy \}$-полный\\
Пример 4\\
$K=\{\overline{x},x\vee y \}$\\
$x\cdot y=\overline{\overline{x\cdot y}}=\overline{\overline{x}\vee \overline{y}}=f(g(f(x),f(y)))$\\
Значит $K^*$- тоже полный
\end{document}
%Оформил Горюнов Дмитрий